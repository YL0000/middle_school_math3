\documentclass[b5paper, openany]{ctexbook}


\usepackage[margin=2.5cm]{geometry}


\usepackage{pifont}
\usepackage[perpage,symbol*]{footmisc}
\DefineFNsymbols{circled}{{\ding{192}}{\ding{193}}{\ding{194}}
{\ding{195}}{\ding{196}}{\ding{197}}{\ding{198}}{\ding{199}}{\ding{200}}{\ding{201}}}
\setfnsymbol{circled}

\usepackage{ulem}

\usepackage{amsmath,amsfonts,mathrsfs,amssymb}
\usepackage{graphicx}

\usepackage[font=bf,labelfont=bf,labelsep=quad]{caption}

\usepackage{tikz}


\usepackage{ntheorem}
\theoremseparator{\;}



\usepackage{blkarray}
\usepackage{bm}
\usepackage[colorlinks=true, linkcolor=black]{hyperref}

\usepackage{enumerate}


\theoremstyle{plain}
\theoremheaderfont{\normalfont\bfseries} 
\theorembodyfont{\normalfont}


\usepackage[framemethod=tikz]{mdframed}


\newtheorem{example}{\bf 例}[chapter]
\newenvironment{solution}{\noindent {\bf 解:}}{}
\newenvironment{analyze}{\noindent {\bf 分析:}}{}
\newenvironment{rmk}{\noindent {\bf 注意:}}{}
\newenvironment{note}{\noindent {\bf 说明:}}{}



\renewcommand{\proofname}{\bf 证明:}
\newenvironment{proof}{{\noindent \bf 证明:}}{}%{\hfill $\square$\par}

\newcommand{\E}{\mathbb{E}}
\renewcommand{\Pr}{\mathbb{P}}
\newcommand{\EP}{\mathbb{E}^{\mathbb{P}}}
\newcommand{\EQ}{\mathbb{E}^{\mathbb{Q}}}
\newcommand{\dif}{\,{\rm d}}
\newcommand{\Var}{{\rm Var}}
\newcommand{\Cov}{{\rm Cov}}
\newcommand{\x}{\times}


 \usepackage{tcolorbox}
 \tcbuselibrary{breakable}
 \tcbuselibrary{most}



\newtcolorbox{ex}[1][]
  {colback = white, colframe = cyan!75!black, fonttitle = \bfseries,
    colbacktitle = cyan!85!black, enhanced,
    attach boxed title to top center={yshift=-2mm},breakable, 
    title=练习, #1}

\newtcolorbox{blk}[2][]
  {colback = white, colframe = magenta!75!black, fonttitle = \bfseries,
    colbacktitle = magenta!85!black, enhanced,
    attach boxed title to top left={xshift=5mm, yshift=-2mm},breakable, 
    title=#2, #1}


\setcounter{tocdepth}{2}

\setcounter{secnumdepth}{3}



\ctexset {
section = {
	name = {第,节},
 	number = \chinese{section}},
subsection = {
	name = {,、\hspace{-1em}},
	number = \chinese{subsection}
},
subsubsection = {
	name = {(,)\hspace{-1em}},
	number = \chinese{subsubsection},
}
}



\renewcommand{\contentsname}{目~~录}

\newcommand{\poly}{\polynomial[reciprocal]}



\usepackage{mathtools}

\setlength{\abovecaptionskip}{0.cm}
\setlength{\belowcaptionskip}{-0.cm}

\usetikzlibrary{decorations.pathmorphing, patterns}
\usetikzlibrary{calc, patterns, decorations.markings}
\usetikzlibrary{positioning, snakes}



\usepackage{yhmath}
\usepackage{longdivision}
\usepackage{polynom}
\usepackage{polynomial}
\usepackage{cancel}
\usepackage{longtable}

\renewcommand{\frac}{\dfrac}
\newcommand{\oc}{$^{\circ}{\rm C}$}
\renewcommand{\Vec}{\overrightarrow}

\usepackage{multicol}
\usepackage{cases}
\begin{document}














\title{中学数学实验教材\\第三册}



\author{中学数学实验教材编写组编}
\date{1983年6月}

\maketitle




\frontmatter

\chapter{前~~言}

这一套中学数学实验教材,内容的选取原则是精简实
用,教材的处理力求深入浅出,顺理成章,尽量作到使人人
能懂,到处有用。

    本教材适用于重点中学,侧重在满足学生将来从事理工
方面学习和工作的需要。

    本教材的教学目的是:使学生切实学好从事现代生产、
特别是学习现代科学技术所必需的数学基础知识;通过对数
学理论、应用、思想和方法的学习,培养学生运算能力,思
维能力,空间想象力,从而逐步培养运用数学的思想和方法
去分析和解决实际问题的能力;通过数学的教学和学习,培
养学生良好的学习习惯,严谨的治学态度和科学的思想方
法,逐步形成辩证唯物主义世界观。

   根据上述教学目的,本教材精选了传统数学那些普遍实
用的最基础的部分,这就是在理论上、应用上和思想方法上
都是基本的、长远起作用的通性、通法。比如,代数中的数
系运算律,式的运算,解代数方程,待定系数法;几何中的
图形的基本概念和主要性质,向量,解析几何;分析中的函
数,极限,连续,微分,积分;概率统计以及逻辑、推理论
证等知识。对于那些理论和应用上虽有一定作用,但发展余
地不大,或没有普遍意义和实用价值,或不必要的重复和过
于繁琐的内容,如立体几何中的空间作图,几何体的体积、
表面积计算,几何难题,因式分解,对数计算等作了较大的
精简或删减。

    全套教材共分六册。第一册是代数。在总结小学所学自
然数、小数、分数基础上,明确提出运算律,把数扩充到有
理数和实数系。灵活运用运算律解一元一次、二次方程,二
元、三元一次方程组,然后进一步系统化,引进多项式运
算,综合除法,辗转相除,余式定理及其推论,学到根式、
分式、部分分式。第二册是几何。由直观几何形象分析归纳
出几何基本概念和基本性质,通过集合术语、简易逻辑转入
欧氏推理几何,处理直线形,圆、基本轨迹与作图,三角比
与解三角形等基本内容。第三册是函数。数形结合引入坐
标,研究多项式函数,指数、对数、三角函数,不等式等。
第四册是代数。把数扩充到复数系,进一步加强多项式理论,
方程式论,讲线性方程组理论,概率(离散的)统计的初步
知识。第五册是几何。引进向量,用向量和初等几何方法综
合处理几何问题,坐标化处理直线、圆、锥线,坐标变换与
二次曲线讨论,然后讲立体几何,并引进空间向量研究空间
解析几何初步知识。第六册是微积分初步。突出逼近法,讲
实数完备性,函数,极限,连续,变率与微分,求和与积分。

本教材基本上采取代数、几何、分析分科,初中、高中
循环排列的安排体系。教学可按初一、初二代数、几何双科
并进,初三学分析,高一、高二代数(包括概率统计)、几
何双科并进,高三学微积分的程序来安排。

    本教材的处理力求符合历史发展和认识发展的规律,深
入浅出,顺理成章。突出由算术到代数,由实验几何到论证
几何,由综合几何到解析几何,由常量数学到变量数学等四
个重大转折,着力采取措施引导学生合乎规律地实现这些转
折,为此,强调数系运算律,集合逻辑,向量和逼近法分别
在实现这四个转折中的作用。这样既遵循历史发展的规律,
又突出了几个转折关头,缩短了认识过程,有利于学生掌握
数学思想发展的脉络,提高数学教学的思想性。

这一套中学数学实验教材是教育部委托北京师范大学、
中国科学院数学研究所、人民教育出版社、北京师范学院、
北京景山学校等单位组成的领导小组组织“中学数学实验教
材编写组”,根据美国加州大学伯克利分校数学系项武义教
授的《关于中学实验数学教材的设想》编写的。第一版印出
后,由教育部实验研究组和有关省市实验研究组指导在北
京景山学校、北京师院附中、上海大同中学、天津南开中
学、天津十六中学、广东省实验中学、华南师院附中、长春
市实验中学等校试教过两遍,在这个基础上编写组吸收了实
验学校老师们的经验和意见,修改成这一版《中学数学实验
教材》,正式出版,内部发行,供中学选作实验教材,教师
参考书或学生课外读物。在编写和修订过程中,项武义教授
曾数次详细修改过原稿,提出过许多宝贵意见。

    本教材虽然试用过两遍,但是实验基础仍然很不够,这
次修改出版,目的是通过更大范围的实验研究,逐步形成另
一套现代化而又适合我国国情的中学数学教科书。在实验过
程中,我们热忱希望大家多提意见,以便进一步把它修改好。

\begin{flushright}
    中学数学实验教材编写组\\
    一九八一年三月
\end{flushright}








\tableofcontents


\mainmatter

% \chapter{指数概念普遍化与对数}

\section{指数概念普遍化}
\subsection{引言}
这一节我们要把指数的概念加以推广,除了第一册学过
的整数指数、
零指数外,还要引入负整数指数,正、负分数
指数,为能够应用对数来简化乘法、除法、乘方、开方的计
算建立初步理论基础。

在推广过程中要注意以下几个方面:
\begin{enumerate}
\item 各种定义的条件。
\item 探索各种定义产生的逻辑过程,
\item 要熟练运用各种定义去计算。
\end{enumerate}


我们来回忆一下正整数指数幂的定义:

\begin{blk}{定义1}
$a^n=\overbrace{a\cdot a\cdots a}^{\text{$n$个}}$ ($n$是大于1的正整数),其中$a$称
为\textbf{底数},$n$称为\textbf{指数},$a^n$称为以$a$为底数、$n$为指数的\textbf{幂}.当$n=1$
时,规定$a^1=a$。
\end{blk}

在定义1中,对于底数$a$,没有任何限制,$a$
可以是正数,也可以是负数,也可以是零;而指数$n$,必须
是正整数,否则定义1就无意义!这样,我们称按定义1定
义的$a^n$为正整数指数幂。

我们可以证明正整数指数幂有下列性质:
\begin{blk}{性质}
\begin{enumerate}
    \item $a^m\cdot a^n=a^{m+n}$ \quad ($m,n$为正整数);
    \item $a^m\div a^n=a^{m-n}$\quad ($m,n$为正整数,且$m>n,\; a\ne 0$);
    \item $(a^m)^n=a^{mn}$\quad ($m,n$为正整数);
    \item $(ab)^n=a^nb^n$ \quad($n$为正整数);
    \item $\left(\frac{a}{b}\right)^n=\frac{a^n}{b^n}$\quad ($n$为正整数,$b\ne 0$)。
\end{enumerate}
\end{blk}

以后将会看到,将指数概念扩充到新的范围以后,这几
条性质依然保留,并且可以适当地合并。

\subsection{零指数与负整数指数}

【I】从正整数指数幂的性质可以看到:
\begin{itemize}
    \item 幂的\textbf{乘法}可以转化为指数\textbf{加法},(性质1)
    \item 幂的\textbf{除法}可以转化为指数\textbf{减法},(性质2)
    \item 幂的\textbf{乘方}可以转化指数\textbf{相乘}。(性质3)
\end{itemize}

要使正整数的加、减、乘、除,尤其是减、除通行无阻,
那么指数只限制在正整数范围是不行的,例如:$a^3\div a^2=
a^{3-2}=a^1=a$, 这个转化是没有问题的,但是
\[a^3\div a^3 \mathop{=}^{?} a^{3-3} \mathop{=}^{?} a^0=?\]
\[a^2\div a^4 \mathop{=}^{?} a^{2-4} \mathop{=}^{?} a^{-2}=?\]

要使指数的减法通行无阻,就会出现零指数和负整数指
数,而零指数和负整数指数是什么,过去从未见过,这就有
必要将指数的概念加以推广,给零指数和负整数下定义。

【II】探索零指数和负整数指数如何定义。

我们知道$a^m\div a^n=a^{m-n}$,
这里限制$a\ne 0$, 并且$m>n$, $m$、$n$均为正整数,现在把$m>
n$的条件取消,这样$m$可以等于$n$. 在$m=n$的时候,上面的公
式就是:
\[a^m\div a^n=a^n\div a^n=a^{n-n}=a^0\quad (a\ne 0)\]
而$a^m\div a^n$的实际内容是
\[a^m\div a^n=\frac{a^m}{a^n}=\frac{a^n}{a^n}=1\]

为了使公式$a^m\div a^n=a^{m-n}$
也适用于$m=n$的情况,就应该
使形式上的运算结果“$a^0$”与实际内容“1”一致起来,我们规
定$a$的零次幂等于1, 即$a^0=1\;\;  (a\ne 0)$就合理了。
用同样的方法,来探索$m<n$时的情况。

当$m<n$时,$a^m\div a^n=a^{m-n}=a^{-(n-m)}$,这里$-(n-m)$是个负整数,我们还没有规定$a^{-(n-m)}$的意
义。但另一方面,在$m<n$的条件下,$a^m\div a^n$的实际意义是:
\[a^m\div a^n=\frac{a^m}{a^n}=\frac{a^m\div a^n}{a^n\div a^n}=\frac{1}{a^{n-m}}\qquad (a\ne 0)\]
这样看来,为了使公式$a^m\div a^n=a^{m-n}$在$m<n$时也适用,规
定:
\[a^{-(n-m)}=\frac{1}{a^{n-m}}\quad (a\ne 0)  \]
就可以了,也就是说,$n$是正整数的时候,应规定:


【III】对零指数和负整数指数下定义

\begin{blk}{定义2}
    \[a^0=1\qquad  (a\ne 0)\]
\end{blk}

\begin{blk}{定义3}
    \[a^{-n}=\frac{1}{a^n} \qquad  (a\ne 0, \quad n\text{是正整数})\]
\end{blk}

这两个定义中特别要注意“$a\ne 0$”这个条件,也就是说零
的零次幂是没有意义的,零的负整数次幂也是没有意义的。

有了定义2和定义3, 对于同底数的整数指数幂相除的
运算法则,就可通行无阻,而不必再局限于$m>n$了,例
如
\[(-2)^2\div (-2)^2=(-2)^0=1\]
\[4^5\div 4^7=4^{5-7}=4^{-2}=\frac{1}{4^2}=\frac{1}{16}\]

定义1、2、3说明现在我们的指数已经扩充到整数范
围了。

【IV】性质的证明

一般来说,当一个概念被推广以后,原来具有的性质可
能有些仍然保持成立,有些就不再成立了。所以必须对原来
的性质逐条加以研究,对那些仍然成立的性质,加以证明肯
定,对那些不再成立的性质也应通过举反例加以否定。另
外,新的概念是否又带来了新的性质,这是研究推广了的概
念应该注意的一个问题,只有通过这样的研究,才能掌握推
广了的概念。

把指数概念推广到整数范围以后,上述五条性质,因条
件起了变化,就成为:

\begin{blk}{}
\begin{enumerate}
    \item $a^m\cdot a^n=a^{m+n}\quad (m,n\in\mathbb{Z},\;\; a\ne 0)$
    \item $a^m\div a^n=a^{m-n}\quad (m,n\in\mathbb{Z},\;\; a\ne 0)$
    \item $(a^m)^n=a^{mn}\quad (m,n\in\mathbb{Z},\;\; a\ne 0)$
    \item $(a\cdot b)^n=a^{n}\cdot b^n\quad (ab\ne 0,\;\; n\in\mathbb{Z})$
    \item $\left(\frac{a}{b}\right)^n=\frac{a^n}{b^n}\quad (ab\ne 0,\;\; n\in\mathbb{Z})$
\end{enumerate}    
\end{blk}

现在我们来证明性质1。

已知:$m,n$是任意两个整数,$a\ne 0$.

求证:$a^m\cdot a^n=a^{m+n}$

我们要对$m,n$分各种情况来讨论,由于$m,n$均可取正整
数、负整数和零,即
\[m=\begin{cases}
    0\\ \text{正整数}\\ \text{负整数}
\end{cases},\qquad n=\begin{cases}
    0\\ \text{正整数}\\ \text{负整数}
\end{cases}  \]
因而从$m$中取出一种情况,从$n$中取出一种情况,组成一种
情况,那么一共有$3\x3=9$种,在这九种情况中,因为$m
=$正整数,$n=$正整数的情况以前已证过,可以略去。又由
于实数乘法适合交换律,因而$m,n$是对称的,这样,我们只
就下面五种情况来证明:
\[\begin{cases}
    m=0\\n=0
\end{cases}\begin{cases}
    m=\text{正整数}\\n=0
\end{cases}\begin{cases}
    m=\text{负整数}\\n=0
\end{cases}\begin{cases}
m=\text{正整数}\\  n=\text{负整数}
\end{cases}\begin{cases}
    m=\text{负整数}\\  n=\text{负整数}
\end{cases}  \]

\textbf{情况1:} 当$m=0$, $n=0$时,由零指数的定义,
\[a^m\cdot a^n=a^0\cdot a^0=1\cdot 1=1=a^0=a^{m+n} \]
$\therefore\quad a^m\cdot a^n=a^{m+n}$成立。

\textbf{情况2:} 当$m$是正整数,$n=0$时,
\[a^m\cdot a^n=a^m\cdot a^0=a^m\cdot 1=a^m=a^{m+0}=a^{m+n}\]
$\therefore\quad a^m\cdot a^n=a^{m+n}$成立。


\textbf{情况3:} 当$m$是负整数,$n=0$时,
\[a^m\cdot a^n=a^m\cdot a^0=a^m\cdot 1=a^m=a^{m+0}=a^{m+n}\]
$\therefore\quad a^m\cdot a^n=a^{m+n}$成立。

\textbf{情况4:} 当$m$是正整数,$n$是负整数,那么$n=-|n|$, $|n|$为正整数。

$\because\quad a^m\cdot a^n=a^m\cdot a^{-|n|}=a^m\cdot\frac{1}{a^{|n|} }=\frac{a^m}{a^{|n|}}=a^{m-|n|}=a^{m+(-|n|)}=a^{m+n} $

$\therefore\quad a^m\cdot a^n=a^{m+n}$成立。

\textbf{情况5:} 当$m,n$都是负整数时,那么
$m=-|m|$, $n=-|n|$。因此:
\[\begin{split}
    a^m\cdot a^n&=a^{-|m|}\cdot a^{-|n|}=\frac{1}{a^{|m|}}\cdot \frac{1}{a^{|n|}}\\
    &=\frac{1}{a^{|m|}\cdot a^{|n|}}=\frac{1}{a^{|m|+|n|}}\\
    &=a^{-(|m|+|n|)}\\
    &=a^{-(|m|)+(-|n|)}=a^{m+n}
\end{split}\]
$\therefore\quad a^m\cdot a^n=a^{m+n}$成立。

综合五种情况及前面的分析,就证明了当$m,n$为任意整
数时,$a^m\cdot a^n=a^{m+n}$成立。

其它几条性质可以类似地证明。应当指出,有了定义2、3以后,对于任何整数$n$, 都有:
\[\frac{1}{a^n}=a^{-n}\qquad (a\ne 0)\]
这是因为:当$n$是正整数时,这就是定义3, 当$n$为零时,
 \[\frac{1}{a^n}=\frac{1}{a^0}=\frac{1}{1}=a^0=a^{-n}\]
 当$n$为负整数时,$n=-|n|$
\[\frac{1}{a^n}=\frac{1}{a^{-|n|}}=a^{|n|}=a^{-n}\]
这样,定义3虽然是对$n$为正整数来说的,有$a^{-n}=\frac{1}{a^n}$,
但实
际上暗示了$n$为任意整数时,$a^{-n}=\frac{1}{a^n}$
都有了确定的意义。

有了这个公式,上面的性质2就可归结到性质1
上。
事实上,只要1成立,那么对于任何整数$m,n$我们
有:
\[a^m\div a^n=a^m\cdot \frac{1}{a^n}=a^m\cdot a^{-n}=a^{m+(-n)}=a^{m-n}\]

有了这个公式,上面的性质5也可归结为性质3和4。
事实上,只要3和4成立,那么对于任何整数$m,n$,
我们有:
\[\begin{split}
    \left(\frac{a}{b}\right)^n&=\left(a\cdot\frac{1}{b}\right)^n=(a\cdot b^{-1})^n=a^n\cdot (b^{-1})^n\\
    &=a^n \cdot b^{-n}=a^n\cdot \frac{1}{b^n}=\frac{a^n}{b^n}
\end{split}\]
这样,五条性质可归结为三条性质:
\begin{enumerate}
    \item $a^m\cdot a^n=a^{m+n}\quad (a\ne 0;\;\;m,n\in\mathbb{Z})$
    \item $(a^m)^n=a^{mn}\quad (a\ne 0;\;\;m,n\in\mathbb{Z})$
    \item $(a\cdot b)^n=a^n\cdot b^n \quad (ab\ne 0;\;\;n\in\mathbb{Z})$
\end{enumerate}


\begin{example}
\begin{multicols}{2}  
    \begin{enumerate}
        \item $2^0=1$
        \item $(0.75)^0=1$
        \item $\left(-\sqrt{3}\right)^0=1$
        \item $0^0$无意义
        \item $10^{-3}=\frac{1}{10^3}=0.001$
        \item $3(-2)^{-4}=\frac{3}{(-2)^{4}}=\frac{3}{16}$
        \item $\left(\frac{1}{2}\right)^{-5}=2^5=32$
        \item $(-0.25)^{-1}=\left(-\frac{1}{4}\right)^{-1}=-4$
    \end{enumerate}
\end{multicols}   
\end{example}

\begin{example}
    把下列单项分式化成整式形式:
    \[\frac{a}{3bc^2}=\frac{1}{3}ab^{-1}c^{-2},\qquad \frac{3}{2a^{-3}b^{-1}c^2}=\frac{3}{2}a^3bc^{-2}  \]
\end{example}

\begin{example}
    把下面的数写成$a\cdot 10^n$的形式,这里$a$是含有一
位整数的数,$n$是任意整数(把一个数写成这种形式叫 科学
记数法)。
    \begin{enumerate}
        \item 1吨(t)$=1000$公斤(kg)$=10^3$公斤(kg)
        \item  1毫米(mm)$=0.001$米(m)$=10^{-3}$米(m)
        \item  1年$=31,556,925,975$秒(s)=$3.1556925975\x10^7$(s)$\approx 3.156\x10^7$(s)
        \item   地球到太阳的距离$=149,640,000$公里(km)
        $=1.4964\x10^8$(km)
        \item    地球的质量$=5,970,000,000,000,000,000,000$(t)
        $=5.97\x10^{21}$(t)
        \item    原子核$U_{288}$的半径$=0.000,000,000,000,93$(cm)$=9.3\x10^{-13}$(cm)
    \end{enumerate}
\end{example}

\begin{example}
\begin{enumerate}
    \item $(b^{-3})^{-2}=b^{(-3)(-2)}=b^6$
    \item \[\begin{split}
        \left(3a^{-2}b^2c^{-3}\right)\left(\frac{4}{5}ab^{-3}c^3\right)&=\frac{12}{5}a^{-2+ 1}b^{2+(-3)}c^{-3+3}\\
        &=\frac{12}{5}a^{-1}b^{-1}\\
        &=\frac{12}{5ab}
    \end{split}\]
    \item \[\begin{split}
        \left(\frac{a+b}{a-b}\right)^{-3}\left(\frac{a-b}{a+b}\right)^{-2}&=\left[\left(\frac{a+b}{a-b}\right)^{-1}\right]^{3} \left(\frac{a-b}{a+b}\right)^{-2}\\
        &=\left(\frac{a-b}{a+b}\right)^{3}\left(\frac{a-b}{a+b}\right)^{-2}\\
        &=\left(\frac{a-b}{a+b}\right)^{3-2}=\left(\frac{a-b}{a+b}\right)^{1}\\
        &=\frac{a-b}{a+b}
    \end{split}\]
\end{enumerate}
\end{example}

    
\begin{example}
\begin{enumerate}
    \item \[\begin{split}
        \frac{\left(a^{-2} b^{-3}\right)\left(-4 a^{-1} b\right)}{12 a^{-4} b^{-2}}&=\frac{-4 a^{-2-1} b^{-3+1}}{12 a^{-4} b^{-2}} \\
    &=\frac{-4 a^{-3} b^{-2}}{12 a^{-4} b^{-2}}\\
    &=-\frac{1}{3} a
    \end{split}\]
    \item \[\begin{split}
        \left(x^{2}-y^{-2}\right)\div\left(x-y^{-1}\right) 
    &=\left[x^{2}-\left(y^{-1}\right)^{2}\right] \div\left(x-y^{-1}\right) \\
    &=\left(x+y^{-1}\right)\left(x-y^{-1}\right)\div \left(x-y^{-1}\right) \\
    &=x+y^{-1}=x+\frac{1}{y}
    \end{split}\]
\end{enumerate}
\end{example}


\section*{习题1.1}
\addcontentsline{toc}{subsection}{习题1.1}

\begin{enumerate}
    \item 换去下列各算式中的负指数:
    \begin{multicols}{2}
        \begin{enumerate}
        \item $4x^{-3}y^3$
        \item $\frac{1}{5c^{-3}}$
        \item $\frac{4a^{-2}}{5b^{-3}}$
        \item $\frac{3a^{-3}x^2}{5b^3y^{-4}}$
    \end{enumerate}
    \end{multicols}
    
    \item 证明下面的运算法则,如果$m,n$为任意整数,而$a\ne 0$,
    则$(a^m)^n=a^{m\cdot n}$
    \item 把下列的数写成$a\cdot 10^n$的形式,这里$a$是含有一位整数的数,$n$是任意整数,
    \begin{multicols}{2}
        \begin{enumerate}
        \item 8900
        \item  3,200,000
        \item 0.000,015
        \item 0.000,000,025
    \end{enumerate}
    \end{multicols}
    
    
\item 下面是物理学常用的长度单位,将它化成mm并以10的幂表
示出来:
\begin{enumerate}
    \item $1\text{微米}(1\mu {\rm m})=\frac{1}{1000}{\rm mm}$
    \item $1\text{毫微米}(1 {\rm nm})=\frac{1}{1,000,000}{\rm mm}$ 
    \item  $1\text{微微米}(1 {\rm pm})=\frac{1}{1,000,000,000}{\rm mm}$ 
\end{enumerate}


\item 以10的幂来表示:
\begin{enumerate}
    \item 1mm是多少cm;
    \item 1${\rm cm}^3$是多少${\rm m}^3$ (Litre即升);
    \item 1g是多少kg, 是多少吨(t)。
\end{enumerate}
\item 将下面的数据用小数形式表示出来:
\begin{enumerate}
    \item 红血球的直径$=0.7\x10^{-8}$(cm);
    \item 最小的细菌的长度$\approx 10^{-4}$(cm);
    \item 钠光(黄色)的波长$=5.89\x10^{-7}$(cm);
    \item 氢原子的直径$\approx 10^{-8}$(cm);
    \item 氢原子的质量$=1.64\x10^{-24}$(g);
    \item 电子的质量$=9\x10^{-28}$(g);
    \item 铀矿石的含镭量$=3.328\x10^{-6}$\%.
\end{enumerate}

\item 完成下列运算并将所得结果中的负指数变换成正指数:
\begin{multicols}{2}
 \begin{enumerate}
    \item $(2x^2)\div (3x^{-3})$
    \item $ax^2\div x^{-1}$
    \item $(a^{-2}x^2)\div a^4$
    \item $(ax)^{-3}\div (bx)^{-3}$
    \item $(2^n)^{n-1} \div (2^{n-1})^{n+1}\quad (n>1)$
\end{enumerate}   
\end{multicols}

\item 利用整数指数幂的指数法则,计算
\begin{multicols}{2}\begin{enumerate}  
\item $\left\{\left[\frac{5}{3}-\left(\frac{6}{5}\right)^{-1}\right]^{-2}-\left(\frac{25}{11}\right)^{-1}\right\}^{-3}$\item $\left[\left(\frac{5 a^{-2} c^{3}}{3 x^{-3} y^{4}}\right)^{-2}\right]^{4}$
\item $\frac{\left(3 x^{2} y^{-3}\right)^4}{\left(-2 x^{-3} y^{2}\right)^{-3}\left(-27 x^{-5} y^{2}\right)}$
\item $\left(\frac{2}{a^{n}+a^{-n}}\right)^{-2}-\left(\frac{2}{a^{n}-a^{-n}}\right)^{-2}$
\end{enumerate}\end{multicols}

\item 求证:
\begin{enumerate}
    \item $\left(a+a^{-1}\right)^{2}\left(a-a^{-1}\right)^{2}=a^{4}-2+\frac{1}{a^{4}}$
    \item $\frac{a^{-3}+b^{-3}}{a^{-1}+b^{-1}}+\frac{a^{-3}-b^{-3}}{a^{-1}-b^{-1}}=2\left(\frac{1}{a^{2}}+\frac{1}{b^{2}}\right)$
\end{enumerate}

\item 化简:
 \begin{multicols}{2}
\begin{enumerate}
    \item  $\frac{a^{2}+a^{-2}-2}{a^{2}-a^{-2}}$
    \item  $\frac{m^{3}+n^{-3}}{m+n^{-1}}+\left(m-n^{-1}\right)^{2}$
\end{enumerate}
\end{multicols}
\item 求$32x^{-6}+12x^{-4}+10x^{-2}-12$除
以$2x^{-2}-1$所得的商式
和余数。
\end{enumerate}

\subsection{$n$次算术根}
为了进一步把指数概念从整数范围扩充到有理数范围,
我们先介绍一下$n$次算术根及其运算性质。

\begin{blk}{定义}
 设$a$是一个实数,$n$是正整数,如果存在着实数
$x$, 使得
$$x^n=a$$
那么,$x$就叫做\textbf{$a$的$n$次方根},$a$叫做\textbf{被开方数},$n$叫做\textbf{根指数}。   
\end{blk}
 
由方根的定义推知:
\begin{enumerate}
    \item 如果$x_1$是正数$a$的偶次方根,那么$x_1$也是$a$的偶次
    方根。

    因为如果存在$x_1$使得$x_1^n=a\; (>0)$, 那么,$(-x_1)^n=(x_1)^n=a$ ($n$为偶数),即$x_1$和$-x_1$都是正数$a$的偶次方根.比如,
$x^4=16$, 则$x=\pm 2$都是16的四次方根。
\item 负数的偶次方根不存在,这是因为任何数的偶次方
都是非负数的缘故。
\item 不等于零的任何数的奇次方根的符号与被开方数的
符号相同。

因为如果被开方数是负数$-a\; (a>0)$, 那么它的奇次方
根,就不能是0或正数,由于0的任何奇次方是0, 正数的
任何奇次方是正数,这就说明负数的奇次方根应该是负数;
同样说明正数的奇次方根应该是正数。
\item 0的方根是0。
\end{enumerate}

在本教程的第六册中,我们将证明下面的存在定理。


\begin{blk}{定理}
    如果$a>0$, 方程$x^n=a$存在唯一的正实数根。
\end{blk}

为明确起见,我们把这个正实数方根叫做$a$的算术根,下
面给出它的定义和记法。


\begin{blk}{定义}
    如果$a$是正实数,那么$a$的正的$n$次方根叫做\textbf{$a$的$n$次
算术根},而零也叫零的$n$次算术根,记为$\sqrt[n]{a}$。$a$的算术平方根
用不带根指数的符号$\sqrt{n}$表示。$\sqrt[1]{a}$这个表示$a$的符号,我们不用。
\end{blk}

例如:$2^6=32$, 2叫做32的5次方根,也是32的5次算术根,
因此$\sqrt[5]{32}=2$; $(\pm 2)^4=16$,$2$和$(-2)$都是16的4次方根,其中2是16的4次算术根,因此$\sqrt[4]{16}=2$; $(-2)$不是16的4次算术根。因此$\sqrt[4]{16}\ne -2$, 但$-2$可写成$-\sqrt[4]{16}$。$(-3)^3=-27$, $-3$
叫做$-27$的3次方根,但不是算术根。有的课本也用符号$\sqrt[n]{a}$
($a<0$, $n$是奇数)表示负数的奇次方根,例如$\sqrt[3]{-27}=-3$。
但是负数的奇次方根总可以用对应的算术根表示出来,例如
$\sqrt[3]{-27}=-\sqrt[3]{27}$, 一般地,$\sqrt[2n+1]{-a}=-\sqrt[2n+1]{a}\quad  (a>0)$。

请注意以下几点:

\begin{enumerate}
    \item 算术根的概念含有两个条件:被开方数是非负的;
算术根只代表方根中的非负值。以后在实数范围内,符号
$\sqrt[n]{a}$($a>0$)只代表算术根,因此,正数$a$的偶次方根的负值
用$-\sqrt[n]{a}$表示。
\item 在等式$x^n=a$中,已知$x$和$n$求$a$的过程叫做乘方运算。
反过来,已知$a$和$n$求$x$的过程叫做开方运算,乘方运算和开
方运算互为逆运算,由于当$a\ge 0$时,$\sqrt[n]{a}$是非负值,因此对
于一切大于1的正整数$n$都有:
\begin{align}
    \left(\sqrt[n]{a}\right)^n&=a\\
    \sqrt[n]{a^n}&=a
\end{align}

事实上,设$\sqrt[n]{a}=x$, 按定义,$x^n=a$, 所以$\left(\sqrt[n]{a}\right)^n=a$。

在$a<0$的场合,可以验知:当$n$是奇数时,等式(1.1)、
(1.2)对于负数的奇次方根仍保持;而当$n$是偶数时,等式(1.1)
无意义,而(1.2)的右端是$|a|$, 不是$a$, 即$a=|a|$, 这是因
为算术根只代表非负数的缘故。

例如:$\sqrt{(-2)^2}+\sqrt[3]{-27}+\sqrt[5]{243}=\sqrt{(-2)^2}+\sqrt[3]{(-3)^3}+\sqrt[5]{3^5}=2-3+3=2$

\item 符号$\sqrt[n]{\quad }$叫做$n$次根号,如果根号下是一个算式,例如$\sqrt{1-a}$,$\sqrt[3]{\frac{x}{y}}$
等等,我们称它为根式。当根号下的算式
的值不为负数时,它的$n$次算术根才有意义。

例如:$\sqrt{(a-2)^2}=\begin{cases}
    a-2 & a\ge 2\\
    2-a & a<2
\end{cases},\quad \sqrt{(1-a)} \text{ 在$a\le 1$的情形下有意义}$

\end{enumerate}

下面我们介绍$n$次算术根的性质。

\begin{blk}{性质1:乘积的开方法则}
设$n$为正整数,$a,b$为正实数,那么
$$\sqrt[n]{ab}=\sqrt[n]{a}\cdot \sqrt[n]{b}$$
\end{blk}

\begin{proof}
    因为$\left(\sqrt[n]{a}\cdot \sqrt[n]{b}\right)^n=\left(\sqrt[n]{a}\right)^n\cdot \left(\sqrt[n]{b}\right)^n=ab$
    
    按方根定义,$\sqrt[n]{a}\cdot \sqrt[n]{b}$是$ab$的$n$次方根,又因为$a,b$是正
    数,当然积$ab$仍是正数,且算术根的积$\sqrt[n]{a}\cdot \sqrt[n]{b}$也是正数,
    这就是说,$\sqrt[n]{a}\cdot \sqrt[n]{b}$是正数$ab$的$n$次算术根;另一方面,$ab$
    的$n$次算术根是$\sqrt[n]{ab}$, 根据$ab$的$n$次算术根的唯一性得到
    $$\sqrt[n]{ab}=\sqrt[n]{a}\cdot \sqrt[n]{b}$$
\end{proof}

\begin{blk}{推论}
\[\sqrt[n]{a_1a_2\cdots a_n}=\sqrt[n]{a_1}\cdot \sqrt[n]{a_2}\cdots \sqrt[n]{a_n}\]
\end{blk}

例如:\[\begin{split}
    \sqrt[3]{60\x 18\x 25}&=\sqrt[3]{(2^2\x 3\x 5)(2\x 3^2)(5^2)}\\
    &=\sqrt[3]{2^3\x 3^3\x 5^3}\\
    &=2\x 3\x 5=30
\end{split}\]

\begin{blk}{性质2:幂的开方法则}
设$m,n$是正整数,$a>0$, 那么,$\sqrt[n]{a^m}=\left(\sqrt[n]{a}\right)^m$。

\end{blk}

\begin{proof}
由性质1的推论得,
\[\sqrt[n]{a^m}=\underbrace{\sqrt[n]{a\cdot a\cdots a}}_{\text{$m$个}}=\underbrace{\sqrt[n]{a}\cdots \sqrt[n]{a}}_{\text{$m$个}}=\left(\sqrt[n]{a}\right)^m\]
\end{proof}

\begin{blk}{性质3:商的开方法则}
    设$n$为正整数,$a,b>0$, 那么,
\[\sqrt[n]{\frac{a}{b}}=\frac{\sqrt[n]{a}}{\sqrt[n]{b}}\] 
\end{blk}

\begin{proof}
$\because\quad \left(\frac{\sqrt[n]{a}}{\sqrt[n]{b}}\right)^n=\frac{\left(\sqrt[n]{a}\right)^n}{\left(\sqrt[n]{b}\right)^n}=\frac{a}{b}$

又$\because\quad a>0,\quad   b>0,\quad \frac{a}{b}>0,\quad \frac{\sqrt[n]{a}}{\sqrt[n]{b}}>0$

$\therefore\quad \frac{\sqrt[n]{a}}{\sqrt[n]{b}}$是$\frac{a}{b}$的$n$次算术根

$\therefore\quad \sqrt[n]{\frac{a}{b}}=\frac{\sqrt[n]{a}}{\sqrt[n]{b}}$

\end{proof}

\begin{blk}{性质4:根式的开方法则}
    设$m,n$是正整数,$a>0$, 那么,
\[\sqrt[m]{\sqrt[n]{a}}=\sqrt[mn]{a}\]
\end{blk}

\begin{proof}
    $\because\quad \left(\sqrt[m]{\sqrt[n]{a}}\right)^{mn}=\left[\left(\sqrt[m]{\sqrt[n]{a}}\right)^m\right]^n=\left(\sqrt[n]{a}\right)^n=a$

    又$\because\quad a>0,\quad \sqrt[m]{\sqrt[n]{a}}>0 $
    
    $\therefore\quad \sqrt[n]{a}$是$a$的$mn$次算术根,

    $\therefore\quad \sqrt[m]{\sqrt[n]{a}}=\sqrt[mn]{a}$
\end{proof}


\begin{blk}{性质5:幂指数与根指数相约法则}
设$m,n,k$为正整数,$a>0$, 那么,
\[\sqrt[nk]{a^{mk}}=\sqrt[n]{a^m} \]  
\end{blk}

\begin{proof}
$\because\quad \left(\sqrt[n]{a^m}\right)^{nk}=\left[\left(\sqrt[n]{a^m}\right)^n\right]^k=\left(a^m\right)^k=a^{mk}$

又$\because\quad a>0,\quad \sqrt[n]{a^m}>0,\quad a^{mk}>0 $
    
$\therefore\quad \sqrt[n]{a^m}$是$a^{mk}$的$nk$次算术根,

$\therefore\quad \sqrt[nk]{a^{mk}}=\sqrt[n]{a^m} $
\end{proof}

最后强调,这些性质是在$a>0$的前提下成立的,对于
$a<0$, 有些根式也许仍有意义,但法则就不适用了,例如,性
质5:

设$a=-8$, 那么,
$\sqrt[6]{(-8)^2}=\sqrt[6]{64}=2$, 而$\sqrt[3]{-8}=\sqrt[3]{(-2)^3}=-2$,
因此,$\sqrt[6]{(-8)^2}\ne \sqrt[3]{-8}$。



\begin{example}
    化简下列各式:
\begin{enumerate}
\item $\sqrt[4]{25 x^{2} y^{2}} \quad(x, y \ge 0)$
\item  $\sqrt[6]{5^{4} a^{4} b^{2}}\quad (a, b \ge 0)$
\item $\sqrt{3 x^{2 a+2}}\quad (x \ge 0)$
\item $\sqrt[6]{27(u+v)^{18}}\quad (u,v\ge 0)$
\item $(a-b) \sqrt{\frac{a^{2}+a b}{a^{2}-2 a b+b^{2}}}\quad (a, b>0)$
\item $\sqrt[3]{16 \sqrt{2}}$
\item $\sqrt[4]{4 x^{6}}\quad (x \in \mathbb{R})$
\end{enumerate}
\end{example}

\begin{solution}
\begin{enumerate}
    \item $\sqrt[4]{25 x^{2} y^{4}}=\sqrt[4]{\left(5 x y^{2}\right)^{2}}=\sqrt{5 x y^{2}}$
    \item $\sqrt[6]{5^{4} a^{4} b^{2}}=\sqrt[6]{\left(5^{2} a^{2} b\right)^{2}}=\sqrt[3]{5^{2} a^{2} b}=\sqrt[3]{25 a^{2} b}$
    \item $\sqrt{3 x^{2 n+2}}=\sqrt{3} \cdot \sqrt{\left(x^{n+1}\right)^{2}}=\sqrt{3} x^{n+1}$
    \item \[\begin{split}
        \sqrt[6]{27(u+v)^{18}}&=\sqrt[6]{3^3\left[(u+v)^{6}\right]^{3}}\\
        &=\sqrt[6]{[3 (u+v)^{6}]^3}=\sqrt{3(u+v)^{6}}
        \\
        &=\sqrt{3}\sqrt{(u+v)^6}=\sqrt{3}(u+v)^3
    \end{split}\]
    \item \[\begin{split}
        (a-b) \sqrt{\frac{a^{2}+a b}{a^{2}-2 a b+b^{2}}}&=(a-b) \frac{\sqrt{a^2+ab}}{\sqrt{(a-b)^2}}\\
        &=(a-b)\frac{\sqrt{a(a+b)}}{|a-b|}\\
        &=\begin{cases}
            \sqrt{a(a+b)}, & a>b>0\\
            -\sqrt{a(a+b)}, & b>a>0
        \end{cases}
    \end{split}\]
\item \[\begin{split}
    \sqrt[3]{16 \sqrt{2}}&=\sqrt[3]{\sqrt{16^2}\cdot \sqrt{2}}\\
    &=\sqrt[3]{\sqrt{16^2 \cdot 2}}=\sqrt[6]{16^2 \cdot 2}\\
    &=\sqrt[6]{2^8\cdot 2}=\sqrt[6]{2^9}=\sqrt{2^3}=2\sqrt{2}
\end{split}\]
\item \[\begin{split}
    \sqrt[4]{4 x^{6}}&=\sqrt[4]{2^2 |x|^{6}}=\sqrt{2|x|^3}\\
    &=\sqrt{2\cdot |x|^2\cdot |x|}=\sqrt{2}|x|\sqrt{|x|}\\
    &=\begin{cases}
        \sqrt{2}x\sqrt{x} , & x\ge 0\\
        -\sqrt{2}x\sqrt{-x} , & x<0
    \end{cases}
\end{split}\]
\end{enumerate}
\end{solution}


\begin{example}
    把$\sqrt{2}$,$\sqrt[3]{-3}$,$\sqrt[4]{4}$化成同次式(指数相同的
    根式叫做同次根式)
\end{example}

\begin{solution}
$\because\quad \sqrt[3]{-3}=-\sqrt[3]{3},\quad \sqrt[4]{4}=\sqrt{2}$

$\therefore\quad $只需要考虑将$\sqrt{2}$, $\sqrt[3]{3}$化成同次根式
\[\begin{split}
\sqrt{2}&=\sqrt[6]{2^3}=\sqrt[6]{8}\\
\sqrt[3]{-3}&=-\sqrt[3]{3}=-\sqrt[6]{3^2}=-\sqrt[6]{9}\\
\sqrt[4]{4}&=\sqrt{2}=\sqrt[6]{8}
\end{split}\]
\end{solution}

\begin{ex}
\begin{enumerate}
    \item 求出下列方根的值:
    \begin{multicols}{2}
\begin{enumerate}
    \item $\sqrt[4]{10000}$
    \item $\sqrt[4]{1}$
    \item $\sqrt[4]{256}$
    \item $\sqrt[4]{\frac{1}{16}}$
    \item $\sqrt[4]{\frac{1}{625}}$
    \item $\sqrt[5]{100000}$
    \item $\sqrt[5]{0}$
    \item $\sqrt[5]{0.00001}$
    \item $\sqrt{\frac{4}{9}}$
    \item $\sqrt[4]{\frac{1}{81}}$
    \item $\sqrt[6]{64}$
    \item $\sqrt[3]{0.064}$
    \item $\sqrt[5]{0.00243}$
    \item $\sqrt[10]{0}$
    \item $\sqrt[3]{-2 \frac{10}{27}}$
\end{enumerate}        
    \end{multicols}

\item 计算下面方根,准确到0.1
\[\sqrt[4]{25},\qquad \sqrt[6]{27},\qquad \sqrt[8]{16},\qquad \sqrt[10]{32} \]

\item 约简下列根式中被开方数的指数和根指数:
\begin{multicols}{2}
\begin{enumerate}
    \item $\sqrt[6]{36 x^{2}}$
    \item $\sqrt[4]{25 y^{2}}$
    \item $\sqrt[8]{2^{4} a^{4} b^{6}}$
    \item $\sqrt[16]{14^{4^{4m} b^{8m}}}$
\end{enumerate}
\end{multicols}
这里,$a>0,\;\; b>0,\quad x,y\in\mathbb{R}$


\item 计算下列各题(题中字母都是正的):
\begin{multicols}{2}
    \begin{enumerate}
    \item $\sqrt[3]{\frac{8 x^{3} y^{6}}{27 x^{6} y^{9}}}$
    \item $\sqrt[n]{\frac{a^{n} b^{2n}}{x^{3 n} y^{n}}}$
    \item $\sqrt{\frac{25(a+b)^{2}}{(c-d)^{4}}}$
    \item $\sqrt[3]{-\frac{(a-b)^{3 n}}{(x+y)^{6n}}}$
    \item $\sqrt[n]{\frac{(a+b)^{2 n}}{a^{3 n}(a-b)^{n}}}$
\end{enumerate}
\end{multicols}

\item 化简下列各式:
\begin{multicols}{2}
    \begin{enumerate}
    \item $\sqrt{2 \sqrt{3}}$
    \item $\sqrt[3]{2 \sqrt{5}}$
    \item $\sqrt{3\sqrt[3]{2}}$
    \item $\sqrt[4]{5 \sqrt{2}}$
    \item $\sqrt{\sqrt[3]{a^{4} b^{2}}}$
    \item $\sqrt{x^{3} \sqrt[3]{x}}$
    \item $\sqrt{\frac{x}{y} \sqrt{\frac{x}{y}}}$
    \end{enumerate}
\end{multicols}
\end{enumerate}
     
\end{ex}


\subsection{最简根式和同类根式}
根式作运算,计算结果用根式表示时,根式应为最简根
式,最简根式指:
\begin{enumerate}
    \item 被开方数的每一个因式的指数都小于根指数;
    \item 根号内的式子不含分母;
    \item 根指数与被开方
数的指数互质。
\end{enumerate}

把根式化为最简根式的方法是:
\begin{enumerate}
    \item 移因式到根号外,例如:\[\sqrt[n]{a^nb}=\sqrt[n]{a^n}\cdot \sqrt[n]{b}=a\sqrt[n]{b}\;\;(a\ge 0)\]
    \item 移因式到根号内,例如:\[a\sqrt[3]{\frac{1}{a^2}}=\sqrt[3]{a^3}\cdot \sqrt[3]{\frac{1}{a^2}}=\sqrt[3]{\frac{a^3}{a^2}}=\sqrt[3]{a}\;\;(a>0)\]
    \item 化去根号内式子的分母,例如
    \begin{enumerate}
        \item 如果$ab>0$, 那么$\sqrt{\frac{a}{b}}=\sqrt{\frac{ab}{b^2}}=\frac{\sqrt{ab}}{\sqrt{b^2}}=\frac{\sqrt{ab}}{|b|}$
        \item 如果$a>0$, $b>0$, 且$n>m$, 那么
        \[\sqrt[n]{\frac{a}{b^m}}=\sqrt[n]{\frac{ab^{n-m}}{b^mb^{n-m}}}=\frac{\sqrt[n]{ab^{n-m}}}{\sqrt[n]{b^n}}=\frac{\sqrt[n]{ab^{n-m}}}{b} \]
    \end{enumerate}
\item 约去根指数与被开方数的指数的公因数,例如,
\[\sqrt[6]{8x^3}=\sqrt[6]{(2x)^3}=\sqrt{2x}\quad (x>0)\]
\end{enumerate}

\begin{example}
    化下面根式为最简根式:
\begin{enumerate}
    \item \[\begin{split}
        \frac{5 a^{2}}{7 b}\sqrt{\frac{49 b^{3}}{5 a}} &=\frac{5 a^{2}}{7 b} \sqrt{\frac{\left(7^{2} b^{2} b\right)(5 a)}{5^{2} a^{2}}} \\
    &=\frac{5 a^{2}}{7 b} \cdot \frac{7 b}{5 a} \sqrt{5 a b} \\
    &=a \sqrt{5 a b} \qquad(a>0, b>0) \\
    \end{split}\]
    \item \[\begin{split}
        \frac{2 a^{2}}{3 b} \sqrt[3]{\frac{b^{3}}{a^{4}}-\frac{b^{5}}{a^{6}}} &=\frac{2 a^{2}}{3 b}\sqrt[3]{\frac{a^{2} b^{3}-b^{5}}{a^{6}}} \\
    &=\frac{2 a^{2}}{3 b} \sqrt[3]{\frac{b^{3}\left(a^{2}-b^{2}\right)}{a^{6}}} \\
    &=\frac{2 a^{2}}{3 b} \cdot \frac{b}{a^{2}} \cdot \sqrt[3]{a^{2}-b^{2}} \\
    &=\frac{2}{3} \sqrt[3]{a^{2}-b^{2}} \qquad(a \neq 0, b>0)
    \end{split}\]
\end{enumerate}
\end{example}

\begin{example}
    作下面根式的乘法和除法:
\begin{enumerate}
    \item $$5\sqrt[4]{2a}\cdot \sqrt[4]{8a^3}=5\sqrt[4]{16a^4}=5\sqrt[4]{(2a)^4}=5\x 2a=10a\qquad (a\ge 0)$$
    \item \[\begin{split}
        \sqrt{\frac{3}{4}} \cdot \sqrt[4]{\frac{4}{3}}&=\sqrt[4]{\left(\frac{3}{4}\right)^{2}} \cdot \sqrt[4]{\frac{4}{3}}=\sqrt[4]{\left(\frac{3}{4}\right)^{2} \cdot \frac{4}{3}}\\
&=\sqrt[4]{\frac{3}{4}}=\sqrt{\frac{3}{2^{2}}}=\sqrt[4]{\frac{3 \times 2^{2}}{2^{4}}}=\frac{\sqrt[4]{12}}{2}
    \end{split}\]
    \item $\frac{16 \sqrt{3}}{\sqrt{2}}=\frac{16 \sqrt{3} \cdot \sqrt{2}}{\sqrt{2} \cdot \sqrt{2}}=8 \sqrt{6}$
    \item $\frac{5}{\sqrt[3]{4}}=\frac{5 \sqrt[3]{2}}{\sqrt[3]{2^2} \sqrt[3]{2}}=\frac{5 \sqrt[3]{2}}{\sqrt[3]{2^{3}}}=\frac{5}{2} \sqrt[3]{2}$
\end{enumerate}
\end{example}

几个根式化成最简根式后,如果根指数相同,根号内的
式子也相同,这几个根式叫做同类根式。


\begin{example}
    $\sqrt[3]{8ax^3}$和$\sqrt[6]{64a^2y^{12}}$是同类根式吗?
\end{example}
    
\begin{solution}
    由于:
\[\begin{split}
    \sqrt[3]{8ax^3}&=2x\sqrt[3]{a}\\
    \sqrt[6]{64a^2y^{12}}&=2y^2\sqrt[6]{a^2}=2y^2\sqrt[3]{a}\qquad (x\ge 0,\;\; a\ge 0)
\end{split}\]    
    $\therefore\quad \sqrt[3]{8ax^3}$和$\sqrt[6]{64a^2y^{12}}$同类根式。
\end{solution}


\begin{example}
    $\sqrt{\frac{2x}{3}}$, $\sqrt{\frac{6}{x}}$, $\sqrt{6x}$是同类根式吗?
\end{example}


\begin{solution}
    由于:
\[\begin{split}
    \sqrt{\frac{2x}{3}}&=\sqrt{\frac{2x\cdot 3}{3\cdot 3}}=\sqrt{\frac{6x}{3^2}}=\frac{\sqrt{6x}}{3}=\frac{1}{3}\sqrt{6x} \\
    \sqrt{\frac{6}{x}}&= \sqrt{\frac{6x}{x^2}}=\frac{\sqrt{6x}}{x}=\frac{1}{x}\sqrt{6x}
\end{split}\]    
    $\therefore\quad \sqrt{\frac{2x}{3}},\; \sqrt{\frac{6}{x}},\;  \sqrt{6x}$是同类根式。
\end{solution}

根式相加减,就是把同类根式分别合并。


\begin{example}
    \[\begin{split}
       &\quad  \frac{2}{3}x\sqrt{9x}+6x\sqrt{\frac{x}{4}}-x^2\sqrt{\frac{1}{x}}\\
&=2x\sqrt{x}+3x\sqrt{x}-x\sqrt{x}\\
&=4x\sqrt{x}
    \end{split}\]
\end{example}
    
\begin{example}
    \[\begin{split}
        &\quad 15\sqrt[3]{4}-3\sqrt[3]{32}-16\sqrt[3]{\frac{1}{16}}-\sqrt[3]{108}\\
        &=15\sqrt[3]{4}-6\sqrt[3]{4}-4\sqrt[3]{4}-3\sqrt[3]{4}\\
        &=2\sqrt[3]{4}
    \end{split}\] 
\end{example}

\begin{ex}
\begin{enumerate}
    \item 说明下面根式是同类根式:
\begin{multicols}{2}
\begin{enumerate}
    \item $\sqrt[3]{24}$和$\sqrt[3]{81}$
    \item $\sqrt[3]{54}$和$\sqrt[3]{16}$
    \item $\sqrt{216}$和$\sqrt{\frac{3}{8}}$
    \item $\sqrt[3]{\frac{72}{343}}$和$\sqrt[3]{41\frac{2}{3}}$
    \item $\sqrt[4]{\frac{1}{27}}$和$\sqrt[4]{0.1875}$
\end{enumerate}
\end{multicols}
    \item  将下面根式化为同次根式:
    \begin{multicols}{2}
\begin{enumerate}
    \item $\sqrt{2},\; \sqrt[3]{5}$
    \item $\sqrt[3]{2},\; \sqrt[4]{3}$
    \item $\sqrt[4]{x^8},\; \sqrt[6]{y^5}$
    \item $\sqrt[3]{xy^2},\; \sqrt{yz},\; \sqrt[4]{xz^3}$
    \item $\sqrt{\frac{1}{a}+\frac{1}{x}},\; \sqrt[3]{\frac{1}{a}-\frac{1}{x}}$
\end{enumerate}
\end{multicols}
    \item 作下面根式的加减法:
\begin{enumerate}
    \item $\sqrt[3]{40}+\left(\frac{3}{2}\sqrt[3]{-5}-2\sqrt[3]{\frac{1}{5}}\right)$
    \item $\left(3\sqrt[3]{32}+\sqrt[3]{\frac{1}{9}}-\sqrt[3]{108}\right)-\left(16\sqrt[3]{\frac{1}{16}}-4\sqrt[3]{\frac{1}{72}}\right)$
\end{enumerate}
    \item 作下面根式的乘除法:
\begin{multicols}{2}
\begin{enumerate}
    \item $\sqrt{a}\cdot \sqrt[4]{\frac{x}{a}}$
    \item $\sqrt{\frac{2}{3}}\cdot \sqrt[3]{\frac{3}{2}}\cdot \sqrt[6]{\frac{1}{2}}$
    \item $\left(\sqrt{2}-\sqrt[3]{4}+\sqrt[4]{8}\right)\cdot \sqrt{2}$
    \item $\sqrt{140}\div \sqrt{20}$
    \item $\sqrt[4]{8}\div \sqrt{2}$
    \item $\sqrt[6]{3}\div \sqrt[12]{18}$
    \item $\left(\sqrt[9]{8}\right)^4$
    \item $\left(\sqrt[9]{27}\right)^4$
\end{enumerate}
\end{multicols}
\end{enumerate}
\end{ex}

\subsection{分数指数幂}
我们再把指数幂的概念由整数指数幂推广到分数指数
幂,先来探索如何合理地定义$a^{\tfrac{1}{4}}$,$a^{\tfrac{3}{4}}$的意义。

假设符号$a^{\tfrac{1}{4}}$,$a^{\tfrac{3}{4}}$
有意义,并且适合整数指数幂法则$(a^m)^n=a^{mn}$,
那么对于$a^{\tfrac{1}{4}}$,$a^{\tfrac{3}{4}}$
应用这个法则就得到
$\left(a^{\tfrac{1}{4}}\right)^4=a$和$\left(a^{\tfrac{3}{4}}\right)^4=a^3$。这就是说,可以把$a^{\tfrac{1}{4}}$,$a^{\tfrac{3}{4}}$
看作方程$x^4=a,\; x^4=a^3$的根。实际上这两个方程的唯一正实数解分别是$\sqrt[4]{a}$和$\sqrt[4]{a^8}$, 即有等式$\left(\sqrt[4]{a}\right)^4=a$, $\left(\sqrt[4]{a^3}\right)^4=a^3$成
立。因此,我们定义$a^{\tfrac{1}{4}}=\sqrt[4]{a}$,$a^{\tfrac{3}{4}}=\sqrt[4]{a^3}$是合理的。

下面给出一般的定义:
\begin{blk}{定义4}
若$a>0$,$m,n$是正整数,我们规定:
 \[a^{\tfrac{m}{n}}=\sqrt[n]{a^m},\qquad a^{-\tfrac{m}{n}}=\frac{1}{a^{\tfrac{m}{n}}}\]
\end{blk}

例如:
\[5^{\tfrac{3}{4}}=\sqrt[4]{5^3},\quad 2^{\tfrac{3}{2}}=\sqrt{a^3},\quad 4^{\tfrac{1}{3}}=\sqrt[3]{4},\quad a^{-\tfrac{4}{3}}=\frac{1}{a^{\tfrac{4}{3}}}=\frac{1}{\sqrt[3]{a^4}} \]

有了定义4, 我们的指数就推广到了有理数了。

对于分数指数,还需要讨论它的合理性问题。这个问题
的提法是这样的。设有一个正有理数$r$, 按照有理数的性质,
一定有两个正整数$m,n$, 使得$r=\frac{m}{n}$,
那么按照定义4,
\[a^r=a^{\tfrac{m}{n}}=\sqrt[n]{a^m}\qquad (a>0)\]
但另一方面,$r$还有其它正分数表示法,例如,
$\frac{2m}{2n}$
就是另一个不同的表示法,设
$\frac{m_1}{n_1}$是$r$的另一个任意正分数表示法,
那么按定义4,
\[a^r=a^{\tfrac{m_1}{n_1}}=\sqrt[n_1]{a^{m_1}}\qquad (a>0)\]

于是,对于同一个正有理数$r$, $a^r$就有很多(实际上是无
穷多)个形式上不同的表示式,而$a^r$当然被规定为一个确定
的数,所以必须证明$a^r$的任何两个不同的表示式是相等的。
这就是下面的定理。

\begin{blk}{定理}
    设$a$是任意给定的正实数,$m,n,m_1,n_1$是正整
    数且$\frac{m}{n}=\frac{m_1}{n_1}$,
    则$a^{\tfrac{m}{n}}=a^{\tfrac{m_1}{n_1}}$   
\end{blk}

\begin{proof}
由于$\frac{m}{n}=\frac{m_1}{n_1}$,因而$mn_1=m_1n$,故
\[a^{mn_1}=a^{m_1n}\]

$\because\quad a^{\tfrac{m}{n}}=\sqrt[n]{a^m}=\sqrt[nn_1]{a^{mn_1}}\qquad 
a^{\tfrac{m_1}{n_1}}=\sqrt[n_1]{a^{m_1}}=\sqrt[nn_1]{a^{m_1n}}$

$\therefore\quad \sqrt[nn_1]{a^{mn_1}}=\sqrt[nn_1]{a^{m_1n}}$,即:
\[a^{\tfrac{m}{n}}=a^{\tfrac{m_1}{n_1}}\]
\end{proof}
    
这就解决了定义的合理性的问题。不仅如此,它还告诉
我们,可以改变有理数的分母以适应各种不同的需要。

例如:$5^{\tfrac{1}{2}}=5^{\tfrac{2}{4}}=5^{\tfrac{3}{6}}=\cdots$

\begin{rmk}
    分指数幂的定义不考虑底是负数的情形,因为这
    时分指数幂不再具有上述的重要性质.例如,$(-1)^{\tfrac{1}{3}}=\sqrt[3]{-1}=-1$, 同时$(-1)^{\tfrac{2}{6}}=\sqrt[6]{(-1)^2}$=1, 所以$(-1)^{\tfrac{1}{3}}\ne (-1)^{\tfrac{2}{6}}$。又$0^{-\tfrac{m}{n}}$没有
意义,因此分指数幂的底限制为正数。
\end{rmk}

有了分数指数幂的定义,上面讲过的三条性质在新的范
围内就可叙述为:
\begin{blk}{性质}
\begin{enumerate}
    \item  $a^r\cdot a'^s=a^{r+s}$\quad ($r,s$是有理数,$a>0$);
    \item  $(a^r)^s=a^{rs}$\quad ($r,s$是有理数,$a>0$);
    \item  $(ab)^r=a^r\cdot b^r$\quad ($r$是有理数,$a,b>0$)。
\end{enumerate}  
\end{blk}

我们只对性质1作出证明,其它性质的证明留给同学
自己去考虑。

性质1的证明:设$a>0$, $r,s$为有理数,我们证明
$a^r a^s=a^{r+s}$

\begin{proof}
\textbf{情形1 } 若$r,s$都是正有理数时,则$r=\frac{m}{n}$, $s=\frac{\ell}{k}$, 
这里$m,n,\ell,k$都是正整数。
\[\begin{split}
a^r a^s&= a^{\tfrac{m}{n}}\cdot a^{\tfrac{\ell}{k}} =\sqrt[n]{a^m}\cdot \sqrt[k]{a^{\ell}}\\
&=\sqrt[nk]{a^{mk}}\cdot \sqrt[nk]{a^{\ell n}}\\
&=\sqrt[nk]{a^{mk}\cdot a^{n\ell}}=\sqrt[nk]{a^{mk+n\ell}}\\
&=a^{\tfrac{mk+n\ell}{nk}}=a^{\tfrac{mk}{nk}+\tfrac{n\ell}{nk}}\\
&=a^{\tfrac{m}{n}+\tfrac{1}{k}}=a^{r+s}
\end{split}\]
$\therefore\quad a^r a^s=a^{r+s}$

\textbf{情形2 } 若$r<0$, $s<0$, 则$r=-|r|$, $s=-|s|$.
\[\begin{split}
a^r\cdot a^s &= a^{-|r|}a^{-|s|}=\frac{1}{a^{|r|}}\cdot \frac{1}{a^{|s|}}\\
&=\frac{1}{a^{|r|}a^{|s|}}=\frac{1}{a^{|r|+|s|}}\\
&=a^{-(|r|+|s|)}=a^{-|r|+(-|s|)}\\
&=a^{r+s}
\end{split}\]
$\therefore\quad a^r a^s=a^{r+s}$


\textbf{情形3 } 若$r>0$, $s<0$, 则$|s|=\frac{\ell}{k}$, $r=\frac{m}{n}$, 
$m,n,\ell,k$都是正整数,则
\[\begin{split}
a^r a^s&=a^r\cdot  a^{-|s|}=\frac{a^r}{a^{|s|}}\\
&=\frac{a^{\tfrac{m}{n}}}{a^{\tfrac{\ell}{k}}}=\frac{\sqrt[n]{a^m}}{\sqrt[k]{a^{\ell}}}\\
&=\frac{\sqrt[nk]{a^{mk}}}{\sqrt[nk]{a^{n\ell}}}=\sqrt[nk]{\frac{a^{mk}}{a^{n\ell}}}\\
&=\sqrt[nk]{a^{mk-n\ell}}=a^{\tfrac{mk-n\ell}{nk}}\\
&=a^{\tfrac{m}{n}-\tfrac{\ell}{k}}=a^{r-|s|}\\
&=a^{r+s}
\end{split}\]
$\therefore\quad a^r\cdot  a^s=a^{r+s}$

(若$r<0$, $s<0$则根据交换律也是成立的)

此外,$r,s$有一个为零的情形,性质1显然成立,故
$a^ra^s=a^{r+s}$对任意有理数$r,s$成立。
\end{proof}

\begin{example}
求下面各分指数幂的值:
\begin{enumerate}
\item $18^{\tfrac{1}{2}}=\left(3^2\cdot 2\right)^{\tfrac{1}{2}}=3\cdot 2^{\tfrac{1}{2}}=3\sqrt{2}$
\item $100^{-\tfrac{3}{2}}=(10^2)^{-\tfrac{3}{2}}=10^{-3}=\frac{1}{10^3}=0.001$
\item $\left(\frac{81}{625}\right)^{-\tfrac{3}{4}}=\left[\left(\frac{3}{5}\right)^4\right]^{-\tfrac{3}{4}}=\left(\frac{3}{5}\right)^{-3}=\left(\frac{5}{3}\right)^3=\frac{125}{27}$
\end{enumerate}
\end{example}
    
\begin{example}
    用分指数幂作下面根式运算:
\begin{enumerate}
    \item $\sqrt[4]{a^3}\div \sqrt[3]{a}=a^{\tfrac{3}{4}}\cdot a^{-\tfrac{1}{3}}=a^{\tfrac{3}{4}-\tfrac{1}{3}}=a^{\tfrac{5}{12}}=\sqrt[12]{a^5}$
    \item $\sqrt{a^3\cdot a\sqrt{a}\cdot a^6 \sqrt[3]{a}}=\left(a^3\cdot a^{1\tfrac{1}{2}}\cdot a^{6\tfrac{1}{3}}\right)^{\tfrac{1}{2}}=\left(a^{10\tfrac{5}{6}}\right)^{\tfrac{1}{2}}=a^{5\tfrac{5}{12}}=a^{5}\cdot \sqrt[12]{a^5}$
\end{enumerate}
    
\end{example}

\begin{example}
    化简下面算式:
\begin{enumerate}
    \item \[\begin{split}
&\frac{5 x^{-\tfrac{2}{3}} y^{\tfrac{1}{2}}}{\left(-\frac{1}{4}x^{-1} y^{-\tfrac{1}{3}}\right)\left(-\frac{5}{6} x^{-\tfrac{1}{3}} y^{-\tfrac{1}{6}}\right)} \\
&=24 x^{-\tfrac{2}{3}+1+\tfrac{1}{3}} y^{\tfrac{1}{2}+\tfrac{1}{3}+\tfrac{1}{6}}=24 x^{\tfrac{2}{3}} y \\
&=24 y \sqrt[3]{x^{2}}
\end{split}\]
\item \[\begin{split}
& \quad   \left(a^{2} x+a x^{1.5} \right)\left(a^{1.5} x^{0.5}+a^{0.5} x\right)^{-1}\\
&=(a x)\left(a+x^{0.5}\right)\left[a^{0.5} x^{0.5} \left(a+x^{0.5}\right)\right]^{-1}\\
&=(a x)(a x)^{-0.5}\left(a+x^{0.5}\right)\left(a+x^{0.5}\right)^{-1}\\
&=(a x)^{0.5}\left(a+x^{0.5}\right)^{0}\\
&=(a x)^{0.5}=\sqrt{a x}
\end{split}\]
\item \[\begin{split}
 &\quad   \frac{m-n}{m^{\tfrac{1}{2}}-n^{\tfrac{1}{2}}}-\frac{m^{\tfrac{3}{4}}+n^{\tfrac{3}{4}}}{m^{\tfrac{1}{4}}+n^{\tfrac{1}{4}}} \\
&=\frac{\left(m^{\tfrac{1}{2}}\right)^2-\left(n^{\tfrac{1}{2}}\right)^2}{m^{\tfrac{1}{2}}-n^{\tfrac{1}{2}}}-\frac{\left(m^{\tfrac{1}{4}}\right)^3+\left(n^{\tfrac{1}{4}}\right)^3}{m^{\tfrac{1}{4}}+n^{\tfrac{1}{4}}} \\
&=m^{\tfrac{1}{2}}+n^{\tfrac{1}{2}}-\left[\left(^{\tfrac{1}{4}}\right)^2- m^{\tfrac{1}{4}}n^{\tfrac{1}{4}}+\left(n^{\tfrac{1}{4}}\right)^2\right]\\
&=m^{\tfrac{1}{4}}n^{\tfrac{1}{4}}=\sqrt[4]{mn}
\end{split}\]
\end{enumerate}
\end{example}

由例1.15看出,有了分数指数,根式就可转化为幂,根式
的运算转化为幂的运算,这就有可能简化计算工作,为以后
的对数计算在理论上作了准备。

\section*{习题1.2}

\addcontentsline{toc}{subsection}{习题1.2}
\begin{enumerate}
    \item 求下面分指数幂的值:
\begin{multicols}{3}
\begin{enumerate}
    \item $8^{\tfrac{4}{3}}$
    \item $16^{-\tfrac{3}{2}}$
    \item $9^{-\tfrac{5}{2}}$
    \item $\left(\frac{27}{8}\right)^{-\tfrac{1}{3}}$
    \item $25^{-\tfrac{1}{2}}$
    \item $\left(2\frac{1}{4}\right)^{-\tfrac{3}{2}}$
    \item $\left(3\frac{3}{8}\right)^{-\tfrac{1}{3}}$
    \item $(0.008)^{-\tfrac{2}{3}}$
    \item $54^{\tfrac{1}{3}}$
    \item $(500)^{\tfrac{1}{3}}$
\end{enumerate}
\end{multicols}
(i),(j)化为最简根式。

\item 按照$\left(\frac{2}{5}\right)^{\tfrac{1}{3}}=\left(\frac{2\cdot 25}{5\cdot 25}\right)^{\tfrac{1}{3}}=\frac{50^{\tfrac{1}{3}}}{5} $ 的样子将分母变成有理数:
\begin{multicols}{3}
\begin{enumerate}
    \item $\left(\frac{2}{3}\right)^{\tfrac{1}{2}}$
    \item $\left(\frac{5}{12}\right)^{\tfrac{1}{2}}$
    \item $\left(\frac{1}{2}\right)^{\tfrac{1}{3}}$
    \item $\left(\frac{4}{3}\right)^{\tfrac{1}{3}}$
    \item $\left(\frac{2}{5}\right)^{\tfrac{1}{4}}$
    \item $\left(\frac{4}{3y^2}\right)^{\tfrac{1}{4}}$
    \item $\left(\frac{1}{10x^3}\right)^{\tfrac{1}{4}}$
    \item $\left(\frac{2x}{3y^2}\right)^{\tfrac{1}{3}}$
\end{enumerate}
\end{multicols}

\item 通过扩分(应用分式的基本性质)将下列分式的分母变成有理数:

例:$\frac{1}{a^{\tfrac{2}{3}}}=\frac{1\cdot a^{\tfrac{1}{3}}}{a^{\tfrac{2}{3}}\cdot a^{\tfrac{1}{3}}}=\frac{a^{\tfrac{1}{3}}}{a}=\frac{\sqrt[3]{a}}{a}$
\begin{multicols}{3}
    \begin{enumerate}
\item $\frac{4}{8^{\tfrac{1}{2}}}$
\item $\frac{10}{5^{\tfrac{1}{3}}}$
\item $\frac{1}{x^{\tfrac{1}{4}}}$
\item $\frac{1}{a^{\tfrac{2}{3}}}$
\item $\frac{10a}{(2a^3)^{\tfrac{1}{3}}}$
    \end{enumerate}
\end{multicols}

\item 按照$2\cdot \left(\frac{1}{4}\right)^{\tfrac{1}{3}}=\left(\frac{8}{4}\right)^{\tfrac{1}{3}}=2^{\tfrac{1}{3}}$的方法做
\begin{multicols}{2}
    \begin{enumerate}
\item $2\cdot \left(\frac{1}{2}\right)^{\tfrac{1}{3}}$
\item $3\cdot \left(\frac{1}{3}\right)^{\tfrac{1}{3}}$
\item $\frac{1}{2} \left(64\right)^{\tfrac{1}{5}}$
\item $a \left(\frac{1}{a}\right)^{\tfrac{3}{2}}$
    \end{enumerate}
\end{multicols}
\item 用根式表示指数幂:
\begin{multicols}{3}
    \begin{enumerate}
\item $a^{\tfrac{4}{5}}$
\item $b^{-\tfrac{1}{2}}$
\item $c^{-\tfrac{3}{5}}$
\item $\left(a^{\tfrac{4}{7}}\right)^{-3}$
\item $a^{\tfrac{1}{3}}b^{\tfrac{2}{3}}c^{-\tfrac{1}{3}}$
    \end{enumerate}
\end{multicols}

\item 用正指数幂表示下列根式:
\begin{multicols}{2}
    \begin{enumerate}
\item $\sqrt[4]{b^{-3}}$
\item $\left(\sqrt{2}\right)^{-\tfrac{3}{5}}$
\item $\left(\frac{1}{\sqrt[4]{x^{-5}}}\right)^{-2}$
\item $\frac{\sqrt[3]{a^2}}{\sqrt[3]{b}}$
    \end{enumerate}
\end{multicols}
\item 用有理指数幂表示下列各式:
\begin{multicols}{2}
    \begin{enumerate}
\item $\sqrt[3]{a^4}$
\item $\sqrt[5]{b^{11}}$
\item $\frac{1}{c\sqrt[5]{c^4}}$
\item $\sqrt{\frac{1}{x^5}}$
\item $\frac{3}{\sqrt[5]{y^3}}$
\item $\frac{a^{\tfrac{1}{2}}}{\sqrt{5x^3}}$
\item $\frac{2\sqrt{a^{-3}}}{3\sqrt[3]{5^5}}$
    \end{enumerate}
\end{multicols}


\item 用分指数幂计算:
\begin{multicols}{2}
    \begin{enumerate}
\item $2\sqrt{2}\cdot \sqrt[4]{2}\cdot \sqrt[8]{2}$
\item $\frac{\sqrt{x}\cdot \sqrt[3]{x^2}}{x\sqrt[6]{x}}$
\item $\frac{a^5\sqrt[3]{a}}{\sqrt{a^3}\sqrt[6]{a^3}}$
\item $\sqrt{\sqrt[3]{4}}$
\item $\sqrt[3]{a\sqrt[4]{a^3}}$
\item $\left(\sqrt[3]{5}-\sqrt{125}\right)\div \sqrt[4]{5}$
\item $\sqrt[3]{a^{-2}}\cdot \sqrt{a^{-3}}$
    \end{enumerate}
\end{multicols}

\item 化简下列各式:
\begin{multicols}{2}
    \begin{enumerate}
\item $a^{\tfrac{1}{4}}a^{\tfrac{1}{3}}$
\item $\left(\frac{1}{2}x^{\tfrac{1}{2}}y^{-\tfrac{1}{2}}z\right)\left(-\frac{2}{3}x^{-1}y^{-\tfrac{1}{3}}z^{\tfrac{1}{2}}\right)$
\item $\left(x^{\tfrac{1}{2}}y^{-\tfrac{2}{3}}\right)^6$
\item $\frac{15a^{\tfrac{1}{2}}b^{\tfrac{1}{3}}c^{\tfrac{3}{4}}}{25a^{-\tfrac{1}{2}}b^{\tfrac{1}{3}}c^{\tfrac{5}{4}}}$
\item $a^{\tfrac{1}{3}}b^{\tfrac{2}{3}}\div a^{-\tfrac{2}{3}}b^{\tfrac{4}{3}}$
\item $\left(a^{\tfrac{1}{2}}\sqrt[3]{b^2}\right)^{-3}\div \sqrt{b^{-4}\sqrt{a^{-2}}}$
\item $\frac{a^{-\tfrac{1}{2}}b^{-\tfrac{3}{4}}}{a^{\tfrac{1}{3}}b^{\tfrac{1}{2}}}\div \frac{a^{\tfrac{5}{6}}}{b^{-\tfrac{1}{6}}}$
\item $\left(8b^{-\tfrac{1}{3}}x^{-\tfrac{4}{3}}b\sqrt[4]{x^{\tfrac{4}{3}}}\right)^{\tfrac{1}{3}}$
\item $\left[\left(x^{\tfrac{1}{m-n}}\right)^{m-\tfrac{n^2}{m}}\right]^{\tfrac{m}{m+n}}$
\item $x\sqrt{x\sqrt{x\sqrt{x}}}$
    \end{enumerate}
\end{multicols}

\item 分解下列各式为因式的积:
\[a^{\tfrac{2}{3}}-b^{\tfrac{2}{3}},\qquad x^{\tfrac{3}{2}}-y^{\tfrac{3}{2}},\qquad x^{-3}-27y^{-3} \]

\item 计算:
\begin{multicols}{2}
    \begin{enumerate}
\item $\left(a^{\tfrac{1}{2}}b^{\tfrac{1}{2}}\right)^2$
\item $\left(a^{\tfrac{1}{3}}-b^{-\tfrac{2}{3}}\right)\left(a^{\tfrac{2}{3}}+b^{-\tfrac{2}{3}}+b^{-\tfrac{4}{3}}\right)$
\item $\left(a^{m}+a^{\tfrac{m}{2}}+1\right)\left(a^{-m}+a^{\tfrac{m}{2}}+1\right)$
\item $\left(m^{\tfrac{3}{2}}+n^{\tfrac{3}{2}}\right)\div \left(m^{\tfrac{1}{2}}+n^{\tfrac{1}{2}}\right)$
\item $\frac{a-b}{a^{\tfrac{1}{3}}-b^{\tfrac{1}{3}}}-\frac{a+b}{a^{\tfrac{1}{3}}+b^{\tfrac{1}{3}}}$
    \end{enumerate}
\end{multicols}
\end{enumerate}

\subsection{无理指数幂}
由于无理指数幂的概念要用到实数完备性或极限存在定
理,我们这里不作详细介绍,只指出$a^{\alpha}$($a>0$, $\alpha$是无理数),例如$2^{\sqrt{2}}$, $10^{\sqrt{8}}$, $3^{\pi}$等等,仍有确定意义,即它仍
代表一个确定的实数,并且也满足指数运算的三个法则:
\[\begin{split}
    a^{\alpha}\cdot a^{\beta}&=a^{\alpha+\beta}\\
    \left(a^{\alpha}\right)^{\beta}&=a^{\alpha\beta}\\
    (ab)^{\alpha}&=a^{\alpha}\cdot b^{\alpha}
\end{split}\]
这里$a$, $b$大于零;$\alpha$, $\beta$是无理数。

于是指数法则可以进一步推广,得到下面的普遍定理。

\begin{blk}{定理}
    指数运算法则$a^{\alpha}\cdot a^{\beta}=a^{\alpha+\beta}$, $\left(a^{\alpha}\right)^{\beta}=a^{\alpha\beta}$, $(ab)^{\alpha}=a^{\alpha}\cdot b^{\alpha}$ ($a,b>0$)对于任何实数$\alpha, \beta$成立。
\end{blk}

在实际应用中,我们常用有理指数幂去近似地代替无理
指数幂,例如$\sqrt{2}\approx 1.414$, $10^{\sqrt{2}}\approx 10^{1.414}$。
因此,在这里
我们只要求同学知道上述结论就可以了。

\section{对数和常用对数}
\subsection{对数的定义}

在前面我们引入了有理指数的概念,并指出对于正实数
$a$与任意实数$\alpha$, $a^{\alpha}$都有明确的意义;在学习指数时,我们回
顾一下,同底指数幂有一些运算公式:
\[\begin{split}
    a^{m}\cdot a^n&=a^{m+n}\\
    a^{m}\div a^n&=a^{m-n}\\
    (a^m)^n&=a^{mn}\\
    \sqrt[n]{a}&=a^{\tfrac{m}{n}}
\end{split}\]
这里幂指数$m,n$为任意实数,根指数$n$是大于1的自然数,
$a$为正实数。

这就是说,同底的幂相乘转化为指数相加;同底的幂相
除转化为指数相减;幂的乘方转化为指数相乘;幂的开方转
化为指数相除,对于数值的计算,加法和减法显然比乘法和
除法容易得多,譬如我们用2的$n$次幂可以简捷地算出式子:
$M=\frac{512^2\x 64}{32^3\x 256}$的值。
\[M=\frac{(2^9)^2\x 2^6}{(2^5)^3\x 2^8}=2^{18+6-15-8}=2^1=2\]
因此,我们自然就希望把这一性质用到实际计算工作中去,
使计算简化。

现在的问题是,任意两个正数相乘,能否应用上述简化
的思想来计算?即,如果$M$和$N$均为任意正实数,能否实现
下述过程:
\begin{center}
\begin{tikzpicture}[>=latex]
\node at (0,0){\Large  $N\x M=a^b\x a^c=a^{b+c}\quad (a>0,\; \text{且} a\ne 1)$};
\draw[->](-4.7,-.3)--(-4.7,-1)--(-2.5,-1)--(-2.5,-.3);
\draw[->](-3.5,.3)--(-3.5,1)--(-1.3,1)--(-1.3,.3);
\node at (-3,0){\Large ?};
\end{tikzpicture}
\end{center}

\begin{rmk}
    因为1以外的正数不可能等于1的任何次幂,所
以这里必须限定$a\ne 1$。
\end{rmk}

要把任意正数的乘法转化为同底的幂的乘法,从而用指
数相加来完成,关键在于任意一个正数$N$能否写成一个已知
数$a$ ($a>0$且$a\ne1$)的幂$a$的形式,换句话说,就是给了一个
不等于1的正实数$a$作为底数,那么对于任意给定的正实数
$N$, 是否有唯一的实数$b$存在,使得$N=a$, 只要这个问题解
决了,上述的想法就可实现,回答是肯定的,这就是:

\begin{blk}{定理}
    设$a>0$且$a\ne 1$, 那么对于任意给定的正实数$N$
存在唯一的实数$b$, 使得
\[a^b=N\]
\end{blk}
 
定理的严格证明需要较多的理论,我们将在第六册给出
它的证明。

现在我们利用这个定理再介绍一个新的概念和一个新的
符号如下:

\begin{blk}{定义}
    设$a$是一个不等于1的正实数,$N$是任意给定的
正实数,如果实数$b$使得等式
\[a^b=N\]
成立,那么$b$就叫做\textbf{以$a$为底$N$的对数},记为$\log_a N=b$, $N$叫
做\textbf{真数}。
\end{blk}
 
从定义里可以看出,下面两个式子是等价的:
\[a^b=N  \Longleftrightarrow \log_a N=b\]
前者叫做\textbf{指数式},后者叫做\textbf{对数式}。

由定义知道,求对数是求方幂的一种逆运算。若给出底数
$a$和指数$b$就是求方幂$N$, 反过来,若给出底数$a$和方幂$N$, 
求指数$b$, 就是求对数$b=\log_a N$; 若给出指数$b$和方幂$N$, 求底数$a$, 这就是求方根$a=N^{\tfrac{1}{b}}$, 这三个等式,$\log_a N=b$, $a^b=N$
和$a=N^{\tfrac{1}{b}}$是等价的,即如果$a$、$b$、$N$三个数满足其中一
个等式,那么它们也满足另外两个等式。




\begin{example}
    将下列指数式换成对数式:
\begin{enumerate}
    \item $10^2=100\quad \Longleftrightarrow \quad \log_{10}100= 2$
    \item $3^5=243\quad \Longleftrightarrow \quad\log_3 243=5$
    \item $4^{\tfrac{1}{4}}=\sqrt{2}\quad \Longleftrightarrow \quad \log_4\sqrt{2}=\frac{1}{4}$
    \item $2^10=1024\quad \Longleftrightarrow \quad \log_2 1024=10$
    \item $3^{-1}=\frac{1}{3}\quad \Longleftrightarrow \quad \log_3\frac{1}{3}=-1$
\end{enumerate}
\end{example}



\begin{example}
    求下面等式中的$x$值:
\begin{enumerate}
    \item $\log_{64}x=-\frac{2}{3}  \quad \Longleftrightarrow \quad x=(64)^{-\tfrac{2}{3}}=\left(4^3\right)^{-\tfrac{2}{3}}=4^{-2}=\frac{1}{16} $
    \item  $ \log_x 8 =6 \quad \Longleftrightarrow \quad x^6=8,\quad x=(2^3)^{\tfrac{1}{6}}=\sqrt{2} $
    \item  $ \log_9 27=x \quad \Longleftrightarrow \quad 9^x=27,\quad 3^{2x}=3^3 $
    
    $\therefore\quad 2x=3,\quad x=\frac{3}{2}$
\end{enumerate}
\end{example}

同学们想一想怎样证明下面两个对数的重要性
质:
\[\log a=1\qquad (a>0,\; a\ne 1)\]
即底的对数恒等于1;
\[\log 1=0\qquad (a>0,\; a\ne 1)\]
即1的对数,对于任何底恒等于0。

对某数连续地完成两个互逆的运算得到的数就是原来这
个数,例如
$$(2+3)-3=2,\qquad (2\x3)\div 3=2$$
因此如
果给出数$N$, 求出以$a$为底$N$的对数以后,接着再求以$a$为底
该对数为指数的幂,结果仍等于$N$. 这一点可将对数式
$\log_a N=b$代入指数式$a^b=N$中的$b$得到下面的恒等式:
\begin{equation}
    a^{\log_a N}=N
\end{equation}

这个恒等式也说明$\log_a N$就是方程$a^b=N$的唯一解。

如果给出数$b$, 我们求出$a$的$b$次幂后,接着再求这个幂
以$a$为底的对数,这相当于把方幂$a^b=N$代入等式$\log_a N=b$中
的真数$N$得到恒等式:
\begin{equation}
    b=\log_a a^b
\end{equation}



\begin{example}
计算
\begin{multicols}{3}
\begin{enumerate}
    \item $3^{\log_3 243}$
    \item $\log_3 27$
    \item $4^{1+\log_4\sqrt{2}}$
    \item $\log_{10}0.1$
    \item $5^{\log_5 2 -1}$
\end{enumerate}
\end{multicols}
\end{example}

\begin{solution}
\begin{enumerate}
    \item $3^{\log_3 243}=243$
    \item $\log_3 27=\log_3 3^3=3$
    \item $4^{1+\log_4\sqrt{2}}=4\cdot 4^{\log_4\sqrt{2}}=4\sqrt{2}$
    \item $\log_{10}0.1=\log_{10}10^{-1}=-1$
    \item $5^{\log_5 2 -1}=\frac{5^{\log_5 2}}{5}=\frac{2}{5}$
\end{enumerate}    
\end{solution}

\begin{example}
 试将以下四式写成2的方幂:
 \[\sqrt{2},\qquad \sqrt{8},\qquad 3,\qquad \sqrt[3]{3}\]   
\end{example}
  
\begin{solution}
\begin{enumerate}
    \item $\sqrt{2}=2^{\tfrac{1}{2}}$
    \item $\sqrt{8}=2^{\tfrac{3}{2}}$
    \item $3=2^{\log_2 3}$
    \item $\sqrt[3]{3}=2^{\log_2 \sqrt[3]{3}}$
\end{enumerate}
\end{solution} 


\begin{ex}
\begin{enumerate}
    \item 计算下列各式的值:
\begin{multicols}{3}
\begin{enumerate}
    \item $2^{0.5}\cdot 8^{0.5}$
    \item $\frac{10^{-2.5}}{10^{0.5}}$
    \item $\left(5^{\tfrac{1}{3}}\right)^3$
    \item $\left(\frac{49}{225}\right)^{-\tfrac{1}{2}}$
    \item $(16\x 81)^{-0.25}$
\end{enumerate}
\end{multicols}   

\item 求下列各式中的$x$, 指出哪个是幂的运算?哪个是求对
数?哪个是开方运算?
\begin{multicols}{3}
    \begin{enumerate}
        \item $3^4=x$
        \item $x^3=1000$
        \item $10^x=0.001$
    \end{enumerate}
    \end{multicols}  
\item 求真数:
\begin{multicols}{3}
\begin{enumerate}
    \item $\log_2 x=3$
    \item $\log_4 x=-2$
    \item $\log_3 N=0$
    \item $\log_3 N=1$
    \item $\log_{\sqrt{2}} x=4$
    \item $\log_{0.1} x=-1$
\end{enumerate}
\end{multicols}  
\item 求底数:
\begin{multicols}{2}
\begin{enumerate}
    \item $\log_x 216 = 3 $
    \item $\log_x \frac{1}{81} = 4 $
    \item $\log_x \frac{1}{64} = -3 $
    \item $\log_x \sqrt{8} = \frac{3}{4} $
\end{enumerate}
\end{multicols}  

\item 求对数:
\begin{multicols}{3}
\begin{enumerate}
    \item $\log_{10} 10000=x$
    \item $\log_{10} 0.001=x$
    \item $\log_4 256=x$
    \item $\log_{\tfrac{1}{3}} 9=x$
    \item $\log_3 \frac{1}{27}=x$
    \item $\log_9\frac{1}{9}=x$
    \item $\log_5 1=x$
    \item $\log_{0.04} 5=x$
    \item $\log_{3\sqrt{3}}\frac{1}{27}=x$
\end{enumerate}
\end{multicols} 

\item 利用恒等式$\log_a a^b=b,\quad (a>0,\; a\ne 1)$计算:
\begin{multicols}{3}
    \begin{enumerate}
        \item $\log_3\sqrt[4]{3}$
        \item $\log_2\sqrt{2}$
        \item $\log_5\frac{1}{\sqrt[3]{5}}$
        \item $\log_3\sqrt{27}$
        \item $\log_a\frac{1}{a^n}$
        \item $\log_a \frac{1}{\sqrt[n]{a}}$
    \end{enumerate}
    \end{multicols} 

\item 求对数$x$:
\begin{multicols}{3}
    \begin{enumerate}
        \item $2^x=8$
        \item $2^x=0.125$
        \item $3^x=1$
        \item $4^x=2$
        \item $4^x=0.5$
        \item $10^x=10\sqrt{10}$
        \item $2^x=\frac{\sqrt[5]{2}}{2}$
        \item $5^x=\sqrt[3]{5^2}$
        \item $10^x=\frac{1}{\sqrt[4]{1000}}$
        \item $2^x=3$
        \item $\left(\sqrt{2}\right)^x=10$
    \end{enumerate}
    \end{multicols} 

\item 根据恒等式$a^{\log_a N}=N$, 求:
\begin{multicols}{3}
    \begin{enumerate}
        \item $2^{\log_2 8}$
        \item $3^{\log_3 7}$
        \item $36^{\log_6 2}$
        \item $25^{\log_5 3}$
        \item $81^{0.5\log_9 7}$
        \item $81^{\tfrac{1}{2}\log_3 7}$
        \item $5^{\log_5 10-1}$
        \item $2^{\log_2 5+1}$
    \end{enumerate}
    \end{multicols} 
\end{enumerate} 
\end{ex}

\subsection{对数的性质}
在前面我们说明了对于每个正实数$x$都有唯一的以$a$为底
的对数$\log_a x$和它对应,于是,
\[\begin{split}
   M(M>0)&\longrightarrow \log_a M,\quad \text{使得 } a^{\log_a M}=M \\
   N(N>0)&\longrightarrow \log_a N,\quad \text{使得 } a^{\log_a N}=N \\
   M\cdot N(M>0,\; N>0)&\longrightarrow \log_a (M\cdot N),\quad \text{使得 } a^{\log_a (MN)}=MN \\
   \frac{M}{N}(M>0,\; N>0)&\longrightarrow \log_a \frac{M}{N},\quad \text{使得 } a^{\log_a \tfrac{M}{N}}=\frac{M}{N} \\
   M^u(M>0,\; u\in\mathbb{R})&\longrightarrow \log_a M^u,\quad \text{使得 } a^{\log_a M^u}=M^u \\
   \sqrt[n]{M}(n\ge 1,\; n\in\mathbb{Z})&\longrightarrow \log_a \sqrt[n]{M},\quad \text{使得 } a^{\log_a \sqrt[n]{M}}=\sqrt[n]{M} \\
\end{split}\]

现在我们来研究怎样求:
\begin{enumerate}
    \item 积的对数;
    \item 商的对数;
    \item $u$次方幂的对数;
    \item $n$次方根的对数。
\end{enumerate}

显然对数的性质与指数法则有密切关系,譬如,对应于
指数法则$a^u\cdot a^v=a^{u+v}$,就有下面的对数法则:

\begin{blk}{性质1}
    两个正数乘积的对数,等于这个积的因数的对
数的和,即
\[\log_a(MN)=\log_a M+\log_a N\]
\end{blk}

\begin{proof}
    根据恒等式,有
    \begin{equation}
        M=a^{\log_a M},\qquad N=a^{\log_a N}
    \end{equation}
    那么将指数法则:$a^u\cdot a^v=a^{u+v}$应用到(1.5), 得到
\[MN=a^{\log_a M}\cdot a^{\log_a N}=a^{\log_a M+\log_a N}\]
    根据对数定义得到
    \[\log_a MN = \log_a M + \log_a N\]
\end{proof}    

对应于指数法则$\frac{a^u}{a^v}=a^{u-v}$,就有对数法则:
    
\begin{blk}{性质2}
    一个商的对数等于分子与分母的对数的差,即
    \[\log_a \frac{M}{N}=\log_a M-\log_a N\]
\end{blk}

\begin{proof}
将指数法则$\frac{a^u}{a^v}=a^{u-v}$
应用到(1.5), 得到:
\[\frac{M}{N}=\frac{a^{\log_a M}}{a^{\log_a N}}=a^{\log_a M-\log_a N}\]
根据对数定义得到:
\[\log\frac{M}{N}=\log_a M -\log_a N\]
\end{proof}    

对应于指数法则$(a^m)^n=a^{mn}$就有对数法则: 
\begin{blk}{性质3}
    一个幂的对数,等于幂底数的对数与指数的
积,即
\[\log_a M^u =u\log_a M\]
\end{blk}
  
\begin{proof}
    将指数法则$(a^m)^n=a^{mn}$应用到(1.5)得到
\[M^u=\left(a^{\log_a M}\right)^u=a^{u\log_a M}\]
根据对数定义得到
\[\log_a M^u = u\log_a M\]
\end{proof}
    
性质3的特殊情况是:

\begin{blk}{性质4}
 $M(M>0)$的$n$次算术根的对数等于根指数的
倒数乘以被开方数的对数,即
\[\log_a\sqrt[n]{M}=\frac{1}{n}\log_a M\]
\end{blk}

\begin{rmk}
    \begin{enumerate}
        \item 由性质2可得$\log_a\frac{1}{M}=-\log_a M\; (M>0)$;
        \item 一般来说,$\log_a(M\pm N)\ne \log_a M\pm \log_a N$;
        \item 上述四个性质说明,如果先对算式取对数,即把算式
        过渡到对数,然后根据对数运算法则,两个数相乘与相除可以
        化为它们的对数相加与相减,一个数的乘方与开方可以化为
        把它的对数乘以或除以指数,这样计算简捷得多,以后我们还
        要进一步说明,如何把计算对数的结果再还原到算式的结果。
    \end{enumerate}
\end{rmk}
    
\begin{example}
已知 $\log _{10} 2=0.3010$, 求 $\log_{10} 5, \log_{10} 40$,
    $\log_{10} 5000$
\end{example}

\begin{solution}
\[\begin{split}
  \log _{10} 5&=\log _{10} \frac{10}{2}=\log _{10} 10-\log _{10} 2=1-0.3010=0.6990 \\
   \log _{10} 40&=\log_{10}(4 \times 10)\\
   &=\log_{10} 2^{2}+\log_{10} 10=2 \log_{10} 2+1\\
    &=2 \times(0.3010)+1=1.6020\\
    \log _{10} 5000&=\log _{10}(5 \times 1000)=\log _{10} 5+\log _{10} 10^{3}\\
    &=\left(\log_{10} 10-\log_{10} 2\right)+3 \log _{10} 10\\
    &=4-0.3010=3.6990
\end{split}\]
\end{solution}


\begin{example}
已知$\log_{10}2=0.3010$,$\log_{10}3=0.4771$,求$\log _{10} 0.2$, $\log _{10} \sqrt{1.5}$, $\log _{10} 450$
\end{example}    

\begin{solution}
    \[\begin{split}
        \log _{10} 0.2 &=\log _{10} \frac{2}{10}=\log _{10} 2-\log _{10} 10\\ 
        &=0.3010-1 =-0.6990\\
        \log _{10} \sqrt{1.5} &=\frac{1}{2} \log _{10} \frac{3}{2}=\frac{1}{2}\left(\log _{10} 3-\log _{10} 2\right)\\
        &=\frac{1}{2}(0.4771-0.3010)=0.08805\\
        \log _{10} 450 &=\log_{10}\frac{3^2\x 10^2}{2}=\log_{10}(3\x 10)^2-\log_{10} 2\\
        &=2\left(\log_{10}3+\log_{10}10\right) -\log_{10}2\\
        &=2(0.4771+1)-0.3010=2.6532
\end{split}\]
注意:在解题时常用到$\log_a a=1$, $\log_a 1=0$。
\end{solution}    

求任何单项式的对数就是求该式中各因式的对数的代数
和。
    
\begin{example}
    已知$y=\frac{(a-b)^3\sqrt[3]{c}}{\sqrt[5]{(a+b)^2d^3}}$,求$\log_c y$
\end{example}

\begin{solution}
\[\begin{split}
    \log_c y&=\log_c \frac{(a-b)^3\sqrt[3]{c}}{\sqrt[5]{(a+b)^2d^3}}\\ 
&=\log_c (a-b)^3\cdot \sqrt[3]{c}  -\log_c \sqrt[5]{(a+b)^2d^3}\\
&=\log_c (a-b)^3+ \log_c \sqrt[3]{c}  -\frac{1}{5}\log_c {(a+b)^2d^3}\\
&=3\log_c (a-b)+\frac{1}{3}\log_c  c-\frac{1}{5}[2\log_c (a+b)+3\log_c d]\\
&=3\log_c (a-b)+\frac{1}{3}-\frac{2}{5}\log_c (a+b)-\frac{3}{5}\log_c d
\end{split}\]
\end{solution}

所谓代数式的还原法就是由一个单项式的对数式反求这
个单项式。

\begin{example}
已知$\log_2 x=\log_2 a+2\log_2 b-\frac{1}{3}\log_2 c$,求$x$。
\end{example}

\begin{solution}
    \[\log_2 x=\log_2 a+\log_2 b^2-\log_2 \sqrt[3]{c}=\log_2 ab^2-\log_2 \sqrt[3]{c}
=\log_2 \frac{ab^2}{\sqrt[3]{c}}\]
根据同底的两个对数相等,则它的真数相等,得出
\[x=\frac{ab^2}{\sqrt[3]{c}}\]
\end{solution}    


\begin{example}
    已知$\log_a N=\log_ac+b$, 求$N$。
\end{example}


\begin{solution}
\[\log_a N=\log_a c+\log_a a^b=\log_a c\cdot a^b\]
根据同底的两个对数相等则其真数相等,得出
\[N=c\cdot a^b\] 
\end{solution}

\begin{blk}{性质5}
    以$b$为底$N$的对数,除以以$b$为底$a$的对数的商,
可以换作以$a$为底$N$的对数
\[\log_a N=\frac{\log_b N}{\log_b a}\]
这个公式叫做换底公式。
\end{blk}


\begin{proof}
    因为$N=a^{\log_a N}$,两边取以b为底的对数,得到
    \[\log_b N = \log_b (a^{log_aN} )\]
    应用对数法则(性质3), 得到
\[\log_b N=(\log_a N)\log_ba\]
即
\[\log_a N=\frac{\log_b N}{\log_b a}\]

\end{proof}


\begin{blk}{推论}
    $$\log_ab=\frac{1}{\log_ba}$$
\end{blk}


\begin{example}
已知$\log_{10}2=0.3010$,求$\log_2 5$。
\end{example}

\begin{solution}
\[\log_2 5=\log_2 \frac{10}{2}=\log_2 10-\log_2 2=\frac{1}{\log_{10}2}-1=\frac{1}{0.3010}-1=\frac{699}{301}\] 
\end{solution}    

\begin{example}
    已知$\log_{10}2=a$, $\log_{10}7=b$,求$\log_{8}9.8$。
\end{example}

\begin{solution}
\[\begin{split}
    \log_{8}9.8&=\frac{\log_{10}9.8}{\log_{10}8}=\frac{\log_{10}\frac{2\x 7^2}{10}}{\log_{10}2^3}\\
    &=\frac{\log_{10}2+2\log_{10}7-\log_{10}10}{3\log_{10}2}\\
    &=\frac{a+2b-1}{3a}
\end{split}\]
\end{solution}

\section*{习题1.3}
\addcontentsline{toc}{subsection}{习题1.3}
\begin{enumerate}
    \item 利用对数性质将下面的式子变形:
    \begin{multicols}{2}
\begin{enumerate}
    \item $\log_a a^2b^2$
    \item $\log_x x\sqrt{y}$
    \item $\log_x\frac{\sqrt[3]{x}}{y^2}$
    \item $\log_a\frac{b^3}{\sqrt{a}}$
    \item $\log_a(ab)^3$
    \item $\log_a\left(\frac{x}{y}\right)^4$
    \item $\log_a\sqrt[4]{\frac{x^3}{y}}$
    \item $\log_a\frac{uv^3}{w^2}$
\end{enumerate}        
    \end{multicols}

\item 求下列各式中$\log_{10}x$的展开式:
\begin{enumerate}
    \item $x=6a\frac{\sqrt{2(a-b)c}}{5(a-b)^2}$
    \item $x=5m^{\tfrac{3}{4}}n^{\tfrac{1}{3}}\sqrt[3]{\frac{2\cos\alpha}{3}},\quad (0^{\circ}\le \alpha\le 90^{\circ})$
    \item $x=\sqrt[5]{\left(\frac{1}{a^2b^2\sqrt[4]{c^3}}\right)^3}$
    \item $x=\sqrt{\frac{\sqrt{ab}}{a^{-1}}\cdot \sqrt[3]{a^{-1}b^{-2}}}$
\end{enumerate}
\item 求证:
\begin{enumerate}
    \item $\log_a (e^x+2+e^{-x})+\log_a(e^x-2+e^{-x})=2\log_a(e^x-e^{-x})$
    \item $\log_a(m^3+3m^2+3m+1)-\log_a(n^3+3n^2+3n+1)=3\log_a\frac{m+1}{n+1}$
\end{enumerate}

\item 已知$\log_{10} x=\log_{10} a+\log_{10} b-3\log_{10} c$, 求$x$。
\item 求下列各式的值:
\begin{multicols}{2}
\begin{enumerate}
    \item $\frac{1}{2}\log_{3} 9$
    \item $\log_{4} 2+\log_{4} 8$
    \item $3\log_{10} 2-\log_{10} 80$
    \item $\log_{5} 10-\log_{5}\frac{2}{\sqrt{5}} $
\end{enumerate}
\end{multicols}
\item 若$\log_{10} 2=0.3010$,
计算$\log_{2} 13+\frac{1}{2}\log_{2} 25+\log_{2} \frac{4}{7}-\log_{2}\frac{13}{35} $
\item 已知$\log_{10} 2=0.3010$, $\log_{10}3=0.4771$,
求$\log_4 3$和$\log_3 2$

\item  试证$\log_{a^k}b=\frac{\log_ab}{k}$

\item 计算:
$(\log_23+\log_49)(\log_34 +\log_92)$
\item 计算:$\log_ab\cdot \log_b c\cdot \log_ca$ ($a,b,c$是不等于1的正
数)。
\end{enumerate}  

\subsection{常用对数}
\subsubsection{近似数计算常识}
本节要介绍常用对数表,对数表所列对数几乎全是近似
数,我们不仅在数学用表中遇到近似数,其实在日常生活
中,有关度量和计算大量物件或计算无理数的的结果,都得
用近似数来表示。因此,在介绍常用对数之前,我们特别介
绍下面几点关于近似数的常识。

在度量和计算中所遇到的数有两种情况:一种是能够
用一个数准确地表示某一个量。例如,在一堂课内出席学生
50人,一星期有7天,这些数都与实际完全符合,叫做该量
的准确量。但是在中午一小时内,经过北京天安门广场的人
数,就未必能用一个数准确地表示,又比如说约2000人出席
全校大会,“约”这个字说明2000这数是近似的.在计算大量数
目的物件时,常常要用近似数来表示。

度量一个量,都不能做到绝对准确,所得结果总会含有
一些误差的,这误差的大小,要看度量仪器的质量以及作度
量的人是否有经验而定。例如,若用最小分度为1毫米的尺
子量一金属杆的长度时,所得结果为1.234米,这表示真实长
度在1.2335米和1.2345米之间.

\begin{blk}{定义1}
    一个量的准确数(有时说准确值)与近似数(或
    近似值)的差,叫做这个\textbf{近似数(值)的误差}。
\end{blk}


设$x$是某量的准确数(值),并且$x=3.283$, 则称$a=3.2$
是$x$的一个不足近似数(值),$b=3.3$是$x$的一个过剩近似数
(值)。把$x$用它的不足近似数$a$来代替,所产生的误差是:
\[x-a=3.283-3.2=0.083\]
如果取$x$的过剩近似数(值),那么这个近似值的误差是:
\[x-b=3.283-3.3=-0.017\]

显然,不足近似值的误差总是正的,而过剩近似值的误
差总是负的。

\begin{blk}{定义2 }
    一个量的准确数与近似数的差的绝对值,叫做
这个\textbf{近似数的绝对误差}。
\end{blk}


在我们所举的例子中,不足近似值$a$的绝对误差:
\[|x-a|=|0.083|=0.083\]
而过剩近似值$b$的绝对误差:
\[|x-b|=|-0.017|=0.017\]

近似数(值)的绝对误差,表示近似数和该准确数相差多
少,绝对误差越小说明这个近似数越精确。

在度量长度时,一个量的准确长度是不知道的,但可以
估计它在哪个范围内。因此近似数的绝对误差的准确值也是
不知道的,但可以估计出近似数的绝对误差不超过什么数。

\begin{blk}{定义3}
    设$x$是某量的准确数,$a$是它的近似的数(值),如
    果这个近似数(值)的绝对误差不超过数$h$, 也就是说满足条
    件$|x-a|\le h$, 那么$a$叫做$x$的\textbf{精确到$h$的近似数(值)}。    
\end{blk}

例如:$|x-a|=0.083<0.1,\quad |x-b|=0.017<0.1$,因此:
3.2和3.3分别叫做3.283的精确到0.1的不足
近似值和过剩近似值。

前面用最小分度为1毫米的尺子度量金属杆所得到的近
似值1.234的绝对误差不超过0.0005(米),因此它是精确到
万分之五(米)的近似值。

\subsubsection{数的四舍五入法和近似数的有效数字}
准确数据四舍五入时,也会出现近似数,如果某校有
2003人,当参观者问校长说“学校有多少人?”时,他大概会
说四舍五入得约数2000人。

把分数$\frac{2}{13}$
化为小数时,得到无穷循环小数$\frac{2}{13}\approx 0.\dot{1}5384\dot{6}$

在实际计算上,我们常把这种小数取到小数点后某一位
为止,比方说取到千分之一位,而按四舍五入法舍去其后各
位数字,于是得:
$\frac{2}{13}\approx 0.154$。

四舍五入法就是略去几位数字时,如果略去的第一位数
字小于5, 那么把留下的最末一位数字保留不动;如果略去
的第一位数字大于或等于5, 那就把留下的最末一位数字加
上1, 用四舍五入法写出来的数,比单单略去最后几位数后写
出的数准确些。

例如,0.154就比0.153更接近$\frac{2}{13}$。

$\because\quad \left|\frac{2}{13}-0.154\right|<0.0002,\quad \left|\frac{2}{13}-0.153\right|<0.0009$

$\therefore\quad $0.151比0.153更接近$\frac{2}{13}$。

估计近似数的准确度的最简便的方法之一,是算出它的
有效数字的个数。

\begin{blk}{定义4}
    若近似数的误差不超过其末一位数字的一-个单
位,那么该数的一切数字;除了左起第一个非零数字之前的
荐以外,都叫有效数字。
\end{blk}


例如,0.154和0.153都是
的取3位有效数字的数,即
$\frac{2}{13}$
的近似值0.154或0.153具有三位有效数字。把准确数
1.9996四舍五入到千分之一位,得2.000, 这时所有四个数字,
连三个零在内,都是有效数字,即它是有四个有效数字的
数。

\begin{blk}{定义5}
    若近似数的误差大于最末一位数字的一个单
    位,那么它的最末一个数字叫不可靠的。  
\end{blk}

把0.1535作为$\frac{2}{13}$
的近似数时,那么它的最末一个数字5
就是不可靠的,因为
$\left|\frac{2}{13}-0.1535\right|>0.0003$

把学校人数2135四舍五入到百位数得2100,这里最末两
个零是不可靠的数字,2100是有两个有效数字的数。

如果近似数末后的零不是有效数字,我们规定把它写得
小一些,这样21{\small00}表示具有两个有效数字的近似数。

用科学记数法表示的数在10的方幂前面的数字代表有效
数字。

例如,0.0000437有三位有效数字,它的标准写法是
$4.37\x10^{-8}$, 前面2100的标准写法是$2.1\x10^3$.

在计算近似数时,我们必须指明精确到几位有效数字。
例如0.09235716, 若取三位有效数字的精确度,我们就写成:
$9.24\x10^{-2}$. 若是$70,454,620,000$取四位有效数字的精确度
就写成$7.045\x10^{10}$.

\begin{rmk}
前面例子有效数字中,最后一位数字都已经过四
舍五入的处理了,如果最后一例中,取二位有效数字的精
度,就要写作$7.0\x10^{10}$, 不可以写作$7\x10^{10}$。  

不要把“小数点后的数字”跟“有效数字”等同起来,数0.000175有三个有效数字,但总共有六个小数点后的数字。
\end{rmk}

关于近似数的计算规则如下:
\begin{blk}{法则1}
    近似数相加(减)时,其中小数位数最少的近似
    数有几个数位,它们的和(差)就保留几个小数数位。  
\end{blk}

\begin{example}
    已知$\pi\approx 3.1416$, $a=2.4$, $b=0.047$。
求$\pi+a+b$。
\end{example}    

\begin{solution}
\begin{enumerate}
\item 找出小数位数最少的近似数,这样的近似数是
    2.4。
    \item 把其它的近似数四舍五入,比所找出的近似数多保
    留一位:
    \[\pi\approx 3.1416\approx 3.14,\qquad 0.047\approx 0.05\]
    \item 作近似数的加法:$3.14+2.47+0.05=5.59$
    \item 把所得结果四舍五入,保留内位数与所找出的近似
    数的位数相同,用四舍五入到十分位:
    \[5.59\approx 5.6\]
    于是
    \[\pi+a+b\approx 5.6\]
    当近似数相减时,可以同样处理。
\end{enumerate}
\end{solution}

\begin{example}
已知$x\approx 6.3\x10^3$, $y=3.26\x10^5$,
求$x-y$。
\end{example}
    
    
\begin{solution}
\[\begin{split}
     x-y&=6.3\x10^3-3.26\x10^5\\
&=10^5(6.3\x10^{-2}-3.26)=(0.063-3.26)\x10^5\\
&=(-3.197)\x10^5\\
&\approx -3.20\x 10^5\\
&=-320000
\end{split}\]   
\end{solution}

\begin{blk}{法则2}
近似数相乘(除)时,有效数字位数最少的因数
有几位有效数字,它们的积(商)就保留几位数字(不算前边
的零)。
\end{blk}


\begin{example}
    已知$u=95.64$, $v=0.28$,
求积$u\cdot v$。
\end{example}


\begin{solution}
\begin{enumerate}
    \item 找出有效数字位数最少的因数,这样的因数是0.28(两位有效数字)。
    \item 把因数95.64四舍五入,使它比有效数字位数最少的
因数多一位有效数字:$95.64\approx 95.6$(四舍五入取到三位有效数字;最后一
个数字是备用数字)。
\item 相乘后得到积:
\[95.6\x0.280=26.768\]
\item 把这个结果四舍五入,使其数字个数与有效数字位
数最少的因数含有的有效数字位数相同:
$26.768\approx 27$。

于是$u\cdot v=27$。
\end{enumerate}
\end{solution}

在计算近似数的商时,应用上面同样的法则。

\begin{example}
设$x\approx 0.823$, $y=1.34$, 计算:
$\frac{xy}{x+y}$的近似值。
\end{example}


\begin{solution}
\begin{enumerate}
    \item 求出积$xy$的近似值:$xy\approx 0.823\x1.34=1.10282\approx 1.103$
    
    (在$x$和$y$的近似值中,都有三位有效数字,所得结果保留四个
数字;有一个数宇是备用数字)。
\item 求出$x+y$的近似值:
$x+y\approx 0.823+1.34=2.163$

($y$的近似数中,小数点后有两位;$x$的近似数中,小数点后
有三位,所得结果保留到小数点后三位,有一个数字是备用数
字)。
\item 求出商$\frac{xy}{x+y}$的近似值:
\[\frac{xy}{x+y}\approx \frac{1.103}{2.163}\approx 0.5099\]
(不算备用数字,在分子与分母的近似值中,都含有三位有效
数字,计算结果,不算整数位上的零,共有四个数字,有一
个数字是备用数字。)

\item 把上述结果四舍五入到三个数字(不算个位数上的零):
\[\frac{xy}{x+y}\approx 0.510\]
\end{enumerate}    
\end{solution}

\begin{ex}
\begin{enumerate}
    \item 下列由四舍五入得到的近似数,各精确到哪一位,各有几个有效数字?
\begin{multicols}{3}
\begin{enumerate}
    \item 306万
    \item 25.7
    \item 0.004
    \item 0.01063
    \item 830000
    \item $2.60\x10^5$
\end{enumerate}
\end{multicols}

    \item 
    用四舍五入法,按要求对下列各数取近似数:
\begin{enumerate}
    \item 0.33428(精确到0.001)
    \item 64.8(精确到1)
    \item 0.05069(保留三个有效数字)
    \item 1.4963(精确到0.01)
    \item 83230(保留1个有效数字)
    \item 80230(保留2个有效数字)
\end{enumerate}

    \item 已知$x$和$y$, 求$x$与$y$的积的近似值:
        \begin{enumerate}
\item $x\approx 15.94,\qquad y\approx  0.85$
\item $x\approx 1.754\x10^8,\qquad y\approx 6.9\x10^{-5}$
\item $x\approx  8.42\x10^{-4},\qquad y\approx 9.81\x10^5$
        \end{enumerate}

    \item 
    已知$x$和$y$的近似值,求商$\frac{x}{y}$
    的近似值:
\begin{enumerate}
    \item $x\approx 18.28,\qquad y\approx 0.54$
    \item $x\approx 0.36,\qquad y\approx 0.0238$
    \item $x\approx 4.5\x10^7,\qquad y\approx 2.79\x10^5$
\end{enumerate}

    \item 求下列各式的近似值:
\begin{enumerate}
    \item $3x-5y$, 其中$x\approx 46.24,\qquad y\approx 25.2$
    \item $\frac{x+y}{x-y}$,其中$x\approx 2.08,\qquad y\approx 10.2$
    \item $x^2-2x+7$, 其中$x\approx 1.44$
\end{enumerate}
\end{enumerate}
\end{ex}

\subsubsection{常用对数}
    以10为底的对数叫做常用对数。在数值的计算上常要用
    到以10为底的对数一些特性。为了书写的方便,我们约定
    把$\log_{10}N$的底10略去不写,用专门的记号“$\lg$”代替“$\log_{10}$”,例
    如
\[ \log_{10} 100=2 \qquad \text{简写成}\qquad  \lg100=2\]
    以后,如果没有指明对数的底,就认为所说的对数是常用对
    数,例如说“2的对数是0.3010”就是指2的常用对数是0.3010。
    也就是
 \[   lg2=0.3010\]
  
 下面我们要看看,常用对数对于数值计算究竟有哪些便
    利的性质。

为了说明常用对数的特性,我们首先把真数用科学记数
法写出来,也就是把它表示为一个1和10之间的数与10的幂的
积,即$u\x10$, 这里$1\le u<10$, $k\in\mathbb{Z}$, 例如光的速度是每秒
186,000公里表示为$1.86\x10^5$每秒公里,一个电子的质量是
0.000,000,000,000,000,000,000,000,000,910,83克表示为
$9.1083\x10^{-28}$克,把一个数用科学记数法写出的规则是:

\begin{enumerate}
    \item 把整数部分有$p+1$位的数$N$的小数点,向左移动$p$
    位得到只有一位整数的数$u$, 这样就有$N=u\x10^p\; (p\ge0)$。
    \item 把左起第一个非零数字前面有$p$个零(包括个位数上
    的一个零)的纯小数N的小数点,向右移动$p$位得到只有一位
    整数的数$u$, 这样就有$N=u\x10^{-p}\; (p\ge1)$。
\end{enumerate}

如果已知$\lg2.23=0.3482$, 我们只须通过观察10的方幂
    的对数,就可以求得和$0.223$的有效数字相同,而只有小数点位置不同的数(即这种形式:$2.23\x10^p$, 或$2.23\x10^{-p}$,
    $p\in\mathbb{N}$的数)的常用对数,这也就是常用对数的好处。例如:
\[\begin{split}
\lg22.3&=\lg(2.23\x 10)=\lg2.23+\lg 10=0.3483+1\\
\lg223&=\lg(2.23\x 10^2)=\lg2.23+\lg 10^2=0.3483+2\\
\lg2230&=\lg(2.23\x 10^3)=\lg2.23+\lg 10^3=0.3483+3\\
\lg22300&=\lg(2.23\x 10^4)=\lg2.23+\lg 10^4=0.3483+4\\
\lg0.223&=\lg(2.23\x 10^{-1})=\lg2.23+\lg 10^{-1}=0.3483-1\\
\lg0.0223&=\lg(2.23\x 10^{-2})=\lg2.23+\lg 10^{-2}=0.3483-2
\end{split}\]

我们定义常用对数中的正的纯小数部分叫做对数的尾
数,而把它的整数部分叫做对数的首数,注意常用对数的尾
数一定是正的,首数可以是任何整数,它的数值等于真数在
科学记数形式下的10的幂指数,在上面的对数中,0.3483是
尾数,整数$+1,+2,+3,+4,-1,-2$是相应对数的首
数,$\lg2.23$的首数是0。

现在我们把常用对数的一些特性总结如下:
\begin{enumerate}
    \item 10的方幂的常用对数是整数。
因为
\[\lg 10^n=n\lg10=n,\qquad n\in\mathbb{Z}\]

看下表:
\begin{center}
\begin{tabular}{c|ccccccccc}
    \hline
$u$ & $\cdots$ &0.001& 0.01&0.1&1&10&100&1000&$\cdots$\\
\hline
$\lg u$ & $\cdots$ &$-3$&$-2$&$-1$&0&1&2&3&$\cdots$\\
\hline
\end{tabular}
\end{center}
 
从表中看出$\lg10^u$的值随幂指数增大而增大。
由此进一步推知:
\begin{enumerate}
    \item 如果$u>1$, $\lg u$便是正的;
    \item  如果$0<u<1$, $\lg u$便是负的;
    \item    如果$u<0$或$u=0$, $\lg u$无意义。
\end{enumerate}


\item 当$u$是10的整数指数幂以外的有理数时,$\lg u$是无理
数(证明略去)。

这也就是说,常用对数表中所列的对数值几乎全是有理
近似值。
\item 如果$u$是一位整数的数,那么它的对数是一个正的纯
小数,它的首数等于零。

因为由上表可以看出,当$1<u<10$时,便有$0<\lg u<1$。
\item 10的整数次幂以外的任何正数的常用对数,可以写
成首数加尾数的形式,这种形式的数是常用对数的人为形
式。
\end{enumerate}

因为
\[\begin{split}
    \lg(u\x10^p)&=\lg u+p \qquad (0<\lg u<1,\; p\in\mathbb{N})\\
&=\text{正的纯小数}+\text{正整数$p$}
\end{split}\]
或者
\[\lg(u\x10^{-p})=\text{正的纯小数}+\text{负整数$(-p)$}\]

利用这个性质,我们可以按下面的规则写对数的首数和
尾数:
\begin{enumerate}
\item 大于1的真数的对数的首数,等于这个真数的整数位
数减1;

小于1的真数的对数的首数是一个负整数,它的绝对值
等于这个真数第一个非零数字前面零的个数(包括个位上的
零在内)。

根据上面把一个数用科学记数法写出的规则和对数的性
质,这个结论是自明的。

\item 真数的对数尾数只决定于它的各个数字的排列顺序
而与小数点的位置无关,因此,只是小数点位置不同的那些
数,它们的对数尾数都相同。
\end{enumerate}

为了书写上的简便,我们把首数和尾数中间的“$+$”号略
去不写,当首数是负数的时候,把“$-$”号放在这个整数的上
面,例如:
\[\begin{split}
    \lg 12.3&=1+0.0899=1.0899\\
    \lg 0.123&=-1+0.0899=\bar{1}.0899\\
    \lg 0.0123&=-2+0.0899=\bar{2}.0899\\
\end{split}\]

\begin{rmk}
   整数部分上的“$-$”号只管整数部分,而不管小数
部分,即:$\bar{1}.0899\ne -1.0899$。
\end{rmk}


\begin{example}
    已知$\lg4=0.6021$, 试举出5个数,它们的对数
    尾数都是0.6021.
\end{example}

\begin{solution}
    因为同一个对数的尾数所对应的真数仅仅是小数点
的位置不同,所以0.04, 0.4, 40, 400, 4000等数的对数尾
数都是0.6021。
\end{solution}

\begin{example}
    已知$\lg x=4.028$, $\lg y=0.1234$, $\lg z=-4.653$,
    分别判断$x,y,z$是大于1还是小于1的数。大于1
    时,有几位整数?小于1时,第一个不等于零的数字前面有
    几个零(包括个位上的零)?
\end{example}

\begin{solution}
    $x$是大于1的数,有5位整数;
$y$是大于1的数,有1位整数;
因为
\[
    \lg z=-4.653=-4+(-0.653)=-5+(1-0.653)=\bar{5}.347<0,
\]
所以$z$是小于1的数,第一个不是零的有效数字前面有5个
零(包括个位数上的零)。
\end{solution}


\begin{example}
    若$\lg4.86=0.6866$, 求$\lg\sqrt[3]{0.0486}$。
\end{example}


\begin{solution}
\[\begin{split}
    \lg\sqrt[3]{0.0486}&=\frac{1}{3}\lg0.0486=\frac{1}{3}(-2+0.6866)\\
    &=\frac{1}{3}(-3+1.6866)\\
    &=-1+0.5622=\bar{1}.5622
\end{split}\]
\end{solution}

\begin{rmk}
    例1.34解释了对于一个有负首数的对数的计算常用
的技巧,假如我们求四次根,那么我们必须把$0.6822-2$写
成$-4+2.6822$, 并且它们被四除。
\end{rmk}

下面再给出几个有负首数的对数作四则运算的例子:
\begin{itemize}
    \item 加法\[\begin{split}
    \bar{2}.3758+\bar{3}.7189&=(-2+0.3758)+(-3+0.7189)\\
&=-5+1.0947=\bar{4}.0947
\end{split}\]
    \item 减法\[\begin{split}
    \bar{3}.7239-\bar{4}.8241&=(-4+1.7239)-(-4+0.8241)\\
&=0.8998
\end{split}\]
    \item 乘法\[\begin{split}
 \bar{3}.2373\x5&=(-3+0.2373)\x5\\
&=-15+1.1865=\overline{14}.1865   
\end{split}\]
    \item 除法\[\begin{split}
\bar{2}.6365\div 4&=(-4+2.6866)\div 4\\
&=-1+0.6717=\bar{1}.6717
\end{split}\]
\end{itemize}

\begin{ex}
\begin{enumerate}
    \item 求下列各数的常用对数:
\[10^{-5},\qquad 10^5,\qquad 0.0000001,\qquad 1000000\]

\item 已知$\lg 3=0.4771$, 求:
\[\lg0.003,\qquad \lg0.03,\qquad \lg0.3,\qquad \lg30,\qquad \lg300,\qquad \lg3000\]

\item \begin{enumerate}
    \item 把下列各数用负数来表示:
    \[\bar{1}.3235,\quad \bar{2}.5638,\quad \bar{1}.8436,\quad \bar{2}.4957,\quad \bar{4}.9879\]
    \item 把下列各数改成负首数与正尾数的和:
    \[-0.3678,\quad-1.8394,\quad-0.2310,\quad-0.0840,\quad-3.0251\]
\end{enumerate}

\item  完成下列运算,用首数$+$尾数表示结果。
\begin{multicols}{2}
\begin{enumerate}
    \item $\bar{1}.5436+2.3892$
    \item $\bar{1}.1546+\bar{2}.9835$
    \item $0.7658-1.3456$
    \item $\bar{1}.2396-0.7459$
    \item $0.1518-\bar{1}.2836$
    \item $0.3215-\bar{2}.9217$
    \item $1-\bar{1}.28123$
    \item $2-\bar{2}.15343$
    \item $\bar{1}.2432\cdot 3$
    \item $0.1546\cdot (-2)$
    \item $\bar{1}.7638\cdot (-3)$
    \item $2.5146\cdot (-1.2)$
    \item $\bar{2}.2817\div 2$
    \item $\bar{2}.1536\div 3$
    \item $\bar{2}.7235\div 4$
    \item $\bar{1}.6274\div (-3)$
    \item $\bar{1}.8236\div (-0.2)$
    \item $\bar{2}.6548\div (-0.3)$  
\end{enumerate}
\end{multicols}

\item 若$\lg6444$的尾数是0.8092, 求:
\[\lg0.6444,\qquad \lg6444000,\qquad \lg0.0006444\]
\item 已知$\lg32.5=1.5119$, 求下面对数式中的真数$x$:
\[\lg x=0.5119,\qquad \lg x=2,5119,\qquad \lg x=\bar{4}.5119\]
\item \begin{enumerate}
\item $3^{10}$是几位整数?
\item $3^{100}$是几位整数?
\item $(0.03)^{10}$在小数点和第一个有效数字之间一共有多少
个零?(提示:$\lg3=0.4771$)
\end{enumerate}
\end{enumerate}
\end{ex}

\subsubsection{对数表}
在上一节里,我们知道任意一个正数的对数都可以分成
两个部分,其中第一部分首数可以从观察中直接求出,第二部
分尾数,只要知道了1到10以内的数的对数,也就可以求出.
但是,怎样来求1到10以内的数的对数呢?

因为10的整数指数幂以外的所有有理数,它们的对数都
是无理数,所以要求一个正数的对数,一般只能得到某一精
度的近似值,为了应用上的方便,人们把1到10以内的数的
对数(也就是对数的尾数)制成了各种精确的对数尾数表,例
如在《四位数学用表》里,就载有具有四位有效数字的对数尾
数表。

这个表里所载的对数尾数,都是用四舍五入的法则求到
小数第四位的,因此,这里的对数尾数是精确到0.0001的近
似值。

现在,我们举例来说明利用这个表找一个数的对数尾数
的方法。

因为仅仅是小数点位置不同的数,它们的对数的尾数都
相同,所以我们从一个数查它的对数的尾数时,可以不管真
数的小数点,只把它看成是若干位的整数,例如,要查
0.005036, 50.36, 0.5036, 503600的对数的尾数,都只要查
5036的对数的尾数就可以了。

求一个数的对数时,要先由真数的小数点的位置确定对
数的首数,然后由真数查出对数的尾数,最后把首数和尾数
合写在一起。

\begin{example}
    求下列各数的对数:
\[523,\qquad 51,\qquad 0.5,\qquad 50.36,\qquad 5.448,\qquad 53842\]
\end{example}

\begin{solution}
\begin{enumerate}
    \item  因为523的整数位数是3, 所以它的对数的首数
    是2, 从对数表可以查得对数尾数是0.7185,

    $\therefore\quad \lg523=2.7185$
    \item  因为51的整数位数是2, 所以它的对数的首数是
    1, 而51的对数尾数和510的对数尾数相同,查表得0.7076,

    $\therefore\quad \lg51=1.7076$
    \item  因为0.5的第一个不为零的数5前面有一个零(包括
    个位数上的零),所以它的对数的首数是$\bar{1}$, 而0.5的对数尾数
    和500的对数尾数相同,查表得0.6990,

    $\therefore\quad \lg0.5=\bar{1}.6990$
    \item  因为50.36的整数位数是2, 所以它的对数的首数
    是1, 而50.36对数尾数和5036的对数尾数相同,我们先查
    前三位数的对数尾数得0.7016, 再加上第四位数字的修正值
    0.0005, 得0.7021,

    $\therefore\quad \lg50.36=1.7021$
    \item  因为5.448的整数位数是1, 所以它的对数的首数
    是0, 查表得5448的对数尾数是0.7362,

    $\therefore\quad \lg5.448=0.7362$
    \item  因为53842的整数位数是5, 所以它的对数的首数是
    4, 要求出它的对数尾数,首先把它四舍五入,变成只有四
    位有效数字的数$5.384\x10^4$, 查表得5384的对数尾数是
    0.7311,

    $\therefore\quad \lg53842\approx \lg53840=4.7311$
\end{enumerate}
\end{solution}    

\begin{example}
    求下列各数的对数:
    \[36.5,\qquad 804.7,\qquad 0.26 ,\qquad 0.00453 ,\qquad 945686\]
\end{example}

\begin{solution}
\[\begin{split}
    \lg 36.5&=1.5623\\
    \lg 804.7&=2.9057\\
    \lg 0.26&=\bar{1}.4150\\
    \lg 0.00453&= \bar{3}.6561\\
    \lg 945686  &\approx  \lg945700=5.9757
\end{split}\]
\end{solution}
    
\subsubsection{反对数表}
已知对数求真数,要用《反对数表》(或者叫做“真数表”)。
反对数表的查法和对数表的查法相类似。



    



\begin{example}
    已知$\lg A=2.2715$, 求$A$.
\end{example}

\begin{solution}
    由于对数的首数只与真数的小数点位置有关,而对
数的尾数只与组成真数的数字(第一个不是零的数字前面的
零不计算在内)有关,所以
\begin{enumerate}
    \item 先由对数尾数0.2715查反对数表,查得1868,这就
    是与所求的真数只有小数点位置不同的整数;
    \item 再由首数确定小数点的位置,因为首数是2, 所以
    真数的整数位数是$2+1=3$, 于是得
    $$A=186.8$$
\end{enumerate}
\end{solution}    

\begin{rmk}
    对数的首数是用来定位,查反对数表时,不要把它计入。
\end{rmk}


\begin{example}
   已知$\lg B=\bar{3}.261$, 求$B$. 
\end{example}

\begin{solution}
由对数尾数0.261, 查反对数表得1824; 又因首数
    是$-3$, 所以$B$的第一个有效数字前面有3个0(包括个位上
    的零),于是得
    \[B=0.001824\]
\end{solution}    

\begin{example}
    已知$\lg x=-0.73248$, 求$x$.
\end{example}

\begin{solution}
$\because\quad -0.73248=\bar{1}.26752\approx \bar{1}.2675$

$\therefore\quad x=0.1881$
\end{solution}  

\begin{ex}
\begin{enumerate}
    \item 查表求下面各数的对数:
\begin{multicols}{4}
\begin{enumerate}
    \item 0.7
    \item 0.015
    \item 2.4
    \item 25.6
    \item 0,146
    \item 0.0084
    \item 1.02
    \item 0.208
    \item 0.0000809
    \item 5436
    \item 342.3
    \item 0.9032
    \item 0.0638153
    \item 236743
    \item 0.547936
\end{enumerate}
\end{multicols}
    \item 求下面对数各相应的真数:
    \begin{multicols}{4}
        \begin{enumerate}
            \item 1.7482
            \item 0.4362
            \item $\bar{1}.6415$
            \item 2.6149
            \item 3.2648
            \item $\bar{2}.18176$
            \item $\bar{3}.9340$
            \item $\bar{4}.7267$
            \item 0.9236
            \item $\bar{1}.1015$
        \end{enumerate}
        \end{multicols}   
    \item 求下面各对数的值:
     \begin{multicols}{2}
        \begin{enumerate}
            \item $\lg\cos60^{\circ}$
            \item $\lg\sin^3 60^{\circ}$
            \item $\lg\sqrt[3]{\cos 45^{\circ}}$
            \item $\lg\sqrt[5]{0.00162}$
        \end{enumerate}
        \end{multicols}    
\end{enumerate}
\end{ex}

\subsubsection{利用对数计算}

\begin{example}
    利用对数计算:$\frac{341\x 0.06794\x 8.7}{0.9832\x 4780}$
\end{example}

\begin{solution}
设$A=\frac{341\x 0.06794\x 8.7}{0.9832\x 4780}$
那么两边取对数
\[\begin{split}
    \lg A&=(\lg 341+\lg0.06794+\lg8.7)-(\lg0.9832+\lg4780)\\
&=(2.5328+\bar{2}.8322+0.9395)-(\bar{1}.9927+3.6794)\\
&=2.3045-3.6721=\bar{2}.6324
\end{split}\]
查反对数表,得到
\[A=0.04289\]    
\end{solution}    


\begin{example}
    已知三角形的面积公式
    \[\Delta=\sqrt{s(s-a)(s-b)(s-c)}\]
    这里$a,b,c$是三角形三条边的长,其中
    $s=\frac{a+b+c}{2}$。

    如果$a\approx 15.37$cm, $b\approx 21.42$cm, $c\approx13.83$cm,
    求三角形$ABC$的面积。
\end{example}

\begin{solution}
由题意可得:
\[s=\frac{1}{2}(15.37+21.42+13.83)=25.31\]
\[s-a=9.94,\qquad s-b=3.89,\qquad s-c=11.48\]
因此:
\[\begin{split}
    \lg \Delta&=\frac{1}{2}\bigl[\lg s+\lg(s-a)+\lg(s-b)+\lg(s-c)\bigr]\\
&=\frac{1}{2}\bigl[1.4033+0.9974+0.5899+1.0599\bigr]\\
&=2.0253
\end{split}\]
$\therefore\quad \Delta=106({\rm cm}^2)$.
\end{solution}    

\begin{example}
    利用对数计算$(-2.31)^3\x\sqrt[5]{0.072}$
\end{example}

\begin{solution}
    设$A=(-2.31)^3\x\sqrt[5]{0.072}=-2.31^3\x\sqrt[5]{0.072}$,那么
\[\begin{split}
    |A|&=2.31^3\x \sqrt[5]{0.072}\\
\lg|A|&=3\lg 2.31+\frac{1}{5}\lg 0.072\\
&=3\x 0.3636+\frac{1}{5} \bar{2}.8573\\
&=1.0908+\frac{1}{5}(0.8573-2)\\
&=0.8623
\end{split}\]
因此$|A|=7.283 \quad \Rightarrow\quad A=-7.283$
\end{solution}    


\begin{example}
    在$\triangle ABC$中,已知$a=62.24$, $b=74.83$, $A=
27^{\circ}18'$, 解这个三角形。
\end{example}

\begin{solution}
    这里$A$是锐角,并且$a<b$, 我们先计算$\lg\sin B$。
\[\sin B=\frac{b\sin A}{a}=\frac{74.83\sin 27^{\circ}18'}{62.24}\]
\[\begin{split}
    \lg\sin B&=\lg 74.83+\lg \sin 27^{\circ}18'-\lg 62.24\\
&=1.8741+\bar{1}.6615-1.7941\\
&=\bar{1}.7415
\end{split}\] 
因为$\lg\sin B<0$,即$0\le \sin B<1$, 且$b>a$, 所以这题有两解,
查表得:
\[B_1=33^{\circ}28',\qquad B_2=180^{\circ}-B_1=146^{\circ}32'\]
\[\begin{split}
    C_1&=180^{\circ}-(A+B_1)=180^{\circ}-60^{\circ}46' =119^{\circ}14' \\
    C_2&=180^{\circ}-(A+B_2)=180^{\circ}-173^{\circ}50'=6^{\circ}10'
\end{split}\]

\[c_1=\frac{a\sin C_1}{\sin A}=\frac{62.24\sin 119^{\circ}14'}{\sin 27^{\circ}18'}\]

\[\begin{split}
    \lg c_1&= \lg 62.24+\lg \sin 119^{\circ}14'-\lg \sin 27^{\circ}18'\\
    &=1.7941+\bar{1}.9365-\bar{1}.6615\\
    &=2.0691
\end{split}\]
$\therefore\quad c_1=117.2$

\[c_2=\frac{a\sin C_2}{\sin A}=\frac{62.24\sin 6^{\circ}10'}{\sin 27^{\circ}18'}\]

\[\begin{split}
    \lg c_2&= \lg 62.24+\lg \sin 6^{\circ}10'-\lg \sin 27^{\circ}18'\\
    &=1.7941+\bar{1}.0311-\bar{1}.6615\\
    &=1.1637
\end{split}\]
$\therefore\quad c_2=14.57$。
\end{solution}  

指数中含有未知数的方程叫做\textbf{指数方程}。一般地,借助
于在指数方程两边取对数,可以求得它的解。

\begin{example}
    若某城市现有人口10万,按照每年1.1\%的比率
    增长,在$t$年后有人口多少?约经过多少年,这个城市的人
    口数就将比现在增加一倍?
\end{example}

\begin{solution}
设$y$代表$t$年后的人口数,于是
\[\begin{split}
    t=0,&\quad y_0=10\\
t=1,&\quad y_1=10+10\x1.1\%=10(1+1.1\%)\\
t=2,&\quad y_2=y_1+y_1\x1.1\%=y_1(1+1.1\%)=10(1+1.1\%)^2\\
t=3,&\quad y_3=y_2+y_2\x1.1\%=y_2(1+1.1\%)=10(1+1.1\%)^3\\
\cdots & \qquad \cdots\cdots\cdots\\
t=t,&\quad y_t=10(1+1.1\%)^t
\end{split}\]

设$t$年后的人口数比现在增加一倍,依题意,$t$满足
下面的指数方程:
\[10\x(1.011)^t=20\]
即:$1.011^t=2$。

在上式两边取对数得:$t\lg 1.011=\lg 2$
\[t=\frac{\lg2}{\lg1.011}\approx\frac{0.3010}{0.0047}\approx 64\]
答:约经过64年人口增加一倍。
\end{solution}

\section*{习题1.4}
\addcontentsline{toc}{subsection}{习题1.4}

\begin{enumerate}
    \item 利用对数进行计算:
\begin{multicols}{2}
\begin{enumerate}
    \item $\frac{154.8 \times 5.436}{12.72}$
    \item $\frac{54.83 \times 1.367}{2.832}$
    \item $\frac{103.8 \times 20.97}{5.174 \times 13.62}$
    \item $\frac{9.738 \times 21.09}{48.72 \times 0.8478}$
    \item $\frac{92.17^{2} \times 5.14^{3}}{2.184^{4} \times 0.5386^{2}}$
    \item $\frac{1.894^{4} \times 23.40^{3}}{44.15^{2} \times 0.9647^{3}}$
    \item  $\frac{8.150^{2} \times \sqrt[3]{14.36}}{24.38 \times \sqrt{8.734}}$
    \item $ \frac{12.48^{3} \times \sqrt[4]{5.760}}{1.842 \times \sqrt[3]{673.8}}$
    \item $(-5.32)^{3}+\sqrt[4]{0.0294}$
    \item $\frac{\sqrt[3]{-0.536}}{3.89^{2} \times 0.924^{2}}$
\end{enumerate}
\end{multicols}
\item 三角形的三条边的长分别是$a\approx 8.975$cm,$b \approx 7.863$cm, $c\approx 6.456$cm, 求它的面积。

\item  计算:
\begin{multicols}{2}
\begin{enumerate}
    \item $\frac{\sin 47^{\circ} 13'}{\tan22^{\circ} 27'}$
    \item $\frac{\sin 34^{\circ} 17' \times \tan82^{\circ} 6'}{\cos 12^{\circ} 37'}$
\end{enumerate}
\end{multicols}
\item 已知$\triangle ABC$中,$a\approx30$m, $B=22.5^{\circ}$, $C=112.5^{\circ}$, 求$\triangle ABC$的面积。

\item 在$\triangle ABC$中,$a=23.64$dm, $A=97^{\circ}15'$, $C=15^{\circ}31'$,
求$b,c$。
\item 
平行四边形的对角线$d\approx 15.67$dm, 它和两个邻边的夹
角$\alpha=47^{\circ}15'$, $\beta=26^{\circ}7'$, 求平行四边形的边长和各角的大小。

\item 已知$\triangle ABC$中,$a=324.1$m, $b=417.2$m, $C=113^{\circ}14'$,
求$c$和它的面积。
\item 已知$\triangle ABC$中,$b=63$, $c=36$, $C=29^{\circ}23'$, 求$B$。
\item 某工厂在五年内总产值每年平均增长19.2\%, 求第五年
的总产值对第一年增长的百分数。
\item 如果劳动生产率平均每年比上一年提高10.4\%, 那么
大约要经过几年可以提高到原来的2倍。
\item 解下面的方程:
\begin{multicols}{2}
\begin{enumerate}
    \item $4^{x}=2^{\tfrac{x+1}{x}}$
    \item $3^{x-3}-3^{x-1}=80$
    \item $5^{2x}-23.5^x-50=0$
    \item $8^{5-3x}=12^{4-2x}$
    \item $5^{1-x}=6^{x-3}$
    \item $\begin{cases}
        2^x\cdot 5^y=1\\
        5^{x+1}\cdot 2^y=2
    \end{cases}$
\end{enumerate}
\end{multicols}

\item 海中有一小岛$B$, 它的周围3海里内都有暗礁,今有一艘
海轮从西向东航行,若它在$A$点测得$\angle BAD=24^{\circ}$, 再
行5.2海里到$D$, 测得$\angle BDC=67^{\circ}$, 问海轮航向不变直向
东行有无触礁危险。
\item 为在河的两岸$A,B$间架一桥,需要精确测算$A$、$B$两
点间的距离,测量人员在
岸边$A$点附近另取一点
$C$, 并量出$AC=30.5$米,分
别测出$\angle A=75^{\circ}12'$, 
$\angle C=58^{\circ}51'$, 求$AB$的长。
\item 一人观测$A,B$两物体,
发现它们分别位于正北和
北$36^{\circ}$西的方向,若此人向
西北与向前进1.2公里,看到$A,B$二物分别位于东北
和正东方向,求$A,B$二物距离。
\item  如图,货轮在海上以35海里/时的速度沿着方位角(从
指北方向顺时针转到目标方向线的水平角)为$148^{\circ}$的方
向航行,为了确定船位,在$B$点观测灯塔$A$的方位角是
$126^{\circ}$, 航行半小时后到达$C$点,观测灯塔$A$的方位角是
$78^{\circ}$。求货轮到达$C$点时与灯塔$A$的距离(精确到1海里)。

\begin{figure}[htp]\centering
    \begin{minipage}[t]{0.48\textwidth}
    \centering
\begin{tikzpicture}[>=latex, scale=.7]
    \fill[rotate=30, cyan!20] (-.5,0.5) rectangle (6.5,2.5);
    \draw[rotate=30] (-.5,.5)--(6.5,.5)    ;
    \draw[rotate=30]  (-.5,2.5)--(6.5,2.5)    ;  
\draw[dashed] (0,0)node[left]{$A$}--(75.2:4.5)node[above]{$B$}--(5,0);
\draw (5,0)node[right]{$C$}--node[below]{30.5m}(0,0);


    \end{tikzpicture}
    \caption{第13题}
    \end{minipage}
    \begin{minipage}[t]{0.48\textwidth}
    \centering
    \begin{tikzpicture}[>=latex, scale=1.5]
    \draw (0,0)node[left]{$B$}--(3,-2)node[right]{$A$}--(1.5,-2.5)node[below]{$C$}--(0,0);
\draw[->] (0,0)--(0,.75)node[right]{北};
\draw[->] (1.5,-2.5)--(1.5,-2.5+.75);
\draw[->] (0,.5)  arc (90:-28:.5);
\node at (31:.5)[right] {$126^{\circ}$};
\draw[->] (0,.25)  arc (90:-56:.25);
\node at (17:.25)[right]{$148^{\circ}$};
\draw[->] (1.5,-2) arc (90:15:.5)node[above=10pt]{$78^{\circ}$};
    \end{tikzpicture}
    \caption{第15题}
    \end{minipage}
    \end{figure}

\item  某气象站每天定时施放气球进行高空观测,为了知道
气球离地面的高度,两观测员在$A,B$两点同时、同向
测得气球仰角$\alpha=45^{\circ}$, $\beta=34^{\circ}36'$, 且知道$A,B$两点相
距118米,求气球离地面的高度,已知测量仪器的高度为
1米(精确到1米)。
\item  在山顶铁塔上$B$处测得地面上一点的俯角$\alpha=54^{\circ}40'$,
在塔底$C$处测得点$A$的俯角$B=50^{\circ}1'$,已知铁塔$BC$部
分高27.3米,求山高$CD$(精确到1米)。

\begin{figure}[htp]
    \centering
\begin{tikzpicture}[scale=1, >=latex]
\draw (0,0)node[left]{$E$} rectangle (5,1);
\draw (0,1)node[left]{$D$}--(0,4)node[left]{$C$}--(5,1)node[right]{$B$};
\draw (0,4)--(3,1)node[above]{$A$}--(3,0);

\draw[<->] (2,1) arc (180:90+45:1)node[left=5pt]{$\alpha$};
\draw[<->] (4,1) arc (180:180-32:1)node[left=5pt]{$\beta$};
\draw[<->] (5,.7) --node[below]{118m}(3,.7);
\end{tikzpicture}
    \caption{第16题}
\end{figure}

\end{enumerate}

\section*{复习题一}
\addcontentsline{toc}{section}{复习题一}
\begin{enumerate}
    \item 计算:
\begin{enumerate}
\item    $3^{0}+3^{-1}-\left(1\frac{7}{9}\right)^{0.5}$
\item   $\left[1-(0.5)^{-2}\right] \div\left(-\frac{27}{8}\right)^{\tfrac{1}{3}}$
\item    $\left(2 \frac{3}{5}\right)^{0}+4^{-2} \times\left(2 \frac{1}{4}\right)^{\tfrac{1}{2}}-(0.01)^{0.5}$
\item   $\left(2 \frac{10}{27}\right)^{-\tfrac{2}{3}}-\left[(0.01)^{\sin\tfrac{\pi}{6}}+\left(6 \frac{1}{4}\right)^{-0.5}\right]$
\end{enumerate}
\item 计算:
\begin{enumerate}
    \item $3^{\log_{3} 9}$
    \item $3^{1-\log _{3} 7}$
    \item $5^{2 \log _{5} 3}-15 \log _{5} 1+\log _{3} \frac{1}{9}$
    \item $4\lg 2+3 \lg 5-\frac{1}{5} \lg 5$
    \item $(\lg 5)^{2}+\lg 2 \cdot \lg 50$
    \item $\lg 5 \cdot \lg 20+(\lg 2)^{2}$
    \item $\lg 12.5-\lg  \frac{5}{8}+\lg\sin 30^{\circ}$
    \item $\log _{2} \cos \frac{\pi}{4}-\log _{2} \sin \frac{\pi}{6}$
    \item $\log _{3} \frac{1}{27}+\log _{\tfrac{1}{2}} 8+\log _{2} 8+\log _{4} 64$
    \item $\frac{5 \lg 6-\lg 3}{1+\frac{1}{2} \lg 0.36+\frac{1}{3} \lg 8}$
    \item $\left(\log _{9} 5\right) \cdot\left(\log _{25} 27\right)$
    \item $2^{\tfrac{1}{5 \log _{5}4}}$
\end{enumerate}

  \item 化简:
 \begin{enumerate}
    \begin{multicols}{2}
     \item $\frac{\left(a^{\tfrac{3}{5}}b^{-\tfrac{6}{5}}\right)^{-\tfrac{1}{2}}\sqrt[5]{a^4}}{\sqrt[5]{b^3}}$
     \item $\left(\frac{8x^{-3}}{\sqrt{y^3z^{-6}}}\right)^{-\tfrac{1}{3}}$
     \item $\sqrt[3]{\frac{8a^3b^6}{27c^3d^9}}$
     \item $8x^{-\tfrac{1}{3}}\sqrt{y^{-\tfrac{1}{3}}x\sqrt[4]{y^{\tfrac{4}{3}}}}$
     \item $\left(2x^{\tfrac{1}{2}}+3y^{-\tfrac{1}{4}}\right)\left(2x^{\tfrac{1}{2}}-3y^{-\tfrac{1}{4}}\right)$
     \item $\left(m^{\tfrac{3}{2}}+n^{\tfrac{3}{2}}\right)\div \left(m^{\tfrac{1}{2}}+n^{\tfrac{1}{2}}\right)$
     \item $\frac{a-b}{a^{\tfrac{1}{3}}-b^{\tfrac{1}{3}}}-\frac{a+b}{a^{\tfrac{1}{3}}+b^{\tfrac{1}{3}}}$
    \end{multicols}
     \item $(1-x^2)^{-\tfrac{1}{2}}-[(1+x)(1-x)]^{\tfrac{1}{2}}-x^2 [(1+x)(1-x)]^{-\tfrac{1}{2}}$
 \end{enumerate} 

 \item 利用对数进行计算:
 \begin{multicols}{2}
\begin{enumerate}
 \item $\frac{8.15^{2} \times \sqrt[3]{14.36}}{24.38 \times \sqrt{8.734}}$
 \item $\sqrt[5]{\frac{2.591^{4} \times \sqrt[3]{0.0836}}{1.147^{2}}}$
 \item $(-5.32)^{3}+\sqrt[4]{0.0294}$
 \item $\sqrt[3]{79.836+\sqrt{156.374}}$
\end{enumerate}
 \end{multicols}

\item \begin{enumerate}
    \item 已知 $\log _{12} 27=a$, 求证 $\log _{6} 16=\frac{4(3-a)}{3+a}$.
    \item 已知 $a^{2}+b^{2}=6 a b\; (a>0, b>0)$,
 求证 $\lg \frac{a-b}{2}=\frac{1}{2}(\lg a+\lg b)$.
\end{enumerate}

\item  解下面方程:
\begin{multicols}{2}
    \begin{enumerate}
        \item  $\begin{cases}
            5^{x}=2^{-y} \\ 
            5^{2+y}=2^{2-x}
        \end{cases}$
        \item  $\begin{cases}
            2^{x}=3^{y}\\ 
            2^{y-1}=3^{x-1}
        \end{cases}$
        \item $x^{\lg x+2}=1000$
        \item $\log_2\log_3\log_5x =0$
        \item $\begin{cases}
            x^5y^3=5\\x^2y^7=11
        \end{cases}$
\end{enumerate}
\end{multicols}

\item 下图表示曲柄机构装置的图样,连接杆$MK$的长$\ell=125$
cm, 曲柄$OK$的长$r=25$cm,连杆与汽缸的轴$OM$成$\alpha$角,
曲柄与同一轴成角$\phi$(例如$\phi=50^{\circ}$), 求活塞$M$的位移$x$
(这里$x$为活塞由左端位置所走的距离)。

\begin{figure}[htp]
    \centering
\begin{tikzpicture}[>=latex, scale=1.7]
    \draw[dashdotted] (-5,0)--(1.2,0);
    \draw (-4.5,.5) rectangle (-2.5,-.5);
    \draw (0,0) circle (.75);
    \node at (-.75,0)[left]{$K_0$};
    \node at (.75,0)[right]{$K_1$};
\draw [fill=gray!20](-3.5,.5) rectangle (-3.6,-.5);
\draw[very thick] (0,0)node[below]{$O$}--node[above]{$r$}(90+45:.75)node[above]{$K$}--node[above]{$\ell$}(-3.5,0)node[below]{$M$};
\draw [dashed](-4.3,.5)node[below]{$M_0$}--(-4.3,-.5);
\draw [dashed](-2.7,.5)node[below]{$M_1$}--(-2.7,-.5);
\draw (-.35,0) arc (180:180-45:.35)node[left=2pt]{$\phi$};
\draw[dashed] (90+45:.75)--(-.75/1.414,0);
\draw[|<->|] (-4.3,-.7)--node[fill=white]{$x$}(-3.5,-.7);

\draw (-2,0) arc (0:10:1.5);
\node at (-2,.2)[right]{$\alpha$};
\end{tikzpicture}
    \caption{第7题}
\end{figure}

\item 如图,两个建筑物水平距离为32.6米,从$A$点观测$B$点
的俯角是$35^{\circ}12'$, 观测$C$点的俯角为$43^{\circ}24'$, 求此二建
筑物的高。

\begin{figure}[htp]
    \centering
\begin{tikzpicture}[>=latex]
\fill [pattern=north east lines] (0,0) rectangle (-.5,4);   
\draw (0,0)--(0,4)node[above]{$A$}--(3,4);
\fill [pattern=north east lines] (5,0) rectangle (5.5,1);   
\draw (5,0)--(5,1)--(5.5,1);
\draw (0,0)--(0,-1);
\draw (5,0)--(5,-1);
\draw[<->](0,-.5)--node[fill=white]{32.6m}(5,-.5);
\node at (0,-.2) [left]{$D$};
\node at (5,-.2) [right]{$C$};
\draw (0,4)--(5,0);\draw (0,4)--(5,1)node[above]{$B$};
\draw(-1,0)--(6,0);

\end{tikzpicture}
    \caption{第8题}
\end{figure}

\item 一只渔船在航行中不幸遇险,发出警报,在遇险南西10
海里处有一具货轮,收到警报后立即侦察,发现这只渔
船的航向是东$15^{\circ}$南,正在用每小时9海里的速度向某
小岛靠近,如果要在40分钟内把这只渔船营救出来,求
货轮航行方向和速度(精确到分)。
\item 
在塔的正西处$A$点测得塔顶的仰角是$45^{\circ}$, 在它的东南
处$B$点测得仰角是$60^{\circ}$, $AB$相距为266尺。求塔高。











\end{enumerate}




%   \chapter{直线、平面坐标化}
\section{直线坐标化}

\subsection{直线的有向化}
在日常生活或科学研究中,往往需要考虑直线的方向.
凡具有方向的直线称为有向直线,给直线以一定的方向就是
将直线有向化,一条直线含有两个相反的方向,所以一条直
线有两种相反的有向化.

例如,北京市的长安街有西、东之分;在街上奔驰的车
辆有往、返之别.

在任一条直线上,可以规定它的某一方向作为正向,它
的另一方向则作为负向.在数学里,一般是规定直线的向上
方向或向右方向为正向,向下方向或向左方向为负向,如图
2.1.
\begin{figure}[htp]
    \centering
\begin{tikzpicture}[>=latex]
\draw[<->] (0,0)node [right]{负向} --(0,3)node [right]{正向};
\draw[<->] (2.5,1.5)node [below]{负向} --(5,1.5)node [below]{正向};
\end{tikzpicture}
    \caption{}
\end{figure}

每条线段都有两个端点,如果把一个端点取做始点,另
一个端点取做终点,由始点到终点的方向叫做该“有向线段”
的方向.如果它的方向和它所位于的有向直线的方向一致,
则它是一个正向线段;如果它们的方向相反则它是一个负向
线段.因此,可以说一条线段有两个相反的方向.凡是考虑
方向的线段,称为有向线段.例如,在有向直线$\ell$上取$A,B$
两点(图2.2),则有向线段$AB$是正向线段,有向线段$BA$是
负向线段.
\begin{figure}[htp]
    \centering
\begin{tikzpicture}[>=latex]
\draw[->] (0,0) --(5,0)node[right]{$\ell$};
\draw (1.5,0)[fill=black] circle (1.5pt)node[below]{$A$};
\draw (3.5,0)[fill=black] circle (1.5pt)node[below]{$B$};
\end{tikzpicture}
    \caption{}
\end{figure}

表示有向线段,一般是将始点写在终点的左边,并在两
个字母上面加一个箭头.如以$A$为始点,$B$为终点的有向线
段记作$\Vec{AB}$, 以$B$为始点.$A$为终点的有向线段记为$\Vec{BA}$.

我们可以用一个有正负号的实数来同时表达在有向直线
上的有向线段的长度和它的正、负方向.这个实数称为该有
向线段的数量.

如图2.2. 设有向线段$\Vec{AB}$的数量是$\alpha\; (\alpha>0)$, 则有向线
段$\Vec{BA}$的数量是$-\alpha$.有时也用$AB$表示有向线段$\Vec{AB}$的数量,用
${BA}$表示有向线段$\Vec{BA}$的数量.显然有向线段$\Vec{AB}$与$\Vec{BA}$的数量
有下列关系:
\[AB=-BA\]
或\[AB+BA=0\]

符号
$|\Vec{AB}|$表示只考虑有向线段$\Vec{AB}$的长度而不考虑方
向.符号$|\Vec{AB}|$也称有向线段$\Vec{AB}$的数量的绝对值.

\subsection{直线的坐标化}
规定了方向,规定了基准点或原点,并规定了长度单位
的直线,称为\textbf{数轴}.这就是直线的坐标化,如图2.3,在日常
生活中,用直尺上的刻度表示长度的大小,用温度计上的刻
度表示温度的高低,这实际上就是把实数$\pm a$对应于数轴上一
点,这点和原点距离等于$a\; (a\ge 0)$, 由于数轴是连续不间断
的,所有的实数,都能用数轴上的点表示出来.在数轴上,
数0用原点表示,正数对应着原点右方的点,负数对应着
原点左方的点.

\begin{figure}[htp]
    \centering
\begin{tikzpicture}[>=latex]
\draw[thick, ->] (-2,0)--(5,0)node[right]{$x$};
\foreach \x in {0,1,3.3}
{
    \draw(\x,0)--(\x,0.2);
}
\node at (0,0) [below]{$O$};
\node at (1,0) [below]{$E$};
\node at (3.3,0) [below]{$P$};
\node at (1,0.2) [above]{$1$};
\node at (3.3,0.2) [above]{$x$};

\end{tikzpicture}
    \caption{}
\end{figure}

\begin{example}
    把下列各数用数轴上的点表示出来:
\[2,\quad -2,\quad -3.5,\quad 3.5,\quad \sqrt{2},\quad \sqrt{3}\]
\end{example}

\begin{solution}
    先画数轴,然后在数轴上找出相应的点.
图2.4中的$A,B,C,D,E,F$分别表示$2,\; -2,\; -3.5,\; 
3.5,\; \sqrt{2},\; \sqrt{3}$.

\begin{figure}[htp]
    \centering
\begin{tikzpicture}[>=latex, scale=1.4]
\draw[->] (-4,0)--(4.5,0) node[right]{$x$};
\foreach \x/\xtext in {2/A,-2/B,-3.5/C,3.5/D,1.414/E,1.732/F}
{
    \draw (\x,0)--(\x,.1);
    \draw (\x,0)[fill=black] circle (1pt);
    \node at (\x,0) [above]{$\xtext$};
}

\foreach \x in {2,-2,-3.5,3.5}
{
    \node at (\x,0) [below]{$\x$};
}

\foreach \x in {-3,-1,0,1,3,4}
{
    \draw (\x,0)--(\x,.1);
    \node at (\x,0) [below]{$\x$};
}
\node at (1.414,0)[below]{$\sqrt{2}$};
\node at (1.732,0)[below]{$\sqrt{3}$};

\draw[dashed] (0,0)--node[left]{$\sqrt{2}$}(1,1)--node[right]{1}(1,0);
\draw[dashdotted] (1.414,0) arc (0:45:1.414);

\draw[dashed]  (1,1)--node[below]{1}(1-1/1.414, 1+1/1.414)--node[left]{$\sqrt{3}$}(0,0);
\draw[dashdotted] (1.732,0) arc (0:80:1.732);
\end{tikzpicture}
    \caption{}
\end{figure}
\end{solution}


反过来,在数轴上每取一点$P$(图2.3), 有向线段$\Vec{OP}$总是
单位有向线段$\Vec{OE}$的实数倍;即存在一个实数$x$, 使得
\[\Vec{OP}=x\Vec{OE} \]
如果$P$点和$E$点在原点同侧,则$x>0$; 如果$P$点和$E$点在原点
异侧,则$x<0$. 从上述两方面看,数轴上的点与实数之间可以
建立一一对应.因此,在直线上建立数轴后,就形成直线的一
个坐标系,实数$x$叫做$P$点的坐标.换言之,数轴上一个点$P$
的坐标就是以原点为始点,以该点为终点的有向线段$\Vec{OP}$的数
量.

如图2.4, $F$点的坐标是$\sqrt{3}$, 而有向线段$\Vec{OF}$的数量是
$\sqrt{3}$; $C$点的坐标是$-3.5$, 而有向线段$\Vec{OC}$的数量是$-3.5$. 即$OF=\sqrt{3}$; $OC=-3.5$.

从图2.4中可以看出,和原点距离相等的两个点,如点$A$
和点$B$, 点$C$和点$D$, 它们所表示的数分别是2和$-2$; $-3.5$
和3.5.这样的数称为相反数.

根据相反数的概念,一个实数的绝对值曾规定如下:
\begin{blk}{}
    正数或零的绝对值是它们自己;
负数的绝对值是它的相反数.
\end{blk}

用符号表示如下:
\[|x|=\begin{cases}
    x, & x\ge 0\\
    -x, & x<0
\end{cases}\]

借助于数轴,我们可以对一个数的绝对值作出几何解
释:把这个数用数轴上的点表示出来,那么这个数的绝对值
就是原点到表示这个数的点的距离.从这里看出,除零以外,
正数和负数的绝对值都是正数,而且两个相反数的绝对值相
等.

前面已说明在数轴上的一点$P$的坐标$x$就是以原点$O$为始
点,以$P$点为终点的有向线段$\Vec{OP}$的数量$OP=x$.

现在进一步研究在数轴上任意一条有向线段的数量表示
法;在数轴上任取不同的两点$A,B$, 设$A$点的坐标为$x_1$, $B$
点的坐标为$x_2$. 如图2.5(1), 现在要求$AB=?$

由图看出:$|\Vec{OB}|=|\Vec{OA}|+|\Vec{AB}|$
故得
\[|\Vec{AB}|=|\Vec{OB}|-|\Vec{OA}|\]
$\because\quad \Vec{AB}, \Vec{OB}, \Vec{OA}$都是正有向线段,

$\therefore\quad AB=OB-OA=x_2-x_1$

当然,$A,B$两点在数轴上的位置不只如图2.5(1)的那
样一种,还有图2.5(2)--(6)几种情况.

\begin{figure}[htp]
    \centering
    \begin{tikzpicture}[>=latex]
\foreach \y in {-1,-2,...,-6}
{
    \draw[->, thick](-4.5,\y)--(5,\y)node[right]{$x$};
}
\foreach \x/\xtext in {-4/O,0/A,3/B}
{
    \draw (\x,-1)--(\x,-1+.1);
    \node at (\x,-1)[below] {$\xtext$};
}
\foreach \x/\xtext in {-3/A,0/O,4/B}
{
    \draw (\x,-2)--(\x,-2+.1);
    \node at (\x,-2)[below] {$\xtext$};
}
\foreach \x/\xtext in {-4/A,0/B,3/O}
{
    \draw (\x,-3)--(\x,-3+.1);
    \node at (\x,-3)[below] {$\xtext$};
}
\foreach \x/\xtext in {-4/O,0/B,3/A}
{
    \draw (\x,-4)--(\x,-4+.1);
    \node at (\x,-4)[below] {$\xtext$};
}
\foreach \x/\xtext in {-2/B,0/O,3/A}
{
    \draw (\x,-5)--(\x,-5+.1);
    \node at (\x,-5)[below] {$\xtext$};
}
\foreach \x/\xtext in {-2/B,0/A,3/O}
{
    \draw (\x,-6)--(\x,-6+.1);
    \node at (\x,-6)[below] {$\xtext$};
}
\node at (-5,-1){(1)};\node at (-5,-2){(2)};
\node at (-5,-3){(3)};\node at (-5,-4){(4)};
\node at (-5,-5){(5)};\node at (-5,-6){(6)};
    \end{tikzpicture}
    \caption{}
\end{figure}



但不管哪一种情况,都可证得$AB=x_2-x_1$.譬如在图2.5
的(2)中,$AB=|\Vec{AB}|$, $OA=-|\Vec{OA}|$, $OB=|\Vec{OB}|$, 而
\[|\Vec{AB}|=|\Vec{OA}|+|\Vec{OB}|\]
$\therefore\quad AB=-OA+OB$,就是
\[AB=OB-OA=x_2-x_1\]

其它情形,同样地可以证明,这就是说,\textbf{在数轴上任意一
条有向线段的数量等于其终点坐标与始点坐标的差}.

\section*{习题2.1}
\addcontentsline{toc}{subsection}{习题2.1}
\begin{enumerate}
    \item 在数轴上$A,B$两点的坐标分别是$x_1,x_2$, 设:
\begin{multicols}{2}
\begin{enumerate}
    \item $x_1=-2,\quad x_2=2$
    \item $x_1=-9,\quad x_2=-11$
    \item $x_1=-3,\quad x_2=5$
    \item $x_1=0,\quad x_2=-8$
\end{enumerate}
\end{multicols}
试求有向线段$AB$与$BA$的数量以及$|AB|,\; |BA|$

\item 若$A,B,C$是数轴上任意三个点,它们的坐标分别是$a,
b,c$, 求证:$AB+BC+CA=0$
\item 若$A,B$两点坐标各为$a,b$, 求:
\begin{enumerate}
    \item 中点$C$的坐标,
    \item 二个三等分点的坐标.
\end{enumerate}

\item 
在数轴上确定坐标是$2+\sqrt{2}$, $3-\sqrt{3}$, $\sqrt{5}$, $\sqrt{6}$的点.
\item 
设数轴上的$P$点的坐标$x$满足条件:
\begin{enumerate}
    \item $|x|=2$
    \item $|x-1|=5$
    \item $|2x+1|=5$
\end{enumerate}
求$P$点的坐标.
\item 如果在一个杆子上的两点$A,B$分别挂上重量等于$p$, $q$公
斤的物品,假设$A,B$的坐标分别是$a,b$, 试问支点(也就
是重心)的坐标应该是多少才能平衡(假定杆子的重量可
以忽略不计).


\end{enumerate}

\section{平面的坐标化}

\subsection{平面的直角坐标化}
在平面上取一定点$O$作为基准点,或称为原点,过原点引
两条互相垂直的数轴.图2.6是常用的取法,把一条数轴取
为水平方向,称为横轴,或叫做$x$轴,以向右为正向;另一
条数轴取为铅垂方向,称为纵轴,或叫$y$轴,以向上为正向.
上述的取法称为在平面内建立直角坐标系,利用直角坐标
系,我们可以用一对有序的实数来表示平面内一个点的位
置.

\begin{figure}[htp]
\centering
\begin{minipage}[t]{0.48\textwidth}
\centering
\begin{tikzpicture}[>=latex, scale=.7]
    \draw[->,thick] (-3,0)--(3,0) node[right]{$x$};
    \draw[->,thick]  (0,-3)--(0,3)node[right]{$y$};
\node at (-.3,-.3){$O$};
\node at (1.5,1.5){一};\node at (-1.5,1.5){二};
\node at (-1.5,-1.5){三};\node at (1.5,-1.5){四};
\node at (1.5,1){$(+,+)$};\node at (-1.5,1){$(-,+)$};
\node at (-1.5,-2){$(-,-)$};\node at (1.5,-2){$(+,-)$};
\end{tikzpicture}
\caption{}
\end{minipage}
\begin{minipage}[t]{0.48\textwidth}
\centering
\begin{tikzpicture}[>=latex]
    \draw[->,thick] (-1,0)--(3,0)node[right]{$x$};
    \draw [->,thick] (0,-1)--(0,3)node[right]{$y$};
    \node at (-.2,-.2){$O$};
\draw (2,0)node[below]{$M$}--(2,2)node[right]{$P(x,y)$}--(0,2)node[left]{$N$};

\end{tikzpicture}
\caption{}
\end{minipage}
\end{figure}



在平面上任取一点$P$(图2.7), 过点$P$分别作$y$轴,$x$轴的平
行线$PM$, $PN$, 于是得到有向线段$\Vec{OM}$, $\Vec{ON}$, 而有向线段$\Vec{OM}$, $\Vec{ON}$分别在$x$轴,$y$轴上的数量是$x,y$, 也就是说,在平面上
任取一点$P$, 可以找到与之对应的一对实数$x,y$.反过来说,
任取一对实数$x,y$, 在$x$轴,$y$轴上可以分别找到与之对应
的以原点$O$为始点,以$M,N$为终点的有向线段$\Vec{OM}$, $\Vec{ON}$,
过$M,N$各作$y$轴、$x$轴的平行线,这两条平行线必交于与之
对应的一点$P$. 这样,平面内的点和所有的有序实数对$(x,y)$之间就建立了一一对应的关系.我们就用这样一对有序实
数$x,y$作为点$P$在平面内位置的标记,称为点$P$的坐标.

有向线段$\Vec{OM}$的数量$x$称为点$P$的横坐标或横标,有向线
段$\Vec{ON}$的数量$y$称为点$P$的纵坐标或纵标,点$P$的坐标记为$(x,
y)$.

横、纵两轴把平面分为四部分,按逆时针次序分别叫做
第一象限、第二象限、第三象限和第四象限.显然,在第一
象限点的横标、纵标都是正数;在第二象限点的横标为负,纵
标为正;在第三象限点的横标、纵标都是负数;在第四象限
点的横标为正、纵标为负.反过来也对(图2.6).

通过坐标系的建立,可以把平面内的点和有序实数对
$(x,y)$一一对应起来,这就有可能把平面内关于点的几何问
题,化成关于这些点的坐标的代数的问题来进行研究.这也
就是用代数方法即解析方法来解决几何问题.

\begin{example}
    菱形的每边长为5, 一条长对角线为8, 如果以
    菱形的对角线作为坐标轴,试求此菱形各顶点的坐标,并描
    绘出菱形的图形.
\end{example}

\begin{solution}
    因为菱形的对角线互相垂直平分.设菱形顶点分别
为$A,B,C,D$. 根据已知条件,设长对角线在$x$轴上,则知
$A,C$坐标各为$(4,0)$, $(-4,0)$. 设$B,D$坐标各为$(0,b)$,
$(0,-b)$.

$\therefore\quad |AB|=BC|=|CD|=|DA|=5$

由勾股定理可求得$B,D$的坐标各为$(0,3)$, $(0,-3)$.

同理可得另一种情形:(图2.8)
\[A(3,0),\quad B(0,4),\quad C(-3,0),\quad D(0,-4)\]
\begin{figure}[htp]
    \centering
    \begin{tikzpicture}[>=latex, scale=.5]
\begin{scope}
    \draw[->](-5,0)--(5,0)node[right]{$x$};
    \draw[->](0,-4)--(0,4)node[right]{$y$};
    \node at (-.4,-.4){$O$};
\draw (-4,0)node[above]{$C$}--(0,3)node[right]{$B$}--(4,0)node[above]{$A$}--(0,-3)node[right]{$D$}--(-4,0);
\end{scope}

\begin{scope}[xshift=12cm]
    \draw[->](-4,0)--(4,0)node[right]{$x$};
    \draw[->](0,-5)--(0,5)node[right]{$y$};  
    \node at (-.4,-.4){$O$};
    \draw (-3,0)node[above]{$C$}--(0,4)node[right]{$B$}--(3,0)node[above]{$A$}--(0,-4)node[right]{$D$}--(-3,0);
\end{scope}
    \end{tikzpicture}
    \caption{}
\end{figure}

\end{solution}

\begin{example}
  一船向北偏东$60^{\circ}$航行40公里,再向北偏东$45^{\circ}$航
行24公里,这时船在起点东面多少公里?北面多少公里(精确
到1公里)?  
\end{example}

\begin{figure}[htp]
    \centering
\begin{tikzpicture}[>=latex]
\draw[->](-1,0)--(6.5,0)node[right]{$x$};
\draw[->] (0,-1)--(0,5)node[right]{$y$};
\draw (2*1.732,0)node[below]{$A_x$}--(2*1.732, 2)node[right]{$A$}--(0,2)node[left]{$A_y$};
\draw (2*1.732+1.2*1.414,0)node[below]{$B_x$}--(2*1.732+1.2*1.414, 2)node[right]{$H$}--(2*1.732+1.2*1.414, 2+1.2*1.414)node[right]{$B$}--(2*1.732,2+1.2*1.414)node[above]{$K$}--(0,2+1.2*1.414)node[left]{$B_y$};
\node at (-.2,-.2){$O$};
\draw (2*1.732+1.2*1.414, 2) rectangle(2*1.732,2+1.2*1.414);

\draw[very thick] (0,0) -- (2*1.732, 2);
\draw[ thick] (0,0) -- (2*1.732+1.2*1.414, 2+1.2*1.414);
\draw[very thick] (2*1.732, 2) -- (2*1.732+1.2*1.414, 2+1.2*1.414);

\draw(1,0)  arc (0:30:1) node[below]{$30^{\circ}$};
\draw(0,.75) arc (90:30:.75)node[left]{$60^{\circ}$};
\draw (2*1.732+.7, 2) arc (0:45:.7) node[right]{$45^{\circ}$};
\draw (2*1.732, 3) arc (90:45:1) node[left]{$45^{\circ}$};


\end{tikzpicture}
    \caption{}
\end{figure}


\begin{solution}
    建立直角坐标系$x-O-y$, 如图2.9.设起点在原点
$O$,$y$轴代表正北方向,$x$轴代表正东方向,船向北偏东$60^{\circ}$航行40公里的位移用有向线段$\Vec{OA}$代表,再向北偏东$45^{\circ}$航行24公
里的位移用有向线段$\Vec{AB}$代表,这时船在$B$点的坐标是$(x,y)$.

因为$A$点的横坐标$x_A$是有向线段$\Vec{OA}$在$x$轴上的射影$\Vec{OA}_x$的数量,即
\begin{equation}
\begin{split}
      x_A=OA&=|OA|\cos(90^{\circ}-60^{\circ})\\
    &=  40\cos 30^{\circ}\\
    &=20\sqrt{3}
\end{split}
\end{equation}
$A$点的纵坐标$y$是有向线段$\Vec{OA}$在$y$轴上的射影$\Vec{OA}_y$的数量,即
\begin{equation}
    \begin{split}
y_A=OA&=|OA|\cos60^{\circ}\\
&=40\cos60^{\circ}\\
&=40\x\frac{1}{2}=20        
    \end{split}
\end{equation}

又有向线段$\Vec{AB}$在$x$轴上的射影$\Vec{A_xB_x}$的数量是
\begin{equation}
A_xB_x=AH=|AB|\cos45^{\circ}=24\x\frac{\sqrt{2}}{2}=12\sqrt{2}        
\end{equation}
有向线段$\Vec{AB}$在$y$轴上射影的数量是
\begin{equation}
    A_yB_y=AK=|AB|\cos45^{\circ}=24\x\frac{\sqrt{2}}{2}=12\sqrt{2}        
    \end{equation}

另一方面,
\[\begin{split}
    A_xB_x&=OB_x-OA_x=x-x_A=x-20\sqrt{3}\\
    A_yB_y&=OB_y-OA_y=y-y_A=y-20 
\end{split}\]
将(2.3), (2.4)分别代入上面两个等式的左端,得
\[\begin{cases}
    12\sqrt{2}=x-20\sqrt{3}\\
    12\sqrt{2}=y-20
\end{cases}\]
因此:
\[\begin{split}
    x&=20\sqrt{3}+12\sqrt{2}\approx 20\x 1.73+12\x 1.41 \approx 51.55\approx 52\\
    y&=20+12\sqrt{2}\approx 20+12\x 1.41\approx 36.9\approx 37
\end{split}\]
答:船在起点东面约52公里,在北面约37公里的地方.
\end{solution}

\begin{ex}
\begin{enumerate}
    \item 正方形的边长为$b$, 对角线在坐标轴上,试求各顶点的
    坐标.
    \item 已知矩形相邻两边的长分别为$a,b$, 各在$x$轴及$y$轴上,
    试求其各顶点坐标.
    \item 已知正方形的边长为$2a$, 一顶点为原点$O\;(0,0)$, 一对角线
    在$x$轴的正方向上,试求其它顶点的坐标.
    \item 正三角形的边长为$b$, 一顶点为原点$O\;(0,0)$, 其高在$y$
    轴上,试求其余两顶点的坐标.
    \item 有一边长是$a$的正三角形$ABC$, $AB$边在$x$轴上,$AB$边的
    中点是原点,试求$\triangle AEC$各顶点的坐标.
    \item 在平行四边形$ABCD$中,$|AB|=8$, $|AD|=5$, $\angle A=
    60^{\circ}$, 如果以点$A$为原点,$AB$所在直线为$x$轴,$C$点所在
    象限为第一象限,试求各顶点的坐标.
\end{enumerate}
\end{ex}

\subsection{两点间的距离}
若两点为$P_1\;(x_1,y_1)$, $P_2\;(x_2,y_2)$, 则这两点间的距离为;
$$|\Vec{P_1P_2}|=\sqrt{(x_2-x_1)^2+(y_2-y_1)^2}$$

\begin{figure}[htp]
    \centering
\begin{tikzpicture}[>=latex,scale=.6]
    \draw[->](-5,0)--(5,0)node[right]{$x$};
    \draw[->] (0,-4)--(0,5)node[right]{$y$};
\draw[dashed](-4,0)node[above]{$M_1$}--(-4,-3)--(0,-3)node[right]{$N_1$}--(3,-3)node[right]{$S$}--(3,0)node[above]{$M_2$}--(3,4)--(0,4)node[left]{$N_2$};
\draw[very thick](-4,-3)node[left]{$P_1$}--(3,4)node[right]{$P_2$};
\node at (-.3,-.3){$O$};
\end{tikzpicture}
    \caption{}
\end{figure}



事实上,由$P_1$, $P_2$引$y$轴、$x$轴的平行线$P_1M_1$, $P_2M_2$,
$P_1N_1$, $P_2N_2$, 延长$P_1N_1$与$P_2M_2$相交于$S$(图2.10), 在直角三角形$P_1SP_2$中,由勾股定理得:
\[|\Vec{P_1P_2}|^2=|\Vec{P_1S}|^2+|\Vec{SP_2}|^2 \]
其中,$|\Vec{P_1S}|=|\Vec{M_1M_2}|=|x_2-x_1|$,$|\Vec{SP_2}|=|\Vec{N_1N_2}|=|y_2-y_1|$.

于是$$|\Vec{P_1P_2}|^2=|x_2-x_1|^2+|y_2-y_1|^2$$
开平方得:
$$|\Vec{P_1P_2}|=\sqrt{(x_2-x_1)^2+(y_2-y_1)^2}$$
这就是平面上任意两点间的距离公式.

有了两点间距离公式,我们还可以把实数的绝对值和算
术平方根的概念联系起来,把两点间距离公式应用到点
$P\; (x,0)$和原点$(0,0)$上,得
\[d=\sqrt{(x-0)^2+(0-0)^2}=\sqrt{x^2}\]
但另一方面,在$x$轴上表示数$x$的$P$点到原点的距离就是$|x|$, 
因此
\[|x|=\sqrt{x^2}\]
这表示一个数的绝对值就等于这个数的平方的算术平方根.

\begin{example}
    试求两点$A(e,a)$, $B(e,b)$间的距离.
\end{example}

\begin{solution}
    设$x_1=e$, $y_1=a$, $x_2=e$, $y_2=b$, 由距离公式可得:
\[|\Vec{AB}|=\sqrt{(e-e)^2+(b-a)^2}=|b-a|\]
\end{solution}

\begin{example}
    试求$P(c,f)$, $Q(d,f)$两点间的距离.
\end{example}


\begin{solution}
    设$x_1=c$, $y_1=f$, $x_2=d$, $y_2=f$, 由两点间的距离
公式可得
\[|\Vec{PQ}|=\sqrt{(d-c)^2+(f-f)^2}=\sqrt{(d-c)^2}=|d-c|\]
\end{solution}

由以上两例可以看出,如果两点间的线段平行于$x$轴或
$y$轴,则其距离分别等于这两点横标、纵标的差的绝对值.


\begin{example}
    已知$\triangle ABC$各顶点为$A(-\sqrt{3},\sqrt{2})$,
$B(-\sqrt{2},\sqrt{3})$, $C(\sqrt{3},\sqrt{2})$, 试求此三角形三边的长.
\end{example}

\begin{solution}
$\because\quad A(-\sqrt{3},\sqrt{2}),\qquad B(-\sqrt{2},\sqrt{3})$

$\therefore\quad |\Vec{AB}|=\sqrt{\left(-\sqrt{2}+\sqrt{3}\right)^2+\left(\sqrt{3}-\sqrt{2}\right)^2}=\sqrt{10-4\sqrt{6}}$

$\because\quad B(-\sqrt{2},\sqrt{3}),\qquad C(\sqrt{3},\sqrt{2})$

$\therefore\quad |\Vec{BC}|=\sqrt{\left(\sqrt{3}+\sqrt{2}\right)^2+\left(\sqrt{2}-\sqrt{3}\right)^2}=\sqrt{10}$

$\therefore\quad |\Vec{CA}|=\sqrt{\left(-\sqrt{3}-\sqrt{3}\right)^2+\left(\sqrt{2}-\sqrt{2}\right)^2}=2\sqrt{3}$

$\therefore\quad \triangle ABC$的三边的长分别为$\sqrt{10-4\sqrt{6}}$, $\sqrt{10}$, $2\sqrt{3}$
\end{solution}


\begin{example}
    试证:矩形的对角线等长.
\end{example}

\begin{solution}
设知形的四个顶点分别为$A(0,0)$, $B(0,b)$, $C(a,b)$, $D(a,0)$,其对角线为$AC$,$BD$.

由两点距离公式,得:
\[\begin{split}
    |\Vec{AC}|&=\sqrt{(a-0)^2+(b-0)^2}=\sqrt{a^2+b^2}\\
|\Vec{BD}|&=\sqrt{(a-0)^2+(0-b)^2}=\sqrt{a^2+b^2}
\end{split}\]
故知:$|\Vec{AC}|=|\Vec{BD}|$.
\end{solution}

\begin{example}
    动点$P(a,b)$与二定点$A(5,7)$, $B(-3,-4)$等距
离,试求动点横、纵坐标应满足的条件.
\end{example}


\begin{solution}
    由距离公式得:
\[\begin{split}
    |\Vec{PA}|&=\sqrt{(5-a)^2+(7-b)^2}\\
    |\Vec{PB}|&=\sqrt{(-3-a)^2+(-4-b)^2}
\end{split}\]
由于动点$P$与定点$A,B$等距离,即$|\Vec{PA}|=|\Vec{PB}|$,故得:
\[\sqrt{(5-a)^2+(7-b)^2}=\sqrt{(-3-a)^2+(-4-b)^2}\]
化简后,得:
\[16a+22b-49=0\]
这就是$a,b$应满足的条件.
\end{solution}

\section*{习题2.2}
\addcontentsline{toc}{subsection}{习题2.2}
\begin{enumerate}
    \item 试求下列两点连接线段的长:
\begin{enumerate}
    \item $A(-4,4),\qquad B(\sqrt{3},\sqrt{2})$
    \item $A(7,1),\qquad B\left(\frac{a}{2},\frac{\sqrt{3}}{2}a\right)$
    \item $M(a+b,a+c),\qquad N(a+c,b+c)$
    \item $P_1(at^2_2, 2at_1t_2),\qquad P_2(at^2_2,0)\quad (a>0)$
\end{enumerate}

\item 证明以$A(3,8)$, $B(-11,3)$, $C(-8,2)$为顶点的三角形
是等腰三角形.
\item \begin{enumerate}
    \item 证明以$A(7,5)$, $B(2,3)$, $C(6,-7)$为顶点的三角
形是一直角三角形;
\item 求这直角三角形的面积.
\end{enumerate}

\item  证明$A(-3,-2)$, $B(5,2)$, $C(9,4)$三个点在一条直线
上.
\item 求这样一点$P$的坐标使它和$A(1,7)$, $B(8,6)$, $C(7,-1)$
的距离相等.
\item 已知$P(b,4)$与$Q(0,-4)$的距离为10, 求$b$的值.
\item 一个动点$P(x,y)$与定点$Q(-3,4)$的距离永远等于5, $x,
y$应当满足什么条件.
\end{enumerate}


\section*{复习题二}

\addcontentsline{toc}{section}{复习题二}

\begin{enumerate}
    \item 设$A_1,A_2,A_3,A_4$是同一条有向直线上的四个点,求
证不论它们的位置怎样,都有:
\[A_1A_2+A_2A_3+A_3A_4+A_4A_1=0\]
\item 设$A,B,C,D$是同一条直线上的四个点,求证不论它
们的位置排列的顺序怎样,关系式
\[AB\cdot CD+BC\cdot AD=AC\cdot BD\]
总是成立的.

(提示:设$A$、$B$、$C$、$D$是同一条直线上的四个点,
其坐标分别是$a,b,c,d$.利用坐标计算)

\item 南北向的直道上有一车站,一人原在车站南面离车站5
里之处,他向北走了9里,后又回过身来向南走了3
里,问此人最后到达的地点离车站多远,又在车站哪一
面?用计算方法并在直线坐标轴上表明此人最后到达的
地点.
\item 若一动点$P(x,y)$与两定点$A(2,-1)$, $B(-7,3)$等距离,
它的坐标应当满足什么条件?
\item  在$y$轴上有一点与$A(2,-1)$, $B(-7,3)$两点等距离,求
它的坐标.
\item  试证以$A(a,0)$, $B(-a,0)$, $C(0,\sqrt{3}a)$为顶点的三角
形是个等边三角形.
\item 证明正方形的对角线相等.
\item  证明$(1,4)$, $(4,1)$, $(5,5)$是一个等腰三角形的三个顶
点.
\item  证明$(-4,-2)$, $(2,0)$, $(8,6)$, $(2,4)$是一个平行四边
形的四个顶点,并求它的两条对角线的长度.

(提示:证明两组对边分别相等)
\item 用解析法证明平行四边形各边平方的和等于对角线平
方的和.
\end{enumerate}


%    \chapter{不等式和解不等式}

\section{大小次序与不等式}

\subsection{大小关系与次序关系}

“数系”的概念由于对于各种“量”的问题作了系统的讨论
而产生.由于计算个数和度量各种量的需要而产生了整数系
和实数系.日常生活的量与量的问题除了可以运算之外,还常
有很自然的“大小”关系,将这些关系抽象化,即得数与数之
间的大小关系,运算与大小关系是数系的两种基本结构.本章
将以数系在大小关系上的基本性质为出发点,逐步讨论不等
式的性质.

在日常生活中,我们说$A$量大于$B$量的意义是:“从$A$量
减去$B$量后还有剩余”,所以在数系中我们定义“$a$大于$b$”的意
义为$a-b$是一个正数.也就是:

\begin{blk}{定义}
\[\begin{split}
    a-b\text{是一个正数}&\Longleftrightarrow a>b\\
    a-b=0  &\Longleftrightarrow a=b\\
    a-b\text{是一个负数}&\Longleftrightarrow a<b
\end{split}\]
所以,$a>b$与$b<a$是同一回事.
\end{blk}


由这个定义,我们有下面的特例:
\[\begin{split}
    a>0 &\Longleftrightarrow a\text{是一个正数}\\
    a<0&\Longleftrightarrow a\text{是一个负数} 
\end{split}\]
这种正负数的表示法,前面已经用过.

我们知道任何一个实数或为正,或为零,或为负,上述
三种关系有且仅有一种成立.于是任意两个实数$a,b$的差$a-
b$也就或为正,或为零,或为负有且仅有一种成立.这也就是
说对于任意两个实数,我们都能比较它们的大小,下列关系
有一种且仅有一种成立:
\[a>b,\quad \text{或}\quad a=b,\quad  \text{或}\quad a<b\]

实数系的大小关系和直线上的点的次序关系具有相同的
构造,即坐标的大小关系就相当于相应点在数轴上的左右关
系.

我们可以这么来比着看:
\begin{enumerate}
    \item 0这个数把实数集$\mathbb{R}$分成三部分:$\mathbb{R}_+$, $\{0\}$,$\mathbb{R}_-$,
原点$O$这个点把数轴$\ell$分成三段:不含原点的正向射线$\ell_+$,原
点和不含原点的负向射线$\ell_-$;
\item 在数轴上任给两个点$A,B$, 我们在第二章1.2中已
经知道有向线段$\Vec{AB}$的数量
\[AB=x_B-x_A\]
这里$x_A,x_B$分别为$A,B$的坐标,于是
\[\begin{split}
    \text{$B$点在$A$点之右}&\Longleftrightarrow AB=x_B-x_A>0\Longleftrightarrow x_B>x_A\\
    \text{$B$点与$A$点重合}&\Longleftrightarrow AB=x_B-x_A=0\Longleftrightarrow x_B=x_A\\
    \text{$B$点在$A$点之左}&\Longleftrightarrow AB=x_B-x_A<0\Longleftrightarrow x_B<x_A\\
\end{split}\]
\end{enumerate}

基于上述实数的大小关系和数轴上点的次序关系之间的密切
对应,一切实数按照由小而大的顺序从左往右排列在数轴
上,这就使得我们可以由坐标的大小来确定直线上点的次序,
反过来,也用数轴上的点的次序来把数的大小关系形象化.

两个数或两个代数式用不等号“$>$”或“$<$”联结起来,以
表示它们的数量关系就构成不等式.

例如$5>3$, $-7<-4$, $x+1>3$, $a>b$等都是不等式.如
果不等式中含有变数,那么使不等式成立的变数值叫做不等
式的解,例如$x=2.1$是不等式$x+1>3$的一个解.使不等式成
立的变数值的全体,称为这个不等式的解集.一般来说,不等
式的解集是实数集的子集.例如不等式$x+1>3$的解集是:
$\{x|x\in\mathbb{R},\;\;x>2\}$.如果不等式的解集是全体实数集$\mathbb{R}$的话,
那么这种不等式称为恒不等式.例如$x^2+1>0$, 就是一个恒
不等式,如果不等式的解集是空集$\emptyset$的话,那么这种不等式是
不成立的或者说是矛盾的不等式,例如$x-1>x+1$, 就是
矛盾的不等式,由上面所说可以明白:一个含有变数的不等
式,只有在它的解集上才是成立的,譬如我们说$x+1>3$,
它只在$x>2$的条件下才是正确的.

有时我们会遇到用不等号“$\ge$”或“$\le$”联结的不等式,例
如$|x|\ge 2$, 其中“$\ge$”表示“$>$或$=$”,即其解集是
$$\{x|x\in\mathbb{R},\;\;|x|>2\}\cup\{x|x\in\mathbb{R},\;\;|x|=2\}$$
也就是$$\{x|x\in\mathbb{R},\;\;x\ge2\}\cup
\{x|x\in\mathbb{R},x\le -2\}$$
同样地“$\le$”表示“$<$或$=$”.我们也常写
$a<b<c$来表示$a<b$和$b<c$; $a<b\le c$表示$a<b$和$b\le c$.

我们可以把不等式解集用数轴或平面上的对应点集直观
地表示出来,如:
不等式$x+1>3$的解集可用图3.1表示,解点集是一条以
坐标是2的点为端点的开射线,由于2不包括在内,故用空
圈表示.

\begin{figure}[htp]
    \centering
\begin{tikzpicture}[>=latex]
    \draw[->] (-4,0)--(4,0)node [right]{$x$};

\foreach \x in {-2,-1,...,2}
{
    \draw (\x,.15)--(\x, 0)node[below]{$\x$};
}
\draw (2,0) circle (2pt);

\draw[thick]  (2,0.1) to [bend left=12] (4,.5);


\end{tikzpicture}
    \caption{}
\end{figure}

不等式$|x|\ge 2$的解集可用图3.2表示,解集是两条射
线,一条是以坐标是2的点为端点的正向射线;另一条是以
坐标是$-2$的点为端点的负向射线.注意$\pm 2$处用实圈表示,
说明这两个点被包括在解集内.
\begin{figure}[htp]
    \centering
\begin{tikzpicture}[>=latex]
    \draw[->] (-4,0)--(4,0)node [right]{$x$};

\foreach \x in {-2,-1,...,2}
{
    \draw (\x,.15)--(\x, 0)node[below]{$\x$};
}
\draw (2,0)[fill=black] circle (1.5pt);  \draw (-2,0)[fill=black]  circle (1.5pt);

\draw[thick] (2,0) to [bend left=12] (4,.5);
\draw[thick]  (-2,0) to [bend right=12] (-4,.5);

\end{tikzpicture}
    \caption{}
\end{figure}


不等式$|x|\le 2$的解集可用图3.3表示,解点集是以$\pm 2$
为坐标的点为端点的线段.
\begin{figure}[htp]
    \centering
\begin{tikzpicture}[>=latex]
    \draw[->] (-4,0)--(4,0)node [right]{$x$};

\foreach \x in {-2,-1,...,2}
{
    \draw (\x,.15)--(\x, 0)node[below]{$\x$};
}
\draw (2,0)[fill=black] circle (1.5pt);  \draw (-2,0)[fill=black]  circle (1.5pt);

\draw[thick]  (-2,0)--(-2,.5)--(2,.5)--(2,0);

\end{tikzpicture}
    \caption{}
\end{figure}

学会不等式的解集合的图示,对以后解不等式是会有很
大好处的.

\begin{ex}
    \begin{enumerate}
    \item 在一条数轴上已知点$B$在$A,C$之间,如何用对应的数的
    大小来表示这种点的次序关系?
    \item 什么叫不等式、恒不等式、不等式的解?
    \item 试画出下列不等式解集图示:
    \begin{enumerate}
        \item $\{x|x\in\mathbb{R},\;\;x\ge -2\}$
        \item $\{x|x\in\mathbb{R},\;\;-1<x\le 3\}$
        \item $\{x|x\in\mathbb{R},\;\;x\ge 3\}\cup\{x|x\in\mathbb{R},\;\; x<-1\}$
        \item $\{x|x\in\mathbb{R},\;\;x\le 2\}\cap \{x|x\in\mathbb{R},\;\; x\le 3\}$
    \end{enumerate}
    
\end{enumerate}
\end{ex}

\subsection{不等式的基本性质}
基于实数系中大于和小于的定义以及实数系中的下列性
质,即
\begin{blk}{}
\begin{enumerate}
    \item 正数加正数仍是正数;
    \item 正数乘正数仍是正数;
    \item 正数乘负数则为负数;
    \item 负数乘负数则为正数;
    \item 任何一数或为正,或为零,或为负,且这三种可能
性有一种且仅有一种成立.
\end{enumerate}
\end{blk}

我们说明不等式的基本性质如下:
\begin{blk}{性质1}
    如果$a>b$, $b>c$, 那么$a>c$ (不等式传递性).
\end{blk}

\begin{proof}
    $a>b$, $b>c$就是$a-b>0$, $b-c>0$.
由于
$$a-c=(a-b)+(b-c)>0$$
这就是说:
$a>c$    
\end{proof}

\begin{blk}{性质2}
    如果$a>b$, 那么对于任意的$c$, 有$a+c>b+c$
(两边同加一个数,不等号方向不变).
\end{blk}

\begin{proof}
    $a>b$,就是$a-b>0$,
    由于
    $$(a+c)-(b+c)=a-b$$
    所以
    $$(a+c)-(b+c)>0$$
    这就是:
$a+c>b+c$
\end{proof}

\begin{blk}{推论1}
不等式中任何一项可以把它的符号变成相反的
符号后,从一边移到另一边.

\end{blk}

\begin{blk}{推论2}
    如果$a>b$, $c>d$, 那么$a+c>b+d$ (同向不等式的两端相加仍得同向不等式).
    \end{blk}

这是因为,根据性质2, 可得$a+c>b+c$,$b+c>b+
d$再根据性质1,可得$a+c>b+d$.

\begin{blk}{性质3}
如果$a>b$,$c>0$,那么$ac>bc$,如果$c<0$,
那么$ac<bc$,(不等式两端乘以正数得同向不等式,乘以负
数得反向不等式).
\end{blk}

\begin{proof}
$a>b$就是$a-b>0$,
由于
$$ac-bc=(a-b)c$$
所以当$c>0$时,$ac-bc>0$, 就是说
$$ac>bc$$
当$c<0$时,$ac-bc<0$, 就是说
$$ac<bc$$
\end{proof}

\begin{blk}{推论1}
   $a>b$和$b<a$是等价的,即:如果$a>b$,那么
$b<a$;反过来,如果$b<a$,那么$a>b$.
\end{blk}

这是因为$a>b$就是$a-b>0$, 不等式两边乘以$-1$, 得
$-(a-b)<0$, 即$b-a<0$, 这就是$b<a$, 同理可证后半个
结论.

\begin{blk}{推论2}
    如果$a>b>0$, $c>a>0$, 那么$ac>bd$.
\end{blk}

 
这是因为$a>b$, $c>0$, 根据性质3, 可得:
\begin{equation}
    ac>bc
\end{equation}
又$c>d$, $b>0$, 同理得
\begin{equation}
    bc>bd
\end{equation}
由(3.1),(3.2)得:
\begin{equation*}
    ac>bd  \tag{性质1}
\end{equation*}

\begin{blk}{推论3}
    如果$a>b$, 并且$a,b$同号,那么$\frac{1}{a}<\frac{1}{b}$
\end{blk}

因为$\frac{1}{b}-\frac{1}{a}=\frac{a-b}{ab}$,再由假设得到:
\[a-b>0,\quad ab>0\]
所以$\frac{1}{b}-\frac{1}{a}>0$,即$\frac{1}{a}<\frac{1}{b}$.

\begin{blk}{推论4}
    如果$a>b>0$, 那么$a^n>b^n$ ($n$是大于1的整
数).
\end{blk}

\begin{itemize}
    \item 当$n=2$时,由$a>b>0$,根据推论2得到:
$a^2>b^2$
\item 当$n=3$时,由$a>b>0$和$a^2>b^2$, 同理得到:
$a^3>b^3$
\item 依此类推,得到:$a^n>b^n$
\end{itemize}
这样正数之间的不等式可以进行$n$次乘方运算,仍得同向不
等式.

\begin{blk}{推论5}
    如果$a>b>0$, 那么$\sqrt[n]{a}>\sqrt[n]{b}$ ($n$是大于1的整
数).
\end{blk}

因为$\sqrt[n]{a}$, $\sqrt[n]{b}$是$n$次算术根,所以它们都是正数.

假设$a^{\tfrac{1}{n}}=b^{\tfrac{1}{n}}$,于是$\left(a^{\tfrac{1}{n}}\right)^n=\left(b^{\tfrac{1}{n}}\right)^n$,因而$a=b$,这就与已知$a>b$矛盾.

假设$a^{\tfrac{1}{n}}<b^{\tfrac{1}{n}}$,于是$\left(a^{\tfrac{1}{n}}\right)^n<\left(b^{\tfrac{1}{n}}\right)^n$,即$a<b$,这又与
已知条件矛盾.但是$\sqrt[n]{a}$和$\sqrt[n]{b}$的大小关系,只有三种可能,
而且仅有一种成立,因此$\sqrt[n]{a}>\sqrt[n]{b}$.
这样正数之间的不等式可以进行开$n$次方运算,仍得同向不
等式.

下面我们从不等式的定义和不等式的基本性质及其推论
出发来证明一些恒不等式.在推导不等式时,我们常利用实
数的平方不会是负的这个事实.我们首先证明下面的命题:

\begin{blk}{命题}
    若$a$是任意实数,那么$a^2\ge 0$.
\end{blk}

事实上,$a$或是正数,或是零,或是负数.如果$a>0$,
那么$a^2=a\x a>0$; 如果$a=0$, 那么$a^2=0\x0=0$; 如果
$a<0$, 于是$a=-|a|$, $a^2=(-|a|)\cdot (-|a|)=|a|^2>0$, 无
论哪种情形都有$a^2\ge 0$.

\subsubsection{比较法}

为证明某一个不等式成立,常用“大于”
定义,即要证$a>b$, 我们常证$a-b>0$, 这种方法叫做比较
法.用比较法证明不等式时,常常要把式子配方或把式子分
解成恒取正值或负值的因式的乘积.

\begin{example}
    若$a,b$是实数,则
    \begin{equation}
        a^2+b^2\ge 2ab
    \end{equation}
\end{example}
\begin{proof}
    $\because\quad a^2+b^2-2ab=(a-b)^2\ge 0$

    $\therefore\quad a^2+b^2\ge 2ab$,当且仅当$a=b$时,取等号.
\end{proof}

    
\begin{example}
    若$a,b,c$是任何实数,求证:
    \begin{equation}
   a^2+b^2+c^2\ge ab+bc+ca     
    \end{equation}
\end{example}

\begin{proof}
    \[\begin{split}
     &\quad   a^2+b^2+c^2-ab-bc-ca \\
      &=\frac{1}{2}( 2a^2+2b^2+2c^2-2ab-2bc-2ca)\\
        &=\frac{1}{2}\left[(a^2-2ab+b^2)+(b^2-2bc+c^2)+(c^2-2ca+a^2)\right]\\
        &=\frac{1}{2}\left[(a-b)^2+(b-c)^2+(c-a)^2\right]\ge 0
    \end{split}\]
   
    这里等于关系成立等同于$(a-b)^2=(b-c)^2=(c-a)^2=0$.
    即$a=b=c$, 因此,
$a^2+b^2+c^2\ge ab+bc+ca $
这里当且仅当$a=b=c$时取等号.

\textbf{另证:} 
\[\begin{split}
    &\quad   a^2+b^2+c^2-ab-bc-ca \\
&=a^2-(b+c)a+b^2+c^2-bc\\
&=\left[a^2-(b+c)a+\left(\frac{b+c}{2}\right)^2\right]+b^2+c^2-bc-\frac{(b+c)^2}{4}\\
&=\left(a-\frac{b+c}{2}\right)^2+\frac{3}{4}(b-c)^2\ge 0
\end{split}\]

这里等于关系当且仅当$b=c$和$a=\frac{b+c}{2}$
时,即$a=b=c$时成
立,因此,
$a^2+b^2+c^2\ge ab+bc+ca $.
\end{proof}




\begin{example}
    若$a,b$是不相等的正数,则
\[\frac{a}{\sqrt{b}}+\frac{b}{\sqrt{a}}>\sqrt{a}+\sqrt{b}\]
\end{example}
    
\begin{proof}
\[\begin{split}
     \frac{a}{\sqrt{b}}+\frac{b}{\sqrt{a}}-\sqrt{a}-\sqrt{b}
    &=\frac{a\sqrt{a}+b\sqrt{b}-a\sqrt{b}-b\sqrt{a}}{\sqrt{ab}}\\
    &=\frac{a\left(\sqrt{a}-\sqrt{b}\right)-b\left(\sqrt{a}-\sqrt{b}\right)}{\sqrt{ab}}\\
    &=\frac{\left(\sqrt{a}-\sqrt{b}\right)(a-b)}{\sqrt{ab}}\\
    &=\frac{\left(\sqrt{a}-\sqrt{b}\right)^2\left(\sqrt{a}+\sqrt{b}\right)}{\sqrt{ab}}>0
\end{split}\]

因为算术根式$\sqrt{ab}$, $\sqrt{a}+\sqrt{b}$在$a,b$是正数的条件
下是正数,又$a\ne b$,因此$\left(\sqrt{a}-\sqrt{b}\right)^2>0$.

所以,$\frac{a}{\sqrt{b}}+\frac{b}{\sqrt{a}}>\sqrt{a}+\sqrt{b}$
\end{proof}

\subsubsection{综合法}
要证明某个不等式,常由已知的或明显
的不等式,根据不等式的性质把它推导出来,这种由因导果
的证法叫做综合证法.
\begin{example}
    如果$a>b>0$, $0<c<d$,那么
    \begin{equation}
        \frac{a}{c}>\frac{b}{d}
    \end{equation}
\end{example}

\begin{proof}
 因为$c,d$同为正数,又$c<d$, 根据性质3的推
论3,得\[\frac{1}{c}>\frac{1}{d}>0\]
又$a>b>0$,
根据性质3的推论2, 得   
\[\frac{a}{c}>\frac{b}{d}\]
\end{proof}
    
\begin{example}
    $a_1$, $a_2$是任意正数,试证:
\begin{equation}
    \frac{a_1+a_2}{2}\ge \sqrt{a_1a_2}
\end{equation}
这里当且仅当$a_1=a_2$时,取“$=$”号.
\end{example}

\begin{proof}
    因为$(a_1-a_2)^2\ge 0$, 由此得到
\[a_1^2-2a_1a_2+a_2^2\ge 0\]
移项:
\[a^2_1+a^2_2\ge 2a_1a_2\]
两边同加正数$2a_1a_2$, 得
\[\begin{split}
    a_1^2+2a_1a_2+a^2_2&\ge 4a_1a_2\\
(a_1+a_2)^2&\ge 4a_1a_2
\end{split}\]

根据性质3的推论5, 得
\[a_1+a_2\ge 2\sqrt{a_1a_2}\]
即:$\frac{a_1+a_2}{2}\ge \sqrt{a_1a_2}$

显然,当且仅当$a_1=a_2$时,不等式取等号.
\end{proof}

这个不等式很重要,我们叫
$\frac{a_1+a_2}{2}$为正数$a_1$, $a_2$的算术
平均值,叫$\sqrt{a_1a_2}$为正数$a_1$, $a_2$的几何平均值.上面的不等
式是说两个正数的算术平均值不小于它们的几何平均值.

这个不等式有下面的几何
意义:

\begin{figure}[htp]
    \centering
\begin{tikzpicture}[>=latex]
    \draw[|<->|] (2,0)--node[fill=white]{$h$}(2,2.4);
 \draw (0,0)node[left]{$A$}--(5,0)node[right]{$B$};
\draw (5,0) arc (0:180:2.5);    
\draw (1.8,2.4)node[above]{$C$}--(1.8,0)node[below]{$D$};
\draw (0,0)--(1.8,2.4)--(5,0);
\draw[|<->|] (0,-.5)--node[fill=white]{$a_1$}(1.8,-.5);
\draw[|<->|] (5,-.5)--node[fill=white]{$a_2$}(1.8,-.5);
\draw (0,0)--(0,-.8);\draw (5,0)--(5,-.8);

\end{tikzpicture} 
    \caption{}
\end{figure}


以$a_1+a_2$为半径作圆
(图3.4), 设$AD=a_1$, $DB=a_2$,
过$D$点作线段$AB$的垂直线交
半圆于$C$点,于是$\angle ACB=
90^{\circ}$, 设$DC=h$, 由勾股定理
知:
\[AD^2+DC^2=AC^2,\qquad DC^2+DB^2=BC^2,\qquad AC^2+BC^2=AB^2\]
即:
$(a_1^2+h^2)+(a_2^2+h^2) =(a_1+a_2)^2$
从而:
\[h=\sqrt{a_1a_2}\]

另一方面,$DC$无论如何是不会超过圆的半径的,即:
\[h=\sqrt{a_1a_2}\le \frac{a_1+a_2}{2}\]
此外,当且仅当$D$点与圆心$O$点重合时,即$a_1=a_2$时,$DC$等
于圆的半径,即$h=\sqrt{a_1a_2}=\frac{a_1+a_2}{2}$    

\begin{example}
    两个正数的和是定值$A$, 求证在所有这样的和
中,当且仅当这两数相等时,它们的乘积最大.
\end{example}

\begin{proof}
设两个正数是$x$和$y$, 由题设知
\begin{equation}
   x+y=A 
\end{equation}
而其积依例3.5应该适合不等式:
\[\sqrt{xy}\le \frac{x+y}{2}\]
也即:$xy\le \left(\frac{x+y}{2}\right)^2$
将条件(3.7)代入上面不等式中,得
\[xy\le \left(\frac{A}{2}\right)^2\]
这表示其和等于$A$的两个正数,无论怎样取法,它们的积都
不大于常数$\left(\frac{A}{2}\right)^2$.
由于这里等式当且仅当两个正数相等时,
即$x=y=\frac{A}{2}$时,才能成立,因此当且仅当两个正数相等时,
它们的积才能达到最大值$\left(\frac{A}{2}\right)^2$.

\end{proof}
    
\begin{example}
$a_1,a_2,a_3$为任意正数,求证:
\begin{equation}
 \frac{a_1+a_2+a_3}{3}\ge \sqrt[3]{a_1a_2a_3}  
\end{equation}
这里的等于关系当且仅当$a_1=a_2=a_3$时成立.
\end{example}

\begin{proof}
    如果令$b_1>0$, $b_2>0$, $b_3>0$, 使得
\[a_1=b_1^3,\qquad a_2=b_2^3,\qquad  a_3=b_3^3\]
这样一来,我们就要证明:
\[b_1^3+b_2^3+b_3^3\ge 3b_1b_2b_3\]
由于
$b_1^3+b_2^3=(b_1+b_2)(b_2^2-b_1b_2+b^2_2)$
利用不等式(3.3)或(3.6),就有:
\[b_1^2 -b_1b_2+b_2^2\ge b_1b_2\]
所以我们有:
\[b^3_1+b_2^3\ge b_1b_2(b_1+b_2)=b_1^2 b_2+b_1b_2^2\]
同样地可证明:
\[b_2^3+b_3^3\ge b_2^2b_3+b_2b^2_3\]
以及
\[b_3^3+b_1^3\ge b_3^2b_1+b_3b^2_1\]
从而
\[\begin{split}
    2(b_1^3+b_2^3+b_3^3) &\ge b_1(b^2_2+b^2_3)+b_2(b^2_3+b^2_1)+b_3(b^2_1+b^2_2)\\
    &\ge b_1(2b_1b_2)+b_2(2b_3b_1)+b_3(2b_1b_2)\\
    &=6b_1b_2b_3
\end{split}\]
即:$b_1^3+b_2^3+b_3^3\ge 3b_1b_2b_3$

因而,$\frac{a_1+a_2+a_3}{3}\ge \sqrt[3]{a_1a_2a_3}  $

此外,从上面的过程可见,等式成立,等同于要求
\[b_1^2-b_1b_2 +b_2^2=b_1b_2,\qquad  b_2^2 -b_2b_3+b_3^2 =b_2b_3,\qquad b_3^2-b_3b_1+b_1^2=b_2b_1\]
三个式子成立,这就等同于要求
$b_1=b_2=b_3$.
\end{proof}

\begin{rmk}
    如果我们注意下面的因式分解:
    \[b_1^3+b_2^3+b_3^3-3b_1b_2b_3=(b_1+b_2+b_3)(b_1^2+b_2^2
    +b^2_3-b_1b_2-b_2b_3-b_3b_1)\]
    那么,我们就立即可得不等式(3.8)的证明.但是这样的因式
    分解不易使人想到,而且也不能进一步推广. 
\end{rmk}

由不等式(3.6)和(3.8)使人猜想,对于$n$个正数:
$a_1,a_2,\ldots,a_n$是不是有:
\[\frac{a_1+a_2+\cdots+a_n}{n}\ge \sqrt[n]{a_1a_2\cdots a_n}\]
其中$n$是大于1的整数,当且仅当$a_1=a_2=\cdots=a_n$时取等号.

我们说这个一般的结论“$n$个正数的算术平均不小于几何
平均”是成立的,不过这里我们略去它的证明.

\begin{example}
    求证在周长都为$2L$的所有三角形中,面积最大
的必是等边三角形.
\end{example}

\begin{proof}
    设三角形边长为$a,b,c$, 则周长为$a+b+c=2L$,
设面积为$S$, 于是依不等式(3.8)
\[\begin{split}
    S^2&=L(L-a)(L-b)(L-c)\\
&\le L\left[\frac{(L-a)+(L-b)+(L-c)}{3}\right]^3\\
&=\frac{L^4}{27}
\end{split}\]

因此这样的三角形面积不会超过$\frac{L^2}{\sqrt{27}}$,
同时要达到这个数
值必须且只须$L-a=L-b=L-c$, 即为等边三角形.
\end{proof}
    
\begin{example}
    若$x,y,z$是不都相等的正数,且$x+y+z=1$

求证$(1-x)(1-y)(1-z)>8xyz$.
\end{example}

\begin{proof}
由于:
\[\begin{split}
    1-z&=x+y\ge 2\sqrt{xy}\\
1-y&=x+z\ge 2\sqrt{xz}\\
1-x&=y+z\ge 2\sqrt{yz}
\end{split}\]   
因为$x,y,z$不都相等,即至少有两个数不相等,所以上面
三个不等式中至少有一个不等式只能成立“$>$”关系,将上面
不等式的两端相乘得到
\[(1-x)(1-y)(1-z)>8\sqrt{x^2y^2z^2}=8xyz\]
\end{proof}

\subsubsection{分析法}
证明不等式也可以用分析法,就是先假
定这个不等式成立,逐步找出使这个不等式成立的充分条
件,直到推导出已知条件或明显的不等式为止.也就是说由
果索因找出证题路线,然后再按分析过程的相反过程写出由
已知条件导出结论的过程.应用分析法时,一定要仔细检查
每步推理是否可逆,也就是不等式在反推时有无不等式的性
质作根据,如果其中某一步推理不可逆,那么用分析法是无
效的.



\begin{example}
    试证$\left(\sqrt{2}+1\right)^2 <3.4\sqrt{3}$

\end{example}

\begin{proof}
要证
\begin{equation}
    \left(\sqrt{2}+1\right)^2 <3.4\sqrt{3}
\end{equation}
成立 即要证:$3+2\sqrt{2}<\frac{17}{5}\sqrt{3}$.

两边乘以5,只须证:
\begin{equation}
    15+10\sqrt{2}<17\sqrt{3}
\end{equation}
两边平方,可证:
\begin{equation}
    225+200+300\sqrt{2}<867
\end{equation}
移项
\begin{equation}
    300\sqrt{2}<442
\end{equation}
两边除以300,即要证:
\begin{equation}
    \sqrt{2}<1.47
\end{equation}
显然(3.13)成立,而由(3.9)推出(3.13)的每一步都可逆,
所以$\left(\sqrt{2}+1\right)^2 <3.4\sqrt{3}$成立.
\end{proof}
    
\begin{example}
    若$a>b>0$, 求证:
    \[\frac{1}{8}\frac{(a-b)^2}{a}<\frac{a+b}{2}-\sqrt{ab}<\frac{1}{8}\frac{(a-b)^2}{b} \]
\end{example}

\begin{proof}
    (分析法)要证:$\frac{a+b}{2}-\sqrt{ab}<\frac{1}{8}\frac{(a-b)^2}{b}$成立,只须证:
\[\frac{\left(\sqrt{a}-\sqrt{b}\right)^2}{2}<\frac{1}{8}\frac{(a-b)^2}{b}\]

$\because \quad a>b>0,\qquad \therefore\quad \left(\sqrt{a}-\sqrt{b}\right)^2\ne 0$,两边除以$\left(\sqrt{a}-\sqrt{b}\right)^2$,可证:
\[\frac{\left(\sqrt{a}+\sqrt{b}\right)^2}{8b}>\frac{1}{2}\]
但是,$\frac{\left(\sqrt{a}+\sqrt{b}\right)^2}{8b}=\frac{1}{8}\left(1+\sqrt{\frac{a}{b}}\right)^2$,即要证:
\[\frac{1}{8}\left(1+\sqrt{\frac{a}{b}}\right)^2>\frac{1}{2},\quad \text{或}\quad \left(1+\sqrt{\frac{a}{b}}\right)^2>4\]

两边开平方取算术根,可证:
\[1+\sqrt{\frac{a}{b}}>2,\quad \text{或}\quad \sqrt{\frac{a}{b}}>1\]
即要证:$a>b$.

但是,$a>b>0$成立,并且上面推理的每一步可逆,所以,
$\frac{a+b}{2}-\sqrt{ab}<\frac{1}{8}\frac{(a-b)^2}{b}$成立.

同理证明:$\frac{a+b}{2}-\sqrt{ab}>\frac{1}{8}\frac{(a-b)^2}{a}$成立.

因此,$\frac{1}{8}\frac{(a-b)^2}{a}<\frac{a+b}{2}-\sqrt{ab}<\frac{1}{8}\frac{(a-b)^2}{b}$
\end{proof}

\begin{rmk}
    此题分析中的关键:
    \begin{enumerate}
        \item 通过恒等变形,将不等
    式两边的正因式$\left(\sqrt{a}-\sqrt{b}\right)^2$约去;
    \item $\frac{\left(\sqrt{a}+\sqrt{b}\right)^2}{8b}=\frac{1}{8}\left(1+\sqrt{\frac{a}{b}}\right)^2$的右端使求证和已知条件的联系明朗化.
    \end{enumerate}
\end{rmk}



\begin{example}
    试证:若$a>0$, $b^2-4ac\le 0$, 则对于任意实数
\begin{equation}
    ax^2+bx+c\ge 0
\end{equation}
\end{example}

\begin{proof}
\[\begin{split}
    ax^2+bx+c&=a\left[x^2+\frac{b}{a}x+\left(\frac{b}{2a}\right)^2\right]+c-\frac{b^2}{4a}\\
    &=a\left(x+\frac{b}{2a}\right)^2+\frac{4ac-b^2}{4a}
\end{split}\]
因为不论$x$为何值,$\left(x+\frac{b}{2a}\right)^2\ge 0$, 又因为:
$b^2-4ac\le 0$,故$4ac-b^2\ge 0$,因而:
\[ax^2+bx+c=a\left(x+\frac{b}{2a}\right)^2+\frac{4ac-b^2}{4a}\]
的值的正、负与$a$的值的正、负相同,并且当且仅当$b^2-4ac=0$和$x=-\frac{b}{2a}$
时,其值为零,故在$a>0$, $b^2-4ac\le 0$的条件下,不论$x$为何值,
$ax^2+bx+c\ge 0$.
\end{proof}

\begin{rmk}
    二次三项式经过配方后,要解决的问题就明朗
    化了.    
\end{rmk}


\begin{example}
    试证:对于任意实数$a_1,a_2,a_3$; $b_1,b_2,b_3$, 有
  \begin{equation}
      (a_1^2+a_2^2+a_3^2)(b_1^2+b_2^2+b_3^2)\ge (a_1b_1+a_2b_2+a_3b_3)^2
  \end{equation} 
\end{example}

\begin{proof}
    \[\begin{split}
&\quad   (a_1^2+a_2^2+a_3^2)(b_1^2+b_2^2+b_3^2)-(a_1b_1+a_2b_2+a_3b_3)^2\\
 & = a_1^2b_1^2+a_1^2b_2^2+a_1^2b_3^2
    +a_2^2b_1^2+a_2^2b_2^2+a_2^2b_3^2
    +a_3^2b_1^2+a_3^2b_2^2+a_3^2b_3^2\\
  &\qquad  + a_1^2b_1^2-a_2^2b^2_2-a_3^2b_3^2
    -2a_1b_1a_2b_2-2a_1b_1a_3b_3-2a_2b_2a_3b_3\\
    & = (a_1^2b_2^2-2a_1b2b_2a_2b_1+a_2^2+a_2^2b_1^2)
    +(a_1^2b_3^2-2a_1b_3a_3b_1+a_3^2b_1^2)\\
    &\qquad     +(a_2^2b_3^2-2a_2b_3a_3b_2+a_3^2b_2^2)\\
    & = (a_1b_2 -a_2b_1)^2+(a_1b_3-a_3b_1)^2+(a_2b_3-a_3b_2)^2\\
    & \ge 0        
    \end{split}\]
    这里不等式当且仅当$a_1b_2 -a_2b_1=a_1b_3-a_3b_1=a_2b_3-a_3b_2=0$   
    时,即在$\frac{a_1}{b_1}=\frac{a_2}{b_2}=\frac{a_3}{b_3}$时取等号. 
\end{proof}

不等式(3.15)可以推广到下面的情形:

\begin{blk}{}
    对于$a_1,a_2,\ldots,a_n$; $b_1,b_2,\ldots,b_n$的一切实数值,下列
    不等式成立:
    \[(a_1b_1+a_2b_2+\cdots+a_nb_n)^2\le (a_1^2+a_2^2+\cdots+a_n^2)(b_1^2+b_2^2+\cdots+b_n^2)\]
    等式当且仅当
    $\frac{a_1}{b_1}=\frac{a_2}{b_2}=\cdots=\frac{a_n}{b_n}$
    时成立,这个不等式称为
    柯西不等式,不等式(3.15)是柯西不等式$n=3$的特例.
\end{blk}



\begin{example}
    试证:如果$a^2+b^2+c^2=1$和$x^2+y^2+z^2=1$, 这
里$a,b,c$不分别等于$x,y,z$,那么,
\[ax+by+cz<1\]
\end{example}

\begin{proof}
根据柯西不等式,有
\[(ax+by+cz)^2<(a^2+b^2+c^2)(x^2+y^2+z^2)\]
(因为$a,b,c$不分别等于$x,y,z$, 故不能取等号).由此
\[(ax+ay+cz)^2<1\]
即:$|ax+by+cz|<1$,但是
\[ax+by+cz\le |ax+by+cz|\]
所以:$ax+by+cz<1$
\end{proof}

\section*{习题3.1}
\addcontentsline{toc}{subsection}{习题3.1}
\begin{enumerate}
    \item 如果$a>b$, $e>f$, $c>0$, 求证$f-ac<e-bc$.
    \item 如果$a>b$, $g<0$, $c$是任何数,求证:
   \[ g(a-c)<g(b-c)\]

   \item   如果$a>b>0$, $c>d>0$, 求证
   $\frac{1}{ac}<\frac{1}{bd}$
   \item 如果$a>b>0$, $c<d<0$, $e<0$, 求证:
   $\frac{e}{a-c}>\frac{e}{b-d}$
   \item 如果$a,b$是正数,求证:
   \[\frac{a+b}{1+a+b}<\frac{a}{1+a}+\frac{b}{1+b}\]
   \item 回答下列问题,并说明理由(“是”的给予证明;“不是”
   的举一反例).
\begin{enumerate}
    \item 如果$a>b$, $c=d$是否一定有$ac>bd$?
    \item 如果$\frac{a}{c^2}<\frac{b}{c^2}$,
    是否一定有$a<b$?
    \item 如果$ac>bc$, 是否一定有$a>b$?
    \item 如果$a>b$, $c>d$, 是否一定有$a-c>b-d$?
    \item 如果$a>b>0$, $c>d>0$, 是否一定有$\frac{a}{c}>\frac{b}{d}$?
    \item 如果$a>b$, $c>d$, 是否一定有$ac>bd$?
    \item 如果$a>b$, 是否一定有$a^2>b^2$?
    \item 如果$a>b$, 是否一定有$a^3>b^3$?
\end{enumerate}

\textbf{用比较法证明下面不等式:}
\item 若$a,b$是正数,求证$\frac{a^2+b^2}{2}\ge \left(\frac{a+b}{2}\right)^2$
\item $1+2x^4\ge x^2+2x^3$
\item 若$a,b$是正数,求证$a^3+b^3\ge a^2b+ab^2$
\item 证明: $a^2+b^2+c^2\ge 2(a+b+c)-3$
\item 若$a,b,c$是正数,求证
$a^2(b+c) +a(b^2+c^2-bc)> 0$


\textbf{用综合法证明下面不等式:}
\item 证明$\frac{a}{b}+\frac{b}{a}\ge 2$, 其中$a,b$是正数.
\item 若$a,b,c$是正数,求证:
\[\frac{ab(a+b)+ac(a+c)+bc(b+c)}{abc}\ge 6\]


\item 设$a,b$是不相等的正数,求证:
\begin{enumerate}
    \item $(a+b)(a^2+b^2)(a^3+b^3)>8a^3b^3$
    \item $(a+b)(a^3+b^3)>(a^2+b^2)$
\end{enumerate}

\item 若$a_1,a_2,a_3,a_4$都是正数,求证:
\[\frac{a_1+a_2+a_3+a_4}{4}\ge \sqrt[4]{a_1a_2a_3a_4}\]
\item 在上题中,令$a_4=\frac{a_1+a_2+a_3}{3}$,
试由\[\left(\frac{a_1+a_2+a_3+a_4}{4}\right)^4\ge a_1a_2a_3a_4\]
推出$\frac{a_1+a_2+a_3}{3}\ge \sqrt[3]{a_1a_2a_3}$

\item 若$a,b,c$是正数,求证:
$a+b+c\ge \sqrt{ab}+\sqrt{bc}+\sqrt{ca}$
\item 若$a,b$是正数,求证:
$\frac{a^3+b^3}{2}\ge \left(\frac{a+b}{2}\right)^3$
\item 若$a,b,c$是不相等的正数,
求证$(a+b+c)\left(\frac{1}{a}+\frac{1}{b}+\frac{1}{c}\right)>9$

\item 若$x>0$, 求证$x+\frac{9}{x}$
的最小值是6.
\item 当$x$为何值时,$2x(9-2x),\quad \left(0<x<\frac{9}{2}\right)$
取最大值,
最大值是多少?
\item 求证:周长一定的所有矩形中以正方形的面积最
大;面积一定的所有矩形中以正方形的周长最短,
这个正方形的边长是多少?
\item  斜边长一定的直角三角形中,求两直角边的和的最大
值,又何时达到最大值.


\textbf{用综合法或分析法证明下面不等式:}

\item 求证:
\begin{multicols}{2}
\begin{enumerate}
    \item $\sqrt{2}+\sqrt{3}<4$
    \item $\sqrt{3}+\sqrt{5}>\sqrt{15}$
    \item $\sqrt{2}+\sqrt{3}<\sqrt{10}$
    \item $\sqrt{5}+\sqrt{7}>1+\sqrt{15}$
\end{enumerate}    
\end{multicols}

\item 下面的差是正的还是负的?
\[\sqrt[3]{3}-\sqrt{2},\qquad \sqrt[4]{3}-\sqrt[5]{4},\qquad \sqrt[5]{2}-\sqrt[6]{3} \]
\item 若$a,b,c$是正整数,求证:
$ab+bc+ac\le 3abc$

\item 求证$a_1+a_2+\cdots+a_n<\frac{1}{1-a}\quad (0<a<1)$

\item 若$|a|<1$, $|b|<1$, 求证:$\left|\frac{a+b}{1+ab}\right|<1$

\item 求证$\frac{x^2+2}{\sqrt{x^2+1}}\ge 2$

\item 若$a,b,c,d$是正数,求证:
$$\sqrt{(a+c)(b+d)}\ge \sqrt{ab}+\sqrt{cd}$$
\item 若$a>b>0$且$m>n$, 求证:
$\frac{a-b}{a^m+b}>\frac{a-b}{a^n+b}$

\item 若$a>0$, $b>0$且$a+b=1$, 求证:
$\left(a+\frac{1}{a}\right)\left(b+\frac{1}{b}\right)\ge \frac{25}{4}$
\end{enumerate}

\section{解不等式}
解不等式就是要找出使不等式成立的变数值的全体,即
不等式的解集,在上节1.1中我们已经知道含有一个变数的
不等式的解集,或是整个实数集$\mathbb{R}$, 或是实数集$\mathbb{R}$的一个子
集,或是空集$\emptyset$.

解不等式的方法如同解方程一样,在于应用不等式的性
质,逐步将它变形到最简单的不等式,在变形过程中要特别注
意合理地应用不等式的性质,以保证每次得到的不等式的解
集和原来的不等式的解集都相同.现在让我们分析一下怎样
的变形可以保证解集相同,设不等式$\alpha(x)>0$和$\beta(x)>0$的对
应解集是$A$和$B$, 这里$\alpha(x)$, $\beta(x)$代表含有一个变数的代数式.

假设任意给出$x\in A$使$\alpha(x)>0$成立,根据不等式的性质
作变形推出$\beta(x)>0$成立,于是$x\in B$, 那么$A\subseteq B$, 反过来,假
设任给$x\in B$使$\beta(x)>0$成立,根据不等式的性质作逆变形推
出$\alpha(x)>0$成立,于是$x\in A$, 那么$B\subseteq A$, 从$A\subseteq B$和$B\subseteq A$都成
立,就可以说$A=B$, 这也就是说$A=B$的充要条件是$\alpha(x)>0$
成立$\Longleftrightarrow\beta(x)>0$成立,所以我们在解不等式的过程中要求
每步推理都是可逆的,这样才能保证它们的解集相同.

问题:说明能否由不等式$\sqrt{x}+x>-3+\sqrt{x}$推出$x>-3$,
又能否由不等式$x>-3$推出$\sqrt{x}+x>-3+\sqrt{x}$, 它们
的解集有何关系?

根据不等式的性质,容易验证下面的不等式的变形是同
解变形,即能保持它们的解集相同.

假设$f_1(x)$和$f_2(x)$都是含有一个变数的代数式,
\begin{enumerate}
    \item 若$g(x)$是整式,那么
    \[f_1(x)>f_2(x)\Longleftrightarrow f_1(x)+g(x)>f_2(x)+g(x)\]
    \item 若数$m>0$, 那么
    \[f_1(x)>f_2(x)\Longleftrightarrow mf_1(x)>mf_2(x)\]
    \item 若数$m<0$, 那么
    \[f_1(x)>f_2(x)\Longleftrightarrow mf_1(x)<mf_2(x)\]
\end{enumerate}

\subsection{一元一次不等式(组)}
形如$ax+b>0$或$ax+b<0\; (a\ne 0)$的不等式,称为一元一次不等式,其中$a,b$是两个已知的实数,$x$是变数.

由于这第二种类型的一元一次不等式$ax+b<0$的两端乘
以$(-1)$后,总可以变成第一种类型的同解不等式$-ax-b>
0$, 所以我们只讨论第一种类型的一元一次不等式的解集:

\begin{itemize}
    \item 当$a>0$时,那么$ax+b>0$的解集是:$\left\{x\Big|x>-\frac{b}{a}\right\}$,解集是数轴上不含点$-\frac{b}{a}$
的正向射线.
\item 当$a<0$时,那么$ax+b>0$的解集是:$\left\{x\Big|x<-\frac{b}{a}\right\}$,解集是数轴上不含点$-\frac{b}{a}$
的负向射线.
\end{itemize}

\begin{example}
    解不等式$\frac{x-1}{6}-\frac{2x+1}{4}<\frac{2x}{15}-1$
\end{example}

\begin{solution}
两边同乘以60,得:
\[\begin{split}
    10(x-1)-15(2x+1)&<8x-60\\
    -20x-25&<8x-60\\
    -28x&<-35
\end{split}
\]
因此:$x>\frac{5}{4}$,所以它的解集是:$\left\{x\Big|x>\frac{5}{4}\right\}$
\end{solution}

\begin{example}
    解不等式:$3(x-5)\ge x^2-5x+3-(x^2-8x)$
\end{example}

\begin{solution}
    有时我们可以采用另一种书写方式,即采用一系列
相等的集合把原不等式的解集找出来.

原不等式的解集:
\[\begin{split}
&\quad    \left\{x|3(x-5)\ge x^2-5x+3-(x^2-8x)\right\}\\
&=\left\{x|3x-15\ge 3x+3\right\}\\
&=\left\{x|0x\ge 18\right\}\\
&=\emptyset
\end{split}\]
“$\emptyset$”表示空集,这说明原不等式不能成立.

解不等式组就是先分别求不等式组中各个不等式的解
集,然后再求这些解集的交集.
\end{solution}

\begin{example}
    解不等式组:
    \[\begin{cases}
        3(x-2)+1<6(x+1)\\
        7-4(x-3)>5x-10
    \end{cases}\]
\end{example}

\begin{solution}
    先解第一个不等式,化简得:
\[\begin{split}
    3x-6+1&<6x+6\\
    3x&>-11
\end{split}\]
所以$x>-\frac{11}{3}$.

再解第二个不等式,化简得
\[\begin{split}
    7-4x+12&>5x-10\\
    9x&<29
\end{split}\]
所以$x<\frac{29}{9}$

最后求两个解集的公共部分 $-\frac{11}{3}<x<\frac{29}{9}$,这就是不等式组的解(图3.5).

\begin{figure}[htp]
    \centering
    \begin{tikzpicture}[scale=1.3,>=latex]
\draw[->] (-4.5,0)--(4.5,0)node[right]{$x$};
\foreach \x in {-3,-2,..., 3}
{
    \draw (\x,0)node[below]{$\x$}--(\x,.1);
}

\draw(-11/3,0)--(-11/3,.5)--(4.5,.5);
\draw(-4.5,1)--(29/9,1)--(29/9,0);
\fill[pattern=north east lines](-11/3,0) rectangle (29/9,.5);

\foreach \x/\xtext in {{-11/3}/-\frac{11}{3},  {29/9}/\frac{29}{9}}
{
    \node at (\x, 0)[below]{$\xtext$};
    \draw[fill=white] (\x, 0) circle (1.5pt);
}
    \end{tikzpicture}
    \caption{}
\end{figure}

求解的过程也可以用集合的符号来书写,即
\[\begin{split}
&\quad \left\{x\Big|\begin{cases}
    3(x-2)+1<6(x+1)\\
    7-4(x-3)>5x-10
\end{cases}  \right\}\\
&=\left\{x\Big|3(x-2)+1<6(x+1) \right\} \cap \left\{x\Big|7-4(x-3)>5x-10 \right\}    \\
&=\{x|3x>-11\}\cap\{x|9x<29\}\\
&=\left\{x\Big|-\frac{11}{3}<x<\frac{29}{9} \right\}
\end{split}\]

\end{solution}

\begin{example}
    解不等式组:
    \[\begin{cases}
        5(x-3)>3(2x-3)\\
        5(x-2)<3(x-1)
    \end{cases}\]
\end{example}

\begin{solution}
\[\begin{split}
&\quad \left\{x\Big|\begin{cases}
    5(x-3)>3(2x-3)\\
    5(x-2)<3(x-1)
\end{cases} \right\}\\
&=\left\{x\Big|5(x-3)>3(2x-3) \right\} \cap \left\{x\Big|5(x-2)<3(x-1) \right\}    \\
&=\{x|x<-6\}\cap\left\{x\Big|x<\frac{7}{2}\right\}\\
&=\{x|x<-6\}
\end{split}\] 

\begin{figure}[htp]
    \centering
\begin{tikzpicture}[>=latex, scale=.7]
\draw [->](-8,0)--(4.5,0)node [right]{$x$};
\foreach \x in {-5,-4,...,3}
{
    \draw (\x,0)node[below]{$\x$}--(\x,.1);
}

\draw (-6,0)--(-6,1)--(-8,1);
\draw (3.5,0)--(3.5,.5)--(-8,.5);
\fill[pattern=north east lines](-8,1) rectangle (-6,0);

\foreach \x/\xtext in {-6/-6,3.5/\frac{7}{2}}
{
    \node at (\x,0)[below]{$\xtext$};
    \draw[fill=white] (\x,0) circle (2pt);
}

\end{tikzpicture}
    \caption{}
\end{figure}


在最后求交集时可借助于如图3.6的图示,用这种图表
交集,直观清晰不易出错.
\end{solution}

\begin{example}
    解不等式组:
\[\begin{cases}
    7x-3>5x+1\\
    \frac{1}{2}x>x+\frac{1}{2}
\end{cases}\]
\end{example}

\begin{solution}
 \[\begin{split}
&  \quad \left\{x\Big|\begin{cases}
    7x-3>5x+1\\
    \frac{1}{2}x>x+\frac{1}{2}
\end{cases} \right\}   \\
&=\{x|7x-3>5x+1 \}\cap\left\{x|\frac{1}{2}x>x+\frac{1}{2}\right\}\\
&=\{x|x>2\}\cap\{x|x<-1\}=\emptyset
 \end{split}\]  
\begin{figure}[htp]
    \centering
\begin{tikzpicture}[>=latex]
\draw[->](-5,0)--(5,0)node[right]{$x$};
\foreach \x in {-1,0,1,2}
{
    \draw (\x,0)node[below]{$\x$}--(\x,.1);
}    
\draw (-5,.3)--(-1,.3)--(-1,0);
\draw (5,.3)--(2,.3)--(2,0);

\foreach \x in {-1,2}
{
    \draw (\x,0) [fill=white]circle(1.5pt);
}

\fill[pattern=north east lines] (-5,.3) rectangle (-1,0);
\fill[pattern=north east lines]  (5,.3) rectangle (2,0);
\end{tikzpicture}
    \caption{}
\end{figure}

 不等式组的解集是空集,说明没有实数能使不等式组的
各不等式同时成立,换句话说,不等式组无解.
\end{solution}

\begin{rmk}
    含有字母系数的不等式,它的解集因字母系数的
取值不同而有不同的情形,因此需要对字母系数高取值分情
况进行讨论.
\end{rmk}

\begin{example}
    解不等式:$mx-2>x-3m$
\end{example}

\begin{solution}
移项、化简得:
\[(m-1)x>2-3m\]
因为$m-1$可能是正数或零或负数,故需分三种情形来讨
论:
\begin{itemize}
    \item 当$m>1$时,则$x>\frac{2-3m}{m-1}$;
\item 当$m=1$时,原不等式化为$0x>-1$, 这个不等式的解集
是所有实数;
\item 当$m<1$时,则$x<\frac{2-3m}{m-1}$.
\end{itemize}

答:不等式的解集是:
\begin{itemize}
    \item 当$m>1$时,$\left\{x\Big| x>\frac{2-3m}{m-1}\right\}$
    \item 当$m=1$时,$\left\{x\Big| x\in\mathbb{R}\right\}$
    \item 当$m<1$时,$\left\{x\Big| x<\frac{2-3m}{m-1}\right\}$
\end{itemize}
\end{solution}

\begin{ex}
\begin{enumerate}
    \item  解下列不等式,并在数轴上表示出它的解集:
    \begin{multicols}{2}
\begin{enumerate}
\item $3 x-2>5 x+8$
\item $5 x-7 \le 33(x-4)$
\item  $0.3 x-0.5>x-1$
\item $0.8-0.2 x \le 0.5 x-0.6$
\item $\frac{1}{3}(x-5)>\frac{1}{4}(1-x)$
\item  $3(x-1)-(x-5)<x-3$
\item $\frac{2}{5}(x-3)+1<\frac{1}{6}(2 x-1) $
\item $\frac{x-3}{4}-\frac{2 x-1}{3}>x$
\item $\frac{2 x-1}{5}-\frac{x-2}{3} \geq \frac{2 x-5}{10}$
\item $\frac{x-1}{2}-\frac{x+1}{2}>\frac{x}{6}$
\end{enumerate}       
    \end{multicols}

\item 解下列不等式组,并把它们的解集在数轴上表示出来:
\begin{multicols}{2}
\begin{enumerate}
\item $\begin{cases}
    \frac{x+2}{6}>\frac{x-9}{6}+\frac{x+5}{2}\\
    6-\left(\frac{x-2}{4}+\frac{2}{3}\right)<\frac{x}{6}
\end{cases}$
\item $\begin{cases}
    \frac{3}{4}x-\frac{2}{3}\le \frac{4x-3}{12}\\
    2x-1>\frac{3x-4}{2}
\end{cases}$
\item $\begin{cases}
    x-1>\frac{7x-2}{3}\\
    4.5x+2.5<3x+2
\end{cases}$
\item $\begin{cases}
2x-3<4-5x\\
\frac{7x-8}{5}\ge x-1    
\end{cases}$
\end{enumerate}
\end{multicols}

\item 解下列不等式:
\begin{multicols}{2}
\begin{enumerate}
    \item $(a-b)x<2ax-b$
    \item $(p-q)x<p^2-q^2$
    \item $3(m+1)x+3m<2mx+3$
    \item $\frac{mx+1}{3}+\frac{4m-x}{2}<\frac{m^2}{6}$
\end{enumerate}
\end{multicols}
\end{enumerate}
\end{ex}    

\subsection{一元一次不等式的应用}

\begin{example}
    为奖励12名运动员需要买一些笔记本和铅笔,每
本笔记本的价格是0.94元,每打铅笔价格是0.76元,如果每
人各得一件奖品且所买物品价格不超过10元,问至多买几本
笔记本?
\end{example}


\begin{solution}
    设买笔记本$x$本,铅笔$(12-x)$打,则
\[0.94x+0.76(12-x)\le 10\]
化简得:
\[0.13x\le 0.88\quad \Rightarrow\quad x\le 4\frac{8}{9}\]
不大于$4\frac{8}{9}$的最大自然数是4.

答:至多买4本笔记本.
\end{solution}
    

    
\begin{example}
    拍照留念像,4吋的一份二张收费2.85元,加印
一张0.48元,预定每人平均出钱不超过1元,并都分到一-张
照片,问参加照像的至少有几人?
\end{example}

\begin{solution}
设参加照像的人数为$x$, 且在二个人以上,于是照像
费为:
\[2.85+0.48(x-2)\]
预定的照像费为:$x$元.
依题意有:
\[2.85+0.48(x-2)\le x\]
化简得
\[\begin{split}
    0.52x&\ge 1.89\\
x&\ge \frac{189}{52}=3\frac{33}{52}
\end{split}\]
大于$3\frac{33}{52}$的最小自然数是4.

答:参加照像的至少得有4人.
\end{solution}
    

\begin{example}
把笔记本123册,铅笔23打,分给三年级一班学
生,如果每人分三本笔记本,则尚余10册以上;每人分给8
支铅笔,则至少缺一人份,问这个班有多少学生?
\end{example}

\begin{solution}
 设这个班有学生$x$人,从人和笔记本册数的关系,得:
 \begin{equation}
     123-3x\ge 10
 \end{equation}
从人和铅笔支数的关系,得
\begin{equation}
    8x-12\x 23\ge 8
\end{equation}
解联立不等式(3.16)和(3.17):

由(3.16)得
\[x\le 37\frac{2}{3}\]

由(3.17)得
\[x\ge 35\frac{1}{2}\]

所以
\begin{equation}
    35\frac{1}{2}\le x\le 37\frac{2}{3}
\end{equation}
表示班级人数的数是自然数,所以使(3.18)成立的自然数
有36和37.

答:36人或37人.
\end{solution}

\begin{example}
浓度8\%的盐水100克与浓度3\%的盐水混合,要
制成的涅合液至少重1200克,含盐至多84克,问掺入浓度3\%
的盐水的重量范围如何?
\end{example}


\begin{solution}
设掺入浓度3\%的盐水$x$克,
由盐水重量关系,得
\begin{equation}
    100+x\ge 1200
\end{equation}
由盐的重量关系,得
\begin{equation}
    100\x8\%+x \x 3\%\le 84
\end{equation}
把(3.19), (3.20)联立起来,由(3.19)得
\[x\ge 1100\]
由(3.20)得
\[0.03x\le 76\quad \Rightarrow\quad 
x\le 2533\frac{1}{3}\]
所以
\[1100\le x\le 2533\frac{1}{3}\]

答:掺入3\%的盐水的重量在1100克到$2533\frac{1}{3}$克的范围
内.
\end{solution}


\begin{ex}
\begin{enumerate}
    \item 某施工队,原规定要在6天内完成300土方的工程,第
    一天完成了60土方,因有其他紧急任务,要求这项工程
    提前2天完成,那么,这个施工队从第二天起平均每天
    至少要完成多少土方?
    \item 旅行者乘摩托艇顺水而下,然后返回原停治处.水流速
    度为2公里/小时,摩托艇在静水中的速度为18公里/小
    时.为了使游览时间不超过3小时,旅行者能走出多少
    路程?
    \item 从A地去相距18km处的B地,先以时速5km,中途改为
    时速4km的速度步行到B地.如果所用的时间不超过4小
    时,问以5km的时速所走的距离不少于多少公里?
    \item 夏令营第一中队野外行军,每小时走3公里,出发后一
    小时,营部有紧急通知,令通讯员骑自行车在40分钟内
    送到,问通讯员骑车的最低速度多大才能在40钟内把
    信送到.
    \item 在比赛时,每名射手各打10枪,每命中一次得5分,每
    脱靶一次扣1分,得到的分数不少于30分的射手算是优
    胜者.射手要成为优胜者至少应该中靶几次?
    \item 
    甲乙两个生产小组分别制定生产计划,甲组计划15天完
    成361个零件,乙组计划10天完成190个零件,头二天,
    甲组每天完成18个,乙组每天完成15个.从第三天起,
    甲乙两组提高了效率,最后都超额完成了生产计划.
    问从第三天起,每天平均至少能完成几个?
    \item 把700个成品装箱,如果每箱平均装15个,则剩余3箱以
上的成品不能装箱,如果每箱装20个,则至少剩下8个
箱子,求箱子的个数.
\item 
含银60\%的合金400g和含银75\%的合金混合,总重量
至少600g, 要制作含银至多420g的合金,问掺入含银
75\%的合金的重量范围如何?
\item 
8点离开家,要在8点30分到8点40分之间到达学校.
从家到学校的路程为2400米,问步行的速度范围如何?
需每分钟不得少于多少米?不得大于多少米?
\end{enumerate}
\end{ex}

\subsection{不等式$(ax+b)(cx+d)>0\; (<0)$,
$\frac{ax+b}{cx+d}>0\; (<0)$的解法举例}
\begin{example}
    解不等式$(0.5x-1)(4-x)>0$.
\end{example}

\begin{solution}
两个因式的积的值为正,当且仅当这两个因式的值
同号,即原不等式成立的充分必要条件是下列不等式组之
一成立:
\[\begin{cases}
    0.5x-1>0\\4-x>0
\end{cases}\quad \text{或}\quad \begin{cases}
    0.5x-1<0\\4-x<0
\end{cases}\]
由此得
\[\begin{cases}
    x>2\\x<4
\end{cases}\quad \text{或}\quad \begin{cases}
    x<2\\x>4
\end{cases}\]

前一个不等式组化为$2<x<4$, 后一个不等式组的解
集合是空集$\emptyset$.

答:原不等式的解集$=\{x|2<x<4\}\cup\emptyset=\{x|2<x<4\}$.
\end{solution}

\begin{example}
    解不等式$\frac{2x-3}{5-4x}<0$.
\end{example}

\begin{solution}
分式的值为负,当且仅当分子和分母的值有不同符
号,即原不等式成立的充分必要条件是下列不等式组之一成
立:
\begin{equation}
    \begin{cases}
        2x-3>0\\5-4x<0
    \end{cases}\Rightarrow\quad \begin{cases}
        x>\frac{3}{2}\\  x>\frac{5}{4}
    \end{cases}
\end{equation}
或
\begin{equation}
    \begin{cases}
        2x-3<0\\5-4x>0
    \end{cases}\Rightarrow\quad \begin{cases}
        x<\frac{3}{2}\\  x<\frac{5}{4}
    \end{cases}
\end{equation}
因为$\frac{3}{2}>\frac{5}{4}$,
不等式组(3.21)的解集$=\left\{x\Big|x>\frac{3}{2}\right\}$, 不
等式组(3.22)的解集$=\left\{x\Big|x<\frac{5}{4}\right\}$.

答:原不等式的解集$=\left\{x\Big|x>\frac{3}{2}\right\}\bigcup\left\{x\Big|x<\frac{5}{4}\right\}$
\end{solution}

\begin{example}
    解不等式$\frac{3x+1}{x-3}<1$.
\end{example}

\begin{solution}
    将这个不等式作同解变形,使不等式的右端为零.

    移项$$\frac{3x+1}{x-3}-1<0$$
化简得$$\frac{2x+4}{x-3}<0$$

同解于:
\begin{equation}
    \begin{cases}
        2x+4>0\\x-3<0
    \end{cases}\Rightarrow\quad \begin{cases}
        x>-2\\  x<3
    \end{cases}
\end{equation}
或
\begin{equation}
    \begin{cases}
        2x+4<0\\x-3>0
    \end{cases}\Rightarrow\quad \begin{cases}
        x<-2\\  x>3
    \end{cases}
\end{equation}
不等式组(3.23)的解集$=\{x|-2<x<3\}$, 不等式组(3.24)的
解集$=\{x|x\le -1\}$.

答:原不等式的解集$=\{x|x>1\}\cup \{x|x\le -1\}$

\end{solution}

\begin{ex}
\begin{enumerate}
    \item 解不等式:
    \begin{multicols}{2}
        \begin{enumerate}
    \item $x(10x-3)>0$
    \item $-x(2-5x)\le 0$
    \item $(10x-1)(5x-2)<0$
    \item $(5-\sqrt{2}x)\left(\frac{1}{2}x-\sqrt{3}\right)\ge 0$
\end{enumerate}
    \end{multicols}

    \item 解不等式:
    \begin{multicols}{2}
        \begin{enumerate}
    \item $\frac{12}{x-7.2}>0$
    \item $\frac{1.75}{3x-0.75}<0$
    \item $\frac{x^2+1}{-x+3}> 0$
    \item $\frac{x^2}{x-1}\ge 0 $
    \item $\frac{x-1}{x^2}\le 0 $
\end{enumerate}
    \end{multicols}
    \item 解不等式:
    \begin{multicols}{2}
        \begin{enumerate}
    \item $\frac{x-17}{9-x}>0 $
    \item $\frac{3x-4}{x}\le 0 $
    \item $\frac{5-x}{3-2x}\le 0$
    \item $\frac{5x-1}{y+2}\ge 1$
    \item $\frac{3y-7}{y+2}\le 2$
    \item $\frac{3}{x-6}>2$
\end{enumerate}
    \end{multicols}

    \item 下列不等式是不是同解不等式:
\begin{enumerate}
    \item $(x-1)(x+2)>0,\qquad \frac{x-1}{x+2}>0$
    \item $(x-3)(x+4)\ge 0,\qquad \frac{x-3}{x+4}\ge 0$
\end{enumerate}
\end{enumerate}
\end{ex}

\subsection{含有绝对值符号的一元一次不等式}

有时我们还会遇到一种含有绝对值符号的不等式.
例如,用车床加工一种底面直径为10cm的圆柱形零件,
如果规定底面直径的绝对误差不超过0.05cm才算合格,那
么合格零件底面直径$x$可以是多少?这就是要解不等式:
\[|x-10|\le 0.05\]

这类不等式称为含有绝对值符号的不等式,或简单地称
为绝对值不等式.

我们先来考虑两个简单的绝对值不等式:
\[|x|<a,\qquad |x|>a\]

我们知道,$|x|$可以看作数轴上表示$x$的点与原点的距
离,由此可知:

如果$a>0$, 那么绝对值不等式$|x|<a$的解就是$-a<x<a$的解(图3.8); $|x|>a$的解就是$x<-a$或$x>a$的解
(图3.9),也就是:
\[\begin{split}
    \{x|\;|x|<a\}&=\{x|\;-a<x<a\}\\
    \{x|\;|x|\ge a\}&=\{x|\;x<-a\}\cup\{x|\;x>a\}
\end{split}\]

\begin{figure}[htp]
    \centering
\begin{tikzpicture}[>=latex]
\draw[->] (-4,0)--(4,0)node[right]{$x$};
\foreach \x/\xtext in {-2/-a,0/0,2/a}
{
    \draw (\x,0)node[below]{$\xtext$}--(\x,.1);
}    
\draw (-2,0)--(-2,.5)--(2,.5)--(2,0);
\fill [pattern=north east lines](-2,.5) rectangle (2,0);
\foreach \x in {-2,2}
{
    \draw (\x,0) [fill=white] circle (2.5pt);
}
\end{tikzpicture}
    \caption{}
\end{figure}

\begin{figure}[htp]
    \centering
\begin{tikzpicture}[>=latex]
\draw[->] (-4,0)--(4,0)node[right]{$x$};
\foreach \x/\xtext in {-2/-a,0/0,2/a}
{
    \draw (\x,0)node[below]{$\xtext$}--(\x,.1);
}    
\draw (-2,0)--(-2,.5)--(-4,.5);
\fill [pattern=north east lines](-4,.5) rectangle (-2,0);
\draw (2,0)--(2,.5)--(4,.5);
\fill [pattern=north east lines](4,.5) rectangle (2,0);
\foreach \x in {-2,2}
{
    \draw (\x,0) [fill=white] circle (2.5pt);
}
\end{tikzpicture}
    \caption{}
\end{figure}

现在让我们转到要解的不等式$|x-10|\le 0.05$上来,其
意义就是数轴上表示$x$的点与坐标为10的点的距离不超过
0.05.根据上面的结论,它和下面不等式同解:
\[-0.05\le x-10\le 0.05\]
也就是
\[9.95\le x\le 10.05\]

\begin{example}
    解不等式
$|x+4|>9$.
\end{example}


\begin{solution}
\[\begin{split}
    &\quad \{x|\;|x+4|>9\}\\
    &=\{x|x+4>9\}\cup \{x|x+4<-9\}\\
    &=\{x|x>5\}\cup\{x|x<- 13\}
\end{split}\]
这个绝对值不等式,其意义就是数轴上表示$x$的点与坐标为$-4$的点的距离大于9, 这些解点只能是以5为端点的
右开射线,和以$-13$为端点的左开射线(图3.10).
\begin{figure}[htp]
    \centering
\begin{tikzpicture}[>=latex]
\draw[->] (-8,0)--(4,0)node[right]{$x$};
\foreach \x/\xtext in {-6.5/-13,-6/-12,-5/-10,-4/-8,-3/-6,-2/-4,-1/-2,0/0,1/2,2/4,2.5/5}
{
    \draw (\x,0)node[below]{$\xtext$}--(\x,.1);
}    
\draw (-6.5,0)--(-6.5,.5)--(-8,.5);
\fill [pattern=north east lines](-8,.5) rectangle (-6.5,0);
\draw (2.5,0)--(2.5,.5)--(4,.5);
\fill [pattern=north east lines](4,.5) rectangle (2.5,0);
\foreach \x in {-6.5,2.5}
{
    \draw (\x,0) [fill=white] circle (2.5pt);
}
\end{tikzpicture}
    \caption{}
\end{figure}
\end{solution}

为了与后面的讨论一致起见,我们还可以这样来解这两
个绝对值不等式:

按照绝对值的定义,
\[|x-10|=\begin{cases}
    x-10, & x\ge 10\\
    -(x-10), & x<10
\end{cases}\]

由此可见,$x=10$是个分界点,我们称它为$x-10$的零点,
零点把数轴分成三段即零点的右开射线,零点和零点的左开
射线.在这里我们约定把零点归于右开射线的左端点,于是零
点左侧任意一点和零点的距离:
\[|x-10|=-(x-10),\qquad x<10\]
零点或零点右侧任意一点和零点的距离:
\[|x-10|=x-10,\qquad x\ge 10\]
这样有下图(图3.11)的情况:
\begin{figure}[htp]
    \centering
\begin{tikzpicture}[>=latex]
\draw[->](0,0)--(10,0)node [right]{$x$};
\foreach \x/\xtext in {3/0,6/10}
{
    \draw (\x, 0) [fill=black] circle (1.5pt)node[below]{$\xtext$}; 
}    
\draw (0,1) to [bend left=15] (6,0);
\draw (10,1) to [bend left=-15] (6,0);
\node at (2,.5){$|x-10|=-(x-10)$};
\node at (9,.5){$|x-10|=x-10$};
\node at (2,-.5){$x<10$};
\node at (9,-.5){$x\ge 10$};

\end{tikzpicture}
    \caption{}
\end{figure}

从而得到不等式$|x-10|\le 0.05$的解集:
\[\begin{split}
  &\quad   \{x|\; |x-10|\le 0.05\}\\
  &=\left\{x\Big| \begin{cases}
    -(x-10)\le 0.05\\x<10
\end{cases}\right\}\bigcup \left\{x\Big| \begin{cases}
    x-10\le 0.05\\x\ge 10
\end{cases}\right\}\\
&=\Big[\{x|-(x-10)\le 0.05\}\cap\{x|x<10\}\Big]\bigcup\Big[\{x|x-10\le 0.05\}\cap\{x|x\ge 10\}\Big]\\
&=\{x|9.95\le x<10\}\cup\{x|10\le x\le 10.05\}\\
&=\{x|9.95\le x\le 10.05\}
\end{split}
    \]

    同理可解例2.28:$|x+4|>9$,在这里零点是$x=-4$, 它把数轴分成两段:
    \begin{itemize}
        \item  当$x\ge -4$时,$|x+4|=x+4$;
        \item  当$x<-4$时,$|x+4|=-(x+4)$.
    \end{itemize}
    于是不等式的解集为:
\[\begin{split}
    &\quad \{x|\; |x+4|>9\} \\
&= \left\{x\Big|\begin{cases}
    -(x+4)>9\\x<-4
\end{cases}\right\}\bigcup \left\{x\Big|\begin{cases}
    x+4>9\\x\ge -4
\end{cases}\right\}\\
    &=\Big[\{x|-(x+4)>9\}\cap\{x|x<-4\}\Big]\bigcup\Big[\{x|x+4>9\}\cap\{x|x\ge -4\}\Big]\\
    &=\{x|x<-13\}\cup\{x|x>5\}
\end{split}\]

当然这种方法对解含一个绝对值的不等式不如前法方
便、但它具有一般性,且它可推广到解含二个、三个……绝
对值的不等式.

\begin{example}
    解不等式$|x-2|+|x+3|>7$.
\end{example}

\begin{solution}
这里$|x-2|$的零点为2; $|x+3|$的零点为$-3$,
把所有零点在数轴上按由小到大的次序排列,这样把数轴分
为下面的三段,而各段上的点和零点2的距离与该点和零
点$-3$的距离的和在脱去绝对值后的情况如图3.12所示.
\begin{figure}[htp]
    \centering
\begin{tikzpicture}[>=latex]
\draw[->] (-6,0)--(6,0)node[right]{$x$};
\foreach \x in {-3,-2,...,2}
{
    \draw (\x,0)node[below]{$\x$}--(\x,.1);
}
\draw (-3,0) [fill=black] circle(1.5pt);
\draw (2,0) [fill=black]circle (1.5pt);

\draw (-3,0)--(-3,.5)--(2,.5)--(2,0);
\draw (-6,1)to [bend left=15] (-3,0);
\draw (6,1)to [bend left=-15] (2,0);

\node at (-5,-.75){$x<-3$}; 
\node at (-.5,-.75){$-3\le x<2$};
\node at (4,-.75){$x\ge 2$};

\node at (-5,.5){$-(x-2)+[-(x+3)]$}; 
\node at (-.5,1){$-(x-2)+(x+3)$};
\node at (4,.5){$(x-2)+(x+3)$};
\end{tikzpicture}
\caption{}
\end{figure}

所以原不等式的解就是这三个解集的并集,即
\[\begin{split}
&\quad \left\{x\Big|\begin{cases}
    -(x-2)+[-(x+3)]>7\\
    x<-3
\end{cases}\right\}\bigcup \left\{x\Big|\begin{cases}
    -(x-2)+(x+3)>7\\ -3\le x<2
\end{cases}\right\}\\
&\qquad \bigcup \left\{x\Big|\begin{cases}
    (x-2)+(x+3)>7\\ x\ge 2
\end{cases}\right\}\\
&=\{x|x<-4\}\cup\emptyset \cup\{x|x>3\}\\
&=\{x|x<-4\}\cup \{x|x>3\}
\end{split}\]

\begin{figure}[htp]
    \centering
\begin{tikzpicture}[>=latex]
\draw[->](-6,0)--(5.5,0)node[right]{$x$};
\foreach \x in {-4,-3,...,3}
{
    \draw(\x,0)node[below]{$\x$}--(\x,.1);
}
\draw (-4,0)--(-4,.5)--(-6,.5);
\draw (3,0)--(3,.5)--(5,.5);
\fill[pattern=north east lines](-6,.5) rectangle (-4,0);
\fill[pattern=north east lines](5,.5) rectangle (3,0);
\draw (-4,0) [fill=white] circle (1.5pt);
\draw (3,0) [fill=white] circle (1.5pt);
\end{tikzpicture}
    \caption{}
\end{figure}
\end{solution}
    
\begin{ex}
\begin{enumerate}
    \item 解下列不等式:
\begin{multicols}{2}
\begin{enumerate}
    \item $|x-2|<5$
    \item $|2x-3|\ge 1$
    \item $|2-3x|\le 3$
    \item $|4x-5|<15$
    \item $\left|\frac{1}{2}x+1\right|<4$
    \item $|2x+5|<\frac{5}{12}$
    \item $\left|\frac{1}{2}x-4\right|+3>|8-x|$
    \item $|x|>2x-1$
\end{enumerate}    
\end{multicols}

\item 解下列不等式:
\begin{multicols}{2}
\begin{enumerate}
    \item $|x+1|+|x-1|>0$
    \item $|x+3|-|x-3|<5$
    \item $|x-5|-|x+3|>3$
    \item $|x-1|-11<2$
\end{enumerate}
\end{multicols}
\item 用宽6cm的铁板,截一个面积为48${\rm cm}^2$的长方形零件,
要求面积的绝对误差不超过0.3${\rm cm}^2$, 长度应在什么范围
内?
\item 解下列不等式:
\begin{enumerate}
    \item $3|x-6|+|x+2|-|x-0.5|\le 7$
    \item $|x|+|x-1|+|x-2|\ge 4$
\end{enumerate}
\end{enumerate}
\end{ex}

\subsection{一元二次不等式}
形如
\[ax^2+bx+c>0 \qquad \text{或}\qquad  ax^2+bx+c<0 \]
的不等式称为一元二次不等式.这里$a\ne 0$.

我们首先说明下面的定理:
\begin{blk}{定理1}
    设$b^2-4ac\le 0$, 则
\begin{enumerate}
    \item 当$a>0$时,不论$x$为何值,$ax^2+bx+c\ge 0$
    \item 当$a<0$时,不论$x$为何值,$ax^2+bx+c\le 0$
\end{enumerate} 
上面两个不等式仅在$b^2-4ac=0$, 且
$x=-\frac{b}{2a}$时等式成立.
\end{blk}

这个定理的证明已在本章的例3.12中用配方法证过了.
建议读者再复习一遍它的证明.

这个定理告诉我们;不管$b^2-4ac<0$还是$b^2-4ac=0$,
二次三项式$ax^2+bx+c$的值除可能是0外,符号总是和二次
项的系数$a$的符号一致,不随$x$的改变而改变的.

根据这个结论,我们知道不等式:
\begin{enumerate}
    \item $ax^2+bx+c>0$ $(a>0,\; b^2-4ac<0)$的解集
    是:$\{x|x\in\mathbb{R}\}$.
    \item $ax^2+bx+c>0$ $(a>0,\;b^2-4ac=0)$的解集是:
    $\left\{x\Big|x\in\mathbb{R},\; x\ne -\frac{b}{2a}\right\}$
    \item $ax^2+bx+c<0$ $(a>0,\; b^2-4ac<0)$的解集是
    空集$\emptyset$.
\end{enumerate}

但当$b^2-4ac>0$时,$ax^2+bx+c$的值的情形是怎样的
呢?这时$ax^2+bx+c$的值的符号要因$x$的值不同而改变,现在
我们就来讨论它的值的符号怎样随$x$的取值改变.

当$b^2-4ac>0$. 而$a\ne 0$时,$ax^2+bx+c=0$有两个不同
的实根$x_1$和$x_2$, 并设$x_1<x_2$, 于是根据因式定理有:
\[ax^2+bx+c=a(x-x_1)(x-x_2)\]

在这里,我们只要记住一个最简单的原则,就可以解决
问题,那就是“奇数个负数相乘是负的,偶数个负数相乘是
正的”,例如当$x_1<x<x_2$时,$x-x_1$是正的而$x-x_2$是负的,于是
$(x-x_1)(x-x_2)<0$, 因此在这时$ax^2+bx+c$的值的符号与$a$
相反,从此我们可以把结果列成一个表.
\begin{center}
\begin{tabular}{c|c|c}
\hline
      & $x<x_1$ & $ax^2+bx+c>0$\\
 $a>0$   &  $x_1<x<x_2$ & $ax^2+bx+c<0$\\
    &  $x_2<x$ & $ax^2+bx+c>0$\\
\hline
  & $x<x_1$ & $ax^2+bx+c<0$\\
$a<0$&  $x_1<x<x_2$ & $ax^2+bx+c>0$\\
&  $x_2<x$ & $ax^2+bx+c<0$\\
\hline
\end{tabular}
\end{center}

我们把上面讨论的结果总结成下面的定理:
\begin{blk}{定理2}
    设$b^2-4ac>0$, 则当$x$的取值在$ax^2+bx+c$的
    二根之间时,$ax^2+bx+c$的值的符号与$a$的符号相反;当$x$的
    取值在二根之间的以外部分时,$ax^2+bx+c$的值的符号与$a$
    的符号相同. 
\end{blk}

\begin{center}
\begin{tikzpicture}[>=latex, scale=.8]
\begin{scope}
\draw[->] (0,0)--(6,0)node[right]{$x$};
\foreach \x/\xtext in {2/x_1,4/x_2}
{
    \draw (\x,0)node[below]{$\xtext$}--(\x,.1);
}
\foreach \x/\xtext in {1/+,3/-,5/+}
{
    \node at (\x,.5){$\xtext$};
}
\node at (3,2){$ax^2+bx+c$};
\node at (3,1.3){$(a>0,\; b^2-4ac>0)$};
\end{scope}
\begin{scope}[xshift=8cm]
\draw[->] (0,0)--(6,0)node[right]{$x$};
\foreach \x/\xtext in {2/x_1,4/x_2}
{
    \draw (\x,0)node[below]{$\xtext$}--(\x,.1);
}
\foreach \x/\xtext in {1/-,3/+,5/-}
{
    \node at (\x,.5){$\xtext$};
}
\node at (3,2){$ax^2+bx+c$};
\node at (3,1.3){$(a<0,\; b^2-4ac>0)$};
\end{scope}
\end{tikzpicture} 
\end{center}

再把定理1和定理2的结论综合起来看,我们说:$ax^2+
bx+c\; (a\ne 0)$的值的符号和它的二次项的系数的符号相同,
除非$x$的值是在它的二根之间或等于它的根.



\begin{example}
 解不等式$-x^2+2x-2>0$
\end{example}

\begin{solution}
 由于$a=-1<0$, 且$b^2-4ac=2^2-4(-1)\cdot(-2)=-4<0$, $-x^2+2x-2$的值对于任何$x$都是负的,
故不等式无解.   
\end{solution}

\begin{example}
   解不等式$2x^2-4x+3>0$ 
\end{example}

\begin{solution}
    由于$a=2>0$, 且$b^2-4ac=(-4)2-4\cdot 2\cdot 3=-8<0$, 故不等式的解集是一切实数.
\end{solution}

\begin{example}
 解不等式$2x^2-4x-5\le 0$   
\end{example}

\begin{solution}
    由于$b^2-4ac=(-4)2-4\x2\x(-5)=56>0$,
所以对应的二次方程有两个实根:
\[x_1=\frac{2-\sqrt{14}}{2},\qquad x_2=\frac{2+\sqrt{14}}{2}\]
又$a=2>0$, 所以要使原不等式成立,它的解必须且只须在
二根之间,或等于二根.因此解集为:
\[\left\{x\Big|\frac{2-\sqrt{14}}{2}\le x\le \frac{2+\sqrt{14}}{2} \right\}\]
\end{solution}

\begin{example}
解不等式$4x^2-12x+9\le 0$
\end{example}

\begin{solution}
    由于$b^2-4ac=(-12)^2-4\x4\x9=0$, 故
\[4x^2-12x+9=(2x-3)^2\]
因此解集只能是$\left\{\frac{3}{2}\right\}$.
\end{solution}


\begin{example}
    $p$为何值时,二次三项式
$x^2+(p-2)x+2p+1$
对于任何$x$都取正值.
\end{example}


\begin{solution}
    只有当判别式$\Delta<0$时,$x^2+(p-2)x+2p+1$
的值才能保持正号,因此
$$\Delta=(p-2)2-4(2p+1)<0$$
即$p(p-12)<0$.

解$p(p-12)=0$, 得到二根$p_1=0$, $p_2=12$, 又$p^2$的系数是
$1>0$, 因此上面不等式的解满足$0<p<12$.

答:在$0<p<12$的条件下,$x^2+(p-2)x+2p+1$
对于任何$x$值都取正值.
\end{solution}


\begin{ex}
\begin{enumerate}
    \item 解下列不等式:
\begin{multicols}{2}
\begin{enumerate}
    \item $(x-2)^2+1>0$
    \item $(x-2)^2-1>0$
    \item $x^2-x+\frac{1}{4}>0$
    \item $2x^2-3x>2$
    \item $x^2>0$
    \item $x^2-4x+6\le 0$
    \item $49x^2+168x+144\le 0$
    \item $x^2+x+\frac{1}{4}\le 0$
    \item $x^2+x+2>0$
    \item $x^2+x-2<0$
\end{enumerate}
\end{multicols}
    \item $m$为何值使得下面的二次三项式的值对于任何$x$值都是
    正的或都是负的:
\begin{multicols}{2}
\begin{enumerate}
    \item $x^2-8x+m+10$
    \item $-x^2-2x+m-6$
    \item $x^2+(m+2)x+3m+1$
    \item $-3x^2+(2m+6)x-m-3$
\end{enumerate}
\end{multicols}
\end{enumerate}    
\end{ex}

\subsection{一元高次不等式}
如果一元较高次多项式能分解成一次或二次的因式的乘
积,上节所述对一元二次不等式的解法的原则,也可以用来
解较高次不等式.

现在我们把上节对解二次不等式所采用的原则概括成以
下的步骤:
\begin{enumerate}
    \item 把不等式化成标准型:$p(x)>0$或$p(x)<0$. 这里
$p(x)$是个一元多项式.
\item 把$p(x)$因式分解
\item 求出各个因式的零点.
\item 把零点在数轴上由小到大排列起来,并把数轴分为
若干段.
\item 在各段上考察各个因式的符号并决定$p(x)$的符号.
\item 找出不等式的解集.
\end{enumerate}


\begin{example}
    解不等式:$x^3+3x^2>2x+6$
\end{example}

\begin{solution}
\begin{enumerate}
    \item 把原不等式化为如下的标准型:$x^3+3x^2-2x-6>0$
    \item 把不等式左端因式分解:
    \[\begin{split}
         x^3+3x^2-2(x+3)&=x^2(x+3)-2(x+3)\\
    &=(x+3)(x^2-2)\\
    &=(x+3)\left(x+\sqrt{2}\right)\left(x-\sqrt{2}\right) 
    \end{split}\]
  
    所以解原不等式,就相当于解不等式
   \[(x+3)\left(x+\sqrt{2}\right)\left(x-\sqrt{2}\right) >0\]
    \item 各个因子的零点是$-3,-\sqrt{2},\sqrt{2}$.
    
    4, 5, 6三步可列表进行:
\begin{center}
\begin{tikzpicture}[yscale=2]
\foreach \x in {0,.5,2}
{
    \draw (0,\x)--(10,\x);
}
\foreach \x/\xtext in {5.5/{},6.5/-3,7.5/-\sqrt{2},8.5/\sqrt{2}}
{
    \draw (\x,0)--(\x,2)node[above]{$\xtext$};
}

\node at (2.5,.25){$(x+8)(x+\sqrt{2})(x-\sqrt{2})$};
\foreach \x/\xtext in {6/-,7/+,8/-,9/+}
{
    \node at (\x,.25) {$\xtext$};
}

\node at (2.5,.75){$x-\sqrt{2}$};
\foreach \x/\xtext in {6/-,7/-,8/-,9/+}
{
    \node at (\x,.75) {$\xtext$};
}

\node at (2.5,1.25){$x+\sqrt{2}$};
\foreach \x/\xtext in {6/-,7/-,8/+,9/+}
{
    \node at (\x,1.25) {$\xtext$};
}

\node at (2.5,1.75){$x+8$};
\foreach \x/\xtext in {6/-,7/+,8/+,9/+}
{
    \node at (\x,1.75) {$\xtext$};
}


\node at (6.5,1.75)[fill=white] {0};
\node at (6.5,.25)[fill=white] {0};
\node at (7.5,.25)[fill=white] {0};
\node at (8.5,.25)[fill=white] {0};
\node at (8.5,.75)[fill=white] {0};
\node at (7.5,1.25)[fill=white] {0};
\end{tikzpicture}
\end{center}
\end{enumerate}

所以原不等式的解集为:
\[\{x|-3<x<-\sqrt{2}\}\cup\{x|x>\sqrt{2}\}\]
\end{solution}

\begin{example}
解不等式$(x^2-3)(x^2-5)\le 0$
\end{example}


\begin{solution}
    解原不等式就相当于解不等式
\[(x+\sqrt{3})(x-\sqrt{3})(x+\sqrt{5})(x-\sqrt{5})\le 0\]
列表解之如下:
\begin{center}
\begin{tikzpicture}[yscale=2, xscale=1.2]
\foreach \x in {0,.5,2.5}
{
    \draw (0,\x)--(10.5,\x);
}
\foreach \x/\xtext in {5.5/{},6.5/-\sqrt{5},7.5/-\sqrt{3},8.5/\sqrt{3},9.5/\sqrt{5}}
{
    \draw (\x,0)--(\x,2.5)node[above]{$\xtext$};
}

\node at (2.5,.25){$(x+\sqrt{3})(x-\sqrt{3})(x+\sqrt{5})(x-\sqrt{5})$};
\foreach \x/\xtext in {6/+,7/-,8/+,9/-,10/+}
{
    \node at (\x,.25) {$\xtext$};
}

\node at (2.5,.75){$x-\sqrt{5}$};
\foreach \x/\xtext in {6/-,7/-,8/-,9/-,10/+}
{
    \node at (\x,.75) {$\xtext$};
}

\node at (2.5,1.25){$x-\sqrt{3}$};
\foreach \x/\xtext in {6/-,7/-,8/-,9/+,10/+}
{
    \node at (\x,1.25) {$\xtext$};
}

\node at (2.5,1.75){$x+\sqrt{3}$};
\foreach \x/\xtext in {6/-,7/-,8/+,9/+,10/+}
{
    \node at (\x,1.75) {$\xtext$};
}

\node at (2.5,2.25){$x+\sqrt{5}$};
\foreach \x/\xtext in {6/-,7/+,8/+,9/+,10/+}
{
    \node at (\x,2.25) {$\xtext$};
}

\foreach \x in {6.5,7.5,8.5,9.5}
{
    \node at (\x,.25)[fill=white] {0};
}
\node at (9.5,.75)[fill=white] {0};
\node at (7.5,1.75)[fill=white] {0};
\node at (6.5,2.25)[fill=white] {0};
\node at (8.5,1.25)[fill=white] {0};
\end{tikzpicture}
\end{center}

所以原不等式的解集为:
\[\{x|-\sqrt{5}\le x\le -\sqrt{3}\}\cup\{x|\sqrt{3}\le x\le \sqrt{5}\}\]
\end{solution}

\begin{example}
    解不等式$(x^3-1)(x^3+2x^2-x-2)>0$
\end{example}

\begin{solution}
    将不等式左侧因式分解后就相当于解不等式:
$$(x-1)^2(x^2+x+1)(x+1)(x+2)>0$$
这里二次三项式$x^2+x+1$的判别式小于零,且首项系数大
于零,故它对一切实数值都大于零,因此在列表时可以把它
略去不计.但因子$(x-1)^2$, 除$x=1$的零点外,$(x-1)^2$恒
大于零,故也可暂不考虑它,不过在最后确定解集时,要把
$x=1$除外的情况考虑进去.这样,就有下表:
\begin{center}
\begin{tikzpicture}[yscale=2, xscale=1.2]
\foreach \x in {0,.5,1.5}
{
    \draw (2,\x)--(8.5,\x);
}
\foreach \x/\xtext in {5.5/{},6.5/-2,7.5/-1}
{
    \draw (\x,0)--(\x,1.5)node[above]{$\xtext$};
}

\node at (3.5,.25){$(x+2)(x+1)$};
\foreach \x/\xtext in {6/+,7/-,8/+}
{
    \node at (\x,.25) {$\xtext$};
}

\node at (3.5,.75){$x+1$};
\foreach \x/\xtext in {6/-,7/-,8/+}
{
    \node at (\x,.75) {$\xtext$};
}

\node at (3.5,1.25){$x+2$};
\foreach \x/\xtext in {6/-,7/+,8/+}
{
    \node at (\x,1.25) {$\xtext$};
}

\node at (6.5,.25)[fill=white] {0};
\node at (7.5,.25)[fill=white] {0};
\node at (7.5,.75)[fill=white] {0};
\node at (6.5,1.25)[fill=white] {0};
\end{tikzpicture}
\end{center}

考虑$(x-1)^2(x^2+x+1)(x+1)(x+2)>0$
的解时,要把$x=1$除外,故解集为:
\[\{x|x<-2\}\cup\{x|x>-1,\; x\ne 1\}\]
或者写成
\[\{x|x<-2\}\cup \{x|-1<x<1\}\cup\{x|x>1\}\]
\end{solution}

\begin{example}
    解不等式$\frac{9-x^2}{x^2-x-2}\ge 0$
\end{example}

\begin{solution}
解法1:用化为不等式组的方法解
\[\begin{split}
&\quad \left\{x\Big| \frac{9-x^2}{x^2-x-2}\ge 0\right\}\\
&=\left\{x\Big|\begin{cases}
    9-x^2\ge 0\\ x^2-x-2>0
\end{cases} \right\}\bigcup \left\{x\Big|\begin{cases}
    9-x^2\le 0\\ x^2-x-2<0
\end{cases} \right\}\\
&=\Big[\{x|-3\le x\le 3\}\cap \{x|x<-1\text{ 或 }x>2\}\Big]\\
&\qquad \bigcup \Big[\{x|x\le-3\text{ 或 }x\ge 3\}\cap \{x|-1<x<2\}\Big]\\
&=\Big[\{x|-3\le x<-1\}\cup\{x|2<x\le 3\} \Big]\cup \big[\emptyset\cup\emptyset\big]\\
&=\{x|-3\le x<-1\}\cup\{x|2<x\le 3\}
\end{split}\]

解法2: 用列表法来解.

事实上,解原不等式就相当于解不等式组
\[\begin{cases}
    (x+3)(x+1)(x-2)(x-3)\le 0\\
    x\ne -1,\quad x\ne 2
\end{cases}\]

\begin{center}
\begin{tikzpicture}[yscale=2, xscale=1.2]
\foreach \x in {0,.5,2.5}
{
    \draw (0,\x)--(10.5,\x);
}
\foreach \x/\xtext in {5.5/{},6.5/-3,7.5/-1,8.5/2,9.5/3}
{
    \draw (\x,0)--(\x,2.5)node[above]{$\xtext$};
}

\node at (2.5,.25){$(x+3)(x+1)(x-2)(x-3)$};
\foreach \x/\xtext in {6/+,7/-,8/+,9/-,10/+}
{
    \node at (\x,.25) {$\xtext$};
}

\node at (2.5,.75){$x-3$};
\foreach \x/\xtext in {6/-,7/-,8/-,9/-,10/+}
{
    \node at (\x,.75) {$\xtext$};
}

\node at (2.5,1.25){$x-2$};
\foreach \x/\xtext in {6/-,7/-,8/-,9/+,10/+}
{
    \node at (\x,1.25) {$\xtext$};
}

\node at (2.5,1.75){$x+1$};
\foreach \x/\xtext in {6/-,7/-,8/+,9/+,10/+}
{
    \node at (\x,1.75) {$\xtext$};
}

\node at (2.5,2.25){$x+3$};
\foreach \x/\xtext in {6/-,7/+,8/+,9/+,10/+}
{
    \node at (\x,2.25) {$\xtext$};
}

\foreach \x in {6.5,7.5,8.5,9.5}
{
    \node at (\x,.25)[fill=white] {0};
}
\node at (9.5,.75)[fill=white] {0};
\node at (7.5,1.75)[fill=white] {0};
\node at (6.5,2.25)[fill=white] {0};
\node at (8.5,1.25)[fill=white] {0};
\end{tikzpicture}
\end{center}

考虑原不等式的解,要把$x=-1$和$x=2$除外,所以原
不等式的解集
\[\{x|-3\le x<-1\}\cup \{2<x\le 3\}\]
\end{solution}

\begin{ex}
 \begin{enumerate}
     \item 解下列不等式:
\begin{enumerate}
    \item $(x-2)(x+3)(x-4)(x+5)>0$
    \item $x^2(x+11)\ge 6(x^2+1)+x-1$
    \item $(x-2)(3x^2+2x-5)<0$
    \item $x^3+7x^2+6x\le 0$
\end{enumerate}
     \item      解下列不等式:
    \begin{multicols}{2}
\begin{enumerate}
    \item $\frac{2x-7}{x-5}>0$
    \item $\frac{(3x+2)(2x+3)}{(3x-2)(2x-3)}\ge 0$
    \item $\frac{x^2+2x-3}{x^2-2x-8}\le 0$
    \item $\frac{x+1}{x-1}<\frac{x-9}{x-3}-2$
\end{enumerate}        
    \end{multicols}
 \end{enumerate}   
\end{ex}

\section*{复习题三}
\addcontentsline{toc}{section}{复习题三}
\begin{enumerate}
\item   设$a,b,m$都是正数,如果$a<b$, 则
$\frac{a+m}{b+m}>\frac{a}{b}$;
如果$a>b$,则$\frac{a+m}{b+m}<\frac{a}{b}$
试证之.

\item  已知$a\ne b$, 求证:
\begin{multicols}{2}
\begin{enumerate}
    \item $a^2+3b^2>2b(a+b)$
    \item $a^4+6a^2b^2+b^4>4ab(a^2 +b^2)$
    \item $(a-b)(a+b)>-2b^2$
    \item $a^4+b^4>a^3b+ab^3$
\end{enumerate}
\end{multicols}

\item 已知$a$、$b$、$c$是不相等的正数,求证:
\begin{enumerate}
    \item $(a+b)(b+c)(c+a)>8abc$
    \item $\frac{b+c-a}{a}+\frac{c+a-b}{b}+\frac{a+b-c}{c}>3$
    \item $(ab+a+b+1)(ab+ac+bc+c^2)>16abc$
\end{enumerate}

\item 已知$a^2+b^2=1$, $x^2+y^2=1$, 求证:$ax+by\le 1$.
\item 若$x,y,z$是正数,且$x+y+z=1$,求证:
$x^2+y^2+z^2\ge \frac{1}{3}$.
\item \begin{enumerate}
    \item 试证:两正数之和与此两正数的倒数之和的乘积不小于4.
    \item 试证:$\frac{\lg^2x+2}{\sqrt{\lg^2x+1}}\ge 2$.
\end{enumerate}
\item 已知$\triangle ABC$的三边为$a$、$b$、$c$且
\[c=\frac{a^2+3}{4},\qquad b=\frac{a^2-2a-3}{4}\]
求证$c$边最大.
\item 
已知直角三角形斜边为定值,求这个直角三角形面积的
最大值,并指出取得最大值的条件.
\item 
解下列关于$x$的不等式:
\begin{multicols}{2}
\begin{enumerate}
    \item $ax+b^2>bx+a^3\quad (a<b)$
    \item $mx-n^3<nx-m^3\quad (m<n)$
    \item $ax+b>cx+d$
    \item $\frac{2x}{2-h}-\frac{h(x+1)}{2-h}<1\quad (h\ne 2)$
    \item $\frac{x-a}{x-b}>0$
\end{enumerate}
\end{multicols}
\item 解不等式组:
\begin{enumerate}
    \item $\begin{cases}
        (x-1)^2<(x+1)^3-4\\(x-1)(x-2)<(x+3)(x-4)+20
    \end{cases}$
    \item $\begin{cases}
        (x+2)^3-(x-1)^3<9x^2 \\(x+2)^2-(x-1)^2>9
    \end{cases}$
\end{enumerate}

\item 求下列不等式组的整数解:
\begin{multicols}{2}
\begin{enumerate}
    \item $\begin{cases}
        2x-7>0\\3x-5<15
    \end{cases}$
    \item $\begin{cases}
        11x-5<13\\3x+2>-5
    \end{cases}$
    \item $\begin{cases}
       3x+10>0\\ \frac{15}{3}x-10<4x
    \end{cases}$
    \item $\begin{cases}
      \frac{2x-11}{4}+\frac{19-2x}{2}>-2x\\
      \frac{2x+15}{9}>\frac{1}{5}(x-1)+\frac{x}{3}
    \end{cases}$
\end{enumerate}
\end{multicols}

\item 解下列不等式,并在数轴上把解集表示出来:
\begin{multicols}{2}
\begin{enumerate}
    \item $|x+2|-|x-3|<4$
    \item $|x-5|-|x-4|\ge 3$
    \item $|x-2|\ge 2x-1$
    \item $|2x^2+5x-2|<1$
\end{enumerate}
\end{multicols}

\item 一个两位数,它的个位数字比十位数字大2, 已知这个
两位数小于45而大于24, 求这数.

\item 一个两位数,它大于22而小于36, 并且它的十位数字比
个位数字大3, 求这数.
\item 一个分数的分子、分母都是自然数,分子比分母小1,
如果分子,分母都加上1, 所得就大于$\frac{1}{2}$,如果分子,
分母都减去1所得就小于$\frac{6}{7}$, 求这个分数.
\item 当$m$是什么实数时,方程
$mx^2-(m+1)x+3=0$
有实数根?没有实数根.
\item $a$是什么实数时,方程$5x-2a=ax-4-x$的解在2和
10之间?
\item $t$是什么实数时,方程
$tx^2-(1-t)x+t=0$
没有实数解?
\item 讨论二次三项式$x^2+2x+m+1$的值的符号.
\item 造一个截面为矩形的水管,要求矩形的长比宽多10cm, 
并且面积不小于250${\rm cm}^2$, 矩形的宽至少应当是多少cm?
\item 要使下列各式有意义,$x$的值应有什么限制?
\begin{enumerate}
    \item $\frac{\sqrt{x-3}}{\sqrt{x^2-3x+2}}$
    \item $\sqrt{|x-2|-3}+\frac{1}{\sqrt[3]{2x+1}}$
\end{enumerate}
\item $m$为何值使得下列不等式对于$x$的一切值都成立:
\begin{enumerate}
    \item $mx^2+12x-5<0$
    \item $(m+3)x^2-5x-4<0$
    \item $(m-2)x^2-2(2m-3)x+5m-6>0$
    \item $(m^2+6m-4)x^2-2(m-1)x+2<0$
    \item $(m^2+4m-5)x^2-2(m+1)x+3>0$
\end{enumerate}

\item 解下列不等式:
\begin{multicols}{2}
\begin{enumerate}
    \item $\frac{(x-1)(x-2)}{x-3}\ge 0$
    \item $\frac{x^2-6x+18}{x-4}<0$
    \item $\frac{x^2+2x-3}{x^2-2x+8}>0$
    \item $\frac{x^2+5x+4}{x^2-5x-6}<0$
    \item $2-\frac{x-3}{x-2}>\frac{x-2}{x-1}$
    \item $3-\frac{2x-17}{x-5}>\frac{x-5}{x+2}$
\end{enumerate}
\end{multicols}

\end{enumerate}








%  \chapter{函数及其图象}

从这一章开始,我们要研究数学中一个很重要的概
念-函数概念.
\section{函数及其图象}
\subsection{常量、变量和函数}
首先来考察这样一个问题,我们对一个铁环加热,据热胀
冷缩原理,铁环就要膨胀,铁环的周长$C$和半径$r$的关系是:
\[C=2\pi r\]
在加热过程中,数值$r$和$C$是变化着的,我们把它们叫做\textbf{变
量},而$2\pi$总是一个固定值,我们称它为\textbf{常量}.

应当注意,一个量是常量还是变量,并不是绝对的,应
根据问题的不同要求作具体的分析.例如,根据欧姆定
律,电压、电流、电阻的关系是$V=IR$, 当电流$I$一定时,
$V,R$是变量,$I$是常量,当电阻$R$一定时,$I,V$是变量,$R$
是常量.当电压$V$一定时,$I,R$是变量,$V$是常量.

下面我们来研究变量之间的关系.先来考察几个简单的
例子.


\begin{example}
    弹性原理(虎克定律)弹簧秤下挂砝码(如图4.1).
随着所悬挂砝码的质量$m$的不同,弹簧的长度$\ell$也随之不同,
$m$越大,$\ell$就越大,$\ell$与$m$之间的关系是:
\[\ell=km+\ell_0\]
\end{example}

\begin{figure}[htp]
    \centering
    \begin{tikzpicture}[>=latex]
    \draw (-.5,0)--(.5,0);
    \draw (4,0)--(3,0);
    \draw [decorate,decoration={coil,segment length=4pt}](0,0)--(0,-3);
    \draw [decorate,decoration={coil,segment length=3pt}](3.5,0)--(3.5,-1.3);
\draw (-.2,-3) rectangle (.2,-3.4);
\node at (.5,-3.2){$m$};

\draw [|<->|](-1,0)--node[fill=white]{$\ell$}(-1,-3);
\draw [|<->|](1,0)--node[fill=white]{$\ell_0$}(1,-1.3);
\draw [|<->|](1,-1.3)--node[fill=white]{$km$}(1,-3);
\draw [|<->|](4.5,0)--node[fill=white]{$\ell_0$}(4.5,-1.3);

    \end{tikzpicture}
 \caption{}
\end{figure}

\begin{example}
    一种保险丝的直径和它所容许通过的额定电流之
    间的数量关系,可以列成下表:
\begin{center}
\begin{tabular}{cccc}
    \hline
    额定电流(I)  &  保险丝直径(d)  &  额定电流(I) &  保险丝直径(d)\\
(安培)  &  (毫米)  &  (安培)  &  (毫米)  \\
\hline
1.0  &  0.28  &  7.5  &  1.25\\
2.0  &  0.52  &  10.0  &  1.51\\
3.0  &  0.71  &  11.0  &  1.67\\
5.0  &  0.98  &  12.0  &  1.75  \\
6.0  &  1.02  &  15.0  &  1.98\\
\hline
\end{tabular}
\end{center}


\end{example}

\begin{example}
    某地气象站的温度自动记录仪描绘了某一天的温
    度变化曲线,如图4.2.
\end{example}

\begin{figure}[htp]
    \centering
\begin{tikzpicture}[>=latex]
\draw[->](-1,0)--(7.5,0)node[right]{$t$小时};
\draw[->](0,-2.5)--(0,2.5)node[right]{$T^{\circ}{\rm C}$};     
\draw [domain=.1:7, samples=1000, ultra thick] plot(\x, {-1.2*sin(\x r)});

\foreach \x in {-6,-4,...,6}
{
    \draw (0,\x*.3)node[left]{$\x$}--(.1,\x*.3);
}
\foreach \x in {2,4,...,24}
{
    \draw (\x*.28,0)node[below]{$\x$}--(\x*.28,0.1);
}


\end{tikzpicture}
    \caption{}
\end{figure}


我们看到在上面的每个例子中都两个变量;例4.1是弹
簧的长度$\ell$和砝码的重量$m$; 例4.2是额定电流$I$和保险丝直
径$d$; 例4.3是时间$t$和温度$T$. 但是在每个例子中的两个变
量并不是孤立的,它们之间有某种关系,如例4.1中,弹簧秤
长度随砝码的重量$m$的变化而变化;在例4.2中,从表中看
出,保险丝直径$d$的大小要随用电器的额定电流$I$的强度
来选择;在例4.3中,由图纸上的曲线看出,随着时间$t$变化,
温度$T$也随之变化,这三个例子用三种不同的形式(解析式、
表格、图象)描述了两个变量之间的依赖关系.

从上面三个例子中可以看到一个共同点,即一个变量可
在某一范围内“自由变化”,我们把这种变量称为自变量(如
例4.1的砝码重量$m$, 在例4.2的表中所列出的额定电流$I$, 例4.3
的时间$t$), 而另一个变量是随着自变量相应地跟着变动的,
这种变量叫做因变量,或称为自变量的函数(如例4.1的弹簧
长度$\ell$,例4.2的保险丝直径$d$, 例4.3的温度$T$都是相应自变量的
因变量,或相应自变量的函数).

“自变量可在某一范围内自由变化”和“因变量是随着自
变量相应地跟着变动”的意思是,“自变量在某一给定的范围
内可以取每一个值,因变量就按照一定的规律取相应的值”.

在如上所述的各种变化过程中,为了把这种动态的事物
的数量变化关系表达出来,在数学上,我们用“变数符号”去表
达变量;用“函数关系”去表达变量之间的相互关联,函数概
念的明确定义如下:

\begin{blk}{定义 }
    如果有两个变量$x$和$y$, 对于变量$x$在某一给定范
围内的每一确定的值,变量$y$按照一个确定的法则有唯一确
定的值和它对应,那么变量$y$就叫做变量$x$的函数,并把$x$叫做
自变量,$y$也可以叫做因变量,自变量的取值范围叫做这个
函数的定义域,在这个定义域上,因变量$y$的取值范围叫做这
个函数的值域.
\end{blk}


这样,在例4.1中弹簧长度$\ell$是砝码重量$m$的函数,例4.2中保
险丝直径$d$是额定电流$I$的函数,例4.3中温度$T$是时间$t$的函数.

这个函数定义包含着三个要素:定义域、对应
法则、值域.

为了正确地理解函数定义,下面我们进一步来剖析一
下:

\subsubsection{函数的定义域}
所谓函数的定义域,就是自变量容许取值的范围,也就
是自变量所取数值的集合.

为以后表述简单起见,在进一步讲解函数的定义域之
前,先介绍一下区间的概念.

我们把数集$\{x|a<x<b\}$记作$(a,b)$, 称为以$a,b$为端点
的开区间,把数集$\{x|a\le x\le b\}$记作$[a,b]$, 称为以$a,b$为端
点的闭区间,而象$(a,b]$和$[a,b)$都称为半开区间,它们分别
是数集$\{x|a<x\le b\}$和$\{x|a\le x<b\}$, 我们把各种区间记号及
其所表示的点(数)集并列如下:
\begin{multicols}{2}
    \begin{enumerate}
    \item $(a,b)=\{x|a<x<b\}$
    \item $ [a,b]=\{x|a\le x\le b\}$
    \item $(a,b]=\{x|a<x\le b\}$
    \item $[a,b)=\{x|a\le x<b\}$
    \item $(a,+\infty)=\{x|x>a\}$
    \item $[a,+\infty)=\{x|x\ge a\}$
    \item $(-\infty,b)=\{x|x<b\}$
    \item $(-\infty,b]=\{x|x\le b\}$
    \item $(-\infty,+\infty)=\mathbb{R}$ (实数集)
\end{enumerate}
\end{multicols}

其中5--9中有无穷大$+\infty$, $-\infty$, 它们是一种记号而
不是一个数,下列记号不论哪一个都是无意义的:$(a,+\infty]$,
$[a,+\infty]$, $[-\infty,b)$, $[-\infty,b]$或$[-\infty,+\infty]$

以后某些数集就可以用区间来表示.

谈到函数,就要指明定义域,离开定义域去谈函数是没
有意义的,定义在$(-\infty,+\infty)$上的关于$x$的函数$y=x^2$和定
义在$[0,+\infty)$上的关于$x$的函数$y=x^2$, 尽管函数表达式是
一样的,但由于定义域不同,故不能认为这两个函数是一样
的.

怎样确定一个函数的定义域呢?一般有两个途径.
\begin{enumerate}
    \item 在实际问题中函数的定义域根据实际问题的意义来
    定.

    例如,例4.3温度自动记录仪记录下的一昼夜的气温$T$的变
    化情况,这里气温$T$是时间$t$的函数,自变量$t$的容许值范围
    是$[0,24]$.

    \item 在理论上研究函数时,定义域或者已经明确给出,
    或者由函数表达式确定.

    如果我们讨论的函数由一个数学算式表示,并且不考虑
    它的实际意义,那么这个函数的定义域,就是指使这个数学
    算式有意义的自变量取值范围,这又叫做自然定义域,通常
    不需要明确指出.

    例如,函数$\frac{1}{x}$
    的定义域是由除零以外的一切实数所组
    成,即$(-\infty,0)\cup(0,+\infty)$, 又如函数$\sqrt{x}$的定义域是一切
    正数和零所组成,即$[0,+\infty)$.
\end{enumerate}


\begin{example}
 在一个边长为30cm
的正方形铁皮上,四角各截去
边长为$x$的小正方形(图4.3),
按虚线折起来成一个无盖盒
子,求这个盒子容积$V$关于自
变量$x$的函数关系,并指明其
定义域.   
\end{example}

\begin{figure}[htp]
    \centering
\begin{tikzpicture}[>=latex]
\draw (0,0) rectangle (4,4);
\draw[pattern=north east lines] (0,0) rectangle (.75,.75);    
\draw[pattern=north east lines] (3.25,0) rectangle (4,.75);  
\draw[pattern=north east lines] (0,3.25) rectangle (.75,4);  
\draw[pattern=north east lines] (3.25,3.25) rectangle (4,4);  
\draw[dashed] (0.75,0.75) rectangle (3.25,3.25);  
\foreach \x in {0,4}
{
    \draw (\x,4)--(\x,4.8);
}
\draw [<->](0,4.4)--node[fill=white]{30}(4,4.4);
\foreach \x in {0,.75,3.25,4}
{
    \draw (\x,0)--(\x,-.8);
}
\draw [<->](0.75,-.4)--node[fill=white]{$30-2x$}(3.25,-.4);
\draw [<->](0.75,-.4)--node[below]{$x$}(0,-.4);
\draw [<->](4,-.4)--node[below]{$x$}(3.25,-.4);
\end{tikzpicture}
    \caption{}
\end{figure}

\begin{solution}
    已知截去小正方形边
    长为$x$cm, 做成的盒子的高是
    $x$cm,盒子的边长是$(30-2x)$cm.
\[\therefore\quad     V=x(30-2x)^2\]
    这就是所求的$V$关于$x$的函数.

    根据实际问题,小正方形的边长不能为零或负数,另一
    方面它又不能等于或大于正方形边长的一半,所以$x$只能在0
    和15之间,故函数的定义域是$(0,15)$.
\end{solution}

\begin{example}    
    求下列$x$的函数的定义域:
    \begin{multicols}{2}
\begin{enumerate}
    \item $y=x+1$
    \item $y=\frac{x^2+1}{x-1}$
    \item $y=\sqrt{3-x}$
    \item $y=\sqrt{3-x}+\sqrt{x+3}$
    \item $\frac{1}{\sqrt[3]{x+4}}$
    \item $\frac{\sqrt{x+1}}{x^2-5x+6}$
    \item $\lg(1-\sqrt{3-x})$
\end{enumerate}        
    \end{multicols}
\end{example}

\begin{solution} 
\begin{enumerate}
    \item $y=x+1$的定义域为实数集$\mathbb{R}$,即:$(-\infty,+\infty)$.
    \item $y=\frac{x^2+1}{x-1}$的定义域为$x\ne 1$的一切实数,即
    \[x\in (-\infty,1)\cup(1,+\infty)\]
\item $y=\sqrt{3-x}$,要使根式有意义,就要$3-x\ge 0$,
即$x\le 3$, 即$x\in (-\infty,3)$.
\item $y=\sqrt{3-x}+\sqrt{x+3}$要使两个根式同时有意义,定义域只能是每个根式的定
义域的交集.
\begin{itemize}
    \item $3-x\ge 0$的解集是$(-\infty,3)$;
    \item  $x+3\ge 0$的解集是$(-3,+\infty)$.
\end{itemize}
因此,$x\in (-\infty,3)\cap (-3,+\infty)=[-3,3]$ (图4.4)
\begin{figure}[htp]
    \centering
\begin{tikzpicture}[>=latex]
\draw[->] (-4,0)--(4,0)node[right]{$x$};
\foreach \x in {-3,0,3}
{
    \draw (\x,0)node[below]{$\x$}--(\x,.1);
}
\draw (-3,0)--(-3,1)--(4,1);
\draw (3,0)--(3,.5)--(-4,.5);
\draw[pattern=north east lines] (-3,.5) rectangle (3,0);
\end{tikzpicture}
    \caption{}
\end{figure}
\item 要使式子$\frac{1}{\sqrt[3]{x+4}}$有意义,就要$x+4\ne 0$, 即$x\ne -4$,所以定义域是$(-\infty,-4)\cup(-4,+\infty)$.
\item 要使式子$\frac{\sqrt{x+1}}{x^2-5x+6}$有意义,必须且只须,
\[\begin{cases}
    x+1\ge 0\\
x^2-5x+6\ne 0
\end{cases}\Rightarrow\quad \begin{cases}
   x\ge -1\\
   x\ne 2\quad x\ne 3 
\end{cases}\]
故定义域是$[-1,2)\cup(2,3)\cup(3,+\infty)$ (图4.5).

\begin{figure}[htp]
    \centering
    \begin{tikzpicture}[>=latex]
        \draw[->] (-2,0)--(5,0)node[right]{$x$};
\foreach \x in {-1,2,3}
{
    \draw (\x,0)node[below]{$\x$}--(\x,.5);
}

\node at (0,0)[below]{0};

\draw (-1,0.5)--(4.5,.5);
\fill[pattern=north east lines] (-1,.5) rectangle (4.5,0);
\draw(-1,0)[fill=black] circle (1.5pt);
\draw(2,0)[fill=white] circle (1.5pt);
\draw(3,0)[fill=white] circle (1.5pt);
    \end{tikzpicture}
    \caption{}
\end{figure}

\item 要使对数$\lg(1-\sqrt{3-x})$的真数中的根式有意
义,只要
$3-x\ge 0$, 
要使对数有意义,又必须且只须真数大于零,即
$1-\sqrt{3-x}>0$,
因此要使对数$\lg(1-\sqrt{3-x})$有意义,必须且只须使
$x$满足下面不等式组:
\[\begin{cases}
    3-x\ge 0\\
\sqrt{3-x}<1
\end{cases}\Rightarrow\quad \begin{cases}
    x\le 3\\
x>2
\end{cases}\]
所以函数$\lg(1-\sqrt{3-x})$的定义域是$(2,3]$. 

\end{enumerate}    
\end{solution}

\subsubsection{对应法则}
对应法则是因变量$y$对自变量$x$的依赖关系(即函数关
系)的具体表现,它是函数概念里最本质的东西,也是区别
各个不同函数的最主要的标志.定义在$(-\infty,+\infty)$上$y$对$x$的
函数:$y=x^2$, 和定义在$(-\infty,+\infty)$上$U$对$V$的函数:$U=V^2$, 
尽管自变量与因变量各用不同的字母,但由于它们的对应法
则是一样的,定义域又相同,故我们不认为它们是不一样的
函数.

应当指出,自变量$x$在某一范围内的每一个值,通过对
应法则,变量$y$有唯一确定的值与之对应,这种对应法则才
体现了因变量$y$与自变量$x$的函数关系,不然的话,尽管给出
了对应法则,但并不构成函数,例如,$x^2+y^2=1$或$y=
\pm\sqrt{1-x^2}$, 当自变量$x$在$-1$到1之间变动时,根据对应法
则,变量$y$就有确定的两个值与之对应,因之对应法则$y=
\pm\sqrt{1-x^2}$, 就不能表现$y$是$x$的函数.但是我们倒可以把这个
对应法则拆成下面两个函数:
\[y=\sqrt{1-x^2},\qquad y=-\sqrt{1-x^2},\quad x\in [-1,1]\]

对应法则的表现形式是各种各样的.一种是用数学公式
来表示函数关系,叫做\textbf{解析式表示法}(如例4.1);一种是列出
对应的数值表来表示函数关系,叫做\textbf{列表表示法}(如例4.2),
还有一种是用图象来表示,叫做\textbf{图象表示法}(如例4.3), 这三
种表示法各有优缺点,解析表示法具有全面、简明的优点,
但自变量与因变量值的对应情况不能直观地反映出来;列
表法虽具有这方面的优点,但对定义域中每个对应值不能完
全列出,不能完全地把函数全貌表达出来;图象法其优点在
于形式简明直观便于比较,易显出数据中的转折点,最高点
或最低点,缺点是不够精确,不便于运算.这里要指出的
是,在实际问题中,这三种方法常常是综合使用的,例如在
作函数图象时,是由解析式→列表→图象,而在根据实
验数据求出一般公式时,常常是把对应数据列成表格,然
后描绘图象,最后根据图象再去寻找相应的公式.

当我们泛指一般函数关系时,就必须舍弃对应法则的具
体表现形式,而只顾及它们的本性——因变量$y$和自变量$x$
之间的依赖关系,采用记号
\[y=f(x)\]
来表达$y$是$x$的函数.注意,记号中的$f$表示某种对应法则.

为简便起见,我们常常用函数的一般记号$y=f(x)$来代
表一个具体函数.如例4.1, 可以用$\ell=f(m)$来表示,不过这里
的$f(m)$应理解为例4.1的具体的对应法则:$f(m)=km+\ell_0$

关于函数的记号,还应注意:
\begin{enumerate}
    \item 例如,圆面积$A$和周长$C$都是半径$r$的函数,如果我
    们都用
 \[   A=f(r),\qquad C=f(r)\]
    来表示,那么这里的$f(r)$是指$2\pi r$还是$\pi r$呢?就分不清了,
    所以,在同时研究这两个函数时,为避免混淆,就须在括号
    外面选用不同的字母,以区别这两个不同的函数关系,比如
    \[   A=f(r),\qquad C=g(r)\]
    \item 认识一个函数,关键就在于认识“$f$”,自变量与因
    变量用什么宇母表示是无关紧要的.

    如果“$f$”是以解析式给出,那么“$f$”就是这个解析式所
    含的一系列按一定顺序的运算.例如,函数
 \[   f(x)=x^2-3x+5\]
    这里“$f$”是指对自变量$x$实行下述运算后求得对应的函数值:
\[\text{(自变量)}^2-3\x \text{(自变量)}+5\longrightarrow \text{对应的函数值}\]
如$f(2)$就是对“2”实行这套运算,即$2^2-3\x 2+5=3$, 
    得到$f(2)=3$.
\end{enumerate}

\subsubsection{函数值域}
当函数的自变量$x$取遍定义域$D$中的一切值时,所对应
的函数值$y$的全体构成集合$R=\{y|y=f(x),\; x\in D\}$. 我们称
集合$R$是这函数的值域.由值域的定义易知对任意的$y_0\in R$,
必有$x_0\in D$与之对应,使关系式$y_0=f(x_0)$成立,但是与这$y_0$对
应的$x$值可能不止一个,例如$y=|x|$, 与$y=4$对应的$x$值有两
个:$x=4$或$x=-4$.再看一例,$y=f(x)=3$, $x\in D=(-\infty,+\infty)$, $R=\{3\}$, 这函数的值域是单元素集,只由一
个数3组成.这个函数称为常值函数,而与$y=3$对应的$x$值有
无穷多个,即一切实数.

\section*{习题4.1}
\addcontentsline{toc}{subsection}{习题4.1}
\begin{enumerate}
    \item     求下列函数的定义域($y$是$x$的函数):
    \begin{multicols}{2}
\begin{enumerate}
    \item $y=2x^2-3$
    \item $y=\sqrt{x}$
    \item $y=\sqrt{x-5}$
    \item $y=\sqrt{x^2-9}$
    \item $y=\lg(-x^2+9)$
    \item $y=\sqrt{x^2+9}$
    \item $y=\frac{x+1}{x-3}$
    \item $y=\sqrt{x^2-8x+15}$
    \item $y=\frac{\sqrt{x-1}}{x-5}$
    \item $y=\sqrt{x-4}+\sqrt{6-x}$
    \item $y=\sqrt{x^2-5x+4}+\lg(x+2)$

\end{enumerate}        
    \end{multicols}

\item 已知$f(x)=7x(2x-5)$, 求函数$f(x)$在$x=0,\frac{1}{2},1,2,
    \frac{1}{2+\sqrt{3}}$处的值.
    
\item 已知自变量$x$与因变量$y$之间有下面的关系,用$x$的代数
    式来表达$y$, 给出函数的定义域.
\begin{multicols}{2}
\begin{enumerate}
    \item $3x+4y=12$
    \item $xy=15$
    \item $(x-2)(y+3)=-6$
    \item $x=\frac{5y+3}{3y+2}$
\end{enumerate}
\end{multicols}

\item 试将$n$边形对角线的数目$N$用边数$n$的函数写出来,并指
出这函数的定义域.
\item 三角形两边的长$a,b$一定,夹角$\theta$不定,写出三角形面积
对于夹角$\theta$的函数关系,并指出这函数的定义域.
\item 圆的半径$R$一定,如果圆内扇形的中心角$\theta$是个变量,
求扇形面积$A$对于中心角$\theta$的函数,这里半径$R$的单位是
厘米,$\theta$的单位是度,又如果扇形中心角的单位改用弧
度,那么这个函数关系表达式有何改变?
\item 人工开凿的直线运河经过相距$d$公里的$A,B$两城(图
4.6).在$B$城垂直于运河的方向上离$B$城$\ell$公里有一个工
厂$C$, 从$A$城运货到工厂,先从水路到一地$M$, 然后走
陆路从$M$到$C$. 假设一吨货物每公里水路运费为$\alpha$元,陆
路运费为$\beta$元,求每吨总运费与$MB$之间的函数关系.
\begin{figure}[htp]
    \centering
\begin{tikzpicture}[>=latex, scale=.8]
\draw (0,0)node[left]{$A$}--(8,0)node[above]{$B$};
\draw (5,0)node[above]{$M$}--(8,4)node[above]{$C$};    
\draw[dashed](8,4)--(8,0);
\foreach \x in {0,5,8}
{
    \draw (\x,0)--(\x,-.8);
}
\draw[<->] (0,-.6)--node[fill=white]{$d$}(8,-.6);
\draw[<->] (5,-.3)--node[fill=white]{$x$}(8,-.3);
\draw[|<->|] (8.3,0)--node[fill=white]{$\ell$}(8.3,4);

\end{tikzpicture}
    \caption{}
\end{figure}


\item  在底为$AC=b$和高为$BD=h$的三角形$ABC$中(图4.7)内
接一个高为$NM=x$的矩形$KLMN$, 把矩形$KLMN$的周长
$P$及其面积$S$表示为$x$的函数.
\begin{figure}[htp]
    \centering
\begin{tikzpicture}[>=latex]
\draw[very thick] (0,0)node[left]{$A$}--(6,0)node[right]{$C$};
\draw[very thick] (0,0)--(2,4)node[above]{$B$}--(6,0);
\draw[very thick] (2,4)--(2,0)node[below]{$D$};  
\draw[<->](2.3,0)--node[fill=white]{$h$}(2.3,4);
\draw (1,0)node[below]{$K$}--(1,2)node[left]{$L$} --(4,2)node[right]{$M$}-- (4,0)node[below]{$N$};
\draw (2,4)--(2.6,4);

\draw[<->](3.6,0)--node[fill=white]{$x$}(3.6,2);
\foreach \x in {0,6}
{
    \draw (\x,0)--(\x,-.8);
}
\draw[<->](0,-.6)--node[fill=white]{$b$}(6,-.6);

\end{tikzpicture}
    \caption{}
\end{figure}

\item 金属切削加工时,刀具一分钟内在零件(工件)表面上所
经过的路程(如铣床),或工件每分钟在刀具上所经过的
路程(车床),叫做切削速度,设:
$D$是刀具或工件直径(毫米);
$n$是刀具或工件每分钟转数;
$v$是切削速度(米/分).
那么切削速度$v$,对于刀具或工件的每分钟转
数$n$的函数关系是:
\[v=\frac{\pi Dn}{1000}\]
你能推导出这个关系式吗?
\item $y=f(x)=\begin{cases}
    \frac{1}{x} & 0<x\le 1\\ 0& x=0
\end{cases}$
是不是定义在$[0,1]$上的函数?为什么?值域是什么?又
$f(x)=\frac{1}{x}$
是不是定义在$[0,1]$上的函数?
\end{enumerate}

\subsection{函数的图象}
我们指出过,几何图形是符合某种条件的点的集合,例
如以$r$为半径的圆$O$是与$O$点距离等于定长$r$的点的集合,线段
的垂直平分线是和$A,B$的距离相等的点的集合.

我们也讲过有序实数对和坐标平面上的点是一一对应
的,即每个有序实数对对应着平面上一个点而且只对应一个
点,反之,平面上的每个点对应着一个有序实数对而且只对
应一个有序实数对.

根据这样的思想,我们来建立函数图象的概念.

已知函数$y=f(x)$, 定义域为$D$.
\begin{figure}[htp]
    \centering
\begin{tikzpicture}[>=latex]
  \draw[->](-.5,0)--(5,0)node[right]{$x$};
    \draw[->] (0,-1)--(0,3.5)node[right]{$y$};
\node at (-.25,-.25){$O$};
\draw (-.5,.5) to [bend right=-14](1,2) to [bend right=-10](4,3)node [right]{$y=f(x)$的图象};
\draw (1,0)node[below]{$x$}--node[right]{$f(x)$}(1,2)node[above]{$(x,f(x))$};
\draw[dashed](1,2)--(0,2)node[left]{$f(x)$};
\end{tikzpicture}
    \caption{}
\end{figure}


对于定义域$D$中任意的$x$, 点$(x,f(x))$表示平面上一个
点.当$x$遍取$D$中的一切值时,点$(x,f(x))$的集合就构 成一个
图形$F$(图4.8). 这个图形$F$就是函数$y=f(x)$的图象.下面给出定义:

\begin{blk}{定义}
函数$y=f(x)$的图象$F$是坐标平面上所有这样的
的图象点的集合:其坐标$(x,y)$由下
列法则给出:
\begin{enumerate}
    \item $x$遍取定义域$D$中
的一切值,
\item $y$由函数关系式$y=f(x)$确定.
\end{enumerate}
用符号表示:
\[F=\{(x,y)|x\in D\; \text{ 且 }\; y=f(x)\}\]    
\end{blk}


由定义知道,要证明图象$F$是函数$y=f(x)$的图象,必
须要证明下面两点:
\begin{enumerate}
    \item  $F$上所有的点的坐标$(x,y)$都能适合$y=f(x)$.
\item 不在$F$上所有的点的坐标$(x,y)$都不适合$y=f(x)$.

(或证2的逆否命题:“坐标能适合$y=f(x)$的关系的点都
在$F$上”也可).
\end{enumerate}

作函数$y=f(x)$的图象就是把$F$上的点都标在坐标平
面上,平面点集$F$一般是个无限集,把$F$的点一个一个地都
标出来是不可能的,在一般情形,我们的办法是先作出图形
$F$的一些点,然后用一条或几条平滑曲线把这些点连接起来
(连结的时候,通常依照自变量由小到大的顺序)而得到图形
$F$.

\begin{example}
描绘函数$y=\frac{1}{2}x^2$的图象.
\end{example}

\begin{solution}
这函数的定义域为一切实数,$x$可取任意实数,如:

取$x=\cdots,\quad -4,\quad -3,\quad -2,\quad -1,\quad 0,\quad 1,\quad 2,\quad 3,\quad 4,\quad \cdots$

计算$y=\cdots,\quad 8,\quad 4.5,\quad 2,\quad 0.5,\quad 0,\quad 0.5,\quad 2,\quad 4.5,\quad 8,\quad \cdots$

得有序对$(x,y)$:$\cdots,\quad (-4,8)$,\quad  $(-3,4.5)$,\quad  $(-2,2)$,\quad  $(-1,0.5)$,\quad  $(0,0)$,\quad  $(1,0.5)$,\quad  $(2,2)$,\quad  $(3,4.5)$,\quad  $(4,8),\quad\cdots$


通常用表格表示:
\begin{center}
\begin{tabular}{c|ccccccccccc}
\hline
$x$ & $\cdots$  &  $-4$  &  $-3$  &  $-2 $ &  $-1$  &  0  &  1  &  2  &  3  &  4  &  $\cdots$\\
\hline
$y$ &$\cdots$  &  8  &  4.5  &  2  &  0.5  &  0  &  0.5  &  2  &  4.5  &  8  &  $\cdots$\\
\hline
\end{tabular}
\end{center}

由这些有序对可在平面上描出对应的点,用平滑曲线把
这些点连接起来,就得到函数$y=\frac{1}{2}x^2$
的图象(图4.9), 这种
描点作图象的方法叫做\textbf{描点法}.
\begin{figure}[htp]
    \centering
\begin{tikzpicture}[>=latex, scale=.7]
    \draw[->](-5,0)--(5,0)node[right]{$x$};
      \draw[->] (0,-1)--(0,9.5)node[right]{$y$};
  \node at (-.35,-.35){$O$};

\foreach \x in {1,2,3,4}
{
    \draw (\x,0)--(\x,.1);
    \draw (-\x,0)--(-\x,.1);
    \draw (0,\x)--(-.1,\x);
    \draw (0,\x+4)--(-.1,\x+4);
}

\foreach \x in {2,4}
{
    \node at (\x,0)[below]{$\x$};
    \node at (-\x,0)[below]{$-\x$};
}

\foreach \x in {2,4,6,8}
{
    \node at (-.1,\x)[left]{$\x$};
}

\draw [domain=-4.2:4.2, samples=100, very thick] plot(\x,{0.5*\x*\x});

\foreach \y in {-4,-3,...,4}
{
    \draw (\y, {0.5*\y*\y})[fill=black] circle (2pt);
}

\node at (5,6){$y=\frac{1}{2}x^2$};
\end{tikzpicture}
    \caption{}
\end{figure}
\end{solution}

应当指出,用描点法作图
象时,我们至多只能描出图象
的有限多个点,而不可能描出
$F$的全部的点,所以这种方法
带有一定盲目性,因为仅有一
些点并不足以确切掌握函数图象全貌,因此我们必须先对这
函数表达式进行详尽的研究,把握这函数图象的某些特点,
如间断点,最高点,最低点,函数变化趋向等,而后根据研
究的结果,结合使用描点法来给出这函数的图象.这是今后要
结合着各种函数逐步深入研究的一个问题.

\begin{ex}
描绘下列函数的图象:
\begin{multicols}{2}
\begin{enumerate}
    \item $y=\frac{1}{2}x^3$
    \item $y=\frac{1}{2}x$
    \item $y=2x+1$
    \item $y=x^2$
    \item $y=\sqrt{x}$
\end{enumerate}
\end{multicols}
\end{ex}

\subsection{正比例函数及其图象}
如果两个变量$x$和$y$之间的相依关系是:当变量$x$依某一
比值变化时,变量$y$按相同比值变化,我们说变量$y$和变量
$x$成正比例变化,或者说变量$y$是变量$x$的\textbf{正比例函数}.

\begin{example}
    我国发射的第一颗人造地球卫星,绕地球一圈平
均运行速度为每秒7.12公里,那么这颗人造地球卫星在2秒钟
内运行了14.24公里,在6秒钟内运行了$7.12\x6=42.72$公
里.如果以公里为单位的路程用$S$表示,以秒为单位的时间
用$t$表示,那么$S$和$t$之间成立下面的等式:
\[S=7.12t\]

容易说明变量$S$和$t$成正比例变化,设$t_1$和$t_2$是时间$t$的任
意两个不等于0的数值,$S_1$和$S_2$是它们的对应值,于是$S_1=
7.12t_1$, $S_2=7.12t_2$, 从而$\frac{S_2}{S_1}=\frac{t_2}{t_1}$
,因此路程$S$是时间$t$的正比例函数.
\end{example}

\begin{example}
    物理学中的虎克定律是“在弹性限度内,弹力跟弹
簧的伸长(或缩短)成正比例”,设弹簧的原长为$\ell_0$(cm), 弹簧
变形后的弹簧长为$\ell$(cm), 当弹簧伸长$\Delta \ell_1=\ell_1-\ell_0$(cm)时,对
应的弹力为$F_1$(kg); 当弹簧伸长$\Delta \ell_2=\ell_2-\ell_0$(cm)时,对应弹
力为$F_2$(kg), 依虎克定律有:
\[\frac{F_2}{F_1}=\frac{\Delta \ell_2}{\Delta \ell_1}\]
也就是:$\frac{F_2}{\Delta \ell_2}=\frac{F_1}{\Delta \ell_1}=k$,这里$k$是常数,在数值上等于弹簧
伸长1cm时对应的弹力数值.因此,弹力$F$与伸长$\Delta \ell$有下面的
关系:
\[F=k\Delta \ell\]
\end{example}

\begin{figure}[htp]
    \centering
\begin{tikzpicture}[>=latex, scale=1.3]
\fill[pattern=north east lines](-1,.25) rectangle  (1,0);
\draw(-1,0)--(1,0);    
\draw [decorate,decoration={zigzag,segment length=8pt}] (0,0)--(0,-1.5);
\draw[dashed](0,-2) circle (.2);
\draw(0,-1.5)--(0,-3);
\draw[dashed](0,-3)--(0,-4);
\draw(0,-3) circle (.2);
\draw[dashed](0,-4) circle (.2);
\draw[dashdotted] (-1,-3)node[left]{$O$}--(1,-3);
\draw (0,-2)--(1,-2);
\draw (0,-4)--(1,-4);
\node at (-.3,-2.75){$m$};
\draw[<->] (-.7,0)--node[fill=white]{$\ell_0$}(-.7,-3);
\draw[<->] (.5,-2)--node[fill=white]{$\Delta \ell$}(.5,-3);
\draw[<->] (.7,-4)--node[fill=white]{$\Delta \ell$}(.7,-3);
\end{tikzpicture}
    \caption{}
\end{figure}
\begin{example}
    让我们考察在弹簧下挂一质量为$m$的小球,如
图4.10, 把小球用力向下拉一段距离后,放开小球就开始振
动.
\begin{itemize}
    \item 如果$\Delta \ell=\ell-\ell_0>0$, $\Delta \ell$是弹
簧伸长量.
\item 如果$\Delta \ell=\ell-\ell_0<0$, $\Delta \ell$是弹
簧压缩量.
\end{itemize}

因为要拉长或压缩弹簧,必
须要有跟拉长或压缩的大小成正
比例(在某一限度内)的力,因此
当小球移到离开平衡位置 $\Delta \ell$的距
离处,这时小球所受恢复力$F$就是:
\[F=-k \Delta \ell\]

式中$k$是弹簧的弹性系数,它在数值上等于弹簧伸长
或压缩单位长度时所产生的弹力.在上式中我们还把在一条
有向直线上的位移的方向和弹力的方向考虑进来,上面等式
中的负号表示力和位移的方向相反.
\end{example}

从上面几个例子得知,正比例函数的一般解析式是:
\[y=kx,\quad  k\ne 0,\quad  x\in(-\infty,+\infty)\]

在函数研究中常用上面的解析式作为正比例函数的定
义.

\begin{blk}{定义}
    函数$y=kx$($k$是不等于零的常数)叫做正比例函
数,这里常数$k$叫做变量$y$对变量的比例系数,它的数值等于自
变量取数值1时,因变量$y$的对应值.
\end{blk}

下面我们来证明正比例函数$y=kx\; (k\ne 0)$的图象是一条
过原点和点$N(1,k)$的直线.

为简单起见,设$k>0$,根据函数图象的定义,要证直线$ON$是$y=kx$的图象,就要证明下面两个命题:
\begin{enumerate}
    \item 在直线$ON$上的每一个点的坐标都满足$y=kx$;
    \item 不在直线$ON$上的任何一点,它的坐标$(x,y)$都不
满足$y=kx$.
\end{enumerate}

我们来证1:
\begin{itemize}
    \item 令$x=0$, 则$y=k\cdot 0=0$,即原点$(0,0)$在函数$y=kx$
的图象上.
\item 令$x=1$,则$y=k\cdot 1=k$,即点$N(1,k)$在$y=kx$的图象上.
\end{itemize}

过原点和$N(1,k)$点作一
条直线(图4.11),现在需要证
明直线$ON$上的每一个点的坐标都满足$y=kx$.

\begin{figure}[htp]
    \centering
\begin{tikzpicture}[>=latex]
    \draw[->](-1.5,0)--(5,0)node[right]{$x$};
    \draw[->] (0,-2)--(0,6)node[right]{$y$};
\node at (.35,-.35){$O$};
\draw[very thick] (-1,-1.5)--(4,6);    
\draw (1,0)node[below]{$N_1$}--(1,1.5)node[left]{$N(1,k)$};
\draw (2,0)node[below]{$P_1$}--(2,3)node[left]{$P(x_0,y_0)$};
\draw (3,0)node[below]{$Q_1$}--(3,4.5)node[right]{$Q_2(x_1,y_2)$};
\node at (3,3)[right]{$Q(x_1,y_1)$};
\draw (3,2.8)[fill=black] circle (1.5pt);
\end{tikzpicture}
    \caption{}
\end{figure}

在直线$ON$上任取一个不同于$O,N$的点$P(x_0,y_0)$, 过$N,P$
两点作直线$N_1N$和$P_1P$平行于$y$轴且和$x$轴分别交于$N_1(1,0)$
和$P_1(x_0,0)$二点,

$\because\quad N_1N\parallel P_1P\qquad \therefore\quad \triangle ON_1N\sim \triangle OP_1P$,故有
\[\frac{|P_1P|}{|OP_1|}=\frac{|N_1N|}{|ON_1|}\]
而$ON_1=1$, $N_1N=k$, 因此
\[\frac{|P_1P|}{|N_1N|}=k\]
又因为$P_1P$, $N_1N$分别是$P$点的纵坐标和横坐标且同号,从而有:
\[\frac{y_0}{x_0}=k\quad \Rightarrow\quad y_0=kx\]
这就证明了$(x_0,y_0)$满足关系$y=kx$.

再来证2:设$Q(x_1,y_1)$点不在直线$ON$上,过$Q$点作直线平行于$y$轴,
交直线$ON$于$Q_2(x_1,y_2)$点,交$x$轴于$Q_1(x_1,0)$点,则$OQ_1=
x_1$. 因为$Q_2$点在直线$ON$上,据1的证明,它的坐标满足$y=
kx$, 即有:
\[\frac{Q_1Q_2}{OQ_1}=\frac{y_2}{x_1}=k\]
另一方面,$Q$点不在直线$ON$上,故$y_2\ne y_1$, 因而
\[\frac{y_1}{x_1}\ne \frac{y_2}{x_1}=k\]
所以 $y_1\ne kx_1$. 这就是说$(x_1,y_1)$不满足$y=kx$.

综合1、2我们得到结论:函数$y=kx$的图象 是过
$O(0,0)$, $N(1,k)$的一条直线.

在$k<0$的情形,用同样的证法得到同样的结论.

函数$y=kx$的图象,以后简称为直线$y=kx$.

现在我们来研究,在比例系数$k$变化的时候,直线$y=kx$
的位置怎样变化.

在画图象时,我们应用上面的结论:正比函数$y=kx$的
图象是通过原点和$N(1,k)$的直线.

对于同一个坐标平面,作函数$y=\frac{1}{2}x$,
$y=x$, $y=2x$
的图象(图4.12),这里三个比例系数都是正的,并且是依
次增加的,从图里可以看出,按照比例系数的增加,函数的图
象渐渐离开$x$轴而接近$y$轴.

对于同一·坐标平面,作函数$y=-\frac{1}{4}x$, $y=-x$, $y=
-3x$的图象(图4.13).这里三个比例系数都是负的,并且
它们的绝对值也是依次增加的.从图里可以看出,按照比例
系数的绝对值增加,函数的图象也渐渐离开$x$轴而接近于$y$轴.
\begin{figure}[htp]\centering
\begin{minipage}[t]{0.48\textwidth}
\centering
\begin{tikzpicture}[>=latex, scale=.8]
    \draw[->](-1.5,0)--(4,0)node[right]{$x$};
    \draw[->] (0,-2)--(0,4)node[right]{$y$};
\draw [thick, domain=-1:3, samples=100] plot(\x, {\x});
\draw [very thick, domain=-.5:2, samples=100] plot(\x, {2*\x});
\draw [ultra thick, domain=-1:3, samples=100] plot(\x, {0.5*\x});
\draw (1,0)node[below]{1}--(1,.1);
\draw (0,1)node[left]{1}--(.1,1);
\node at (2,4) [right]{$y=2x$};
\node at (2.5,2.5) [right]{$y=x$};
\node at (2.4,1) [right]{$y=\frac{1}{2}x$};
\node at (.25,-.25){$O$};
\end{tikzpicture}
\caption{}
\end{minipage}
\begin{minipage}[t]{0.48\textwidth}
\centering
\begin{tikzpicture}[>=latex, scale=.8]
    \draw[->](-3,0)--(3,0)node[right]{$x$};
    \draw[->] (0,-3)--(0,3)node[right]{$y$};
    \node at (-.25,-.25){$O$};
    \draw (1,0)node[below]{1}--(1,.1);
    \draw (0,1)node[left]{1}--(.1,1);
    \draw [thick, domain=-2.5:2.5, samples=100] plot(\x, {-\x});
    \draw [very thick, domain=-3:3, samples=100] plot(\x, {-0.25*\x});
    \draw [ultra thick, domain=-1:1, samples=100] plot(\x, {-3*\x});
    \node at (3,-.75) [right]{$y=-\frac{1}{4}x$};
    \node at (2.5,-2.5) [right]{$y=-x$};
    \node at (1,-3) [right]{$y=-3x$};
\end{tikzpicture}
\caption{}
\end{minipage}
\end{figure}


因此,比例系数$k$和直线$y=kx$与$x$轴正方向所成的角有
关,$k$叫做直线$y=kx$的\textbf{斜率}.

不难证明$k=\tan\alpha,\; 0\le \alpha<180^{\circ},\; \alpha\ne 90^{\circ}$. 这里$\alpha$是直线
向上方向和$x$轴正方向所成的角.

事实上,当$k>0$时,图象为过$O(0,0)$, $N(1,k)$二点,在
一、三象限的直线(图4.14), 那么显见$\tan\alpha=\frac{k}{1}=k$.

当$k<0$时,图象为过$O(0,0)$, $N(1,k)$二点,在二、四象
限的直线(图4.15),这时直线向上方向与$x$轴正的方向所成
的角$\alpha$为钝角.设$\alpha$的补角为$\alpha'$, $k$的相反数$k'=-k>0$, 由
图可见,由于$\triangle ONN_1\simeq \triangle ON'N_1$, $\angle N_1ON'=\angle N_1ON=\alpha'$, $\tan\alpha'=\frac{k'}{1}=k'$, 但
$$\tan\alpha =\tan(180^{\circ}-\alpha')=
-\tan\alpha'=-k'=-(-k)=k$$
 所以$k=\tan\alpha$.

这就是说直线$y=kx$的向上方向与$x$轴正方向所成的角$\alpha$
无论是锐角还是钝角,$\tan\alpha$永远等于比例系数$k$.

\begin{figure}[htp]\centering
\begin{minipage}[t]{0.48\textwidth}
\centering
\begin{tikzpicture}[>=latex, scale=.8]
    \draw[->](-1.5,0)--(4,0)node[right]{$x$};
    \draw[->] (0,-2)--(0,4)node[right]{$y$};
\draw[very thick] (-1,-1)--(3,3)node[right]{$y=kx$};
\draw (2,0)node[below]{1}--(2,.1);
\draw (2,0)--node[right]{$k$}(2,2)node[right]{$N(1,k)$};
\node at (.25,-.25){$O$};
\draw (.5,0) arc (0:45:.5); 
\node at (22.5:.7){$\alpha$};
\end{tikzpicture}
\caption{}
\end{minipage}
\begin{minipage}[t]{0.48\textwidth}
\centering
\begin{tikzpicture}[>=latex, scale=.8]
    \draw[->](-3,0)--(3,0)node[right]{$x$};
    \draw[->] (0,-3)--(0,3)node[right]{$y$};
    \node at (-.25,-.25){$O$};
    \draw[very thick] (0,0)--(-45:3)node[right]{$y=kx$};
    \draw[very thick] (0,0)--(135:3);
\draw [dashed, thick](0,0)--(45:2)node[right]{$N'(1,k')$}--(-45:2);
\draw[thick](1.414,0)--(1.414,-1.414)node[right]{$N(1,k)$};
\node at (1.7,0) [below]{$N_1$};
\node at (2.5,.75){$k'=k$};
\draw (.5,0) arc (0:135:.5);
\draw (.8,0) arc (0:45:.8);
\draw (1,0) arc (0:-45:1);

\node at (22.5:1){$\alpha'$};
\node at (-22.5:1.2){$\alpha'$};
\node at (67.5:.8){$\alpha$};

\end{tikzpicture}
\caption{}
\end{minipage}
\end{figure}

如果函数的值随着自变量的值增加而增加,用算式表示
就是:当$x_1<x_2$时,有不等式$f(x_1)<f(x_2)$, 那么函数$f(x)$称为\textbf{递增函数}.

如果函数的值随着自变量的值增加而减少,即当$x_1<x_2$
时,有$f(x_1)>f(x_2)$, 那么函数$f(x)$称为\textbf{递减函数}.

从函数$y=kx$图象可以看出,当$k>0$时,$y=kx$是递增
函数,当$k<0$时,$y=kx$是递减函数.

\section*{习题4.2}
\addcontentsline{toc}{subsection}{习题4.2}
\begin{enumerate}
    \item   下面两个变量是否成正比例变化,试说明之.
\begin{enumerate}
\item 矩形的一条边长固定,它的面积和另一边的长,比
    例系数得什么值.
    \item 圆周上的弧和它所对圆心角.
    \item 正方形的面积和它的边长.
    \item 定圆内的弦长和这弦所对的弧的度数.
    \item $\log_a N$和$\log_b N$, 这里$N>0$.
\end{enumerate}

    \item 圆周长由公式$C=2\pi R$表示,其中$\pi$是无理常数,$R$是
    圆周半径,圆周长与该圆半径成正比例吗?比例系数等
    于什么?怎样用直径表示圆周长?这情形中比例系数得
    什么值?
    \item 水银注入试管时,管底所受压强$p$与注入水银的深度$h$
    成正比,当5厘米深时一平方厘米所受压力等于68克,
    用公式表示$p$因$n$而变的关系.这种情形中的比例系数
    有什么意义?
    \item 
    假如重量不大于100克,也不小于1克,把它悬在钢丝弹
    簧上,弹簧拉长的距离与所悬重量成正比例,当悬挂20
    (g)重物时,弹簧伸长6(mm)
\begin{enumerate}
    \item 用公式表示重量$p$(g)与弹簧加长量$\ell$(cm)间的关系;
    \item 所得公式可以用于
    怎样大的重量.
\end{enumerate}    

    \item 一物体从静止自由落下,从起点所走距离与经过时间的
    平方成正比.如果物体从开始在30秒内落下4410米,在
    一分钟内落下多少米?如果它的速度和经过的时间成正
    比,在2秒末的速度是每秒19.6米,求在10秒末的速
度:
\item \begin{enumerate}
    \item 在同一个坐标系内作下面函数的图象.
    \[y=\frac{1}{3}x,\qquad     y=x,\qquad y=2\frac{1}{2}x,\qquad     y=-3x\]
    \item 求上面各直线的向上方向与$x$轴正方向所成角的大小.
\end{enumerate} 


\item 把汽油用均匀的速度注入桶里,注入的时间和注入的油
量如下:
\begin{center}
\begin{tabular}{c|cccccccc}
    \hline
    注入的时间$t$(分)&1&2&3&4&5&6&7&8\\
   \hline
注入的油量$q$(升)   &2&4&6&8&10&12&14&16\\
   \hline
\end{tabular}
\end{center}
\begin{enumerate}
    \item 找出$t$的任意两个值的比等于$a$的对应的两个值的
    比;
    \item 找出$q$的任意一个值和对应的$t$的值的比;
    \item 用公式表示$q$和$t$间的函数关系;
    \item $q$和$t$是什么关系?
    \item 作出$q$和$t$间的函数关系的图象.
\end{enumerate}
\end{enumerate}


\subsection{反比例函数及其图象}
如果变量$y$和变量$x$的倒数成正比例变化,那么变量$y$叫
做变量$x$的\textbf{反比例函数}.


\begin{example}
    一个物体作匀速运动,行程120米,则运动速度$v$
(米/秒)与所需时间$t$(秒)之间有关系:
\[v=\frac{120}{t}\]
由这关系推知$v$和$t$成反比例变化,事实上,设$t_1$, $t_2$是变量$t$
的任意两个不等于零的数值,$v_1$和$v_2$分别是它们的对应值,于是
\[\frac{v_2}{v_1}=\frac{\frac{120}{t_2}}{\frac{120}{t_1}}=\frac{\frac{1}{t_2}}{\frac{1}{t_1}}\quad \Rightarrow\quad \frac{v_2}{v_1}=\frac{t_1}{t_2}\]
这就是说,变量$v$和变量$t$的倒数成正比例,因此$v$是$t$的反比例
函数,它的解析式是:
\[v=\frac{120}{t}\qquad (t\ne 0)\]
由上式得到$vt=120$, 或$v_2t_2=v_1t_1=$常数,这也就是说,如
果两个变量成反比例变化,那么这两个变量对应值的乘积恒
等于常数,反过来也对.
\end{example}


\begin{example}
    波义耳定律:当温度不变时,一定质量气体的压
强与它的体积成反比.

令体积$V_1$时的压强是$P_1$, 体积$V_2$时的压强是$P_2$. 波义耳
定律的公式表示是:
\[\frac{P_2}{P_1}=\frac{V_1}{V_2}\]
由此得
\[P_1V_1=P_2V_2,\qquad \text{或}\qquad PV=\text{常量}k\]
从而得
\[P=\frac{k}{V}\qquad (V\ne 0)\]

从上面的例子,我们知道反比例函数的一般解析式是:
\[y=\frac{k}{x},\qquad (x\ne 0,\quad k\ne 0)\]

在函数研究中,常用上面的解析式作为反比例函数的定
义.
\end{example}

\begin{blk}{定义}
函数$y=\frac{k}{x}\; (x\ne 0)$叫做反比例函数,这里$k$是
不等于零的常数.
\end{blk}

下面我们来研究函数$y=\frac{k}{x}$的图象.
先来讨论$k$是正数的情形,例如$k=6$.

任意取自变量$x$的一些值(除零以外),算出$y$的对应值,
列表如下:
\begin{center}
\begin{tabular}{c|cccccccccccccccc}
\hline
$x$ & $-8$   &    $-7$   &    $-6$   &    $-5$   &    $-4$   &    $-3$   &    $-2$   &    $-1$   &    $1$   &    $2$   &    $3$   &    $4$   &    $5$   &    $6$   &    $7$   &    $8$\\
\hline
$y$ &$-\frac{3}{4}$&$-\frac{6}{7}$&$-1$&$-1\frac{1}{5}$&$-1\frac{1}{2}$&$-2$&$-3$&$-6$&6&3&2&$1\frac{1}{2}$&$1\frac{1}{5}$&1&$\frac{6}{7}$&$\frac{3}{4}$\\
\hline
\end{tabular}
\end{center}

\begin{figure}[htp]
    \centering
\begin{tikzpicture}[>=latex,scale=.4]
\draw[->] (-7,0)--(7,0)node[right]{$x$};
\draw[->] (0,-7)--(0,7)node[right]{$y$};
\foreach \x in {-6,-5,...,-1,1,2,...,6}
{
    \draw (\x,0)node[above]{$\x$}--(\x,-.1);
    \draw (0,\x)node[right]{$\x$}--(-.1,\x);
}
\draw [domain=-6:-1, samples=100, very thick] plot(\x, {6/\x});
\draw [domain=1:6, samples=100, very thick] plot(\x, {6/\x});
\node at (-.5,-.5){$O$};

\end{tikzpicture}
    \caption{}
\end{figure}

用表里各组对应值作为点
的坐标,在坐标平面上标出各
个点(图4.16).很明显,横坐标
是正的点都在第一象限,横坐
标是负的点都在第三象限(因
为纵坐标与横坐标符号相同),
并且图象与$x$轴和$y$轴都不相交
(因为不可以取$x=0$, 并且不
论取$x$等于什么值,$y$的值都不
会是0).



顺次连结第一象限里的各个点,得到图象的一个分支,
顺次连结第三象限里的各个点,得到图象的另一个分支,这
两个分支合起来就是函数$y=\frac{6}{x}$
图象.

现在我们再来研究$y=\frac{6}{x}$
的图象的一些特点.

当$x$的绝对值逐渐扩大的时候,$y$的绝对值逐渐缩小,但
是不论取$x$等于什么值,$y$的值都不能是零,因此,$y=\frac{6}{x}$的
图象向右和向左都逐渐接近于$x$轴,但是无论什么时候也不
能达到$x$轴.

同样当$x$的绝对值逐渐缩小的时候,$y$的绝对值逐渐扩大,
但是$x$的值不能是零.因此$y=\frac{6}{x}$
的图象向上和向下都逐渐
接近于$y$轴,但是无论什么时候也不能达到$y$轴.

如果$k$是负数,函数$y=\frac{k}{x}$
的图象也有同样的特点,不过
这时图象一个分支在第二象限,另一个分支在第四象限(图
4.17).

$y=\frac{k}{x}$的图象叫做双曲线.

我们也容易看出:如果$k>0$, 它在区间$(-\infty,0)$或$(0,
+\infty)$上是递减函数;如果$k<0$, 它在区间$(-\infty,0)$或$(0,
+\infty)$上是递增函数.

\begin{figure}[htp]
    \centering
\begin{tikzpicture}[>=latex,scale=.4]
\draw[->] (-7,0)--(7,0)node[right]{$x$};
\draw[->] (0,-7)--(0,7)node[right]{$y$};
\foreach \x in {-6,-5,...,-1,1,2,...,6}
{
    \draw (\x,0)node[below]{$\x$}--(\x,.1);
    \draw (0,\x)node[left]{$\x$}--(.1,\x);
}
\draw [domain=-6.5:-.75, samples=100, very thick] plot(\x, {-5/\x});
\draw [domain=.75:6.5, samples=100, very thick] plot(\x, {-5/\x});
\node at (.5,.5){$O$};

\end{tikzpicture}
    \caption{}
\end{figure}

\begin{ex}
\begin{enumerate}
    \item 下列各种关系里,哪些是正比例关系?哪些是反比例关
    系,哪些都不是,为什么?
    \begin{enumerate}
    \item 完成一定工作的时间和人数(假定每人的工作能力
    相同).
    \item 面积一定的时候,菱形的两条对角线的长.
    \item 被除数相同的时候,除数和商.
    \item 在机车牵引力不变时,消耗的功率与速度.
    \item 在速度维持不变时,消耗的功率与机车牵引力.
    \item 当机车的功率到达定额时,牵引力与速度.
    \item 重量一定的时候,物体的比重和体积.
    \item 体积一定的时候,物体的重量和比重.
    \item 定圆内的弦长和弦到圆心的距离.
    \item 度量一段距离,单位长度与量数.        
    \end{enumerate}

    \item 对于同一个坐标平面,作下列各函数的图象:
    \[y=\frac{8}{x},\qquad y=\frac{4}{x},\qquad y=\frac{2}{x}\]
    \item  对于同一个坐标平面,作下列各函数的图象:
    \[y=\frac{5}{x},\qquad y=-\frac{5}{x}\]
    \item 一工作需要$x-1$个人,用$x+1$天完成,另一工作需要
    $x+2$个人,用$x-1$天完成,且知前一个工作量与后一
    个工作量的比是$9:10$, 求$x$.
\item 从装满纯酒的桶中倒出9升酒后,用水把桶灌满,然后
又倒出9升混合溶液,再用水把桶灌满,这时桶内酒的
体积和水的体积的比是$16:9$, 问桶的容量是多少?
\end{enumerate}
\end{ex}

\section{一次函数(线性函数)}
\subsection{一次函数及其图象}
一次函数的实际例子很多,例如:


\begin{example}
一根长10厘米的弹簧,挂的重量每增加1公斤,
就拉长$\frac{1}{2}$
厘米,那么在弹性限度内,弹簧长度$y$(厘米)与挂
的重量$x$(公斤)之间的函数关系就是:
\begin{equation}
    y=\frac{1}{2}x+10
\end{equation}
\end{example}


\begin{example}
    从北京到广州的包裹邮费为每公斤0.9元,每件另
加手续费0.2元,那么总邮费$y$(元)与包裹重量$x$(公斤)之间
的函数关系就是:
\begin{equation}
  y=0.9x+0.2  
\end{equation}
\end{example}


\begin{example}
    匀速运动中物体离一个定点$O$的距离$S$是用公式:
\[S=vt+S_0\]
表示的,这里$S_0$是物体开始运动的时候离开定点$O$的距离,
$v$是物体运动的速度,它们都是常量,$t$是物体运动的时间,它是自变量,$S$是$t$的一次函数(图4.18).
\begin{figure}[htp]
    \begin{center}
    \begin{tikzpicture}[>=latex]
    \draw[->] (0,0)--(9,0)node[right]{$S$};
    \draw (0,0)node[below]{$O$}--(0,1);
    \draw (3,0)node[below]{$A$}--(3,.5);
    \draw (8,0)node[below]{$B$}--(8,1);
    \draw [<->] (0,.3)--node[fill=white]{$s_0$}(3,.3);
    \draw [<->] (8,.3)--node[fill=white]{$vt$}(3,.3);
    \draw [<->] (0,.8)--node[fill=white]{$vt+s_0$}(8,.8);
    \end{tikzpicture}
    \end{center}
    \caption{}
\end{figure}
\end{example}

这三个例子的函数形式都是:
\[y=kx+b\]
其中,$k,b$是两个常量,$k\ne 0$, 这种形式的函数称为\textbf{一次
函数}(或线性函数).其定义域是$(-\infty,+\infty)$, 值域也是
$(-\infty,+\infty)$.

如果$b=0$, 那么$y=kx+b$就成为$y=kx$, 所以正比例
函数是一次函数的特例.

下面讲一次函数的图象是一条直线.

因为正比例函数是一次函数的特例,所以我们可以通过
正比例函数的图象来认识一次函数的图象,并由此认识一般
的图象的平移原理.

为简单起见,假设$b>0$, 从解析式$y=kx+b$和$y=kx$明显
地看出,对于自变量的相同值,一次函数的对应值$y=kx+b$,
总可以由正比例函数$y=kx$的对应值加上$b$得到,这表示
$y=kx+b$的图象上的一切点比直线$y=kx$上具有相同横坐标
的点高出$b$个单位,因此,$x=kx+b$的图象可以由直线沿着$y$
轴向上平移$b$个单位得到,为此在$y$轴上截取线段$OA=b$, 过
$A$点作直线$\ell'$平行于直线$\ell:\; y=kx$ (图4.19).

\begin{figure}[htp]
    \centering
\begin{tikzpicture}[>=latex]
\draw[->] (-4,0)--(4,0)node[right]{$x$};
\draw[->] (0,-2)--(0,5)node[right]{$y$};
\node at (.25,-.25){$O$};

\draw [domain =-3:4, samples=10, thick]plot(\x,{.5*\x});
\draw [domain =-4:3.8, samples=10, very thick]plot(\x,{.5*\x+1.7});
\foreach \x/\xtext in {1.2/Q_2,3/P_2}
{
    \draw (\x,0)node[below]{$\xtext$}--(\x, 4.5);
}

\draw (-.5,1.7)--(0,1.7);
\draw[<->](-.25,1.7)--node[fill=white]{$b$} (-.25,0);
\node at (0,2)[left]{$A$};
\node at (1.2,1)[right]{$Q_1$};
\node at (1.2,2.7)[right]{$Q_0$};
\node at (3,1.3)[right]{$P_1$};
\node at (3,3)[right]{$P(x_0,y_0)$};

\node at (4,2)[right]{$\ell:\; y=kx$};
\node at (4,3.7)[right]{$\ell':\; y=kx+b$};

\draw (.5,0) arc (0:26.56:.5)node[right]{$\alpha$};
\draw (-3.4+.5,0) arc (0:26.56:.5)node[right]{$\alpha$};

\node at (1.2,4)[right]{$Q(x_1,y_1)$};


\end{tikzpicture}
    \caption{}
\end{figure}

下面我们来证明,这条直线$\ell'$就是$y=kx+b$的图象.

先证直线$\ell'$上的每个点,它的坐标都适合关系式
\[y=kx+b\]

设$P(x_0,y_0)$是$\ell'$上任意一点,过$P$作$x$轴垂线交直线$\ell$于
$P_1$, 交$x$轴于$P_2$, 因$\ell'\parallel \ell$,在$\ell$和$\ell'$之间的所有与$y$轴平行的线
段都等于$b$, 所以$P_1P=b$, 又因为$P_1$在$\ell$上,所以$P_2P_1=kx$.
\[\therefore\quad y_0=P_2P_1+P_1P=kx_0+b\]
这就是说$(x_0,y_0)$适合关系式$y=kx+b$.

再证不在直线$\ell'$上的每个点,它的坐标都不适合关系式
$y=kx+b$. 为此,在$\ell'$外面任取一点$Q(x_1,y_1)$, 过$Q$作$x$轴垂
线交$\ell'$于$Q$, 交$\ell$于$Q_1$, 交$x$轴于$Q_2$, 于是,
\[y_1=Q_2Q_1+Q_1Q=kx_1+Q_1Q\]
因为$Q$在$\ell'$外面,$Q_1Q\ne Q_1Q_0=b$, 所以
\[y_1\ne kx_1+b\]
这就是说,$(x_1,y_1)$不适合关系式$y=x+b$.

综合上面两方面的证明,我们证明了:$\ell'$是函数$y=kx+b$
的图象.

如果$b<0$, 则$y=kx+b$的图象,由直线$y=kx$沿 着$y$轴
向下平移$|b|$个单位得到.

一般结论:一次函数$y=kx+b$的图象是一条直线,它是
由正比例函数$y=kx$的图象沿$y$轴平移而来的;若$b>0$, 则
向上($y$轴正向)移动$b$个单位;若$b<0$, 则向下($y$轴负向)移
动$|b|$个单位.

因为直线$y=kx+b$平行于直线$y=kx$, 所以直线
$y=kx+b$向上的方向和$x$轴正方向所成的角,等于直线
$y=kx$向上的方向和$x$轴正方向组成的角(图4.19),由此可
知,这个角$\alpha$和$k$有关,$k$叫直线$y=kx+b$的\textbf{斜率}.据前面讨
论我们知道,$k=\tan \alpha$.

当$x=0$时,$y=b$, 这就是说$b$表示直线和$y$轴交点的纵
坐标,所以$b$叫做直线$y=kx+b$在$y$轴上的\textbf{截距}.

这样,我们可以说:一次函数$y=kx+b$的图象是一条具
有斜率$k$, 而在$y$轴上的截距为$b$的直线.

\begin{example}
作以下函数的图象
\[y=2x-1,\qquad y=\dfrac{1}{2}x+3\]
\end{example}

\begin{solution}
因为一次函数的图象是直线,只要定出直线上两个
点,就可画出直线,所以对于每个函数,我们只求两个点.
\begin{enumerate}
    \item 在$y=2x-1$中,
    \begin{itemize}
        \item 令$x=0$, 得$y=1$, 得到一点$A(0,-1)$;
        \item 令$x=2$, 得$y=3$, 得到另一点$B(2,3)$.
    \end{itemize}
    \item 在$y=\frac{1}{2}x+3$中,
\begin{itemize}
    \item 令$x=0$, 得$y=3$, 得一点$C(0,3)$;
    \item 令$x=-6$, 得$y=0$, 得一点$D(-6,0)$.
\end{itemize}
分别联$A,B$和$C,D$得所求直线(图4.20).
\end{enumerate}

\begin{figure}[htp]
    \centering
\begin{tikzpicture}[>=latex, scale=.7]
    \draw[->] (-7,0)--(5,0)node[right]{$x$};
    \draw[->] (0,-4)--(0,5)node[right]{$y$};
    \draw [domain =-1.5:3, samples=10, thick]plot(\x,{2*\x-1});
\draw [domain =-7:4, samples=10, very thick]plot(\x,{.5*\x+3});

\foreach \x in {-6,-5,...,-1,1,2,...,4}
{
    \draw (\x,0)node[below]{$\x$}--(\x,.2);
}
\foreach \x in {-3,-2,-1,1,2,3,4}
{
    \draw (0,\x)node[left]{$\x$}--(.2,\x);
}
\node at (.3,.3){$O$};
\draw (0,-1) [fill=black] circle (2pt)node[right]{$A$};
\draw (2,3) [fill=black] circle (2pt)node[right]{$B$};
\draw (0,3) [fill=black] circle (2pt)node[right]{$C$};
\draw (-6,0) [fill=black] circle (2pt)node[above]{$D$};
\end{tikzpicture}    
    \caption{}
\end{figure}
\end{solution}

\begin{ex}
\begin{enumerate}
    \item \begin{enumerate}
        \item 对于同一坐标系画出下列直线:
\[y=\frac{1}{2}x+4,\quad y=\frac{1}{2}-4,\quad y=-\frac{1}{2}x+4,\quad y=-\frac{1}{2}-4\]
\item 求每一条直线的斜率和在$y$轴上的截距.
\item 设$y=3$, 求各式中$x$的对应值,再用图象检验所得
结果.
    \end{enumerate}
    \item 作下列各直线:
\begin{multicols}{2}
    \begin{enumerate}
        \item $y=\frac{2}{3}x-5$
        \item $y=-3x+5$
        \item $y=-\frac{1}{2}x-\frac{9}{2}$
        \item $y=\frac{3}{4}x$
        \item $3x-5y=15$
        \item $x+2y+6=0$
    \end{enumerate}
\end{multicols}

\item \begin{enumerate}
    \item 已知一次函数$y=kx+b$在$x=-4$时,$y=9$, 在$x=6$
时,$y=3$, 求$k$和$b$.
\item 已知直线$y=kx+b$经过$(-4,9)$和$(6,3)$, 求$k$和$b$.
\end{enumerate}

\item 已知$y+b$与$x+a$成正比,$a,b$是常数,求证$y$是$x$的一次
函数,如果$x=3$时,$y=5$; $x=2$时,$y=2$, 求出表
示$y$是$x$的函数的解析式.
\item 直线$y=k_1x+b_1$和$y=k_2x+b_2$相交于$x$轴上同一点
$A(a,0)$, 求证在两直线上横坐标相同的点的纵坐标成正
比例变化.
\end{enumerate}
\end{ex}

\subsection{一次函数的性质}
我们在前面的内容中,曾直观地描述过函数的递增性和递减
性,当时是这样说的:“如果函数的值随着自变量的值增加而
增加,那么这个函数叫做\textbf{递增函数};如果函数的值随着自
变量的值增加而减少,那么这个函数叫做\textbf{递减函数}”.

现在让我们用较精确的语言,再来描述一下这个概念.

如果对于开区间$(a,b)$(或者闭区间$[a,b]$)上的任意两
个自变量的值$x_1$和$x_2$, 若
$x_1<x_2$,
则
$f(x_1)<f(x_2)$, 
那么函数$f(x)$称为在开区间$(a,b)$(或者闭区间$[a,b]$)上的
递增函数(图4.21).
\begin{figure}[htp]
    \centering
\begin{tikzpicture}[>=latex]
\begin{scope}
\draw[->](-2,0)--(2,0)node[right]{$x$};  
\draw[->](0,-1)--(0,3)node[right]{$y$};

\draw (-1.5,.5) to [bend left=15] (0,1.5) to [bend right=15] (1.5,2.5);
\draw (-1.5,.5)--(-1.5,0)node[below]{$a$};
\draw (1.5,2.5)--(1.5,0)node[below]{$b$};

\draw (-1.5,.5) [fill=white]circle(1.5pt) ;
\draw (1.5,2.5) [fill=white]circle(1.5pt) ;

\draw (-.8,1.1)node[above]{$f(x_1)$}--(-.8,0)node[below]{$x_1$};
\draw (.8,1.85)node[above]{$f(x_2)$}--(.8,0)node[below]{$x_2$};
\node at (-.25,-.25){$O$};
\end{scope}
\begin{scope}[xshift=5cm]
    \draw[->](-1,0)--(3,0)node[right]{$x$};  
\draw[->](0,-1)--(0,3)node[right]{$y$};
\draw (.5,-.8) to [bend left=-25] (2.5,2.5);
\draw (.5,-.8)--(.5,0)node[above]{$a$};
\draw (2.5,2.5)--(2.5,0)node[below]{$b$};
\node at (-.25,-.25){$O$};


\draw (.8,-.6)node[right]{$f(x_1)$}--(.8,0)node[above]{$x_1$};
\draw (2,.8)node[left]{$f(x_2)$}--(2,0)node[below]{$x_2$};
\draw (.5,-.8) [fill=white]circle(1.5pt) ;
\draw (2.5,2.5) [fill=white]circle(1.5pt) ;
\end{scope}    
\end{tikzpicture} 
\caption{}
\end{figure}


如果对于开区间$(a,b)$(或者闭区间$[a,b]$)上的任意两
个自变量的值$x_1$和$x_2$,若
$x_1<x_2$,
则
$f(x_1)>f(x_2)$,
那么函数$f(x)$称为在开区间$(a,b)$(或者闭区间$[a,b]$)上的
递减函数(图4.22).

\begin{figure}[htp]
    \centering
\begin{tikzpicture}[>=latex]
\begin{scope}
    \draw[->](-2,0)--(2,0)node[right]{$x$};  
\draw[->](0,-1)--(0,3)node[right]{$y$};
\node at (-.25,-.25){$O$};
\draw (-1.5,2.5) to [bend left=-15] (1.5,.5);
\draw (-1.5,2.5)--(-1.5,0)node[below]{$a$};
\draw (1.5,.5)--(1.5,0)node[below]{$b$};

\draw (-.8,1.8)node[above]{$f(x_1)$}--(-.8,0)node[below]{$x_1$};
\draw (.8,.75)node[above]{$f(x_2)$}--(.8,0)node[below]{$x_2$};




\draw (-1.5,2.5) [fill=white]circle(1.5pt) ;
\draw (1.5,.5) [fill=white]circle(1.5pt) ;
\end{scope}
\begin{scope}[xshift=6cm]
    \draw[->](-2,0)--(2,0)node[right]{$x$};  
    \draw[->](0,-1)--(0,3)node[right]{$y$};
    \node at (-.25,-.25){$O$};
    \draw (-1.5,2) to [bend left=15] (1.5,-.5);
    \draw (-1.5,2)--(-1.5,0)node[below]{$a$};
    \draw (1.5,-.5)--(1.5,0)node[above]{$b$};
    \draw (-.8,1.65)node[above]{$f(x_1)$}--(-.8,0)node[below]{$x_1$};
    \draw (.8,.4)node[above]{$f(x_2)$}--(.8,0)node[below]{$x_2$};
    

    \draw (-1.5,2) [fill=white]circle(1.5pt) ;
    \draw (1.5,-.5) [fill=white]circle(1.5pt) ;  
\end{scope}    
\end{tikzpicture} 
\caption{}
\end{figure}

下面我们来阐明一次函数$y=kx+b$的增减性,当$k>0$
时,一次函数$y=kx+b$在定义域上是递增的;当$k<0$时,
一次函数$y=kx+b$在定义域上是递减的.现在证明上述结
论:

在定义域中任取两数$x_1,x_2$且$x_1<x_2$, 那么
\[y_2-y_1=(kx_2+b)-(kx_1+b)=k(x_2-x_1)\]
因$x_1<x_2$, 即$x_2-x_1>0$, 故:
\begin{itemize}
    \item 若$k>0$, 则$k(x_2-x_1)>0$, 即$y_2>y_1$, 所以$y=kx+b$
是递增的.
\item 若$k<0$, 则$k(x_1-x_2)<0$, 即$y_1<y_2$, 所以$y=kx+b$
是递减的.
\end{itemize}

下面我们来研究一次函数的变化率:

设函数$y=kx+b$的自变量从$x_1$变到$x_2$, 因变量则从$y_1$变
到$y_2$. 设$\Delta x=x_2-x_1$为自变量的改变量,$\Delta y=y_2-y_1$为因变
量的改变量.我们称函数改变量与自变量改变量的比
$\frac{\Delta y}{\Delta x}$
为函数$y$对于自变量$x$的\textbf{平均变化率}.

由$y_1=kx_1+b$和$y_2=kx_2+b$得:$y_2-y_1=k(x_2-x_1)$,即:
\[\Delta y=k\Delta x\]
这表示一次函数的改变量与自变量的改变量成正比例变化,
比例常数$k$就是一次函数的斜率,因此
\[\frac{\Delta y}{\Delta x}=k\]
即一次函数的变化率总是常数,等于一次函数的斜率$k$(图
4.23).

\begin{figure}[htp]
    \centering
\begin{tikzpicture}[>=latex, xscale=.8, yscale=1]
\begin{scope}
    \draw[->](-2,0)--(4,0)node[right]{$x$};  
    \draw[->](0,-1)--(0,4)node[right]{$y$};
    \node at (-.25,-.25){$O$};
\draw [domain=-2:4, samples=10, very thick] plot(\x, {0.5*\x+1.5});
\draw[dashed](-1,1)node[below]{$(x_1,y_1)$}--node[above]{$\Delta x$}(3,1)node[below]{$(x_2,y_1)$}--node[right]{$\Delta y$}(3,3)node[right]{$(x_2,y_2)$};
\node at (0,2.5){$y=kx+b\; (k>0)$};
\node at (1,-1.5){$\frac{\Delta y}{\Delta x}=\frac{y_2-y_1}{x_2-x_1}=k>0$};

\end{scope}
\begin{scope}[xshift=9.5cm]
    \draw[->](-3,0)--(3,0)node[right]{$x$};  
    \draw[->](0,-1)--(0,4)node[right]{$y$};
    \node at (-.25,-.25){$O$};
    \draw [domain=-3:2.5, samples=10, very thick] plot(\x, {-0.5*\x+1.5});
    \draw[dashed](-2,2.5)node[right]{$(x_1,y_1)$}--node[left]{$\Delta y$}(-2,.5)node[below]{$(x_1,y_2)$}--node[above]{$\Delta x$}(2,.5)node[right]{$(x_2,y_2)$};
    \node at (0,2.5)[right]{$y=kx+b\; (k<0)$};
    \node at (0,-1.5){$\frac{\Delta y}{\Delta x}=\frac{y_2-y_1}{x_2-x_1}=k<0$};
\end{scope}
\end{tikzpicture}    
    \caption{}
\end{figure}

变化率为常数说明:
\begin{enumerate}
    \item 不管$x_1,x_2$在区间的什么位置,
即$\Delta x$不管在何处产生,变化率$\frac{\Delta y}{\Delta x}$
是一样的;
\item 变化时时
刻刻在发生,不管$\Delta x$多小,变化率也是一样的,这就表明,
函数$y$随着$x$而均匀变化.
\end{enumerate}

反过来看,如果一个函数的平均变化率等于一个常量$k$, 
那么这个函数是一次函数.

事实上,任意取定函数$y=f(x)$的一对对应值$x_1$和$y_1=
f(x_1)$, 设$\Delta x=x-x_1$, $\Delta y=y-y_1$, 因为函数$y=f(x)$的变
化率等于常数$k$, 因此有
\[\frac{\Delta y}{\Delta x}=k\]
即
\[\frac{y-y_1}{x-x_1}=k\]
化简得
\[y=kx-kx_1+y_1\]
也即:$y=kx+b$, 这里$b=-kx_1+y_1$(常数).

因此函数$y=f(x)$是$x$的一次函数.    

\begin{example}
    一定量的酒精,在温度$t=0^{\circ}{\rm C}$时,体积$V=5.25$升,实验表明,温度每升高$10^{\circ}{\rm C}$, 体积增大0.06升,求体积$V$与温度$t$的变化关系.
\end{example}    

\begin{solution}
因为这里体积的增量与温度的增量成正比:
$\frac{\Delta V}{\Delta t}= 0.006$, 所以$V$与$t$的关系可表为一次函数$V=0.006t+b$, 只
    需要再确定常数$b$.

    因为$t=0$时,$V=5.25$, 所以有
   \[ 5.25=0.006\x 0+b\quad \Rightarrow\quad b=5.25\]
   故所求函数关系是$V=0.006t+5.25$.
\end{solution}


\section*{习题4.3}
\addcontentsline{toc}{subsection}{习题4.3}
\begin{enumerate}
\item 试证正比例函数$y=kx$, 当$k>0$时,在定义域上是递
增的,当$k<0$时,在定义域上是递减的.

\item 试证反比例函数$y=\frac{k}{x}$,
当$k>0$时,在$(0,+\infty)$上是
递减的,当$k<0$时,在$(0,+\infty)$上是递增的.
\item  \begin{enumerate}
    \item 对于同一坐标系,作函数$y=3x+2$和$y=-3x+2$
的图象.
\item 所得的两个图象有哪些相同的地方?有哪些不同的
地方?
\item 求证:$x$的值每次增加一个相同的数,函数$y=3x+2$
的值也每次增加一个相同的数,而函数$y=-3x-2$的
值却每次减少一个相同的数.
\item 求证:函数$y=3x+2$所增加的量和对应的自变量
$x$所增加的量的比等于3, 而对于函数$y=-3x+2$,
这个比等于$-3$.
\end{enumerate} 

\item 
在匀速运动中,某物体所行路程与行动时间的关系由
关系式$s=2t+40$表示.$t$与$s$的大小是成正比例吗?如
果$s$的单位是米,$t$的单位是秒,问每经过一秒这物体就
前进多少米?
\item 
中夜的气温是$5^{\circ}$C,到明天下午2时以前气温按每小时$0.2^{\circ}$C均匀地升高,用公式表示气温$y$与时间$x$的函数关系,写出这函数的定义域.

\item 弹簧的原长3厘米,在由0.5(kg)到3.5(kg)的载重限
度内,每加重一千克弹簧伸长0.5毫米,用公式表示弹
簧长度与载重量间的函数关系.这函数的定义域是什么?
\item 钢杆由于温度上升$1^{\circ}$C所增加的长度和$0^{\circ}$C时的长度比
等于0.000011每度,用公式表示钢杆在$t^{\circ}$C时的长度与温
度$t$的函数关系.
\item 钢桥在$0^{\circ}$C时的长度是400米.求温度从$-20^{\circ}$C上升到
$40^{\circ}$C时钢桥的长度变化了多少?
\item 当压强不变时,一切气体的体积由于温度上升$1^{\circ}$C所增加的体积和$0^{\circ}$C时的气体的体积的比大致是相等的,并等于$\frac{1}{273}$每度.
这就是盖吕萨克定律,写出在$t^{\circ}$C时的体积$v$和温度$t$的函数关系.

\end{enumerate}

\subsection{方程$ax+by+c=0$的图象}
在前面的内容中,我们已经知道函数$y=kx+b\; (k\ne 0)$的图象
是一条直线,但是无论$k$和$b$取什么数值,一次函数的图象仅
是坐标平面上直线集合中的一部分,因为$k\ne 0$, 所以直线
$y=kx+b\; (k\ne 0)$, 不能表示$x$轴以及和$x$轴平行的任何直线.
又当直线向上方向与$x$轴正方向所成角$\alpha=90^{\circ}$时,$k=\tan\alpha$不存
在,这就是说直线$y=kx+b\; (k\ne 0)$, 不能表示$y$轴以及和$y$
轴平行的直线.我们把函数关系$y=kx+b$, 也可以看做是关
于$x$和$y$的一次方程
\[kx-y+b=0\]
这样,我们也就可以知道二元一次方程
\[kx-y+b=0\]
的图象是一条直线.

现在,我们来研究一般的情况,证明任何一个二元一次
方程
\[ax+by+c=0\qquad  \text{($a,b$不同时为零)}\]
的图象都是直线.

根据$a,b$可取值的条件,可以看出要证明这个结论应该
分三种情况,就是:
\begin{itemize}
    \item $a\ne 0,\quad b\ne 0$
    \item $a=0,\quad b\ne 0$
    \item $a\ne 0,\quad  b=0$
\end{itemize}

\begin{enumerate}
    \item $a\ne 0,b\ne 0$, 这时方程可以化为:
    \begin{equation}
        y=-\frac{a}{b}x-\frac{c}{b}
    \end{equation}
这是$x$的一次函数,我们已经知道它的图象是一条直线,这条
直线的斜率是$-\frac{a}{b}$,
在$y$轴上的截距是$-\frac{c}{b}$.如果$c=0$, 那
么这条直线就经过原点.

\item $a=0,b\ne 0$, 这时方程可化为:
\begin{equation}
    y=0x-\frac{c}{b}
\end{equation}
这里可以看到,不论变量$x$取什么实数值,和它对应的$y$的值
总等于$-\frac{c}{b}$.
所以它的图象是平行于$x$轴,并且和$x$轴的距离等
于$-\frac{c}{b}$的直线.当$-\frac{c}{b}>0$时,直线在$x$轴上方;当$-\frac{c}{b}<0$时,直线在$x$轴下方,特别,当$c=0$时,$-\frac{c}{b}=0$, 这时
$y=0$, 它的图象就是$x$轴(图4.24).

\item $a\ne 0,b=0$, 这时方程可化为:
\begin{equation}
    x=0\cdot y-\frac{c}{a}
\end{equation}

可以看到,不论变量$y$取什么数值,和它对应的$x$的值总
等于$-\frac{c}{a}$.
所以它的图象是平行$y$轴,并且和$y$轴的距离是
$-\frac{c}{a}$的一条直线.当$-\frac{c}{a}>0$
时,直线在$y$轴的右边,当
$-\frac{c}{a}<0$时,直线在$y$轴的左边,特别,当$c=0$时,$-\frac{c}{a}=0$,
这时$x=0$, 它的图象就是$y$轴(图4.25).

\begin{figure}[htp]\centering
    \begin{minipage}[t]{0.48\textwidth}
    \centering
    \begin{tikzpicture}[>=latex, scale=.7]
        \draw[->](-3,0)--(3,0)node[right]{$x$};
        \draw[->] (0,-3)--(0,3)node[right]{$y$};
    \draw[very thick] (-3,1.5)--(2,1.5)node[right]{$y=-\frac{c}{b}>0$};
    \draw[very thick] (-3,-1.5)--(2,-1.5)node[right]{$y=-\frac{c}{b}<0$};
    \node at (-2,0)[above]{$y=0$};
    \node at (.25,-.25){$O$};
    \end{tikzpicture}
    \caption{}
    \end{minipage}
    \begin{minipage}[t]{0.48\textwidth}
    \centering
    \begin{tikzpicture}[>=latex, scale=.7]
        \draw[->](-3,0)--(3,0)node[right]{$x$};
        \draw[->] (0,-3)--(0,3)node[right]{$y$};
        \node at (.25,-.25){$O$};
        \draw[very thick] (1.5,-3)--(1.5,3)node[right]{$x=-\frac{c}{a}>0$};
        \draw[very thick] (-1.5,-3)--(-1.5,3)node[left]{$x=-\frac{c}{a}<0$};
\node at  (0,-3)[fill=white]{$x=0$};
    \end{tikzpicture}
    \caption{}
    \end{minipage}
    \end{figure}
\end{enumerate}

总结上面这三种情况,我们得到:
方程$ax+by+c=0$ ($a,b$不同时等于零)的图象是一条直
线.

以后我们把这个图象,简称为直线$ax+by+c=0
$.



\begin{example}
    在同一坐标系里作以下方程的图象:
\[x-y+2=0,\qquad 2x+y+1=0\]
\end{example}

\begin{solution}
列表:
\begin{multicols}{2}
    \begin{center}
        \begin{tabular}{c|cc}
           \hline
            $x$&0&$-2$\\
            \hline
            $y$&2&0\\
            \hline
        \end{tabular}
    \end{center}
    \begin{center}
        \begin{tabular}{c|cc}
            \hline
            $x$&0&$-\frac{1}{2}$\\
            \hline
            $y$&$-1$&0\\
            \hline
        \end{tabular}
    \end{center}
\end{multicols}
图象如图4.26

\begin{figure}[htp]
    \centering
\begin{tikzpicture}[>=latex]
\draw[->] (-3,0)--(3,0)node[right]{$x$};
\draw[->] (0,-2)--(0,3)node[right]{$y$};
\node at (.25,-.25){$O$};    
\draw [domain=-2.5:1, samples=10, thick]plot(\x, {\x+2});
\draw [domain=-2:.5, samples=10, thick]plot(\x, {-1-2*\x});
\node at (-2,3)[fill=white]{$2x+y+1=0$};
\node at (1,3)[fill=white, right]{$x-y+2=0$};
\node at (0,2)[right]{$A$};\node at (0,-1)[right]{$D$};
\node at (-2,0)[above]{$B$};
\node at (-.5,0)[above]{$C$};
\node at (-1,1)[above]{$P$};

\end{tikzpicture}
    \caption{}
\end{figure}

如果在同一坐标系要画出二元一次方程组里两个方程的
图象,就可以根据图象求出方程组的解.这种解法叫做二元
一次方程组的图象解法.现在举例说明如下:
\end{solution}

\begin{example}
    用图象法解方程组:
    \begin{numcases}{}
        x-y+2=0\\
        2x+y+1=0
    \end{numcases}
\end{example}

\begin{solution}
在例4.17里我们已经画过这两个方程的图象,它们是
两条相交直线$AB$和$CD$(图4.26),这个交点$P$的坐标是$x=-1$, $y=1$, 它就是所求方程组的解.

这是因为$P$点既在方程(4.6)的图象$AB$上,又在方程(4.7)
的图象$CD$上,所以它的坐标$x=-1$, $y=1$, 既适合方程
(4.6)又适合方程(4.7),因此$x=-1$, $y=1$是方程组的解.

反过来,方程组的解,必须同时适合方程(4.6)和(4.7),
所以表示它的点必须既在方程(4.6)的图象$AB$上,又在方程
(4.7)的图象$CD$上,因此它必须是$AB$和$CD$的交点$P$.这就是
说,除去$x=-1$, $y=1$以外,方程组不再有其它的解.

\end{solution}

\begin{example}
    用图象法解下面的方程组:
    \begin{numcases}
    2x-3y+4=0\\
    4x-6y+8=0        
    \end{numcases}
\end{example}


\begin{solution}
 列表
 \begin{multicols}{2}
    \begin{center}
        \begin{tabular}{c|cc}
           \hline
            $x$&1&$-2$\\
            \hline
            $y$&2&0\\
            \hline
        \end{tabular}
    \end{center}
    \begin{center}
        \begin{tabular}{c|cc}
            \hline
            $x$&1&$-2$\\
            \hline
            $y$&$2$&0\\
            \hline
        \end{tabular}
    \end{center}
\end{multicols}

\begin{figure}[htp]
    \centering
\begin{tikzpicture}[>=latex]
    \draw[->] (-3,0)--(3,0)node[right]{$x$};
    \draw[->] (0,-1)--(0,3)node[right]{$y$};
    \node at (.25,-.25){$O$};   
\draw[domain=-3:2, samples=10, very thick]plot (\x, {2*\x/3+4/3});
\node at (3,2.5)[below]{$4x-6y+8=0$};
\node at (3,2.5)[above]{$2x-3y+4=0$};
\draw (1,0)node[below]{1}--(1,.1);
\draw (0,1)--(.1,1)node[right]{1};

\end{tikzpicture}
    \caption{}
\end{figure}

    这两个方程的图象是同一条直线(图4.27),所以在直线
    上的每一个点的坐标都是这两个方程的公共解,方程组有无
    穷多解.     
\end{solution}



\begin{example}
    用图象法解下面的方程组:
\begin{numcases}{}
    2x-3y+4=0\\
    4x-6y-2=0
\end{numcases}
\end{example}

\begin{solution}
    列表
    \begin{multicols}{2}
       \begin{center}
           \begin{tabular}{c|cc}
              \hline
               $x$&1&$-2$\\
               \hline
               $y$&2&0\\
               \hline
           \end{tabular}
       \end{center}
       \begin{center}
           \begin{tabular}{c|cc}
               \hline
               $x$&$-1$&2\\
               \hline
               $y$&$-1$&1\\
               \hline
           \end{tabular}
       \end{center}
   \end{multicols}
   \begin{figure}[htp]
    \centering
\begin{tikzpicture}[>=latex]
    \draw[->] (-3,0)--(3,0)node[right]{$x$};
    \draw[->] (0,-2.5)--(0,3)node[right]{$y$};
    \node at (-.25,-.25){$O$};   
\draw[domain=-3:2, samples=10, very thick]plot (\x, {2*\x/3+4/3});
\draw[domain=-3:2, samples=10, very thick]plot (\x, {2*\x/3-1/3});
\node at (2,8/3)[right]{$2x-3y+4=0$};
\node at (2,1)[right]{$ 4x-6y-2=0$};
\draw (1,0)node[below]{1}--(1,.1);
\draw (0,1)--(.1,1)node[right]{1};

\end{tikzpicture}
    \caption{}
\end{figure}

由图4.28看出,当两条直线方程的斜率相同而截距不同
时,这两个方程的图象是两条互相平行的直线,它们没有公
共的点,所以这个方程组没有解.
\end{solution}


从例4.18、4.19、4.20可以知道,二元一次方程组的解可以有三种情况:
\begin{enumerate}
   \item 有一组且只有一组解(方程组里两个方程的图象是
两条相交直线)、
\item 有无穷多组解(方程组里两个方程的图象是同一条
直线).
\item 没有解(方程组里两个方程的图象是两条互相平行
的直线). 
\end{enumerate}

因为平面上两条直线的相互位置关系,只有相交、重
合、平行这三种情况,所以二元一次方程组的解,也只可能
有这三种情况.

\section*{习题4.4}
\addcontentsline{toc}{subsection}{习题4.4}
\begin{enumerate}
    \item 
    求作下列各方程的图象:
\begin{multicols}{2}
\begin{enumerate}
    \item $3x+2y=6$
    \item $4x+6y+9=0$
    \item $3y-4=5x$
    \item $x+y=0$
    \item $3x-4y=0$
\end{enumerate}
\end{multicols}
 
 \item 对于同一坐标系,作下列各方程的图象:
 \begin{multicols}{2}
    \begin{enumerate}
        \item $y=2$
        \item $y=-2$
        \item $x=4$
        \item $x=-4$
        \item $x=0$
        \item $y=0$
    \end{enumerate}
    \end{multicols}
\item \begin{enumerate}
   \item 已知函数$y=3x-2$和$y=2x+3$, 求作这两个函
    数的图象;
    \item 根据这两个图象,求$x$等于什么数值的时候,函数$y=3x-2$和$y=2x+3$有相同的值;
    \item 用图象法解方程$6x+3=4x-7$. 
\end{enumerate}

\item     用图象法解下列各方程组:
  \begin{multicols}{2}
\begin{enumerate}
    \item $\begin{cases}
        y=-\frac{2}{3}x+5\\
        y=\frac{3}{2}x-8
    \end{cases}$
    \item $\begin{cases}
        x+y=5\\x-y=1
    \end{cases}$
    \item $\begin{cases}
        3x+y=9\\x+2y=-2
    \end{cases}$
    \item $\begin{cases}
        2x-5y+16=0\\3x+4y+1=0
    \end{cases}$
\end{enumerate}
\end{multicols}
    \item 讨论下列各方程组的解,并且用图象表示出来:
    \begin{multicols}{2}
        \begin{enumerate}
\item $\begin{cases}
    x-y-5=0\\2x+2y-9=0
\end{cases}$    
\item $\begin{cases}
     x+y=3\\2x+2y=7
\end{cases}$ 
\item $\begin{cases}
    x-3y+2=0\\3x-9y+6=0
\end{cases}$ 
\item $\begin{cases}
    x-4=y\\2x-2y=5
\end{cases}$ 
        \end{enumerate}
\end{multicols}
   
\item 四条直线:$2x+y=6$, $x+2y=6$, $x$轴和$y$轴围成-个四边形,求这四边形的面积.

\end{enumerate}

\subsection{二元一次不等式的图象}
前面我们已经研究过$ax+by+c=0$ ($a,b$不同时为零)
的图象是这样一条直线,它上面的点的坐标是适合方程的所
有有序对$(x,y)$的集合.

现在来研究$ax+by+c>0$或$ax+by+c<0$($a,b$不同时
为零)的图象,类似上面这种含有二个一次变数的不等式叫做
二元一次不等式,使二元一次不等式成立的有序实数对$(x,y)$
叫做 二元一次不等式的解.例如:
\begin{equation}
 3x+y+5>0   
\end{equation}
对于$(-3,5)$有$3\x(-3)+5+5=
1>0$, (4.12)成立,因此$(-3,5)$是它的解,又$(-5,7)$
使$3(-5)+7+5=-3<0$, 所以$(-5,7)$不是(4.12)
的解.

根据图象的意义,$ax+by+c>0$ (或$<0$) (这里$a^2+b^2>
0$) 的图象就是所有适合$ax+by+c>0$ (或$<0$) 的有序对的
集合,以这样的有序对为坐标的所有点的集合表示平面上的
什么图形呢?我们从一个具体例子入手,看看$3x-y+5>0$
的图象是什么,也就是要看看$y<3x+5$的图象是什
么?

现在在直线$y=3x+5$上取一点,若$x=1$则$y=8$. 因此
$A(1,8)$是在直线$y=3x+5$上的点,如果在点$A$的正下方取
点$A_1,A_2,\ldots$, 因为这些点的$x$坐标仍为1, 但$y$的坐标小于
8, 这样把$x=1$与小于8的任意的$y$配对,如$(1,7)$, 
$\left(1,\frac{15}{2}\right)$, $(1,-2000),\ldots$都适合$y<3x+5$, 因此都
是$y<3x+5$图象上的点,再在点
$A$的正上方取某些点,则这
些点的坐标不适合$y<3x+5$.同样,在直线$y=3x+5$上另
取一点$B(-2,-1)$, 这样把$x=-2$与小于$-1$的任意
$y$配对都是$y<3x+5$图象上的点,而把$x=-2$与大于$-1$
的任意$y$配对就都不适合$y<3x+5$, 由此可见,$y=3x+5$的
图象是过$A,B$的一条直线,而$y<3x+5$的图象是以直线
$AB$为界的开的半平面(不包括直线$AB$),如图4.29中的阴影部分.

\begin{figure}[htp]
    \centering
\begin{tikzpicture}[>=latex, yscale=.5]
\fill [gray!20] (-2.5,-2.5)--(1,8)--(2.75,8)--(2.75,-2.5)--(-2.5,-2.5); 
   \draw[->] (-3,0)--(3,0)node[right]{$x$};
    \draw[->] (0,-3)--(0,8)node[right]{$y$};
    \node at (-.25,-.5){$O$};   
\foreach \x in {-2,-1,1,2}
{
    \draw (\x,0)node[below]{$\x$}--(\x,.2);
}
\draw [domain=-2.5:1, samples=10, very thick] plot(\x, {3*\x+5});
\foreach \y in {2,4,6}
{
    \draw (0,\y)node[left]{$\y$}--(.1,\y);
}

\end{tikzpicture}
    \caption{}
\end{figure}

现在的问题是,直
线$3x-y+5=0$把平
面分成除直线外的两个
半平面,$3x-y+5>0$
表示两个半平面之一,
我们如何较方便地而不
必把不等式就$y$来解就
可以确定$3x-y+5>0$
是哪个半平面?这是非
常容易确定的,因为
$3x-y+5>0$所确定的
那个半平面内,一切点的坐标都满足$3x-y+
5>0$. 故我们只要任意选一点,譬如说原点$(0,0)$, 把它
代到$3x-y+5$中,如果结果大于零,那么$3x-y+5>0$
就是包含原点的那个平面,不然就是不包含原点的那个平面,
现在实际代入的结果$3\x0-0+5>0$, 故$3x-y+5>0$的
图象就是包含原点的那个半平面了,另一个半平面就是
$3x-y+5<0$的图象了.二元一次不等式的图象也叫做不等
式的表示区域.

\begin{example}
图示下列不等式组的图象:
\[\begin{cases}
  2x+3y-6\ge 0\\
2x-3y-6\le 0 
\end{cases}\]
 \end{example}

\begin{solution} 
$2x+3y-6\ge 0$的图象是以直线$m:\; 2x+3y-6=0$
为界包括$m$不含$(0,0)$的半平面;
$2x-3y-6\le 0$的图象是以直线$h:\; 2x-3y-6=0$为
界包括$h$包含$(0,0)$的半平面.不等式组的解集的图象是这
两个半平面的公共部分.换言之,不等式组的解集图象是这个
不等式组中各个不等式图象的交集(图4.30).

\begin{figure}[htp]
    \centering
\begin{tikzpicture}[>=latex, scale=.7]
\fill [gray!20](-2,10/3)node[above, black]{$2x+3y-6= 0$}--(-2,4)--(5.5,4)--(5.5,5/3)node[above, black]{$2x-3y-6= 0$}--(3,0)--(-2,10/3);
\node at (-2,-10/3) [below]{$h$};
\node at (5.5,-5/3) [below]{$m$};
    \draw[->] (-3,0)--(6,0)node[right]{$x$};
    \draw[->] (0,-4)--(0,4)node[right]{$y$};
    \node at (-.25,-.5){$O$};   
\foreach \x in {-2,-1,1,2,...,5}
{
    \draw (\x,0)node[below]{$\x$}--(\x,.1);
}
\draw [domain=-2:5.5, samples=10, very thick] plot(\x, {2-2*\x/3});
\draw [domain=-2:5.5, samples=10, very thick] plot(\x, {2*\x/3-2});

\foreach \y in {-3,-2,-1,1,2,3}
{
    \draw (0,\y)node[left]{$\y$}--(.1,\y);
}


\end{tikzpicture}
    \caption{}
\end{figure}
\end{solution}



\begin{example}
    图示不等式组的图象:
\[\begin{cases}
   x\ge  0\\
    y\ge -1\\
    x+y\le 3\\
    x\le 2
\end{cases}\]
\end{example}



\begin{solution}
不等式组的解集是由
四条直线所围成的,在图4.31
中阴影部分的四边形$ABCD$的
周界以及它的内部.因此不等
式组的图象是一个凸区域.
\end{solution}

我们已经知道二元一次不
等式组的解集的图象是一个凸
区域.是不是二元一次不等式组总表示一个凸区域呢?下面
的例子指出存在二元一次不等
式组不构成区城的情况.

\begin{example}
   不等式组:
   \[\begin{cases}
    y\ge 0\\
    x-y\ge 3\\
    5x+2y\le 10    
   \end{cases}\]
没有解. 
\end{example}

\begin{solution}
    因从图象上来看(图4.32),
对应于这三个不等式的半平面
没有公共部分.


\end{solution}

\begin{figure}[htp]\centering
    \begin{minipage}[t]{0.48\textwidth}
    \centering
    \begin{tikzpicture}[>=latex, scale=.7]
        \fill [gray!20](0,-1)--(2,-1)--(2,1)node[right, black]{$D$}--(0,3)node[right, black]{$A$}--(0,-1);
    
    
        \draw[->] (-2,0)--(5,0)node[right]{$x$};
        \draw[->,very thick] (0,-3)node[below]{$x=0$}--(0,4)node[right]{$y$};
        \node at (-.25,-.5){$O$};   
        \foreach \x in {1,2,3}
        {
            \draw (\x,0)node[below]{$\x$}--(\x,.1);
            \draw (0,\x)node[left]{$\x$}--(.1,\x);
        }
        \draw [domain=-1:5, samples=10, very thick] plot(\x, {3-\x});
        \draw[very thick] (-2,-1)node[below]{$y=-1$}--(5,-1);
        \draw[very thick] (2,-3)--(2,4)node[right]{$x=2$};
    \node at (5,-2) [fill=white]{$x+y=3$};
    \node at (0-.25,-1-.25){$B$};
    \node at (2-.25,-1-.25){$C$};
    \end{tikzpicture}
    \caption{}
    \end{minipage}
    \begin{minipage}[t]{0.48\textwidth}
    \centering
    \begin{tikzpicture}[>=latex, scale=.6]
        \draw[->,very thick] (-2,0)node[above]{$y=0$}--(5,0)node[right]{$x$};
        \draw[->] (0,-4)--(0,7)node[right]{$y$};
        \foreach \x in {-1,1,2,...,4}
        {
            \draw (\x,0)node[below]{$\x$}--(\x,.1);
        }
    \foreach \y in {-3,-2,-1,1,2,...,5}
    {
        \draw (0,\y)node[left]{$\y$}--(.1,\y);
    }
    \draw [domain=-1:5, samples=10, very thick] plot(\x, {\x-3});
    \draw [domain=-.5:3.3, samples=10, very thick] plot(\x, {5-2.5*\x});
    
    \node at (2,4){$5x+2y=10$};
    \node at (5,2.3){$x-y=3$};
    
    \end{tikzpicture}
    \caption{}
    \end{minipage}
    \end{figure}



\begin{example}
    王明购买价格是3
角和5角的笔记本若干本,每
种至少买一本,但不得超过2.6元,问有多少种买法?指出正好花完2.6元的情形.
\end{example}

\begin{solution}
    设购买3角的笔记本$x$本,5角的笔记本$y$本,依题
意列出不等式组:
\[\begin{cases}
    0.3x+0.5y\le 2.6\\
x\ge 1\\
y\ge 1
\end{cases}\]
其中$x,y$是整数.

所以只要在$\triangle ABC$围成的凸区域,包括周界,数出坐标是整
数的点的个数即可,由图
4.33可以看出,这样的点
共17个,每个点代表一种
买法,其中在直线$0.3x+5y=2.6$上的点共有两
个,即$(2,4)$和$(7,1)$, 这表示若买0.3元的2本,
0.5元的4本或者买0.3元的7本,0.5元的1本,则
恰好花完2.6元,其它情形的买法都未花完2.6元.

\begin{figure}[htp]
    \centering
\begin{tikzpicture}[>=latex, scale=.8]
    \draw[->] (-1,0) --(10,0)node[right]{$x$};
    \draw[->] (0,-1)--(0,7)node[right]{$y$};
    \foreach \x in {1,2,...,8}
    {
        \draw (\x,0)node[below]{$\x$}--(\x,.1);
    }
\foreach \y in {1,2,...,5}
{
    \draw (0,\y)node[left]{$\y$}--(.1,\y);
}
\draw [domain=-1:9.5, samples=10, very thick] plot(\x, {5.2-0.6*\x});
\draw[very thick](-1,1)--(10,1)node[above]{$y=1$};
\draw[very thick](1,-1)--(1,7)node[right]{$x=1$};
\node at (9.5,-1){$0.3x+0.5y=2.6$};
\node at (1-.3,1-.3){$B$};
\node at (1,4.6)[right]{$A$};
\node at (7,1)[above]{$C$};

\foreach \x in {1,2,...,7}
{
    \draw (\x,1) [fill=black] circle(2pt);
}
\foreach \x in {1,2,...,5}
{
    \draw (\x,2)[fill=black] circle (2pt);
}
\foreach \x in {1,2,...,3}
{
    \draw (\x,3)[fill=black] circle (2pt);
}
\foreach \x in {1,2}
{
    \draw (\x,4) [fill=black] circle(2pt);
}

\end{tikzpicture}
    \caption{}
\end{figure}
\end{solution}

\begin{example}
 某停车场共占地36个单位,每一小汽车空位占地
2个单位,而每一大汽车空位占地3个单位,并且要使小汽
车停放辆数至少是大汽车停放辆数的两倍,问:
\begin{enumerate}
    \item 这个停车场最多可以停放多少辆汽车?
    \item 在满足上述条件下,尽可能多地停放大汽车,那么这
个停车场最多放大、小汽车各几辆?
\end{enumerate}
\end{example}


\begin{solution}
设停车场停放小汽车$s$辆,大汽车$\ell$辆.依题意$s$和$\ell$必
须满足不等式组: 
\[\begin{cases}
2s+3\ell\le 36\\
s\ge 2\ell    
\end{cases}\]
其中:$s,\ell\ge 0$,且$s,\ell$是整数.作不等式组图象,它的解集是图4.34的$\triangle OAB$围成的凸区域
(包括周界)中坐标是整数的点,由图可以看出在$A(18,0)$
点,停车场停放车辆最多,共放小汽车18辆,我们可以这样
来说明:

直线$2s+3\ell =36$,即:$s=18-\frac{3}{2}\ell$
的斜率是$-\frac{3}{2}$,因此
\[\frac{\Delta s}{\Delta \ell}=-\frac{3}{2}\]
这表示在停车场停满车辆的情形下,每次增停大汽车2辆,
小汽车就要减停3辆.于是停车场的停车总数$s+\ell=\left(18-\frac{3}{2}\ell\right)+\ell=18-\frac{1}{2}\ell$
将因$\ell$的增加而减小,故当$\ell=0$时,
$s+\ell$有最大值18.

\begin{figure}[htp]
    \centering
\begin{tikzpicture}[>=latex, scale=1.3]
\fill[gray!20] (0,0)--(2.06,1.03)--(3.6,0)--(0,0);

    \draw[->] (0,0) --(4.5,0)node[right]{$s$};
    \draw[->] (0,0)node[left]{$O$}--(0,3.5)node[right]{$\ell$};
\foreach \x in {5,10,15}
{
    \draw(\x/5,0)node[below]{$\x$}--(\x/5,.1);
    \draw(0,\x/5)node[left]{$\x$}--(.1,\x/5);
}
\draw(4,0)node[below]{20}--(4,.1);
\node at (0.25,2.4)[right]{$2s+3\ell=36$};
\draw[very thick](0,2.4)node[left]{12}--(3.6,0)node[below]{18};
\node at (3.6,0)[above]{$A$};
\draw[very thick] (0,0)--(4,2)node[right]{$\ell=\frac{1}{2}s$};
\node at (2.06,1.03)[above]{$B$};
\end{tikzpicture}
    \caption{}
\end{figure}



再从图看出$B$点最高,$B$点的坐标应满足方程组:
\[\begin{cases}
    2s+3\ell=36\\
s=2\ell
\end{cases}\]
解得:$\ell=5\frac{1}{7},\quad s=10\frac{2}{7}$.

在$\triangle OAB$区域内靠近$B\left(10\frac{2}{7},\; 5\frac{1}{7}\right)$点的整数值坐标的点是$(10,5)$.此点坐标符合所列不等式组,因此大汽车最多可
停放5辆,此时小汽车最多可以停放10辆.
\end{solution}
   
\section*{习题4.5}

\addcontentsline{toc}{subsection}{习题4.5}

\begin{enumerate}
    \item 画出下列不等式的图象:
\begin{multicols}{2}
    \begin{enumerate}
        \item $2x+3y\ge 6$
        \item $3x-4y\le 2$
        \item $5x-2y\ge 0$
        \item $4x+3y\le 0$
    \end{enumerate}
\end{multicols}

 \item 画出下列不等式组的图象:
 \begin{multicols}{2}
    \begin{enumerate}
        \item $\begin{cases}
            y-2x>0\\x-y-2<0
        \end{cases}$
        \item $\begin{cases}
            x+1\ge 0\\y+1\ge 0\\x+y\le 1\\ x-y\le 1
        \end{cases}$
        \item $\begin{cases}
            y-1\le 0\\2x+3y\ge 0
        \end{cases}$
        
        \item $\begin{cases}
            x-y+2\ge 0\\x-y-2\le 0\\x+y+2\le 0\\x+y-2\ge 0
        \end{cases}$
    \end{enumerate}
\end{multicols}

\item 当$(x,y)$同时满足下面不等式组时,求$x$的最大值与$y$的
    最大值,
    \[\begin{cases}
     x-y+4\ge 0\\
2x+y-4\le 0\\
x+2y+1\ge 0       
    \end{cases}\]

\item 
30个人要分乘5座与6座的出租汽车出游,车库6座车
至多只有两辆,5座车数量很多,车上有空座也无妨:
\begin{enumerate}
    \item 画出图象表示派车方案;
    \item 哪些派车办法使用车
的数目最少?在这些办法中分别有几个空座?
\end{enumerate}

\item 某小工厂每生产一件产品A获利4元,每生产一件产品
B获利5元,该厂每周至多能生产A35件,B20件,要求
每周至少获利200元,应该怎样安排生产计划?
\begin{enumerate}
    \item 写出必须满足的关系式,说明你所用字母的意义;
    \item 画出解的集合的图象;
    \item 找出使工厂盈利超过200元而生产B尽量少的各种
办法.
\end{enumerate}
\end{enumerate}

\subsection{再谈函数及其图象}
画函数图象要注意函数的定义域,如

\begin{example}
    画出下列函数的图象:
    \begin{enumerate}
        \item $y=2x,\qquad (x\in\mathbb{R})$;
        \item $y=2x,\qquad (0\le x\le 5)$;
        \item $y=2x$,\qquad ($x$是整数).
    \end{enumerate}
\end{example}


\begin{solution}
1的图象是图4.35(a)的直线$AB$; 2的图象是
图4.35(b)的线段$CD$; 3的图象是图4.35(c)的一些离散
的点.

\begin{figure}[htp]
    \centering
\begin{tikzpicture}[>=latex, scale=.7]
\begin{scope}
    \draw[->] (-2,0) --(2,0)node[right]{$x$};
    \draw[->] (0,-3)--(0,4)node[right]{$y$};
    \draw[very thick](-1.5,-3)node[left]{$A$}--(1.5,3)node[right]{$B$};
    \draw (0,1)node[left]{1}--(.1,1);
    \draw (1,0)node[below]{1}--(1,.1);
    \node at (.25,-.25){$O$};

    \node at (0,-3.5){(a)};
\end{scope}
\begin{scope}[xshift=5cm, yshift=-2cm]
    \draw[->] (-1,0) --(3,0)node[right]{$x$};
    \draw[->] (0,-1)--(0,6)node[right]{$y$};
    \draw[very thick](0,0)--(2.5,5)node[right]{$D$};
\foreach \x in {1,2,...,5}
{
    \draw(\x/2,0)node[below]{$\x$}--(\x/2,.1);
}
\foreach \x in {2,4,...,10}
{
    \draw (0,\x/2)node[left]{$\x$}--(.1,\x/2);
}
\node at  (-.3,-.3){$C$};
\node at  (-.3,.3){$O$};
\node at (1,-1.5){(b)};
\end{scope}
\begin{scope}[xshift=12cm]
    \draw[->] (-2,0) --(2,0)node[right]{$x$};
    \draw[->] (0,-3)--(0,4)node[right]{$y$};
    \foreach \x in {1,2}
    {
        \draw(\x/2,0)node[below]{$\x$}--(\x/2,.1);
    }
    \foreach \x in {1,3,5,7}
    {
        \draw (0,\x/2)node[left]{$\x$}--(.1,\x/2);
    }
\foreach \x in {-3,-2,...,3}
{
    \draw (\x/2,\x) [fill=black] circle(2pt);
}
\node at (0,-3.5){(c)};
\end{scope}

\end{tikzpicture}
    \caption{}
\end{figure}
\end{solution}

从这里我们再次体会到,当谈论函数时是离不开定义域
的,因为若函数表达式一样,而其定义域不同,则它们的图象
会差之干里,故例中1、2、3三个函数不能认为是相
同的函数.

有时用公式表示的函数,在它的定义域的不同部分可以
用不同的公式表示,即可用若干公式表示变量间的关系,请
看下例:


\begin{example}
 火车在9小时内从A行驶到B, 在最初三小时内,
它的行驶速度为50公里/小时,接下来它停止了两小时,在最
后的四小时内,它以速度60公里/小时行驶到达B. 试表示行
车路程和时间的关系.
\end{example}


\begin{solution}
以$x$表示时间,单位为小时;以$y$表示走过的路程,
单位为公里,我们就得到下面的函数式$y=f(x)$:
\[y=f(x)=\begin{cases}
   50x& 0\le x\le 3\\
150 &3\le x\le 5\\
150+60(x-5)& 5\le x\le 9 
\end{cases}\]
显见,对于定义域$[0,9]$内每一个$x$值,$y$就有唯一确定的值
和它对应,因而$f(x)$是个定义在$[0,9]$上的函数,但这个函
数是用几个不同的式子给出来的,这个函数的图象画在图
4.36中.

\begin{figure}[htp]
    \centering
\begin{tikzpicture}[>=latex,scale=.7]
    \draw[->] (-1,0) --(9,0)node[right]{$x$(小时)};
    \draw[->] (0,-1)--(0,9)node[right]{$y$(公里)};
\foreach \x/\y in {1/50,2/100,3/150,4/200,5/250,6/300,7/350,8/400}
{
    \draw (\x,0)node[below]{$\x$}--(\x,.1);
    \draw (0,\x)node[left]{$\y$}--(.1,\x);
}
    \node at (-.5,-.5){$O$};
\draw[very thick](0,0)--node[rotate=45, above]{$y=50x$}(3,3)--node[above]{$y=50$}(5,3)--node[rotate=52, above]{$y=150+60(x-5)$}(9,7.8);

\end{tikzpicture}
    \caption{}
\end{figure}
    
\end{solution}


\begin{example}
    画出函数$y=|x|$的图象.
\end{example}




\begin{solution}    
这个函数的定义域是一切实数,按照绝对值的定义
    我们有:
\[y=|x|=\begin{cases}
    x&x\ge 0\\
    -x&x<0
\end{cases}\]
图象是折线(图4.37).
\begin{figure}[htp]
    \centering
\begin{tikzpicture}[>=latex]
    \draw[->] (-3,0) --(3,0)node[right]{$x$};
    \draw[->] (0,0)--(0,3)node[right]{$y$};
    \draw (1,0)node[below]{$1$}--(1,.1);
    \draw (0,1)node[left]{$1$}--(.1,1);
    \node at (0,0)[below]{$O$};
\draw[very thick] (-2,2)--(0,0)--(2,2);
\end{tikzpicture}
    \caption{}
\end{figure}
\end{solution}    

\begin{example}
    已知$f(x)=|x+1|+\sqrt{(x-2)^2}$
\begin{enumerate}
    \item 求函数的定义域;
    \item 当$-1\le x<2$时,化简函数的解析式;
    \item 作出函数的图象,并说明函数的值域是什么?
\end{enumerate}
\end{example}    

\begin{solution}
\begin{enumerate}
    \item 对于$|x+1|$, $x$可取一切实数;对于$\sqrt{(x-2)^2}$,
    $x$必须满足$(x-2)\ge 0$, 这个不等式对于一切实数都成立,
    所以函数的定义域是一切实数.
\item $\sqrt{(x-2)^2}=|x-2|$, $f(x)=|x+1|+|x-2|$,
    要脱掉绝对值符号需分段讨论.我们知道$x=-1$时,
    $|x+1|=0$; $x=2$时,$|x-2|=0$, 故$-1,2$把数轴
    分为三段:$(-\infty,-1)$, $[-1,2)$, $[2,+\infty)$.
\begin{itemize}
    \item 当$x\in (-\infty,-1)$时,
    $$f(x)=|x+1|+|x-2|=-(x+1)-(x-2)=-2x+1$$
    \item 当$x\in[-1,2)$时,$f(x)=(x+1)-(x-2)=3$
  \item   当$x\in [2,+\infty)$时,$f(x)=(x+1)+(x-2)=2x-1$
\end{itemize}
即:\[f(x)=\begin{cases}
   -2x+1  & x<-1\\
   3& -1\le x<2\\
   2x-1 & x\ge 2 
\end{cases}\]
\item 画出的图象是图4.38.
由观察图象知:
\begin{itemize}
    \item 当$x<-1$时,$f(x)=-2x+1>3$;
    \item 当$-1\le x<2$时,$f(x)=3$;
    \item 当$x\ge 2$时,$f(x)=2x-1\ge 3$
\end{itemize}
所以函数的值域是$f(x)\ge 3$的一切实数.
\end{enumerate}

\begin{figure}[htp]
    \centering
\begin{tikzpicture}[>=latex, scale=.7]
    \draw[->] (-4,0) --(5,0)node[right]{$x$};
    \draw[->] (0,-1)--(0,7)node[right]{$y$};
\foreach \x in {-3,-2,-1,1,2,3,4}
{
    \draw (\x,0)node[below]{$\x$}--(\x,.1);
}
\foreach \y in {1,2,3}
{
    \draw (0,\y)node[left]{$\y$}--(.1,\y);
}

\draw [very thick](-3,7)--node[rotate=-60, above]{$y=-2x+1$}(-1,3)--node[above]{$y=3$}(2,3)--node[rotate=60, above]{$y=2x-1$}(4,7);


    \node at (0.25,-0.25){$O$};

\end{tikzpicture}
    \caption{}
\end{figure}


\end{solution}

下面我们再来看一类函数作为本章的结束.

\begin{example}
邮局规定,寄往外埠普通信件重量不超过20克者
邮资8分,重量超过20克但不超过40克者,邮资1角6分,
重量超过40克但不超过60克者,邮资2角4分,依此每增加
20克邮资增加8分,因此邮资是重量$x$的函数,函数的图象
为图4.39.
\begin{figure}[htp]
    \centering
\begin{tikzpicture}[>=latex]
    \draw[->] (-1,0) --(6,0)node[right]{$x$(克)};
    \draw[->] (0,-1)--(0,6)node[right]{$y$(分)};
    \foreach \x in {20,40,...,100}
    {
        \draw (\x/20,0)node[below]{$\x$}--(\x/20,.1);
    }
    \foreach \y in {8,16,...,40}
    {
        \draw (0,\y/8)node[left]{$\y$}--(.1,\y/8);
    }
    \node at (-0.25,-0.25){$O$};
\draw[very thick]  (0,1)--(1,1);
\draw[very thick]  (1,2)--(2,2);
\draw[very thick]  (2,3)--(3,3);
\draw[very thick]  (3,4)--(4,4);
\draw[very thick]  (4,5)--(5,5);
\foreach \x in {1,2,3,4}
{
    \draw (\x, \x+1) [fill=white] circle (1.5pt);
}
\end{tikzpicture}
    \caption{}
\end{figure}

这种类型的函数称为阶梯函数,它的特点是:自变量$x$
的变化范围分成若干区间,在每个区间中,因变量$y$的值是
不变的,但对应不同区间,$y$值是可以不同的,在每个区间
的端点处,因变量$y$的值有一个跳跃,也就是说函数的图象
不是连续的,而有间断的地方.

例4.27和例4.30中的函数从总体看也是递增变化的,但有时
不增也不减处在平稳状态,这样的函数称为不减的.

    
\end{example}

\begin{blk}{定义}
    如果对于开区间$(a,b)$(或闭间$[a,b]$)的任意两个
  自变量的值$x_1$和$x_2$:
  \begin{itemize}
      \item 当$x_1<x_2$时,可以推出$f(x_1)\le f(x_2)$, 那么
  函数$f(x)$称为在开区间$(a,b)$(或闭区间$[a,b]$上)\textbf{不减}.
\item  当$x_1<x_2$时,可以推出$f(x_1)\ge f(x_2)$, 那么函数$f(x)$称
  为在开区间(或闭区间$[a,b]$上\textbf{不增}.   
  \end{itemize}
  \end{blk}
  
  
  



\section*{习题4.6}

\addcontentsline{toc}{subsection}{习题4.6}

\begin{enumerate}
    \item 作函数图象:
   \[y=f(x)=\begin{cases}
       x& 0\le x<2\\
       2& 2\le x<4\\
       -x+6 & 4\le x<6
   \end{cases}\]
   \item 作函数图象:
    $y=-|x+1|$
    \item 作函数图象:
   \[f(x)=\begin{cases}
       2& x\le -1\\
       1-x& -1<x\le 0\\
       1+x & 0<x\le 1\\
       2& x>1
   \end{cases}\]
    \item 一列火车在$t=0$时由A地开出,速度是每小时100公里,
    行驶2小时到达B地;在那里停车1小时后,以每小时
    80公里的速度继续向前行驶3小时,表示火车在时刻$t$与
    A地的距离(单位:公里)的函数.
    \item 作出函数$y=|x-2|-2$的图象,并求出它与$x$轴所围
    成的图形的面积.
\end{enumerate}

\section*{复习题四}
\addcontentsline{toc}{section}{复习题四}

\begin{enumerate}
    \item 叙述函数的定义域与值域.
    \item 指出下列函数的定义域与值域:
\begin{multicols}{2}
\begin{enumerate}
    \item $f(x)=\sqrt{x}$
    \item $f(x)=1-|x|$
    \item $f(x)=\frac{1}{\sqrt{x^2+1}}$
    \item $f(x)=\sqrt{4x-1}$
\end{enumerate}
\end{multicols}
    
    \item 指出下列函数的定义域:
\begin{multicols}{2}
\begin{enumerate}
\item $f(x)=\frac{\sqrt{x+1}}{\sqrt{x-1}}$
\item $f(x)=\frac{\sqrt{x+4}}{x-5}$
\item $f(x)=\frac{6}{x^2-3x+2}$
\item $f(x)=\frac{\sqrt{4x+8}}{3x-2}$
\item $f(x)=\sqrt{2x-1}+1-2x+x^2$
\item $f(x)=\frac{1}{x+5}+\lg(-x)$
\end{enumerate}
\end{multicols}
    \item 作出下列函数的图象:
\begin{enumerate}
    \item $f(x)=2x,\qquad x\in\{-2,-1,0,1,2\}$
    \item $f(x)=-2x+1,\qquad x\ge 1$
    \item $f(x)=-x-2,\qquad -2<x\le 2$
    \item $f(x)=\begin{cases}
        1&x>0\\-1&x\le 0
    \end{cases}$
    \item $f(x)=\begin{cases}
        -x+6  &  x>2\\
        x^2& -2\le x\le 2\\
        x+6& x<-2
    \end{cases}$
\end{enumerate}

\item 在同一坐标系里,作出下面三个函数的图象,并说明这
三个图象的形状和位置有何关系:
\[y=7x;\qquad y=7x+3;\qquad y=7x-3\]
\item 
按照下列各组条件,确定一次函数$y=f(x)$:
\begin{multicols}{2}
\begin{enumerate}
    \item $f(0)=3,\qquad f(-1)=-1$
    \item $f(1)=2,\qquad f(-1)=1$
    \item $f(0)=-3,\qquad f\left(\frac{3}{2}\right)=0$
    \item $f(1)=1,\qquad f(0)=-2$
    \item $f(0)=0,\qquad f(5)=-7$
\end{enumerate}
\end{multicols}
\item 已知$P_1(1,1)$及$P_2(5,13)$两点在直线$y=kx+b$上,
试求出$k$和$b$, 并画出图象,点$P(3,7)$是否在这条直线上?
\item 长为12cm的弹簧,能承受的重量在0—15千克的范围内,
重量每增加1千克,弹簧被拉长0.5cm, 试写出承受的
重量$x$(千克)与弹簧长$y$(厘米)之间的函数关系式,并作
出图象.
\item 
行星绕太阳一周所需日数的平方,与行星离太阳的距离
的立方成正比例,已知地球绕太阳一周约需365日,水星
绕太阳一周约需88日,地球离太阳约$1.5\x10^8$公里,求
水星离太阳约多少公里?
\item 海上所能看到的水面距离,与眼晴离海平面的高的平
方根成反比例.如果眼晴离海平面高2米时,所能看到的
水面距离是5公里,那么眼晴离海平面高是8米时,能
看到的距离是多少公里?
\item 如果变量$y$与$x$成反比例,变量$z$与$y$也成反比例,求证
变量$z$与$x$成正比例.
\item 在同一坐标系里,对$k,b$的下列数值,作出函数
$y=k|x|+b$的图象,并说明这些图象的形状和位置有
何关系?
\begin{multicols}{2}
\begin{enumerate}
    \item $k=3,\qquad b=5$
    \item $k=-3,\qquad b=5$
    \item $k=4,\qquad b=0$
    \item $k=-4,\qquad b=0$
    \item $k=\frac{1}{2},\qquad b=3$
    \item $k=-\frac{1}{2},\qquad b=4$
\end{enumerate}
\end{multicols}

\end{enumerate}


%     \chapter{二次函数}

\section{二次函数}
\subsection{函数的奇偶性}

在研究二次函数之前,我们先来研究函数的一个性
质——函数的奇偶性。

我们先来描绘$y=x^2$的图象。

先作出下面的数值表:

\begin{center}
\begin{tabular}{c|ccccccccccc}
    \hline
    $x$   &$\cdots$&   $-2$   &   $-1.6$   &   $-1$   &   $-0.5$   &   $0$   &   $0.5$   &   $1$   &   $1.5$   &   $2$   &   $\cdots$      \\
    \hline
       $y$  &$\cdots$ &   $4$   &   $2.25$   &   $1$   &   $0.25$   &   $0$   &   $0.25$   &   $1$   &   $2.25$   &   $4$   &   $\cdots$\\
       \hline
\end{tabular}
\end{center}

用表里各组对应值作为点的坐标,作出各个点,然后用
平滑的曲线把它们连结起来,就得出$y=x^2$的图象(图5.1),
这个图象叫做抛物线。函数$y=x^2$的图象,以后简称为抛物线
$y=x^2$。

\begin{figure}[htp]
    \centering
\begin{tikzpicture}[>=latex, scale=.8]
\draw[->](-3,0)--(3,0)node[right]{$x$};
\draw[->](0,-1)--(0,5)node[right]{$y$};
\foreach \x in {-2,-1,1,2}
{
    \draw (\x,0)node[below]{$\x$}--(\x, 0.1);
}
\foreach \x in {1,2,...,4}
{
    \draw (0,\x)node[left]{$\x$}--(.1,\x);
}

\draw[domain=-2:2, samples=100, very thick] plot(\x,{\x*\x});

\end{tikzpicture}
    \caption{}
\end{figure}


从上面表格中可以看到这个函数有一个特点:当自变量
取绝对值相等而符号相反的两个值时(如$x$取1.5和$-1.5$),
它们对应的函数值相等($y$都取2.25),这说明$y$轴垂直平分以
点$(x,f(x))$, $(-x,f(-x))$为端点的线段,换句话说,点
$(x,f(x))$, $(-x,f(x))$是关于$y$轴对称的,因此抛物线$y=x^2$
是关于$y$轴对称的。

我们把具有这种特征
的函数叫做偶函数。$f(x)$
是偶函数的标志是:当自
变量$x$取一对互为相反
数的值时,函数的值不
变,就有$f(x)=f(-x)$。

一般地说,对于函数
$f(x)$, 设$x$和$-x$都属于函
数的定义域,如果
\[f(-x)=f(x)\]
那么函数$f(x)$叫做\textbf{偶函数},偶函数的图象关于$y$轴对称。

我们再来画函数$y=\frac{1}{8}x^3$的图象

先作出下面的数值表:
\begin{center}
\begin{tabular}{c|ccccccccccc}
    \hline
    $x$ &$\cdots$&$-4$&$-3$&$-2$&$-1$&0&1&2&3&4&$\cdots$\\
\hline
$y$ &$\cdots$&$-8$&$-3\tfrac{3}{8}$&$-1$&$-\tfrac{1}{8}$&0&$\tfrac{1}{8}$&1&$3\tfrac{3}{8}$&8&$\cdots$\\
\hline
\end{tabular}
\end{center}

根据表里这些对应值,作出函数$y=x^3$的图象如图
5.2。这个图象称为立方抛物线。
\begin{figure}[htp]
    \centering
\begin{tikzpicture}[>=latex, scale=.5]
\draw[->](-5,0)--(5,0)node[right]{$x$};
\draw[->](0,-9)--(0,9)node[right]{$y$};
\foreach \x in {1,2,3,4}
{
    \draw (\x,0)node[below]{$\x$}--(\x, 0.1);
}
\foreach \x in {-2,-4}
{
    \draw (\x,0)node[above]{$\x$}--(\x, -0.1);
}
\foreach \x in {-8,-6,...,-2,2,4,...,8}
{
    \draw (0,\x)node[left]{$\x$}--(0.1,\x);
}
\node at (-.5,-.5){$O$};
\draw[domain=-4:4, samples=100, very thick] plot(\x,{\x*\x*\x/8});

\end{tikzpicture}
    \caption{}
\end{figure}

从上面表格中可以看到这个函数也有一个特征:因为,
$\frac{1}{8}(-x)^3=-\frac{1}{8}x^3$, 所以当自变量取两个互为相反数的值时,
对应的函数值也是互为相反数。所以如果点$(x,f(x))$在函
数的图象上,那么必有另一点$(-x,-f(x))$也在函数的图象
上,而原点恰是以$(x,f(x))$, $(-x,-f(x))$为端点的线段
的中点,换句话说,点$(x,f(x))$, $(-x,-f(x))$是关于原
点对称的。因此立方抛物线
$y=\frac{1}{8}x^3$是关于原点对称的,我
们把具有这种特征的函数叫做
奇函数,$f(x)$是奇函数的标志
是:当自变量$x$取一对互为相
反数的值时,函数的值也是
互为相反数,就是$f(-x)=-f(x)$。

一般地说,对于函数$f(x)$, 设$x$与$-x$都属于函数的定义
域,如果$$f(-x)=-f(x)$$ 那么函数$f(x)$叫做\textbf{奇函数}。奇函数的图象关于原点对称。

考虑一个函数是偶函数、奇函数,或者既不是偶函数又
不是奇函数,叫做研究函数的奇偶性,对于一个奇函数或者
偶函数,要了解它的性质和图象,只要了解当自变量取正
值时的性质和图象就可以了。例如,要作函数$y=\frac{1}{8}x^3$的图
象,因为它是奇函数,所以只要作出自变量取正值时的函数
图象,就可以利用奇函数的图象必定关于原点对称这一特点,
作出自变量取负值时的图象。

\begin{example}
研究下列函数的奇偶性:
\[f(x)=x^4+x^2,\qquad f(x)=x^3+x,\qquad f(x)=x+1\]   
\end{example}

\begin{solution}
\begin{enumerate}
    \item 对于$f(x)=x^4+x^2$, 我们有:
    \[f(-x)=(-x)^4+(-x)^2=x^4+x^2=f(x)\]
    $\therefore\quad $函数$f(x)=x^4+x^2$是偶函数。
    \item 对于$f(x)=x^3+x$, 我们有:
    \[f(-x)=(-x)^3+(-x)=-(x^3+x)=-f(x)\]
    $\therefore\quad $函数$f(x)=x^3+x$是奇函数。
    \item 对于$f(x)=x+1$, 我们有:
    \[f(-x)=-x+1\]
    这里$f(-x)\ne f(x)$, 并且$f(-x)\ne -f(x)$, 所以函数$f(x)=
x+1$既不是偶函数又不是奇函数。
\end{enumerate} 
\end{solution}

\begin{ex}
\begin{enumerate}
    \item  研究下面函数的奇偶性:(其中$k,b$是常数)
\begin{multicols}{2}
\begin{enumerate}
    \item $y=kx\quad (k\ne 0)$
    \item $y=\frac{k}{x}\quad (k\ne 0)$
    \item $y=kx+b\quad (k\ne 0)$
    \item $y=\sqrt{1+x^2}$
    \item $y=|x|$
    \item $y=\sqrt[3]{x}$
    \item $y=\sqrt{x}+1$
    \item $y=x^2+x+1$
\end{enumerate}
\end{multicols}
\item 证明:
\begin{enumerate}
    \item 两个偶函数的和是偶函数;
    \item 两个奇函数的和是奇函数;
    \item 两个偶函数的乘积是偶函数,
    \item 两个奇函数的乘积是偶函数;
    \item 偶函数与奇函数的乘积是奇函数。
\end{enumerate}
\end{enumerate}
\end{ex}

\subsection{函数$y=ax^2\; (a\ne 0)$的图象和性质}
在上一章里,我们研究了$x$的一次函数$y=ax+b$, 现在
我们要研究另一类重要的函数,这类函数的解析式是$x$的二
次式,我们把它叫做$x$的二次函数。

先看几个实例:

\begin{example}
    一石块离地面高为$h$, 设其速度为零,自由地落
到地面,运动的时间为$t$, 如果不考虑空气的阻力,于是$h$与
$t$之间将有函数关系:
\begin{equation}
 h=\frac{1}{2}gt^2   
\end{equation}
这里$g$是重力加速度。
\end{example}

\begin{example}
    农机厂第一个月水泵的产量为50(台),第三个月
的产量为$y$(台),与月平均增长
率之间的关系是:$y=50(1+x)^2$,即:
\begin{equation}
  y=50x^2+100x+50  
\end{equation}
\end{example}

\begin{example}
    设在半径是20厘米的
圆面上,从中心挖去一个半径为$x$厘米的圆面(图5.3),剩下的圆
环面积是$y$平方厘米,那么变量$y$和$x$间有下面的函数关系:
\begin{equation}
    y=400\pi-\pi x^2
\end{equation}
\end{example}

\begin{figure}[htp]
    \centering
\begin{tikzpicture}[>=latex]
 \draw [pattern=north east lines] (0,0) circle (1.5);
    \draw (0,0)[fill=white] circle (1);
\draw[->] (0,0)node[below]{$O$}--node[left]{$x$}(45:1);
\draw (0,0) -- (15:1.5);
\end{tikzpicture}    
    \caption{}
\end{figure}

从上面这些例子可以看出,它们有一个共同特点,那就
是每一个函数关系中,等号右边都是自变量的二次式。这些
函数都可以用
\begin{equation}
    y=ax^2+bx+c
\end{equation}
来表示,这里$a$是不等于零的实数,$b,c$是任意实数。

我们把函数$y=ax^2+bx+c\; (a\ne 0)$叫做$x$的二次函数。
下面我们先从最简单的情况开始研究,即研究在(5.4)中
取$b=0,c=0$的二次函数$y=ax^2\; (a\ne 0)$。

先来看$a>0$的情形。

我们已经在上一节和第三章中画过$y=x^2$和$y=\frac{1}{2}x^2$的图
象,现在把这两个图象和$y=2x^2$的图象都画在同一个坐标系
里。

先作下面的表:
\begin{center}
\begin{tabular}{c|ccccccccc}
    \hline
$x$ &$\cdots$ & $-2$  & $-\tfrac{3}{2}$  & $-1$   & 0   & 1&$\tfrac{3}{2}$& 2&$\cdots$\\
\hline
$y=2x^2$ &$\cdots$ & 8&$\tfrac{9}{2}$ &2  & 0   & 2&$\tfrac{9}{2}$ &8&$\cdots$\\
$y=x^2$ &$\cdots$ & 4&$\tfrac{9}{4}$&1  & 0   & 1&$\tfrac{9}{4}$&4&$\cdots$\\
$y=\tfrac{1}{2}x^2$ &$\cdots$ & $\tfrac{9}{8}$&$\tfrac{1}{2}$  & 0   & $\tfrac{1}{2}$  & $\tfrac{9}{8}$&2&$\cdots$\\
\hline
\end{tabular}    
\end{center}

\begin{figure}[htp]
    \centering
\begin{tikzpicture}[>=latex]
    \draw[->] (-3,0)--(3,0)node[right]{$x$};
    \draw [->] (0,-1)--(0,8)node[right]{$y$};
\foreach \y in {1,2,...,7}
{
    \draw (0,\y)--(.1,\y)node[right]{$\y$};
}
\foreach \x/\xtext in {-4/-2,-3/-\frac{3}{2},-2/-1,-1/-\frac{1}{2},1/\frac{1}{2},2/1,3/\frac{3}{2},4/2}
{
    \draw (\x/2,0)node[below]{$\xtext$}--(\x/2,.1);
}

\draw [domain=-3:3, samples=100, very thick]plot(\x,{\x*\x*0.5});
\draw [domain=-2.2:2.2, samples=100, very thick]plot(\x,{\x*\x});
\draw [domain=-2:2, samples=100, very thick]plot(\x,{\x*\x*2});
\node at (-.25,.25){$O$};
\foreach \x in {.5,1,...,2}
{
    \draw[dashed] (\x,0)--(\x,2*\x*\x);
}
\node at (2,8)[right]{$y=2x^2$};
\node at (2.5,5)[above]{$y=x^2$};
\node at (3,4.5)[right]{$y=\frac{1}{2}x^2$};
\end{tikzpicture}
    \caption{}
\end{figure}

从这个表可以看到,对于同一个$x$值,函数$y=2x^2$所对应
的值是函数$y=x^2$所对应的值的2倍。所以要画出函数$y=2x^2$
的图象,可以用$y=x^2$的图象为基础。

除了让这图象上的原点不动外,其它每一点的纵坐标都
拉长到原来的2倍,这样得到的新的点集就是$y=2x^2$的图象。
作图时我们只描出图象上几个关于$y$轴对称的点,如上表所
示,然后用平滑的曲线把它们连接起来。

同理,要作出函数$y=\frac{1}{2}x^2$的图象,也可以用$y=x^2$的图
象为基础。除了让$y=x^2$的图象上的原点不动外,其它每一点
的纵标都压缩到原来的$\frac{1}{2}$,便得到$y=\frac{1}{2}x^2$的图象。作图时
我们只描出图象上关于$y$轴对称的点,如上表所示,然后用
平滑曲线把它们连接起来。这样,就得到这三个函数的图象如图5.4。

再来看$a<0$的情形:

例如,我们要画函数$y=-x^2$的图象,也可以在函数$y=x^2$
的图象的基础上来研究。

作下面的表:
\begin{center}
\begin{tabular}{c|ccccccccc}
    \hline
$x$ &$\cdots$ & $-2$  & $-\tfrac{3}{2}$  & $-1$   & 0   & 1&$\tfrac{3}{2}$& 2&$\cdots$\\
\hline
$y=x^2$ &$\cdots$ & 4&$\tfrac{9}{4}$&1  & 0   & 1&$\tfrac{9}{4}$&4&$\cdots$\\
$y=-x^2$ &$\cdots$ & $-4$&$-\tfrac{9}{4}$&$-1$  & 0   & $-1$&$-\tfrac{9}{4}$&$-4$&$\cdots$\\
\hline
\end{tabular}    
\end{center}

从这个表可以看到,对于同一个$x$值,函数$y=-x^2$所对
应的值,恰巧是函数$y=x^2$所对应的值的相反数,当$x$遍取一
切实数值时,把函数$y=x^2$图象上的每一点纵坐标改为它的
相反数就得到函数$y=-x^2$的图象上的点,而以$(x,-x^2)$和
$(x,x^2)$为坐标的点是关于$x$轴的对称点,因此把图象$y=x^2$沿
$x$轴折转过来就可以得到$y=-x^2$的图象.$y=-x^2$的图象是在
$x$轴下方,开口向下(图5.5)。

同样,从函数$y=2x^2$和$y=4x^2$的图象可得出函数$y=
-2x^2$和$y=-4x^2$的图象(图5.6)。这些图象在x轴下方,开
口向下。

\begin{figure}[htp]\centering
    \begin{minipage}[t]{0.48\textwidth}
    \centering
\begin{tikzpicture}[>=latex, scale=.6]
      \draw[->] (-4,0)--(4,0)node[right]{$x$};
    \draw [->] (0,-8.5)--(0,8.5)node[right]{$y$};

\foreach \x in {-3,-2,-1,1,2,3}
{
    \draw (\x,0)node[below]{$\x$}--(\x,.2);
}

\draw [domain=-2.7:2.7, samples=100, dashed, very thick]plot(\x,{\x*\x});
\draw [domain=-2.7:2.7, samples=100, very thick]plot(\x,{-\x*\x});
\node at (.4,-.4){$O$};
\node at (2.5,6.25)[right]{$y=x^2$};
\node at (2.5,-6.25)[right]{$y=-x^2$}; 
    \end{tikzpicture}
    \caption{}
    \end{minipage}
    \begin{minipage}[t]{0.48\textwidth}
    \centering
    \begin{tikzpicture}[>=latex, scale=.6]
\draw[->] (-4,0)--(4,0)node[right]{$x$};
    \draw [->] (0,-8)--(0,8)node[right]{$y$};
\foreach \y in {-8,-7,...,-1,1,2,...,7}
{
    \draw (0,\y)--(.1,\y)node[right]{$\y$};
}
\foreach \x in {-3,-2,-1,1,2,3}
{
    \draw (\x,0)node[below]{$\x$}--(\x,.2);
}

\draw [domain=-3:3, samples=100, very thick, dashed]plot(\x,{\x*\x*0.5});
\draw [domain=-2:2, samples=100, very thick, dashed]plot(\x,{\x*\x*2});

\draw [domain=-3:3, samples=100, very thick]plot(\x,{-\x*\x*0.5});
\draw [domain=-2:2, samples=100, very thick]plot(\x,{-\x*\x*2});
\node at (-.4,.4){$O$};
\node at (2,8)[right]{$y=2x^2$};
\node at (3,4.5)[right]{$y=\frac{1}{2}x^2$};
\node at (2,-8)[right]{$y=-2x^2$};
\node at (3,-4.5)[right]{$y=-\frac{1}{2}x^2$};      
    \end{tikzpicture}
    \caption{}
    \end{minipage}
    \end{figure}

总结上面这两种情况,我们知道函数$y=ax^2$的图象是一
条抛物线。

从图象上我们能看到二次函数$y=ax^2$的下面一些性质:

\begin{blk}{性质1}
抛物线$y=ax^2$可向$x$轴左右方向无限延伸。这就是说
函数$y=ax^2$的定义域为实数集$\mathbb{R}$.
\end{blk}

\begin{blk}{性质2}
抛物线$y=ax^2$在$a>0$时,在$x$轴上方且在$y$轴的左
右两侧同时向上无限延伸,这就是说函数$y=ax^2$在$a>0$时,
函数值域为非负实数,即$\mathbb{R}^{+}\cup \{0\}$; 在$a<0$时,抛物线
$y=ax^2$在$x$轴下方且在y轴两侧同时向下无限延伸,这就是说
函数$y=ax^2$在$a<0$时,函数值域为非正实数,即$\mathbb{R}^{-}\cup \{0\}$.
\end{blk}

\begin{blk}{性质3}
抛物线$y=ax^2$在$a>0$时开口向上,在$a<0$时开口
向下,且$|a|$越大开口就越小。
\end{blk}

\begin{blk}{性质4}
抛物线$y=ax^2$关于$y$轴对称,这就是说函数$y=ax^2$是
个偶函数,事实上这个性质是可以证明的,即由于$f(-x)=
a(-x)^2=ax^2=f(x)$, 故函数$y=ax^2$是个偶函数.我们把$y$轴称
为抛物线$y=ax^2$的对称轴,其方程是$x=0$。
\end{blk}

\begin{blk}{性质5}
抛物线$y=ax^2$当$a>0$时,图象在$(-\infty,0)$是下
降的,在$(0,+\infty)$是上升的。这就是说函数$y=ax^2$当$a>0$
时,在$(-\infty,0)$是递减的;在$(0,+\infty)$是递增的(图5.7)。

抛物线$y=ax^2$当$a<0$时,图象在$(-\infty,0)$是上升的,
在$(0,+\infty)$是下降的,这就是说函数$y=ax^2$当$a<0$时,
在$(-\infty,0)$是递增的;在$(0,+\infty)$是递减的(图5.8)。
\end{blk}

\begin{figure}[htp]\centering
    \begin{minipage}[t]{0.48\textwidth}
    \centering
\begin{tikzpicture}[>=latex, scale=.7]
    \draw[->] (-3.5,0)--(3.5,0)node[right]{$x$};
    \draw [->] (0,-1)--(0,5)node[right]{$y$};
    \draw [domain=-3:3, samples=100, very thick]plot(\x,{\x*\x*0.5});
    \node at (2.5,4.5)[above]{$y=ax^2,\; (a>0)$};
    \node at (.3,-.3){$O$};
    \end{tikzpicture}
    \caption{}
    \end{minipage}
    \begin{minipage}[t]{0.48\textwidth}
    \centering
    \begin{tikzpicture}[>=latex, scale=.7]
        \draw[->]  (-3.5,0)--(3.5,0)node[right]{$x$};
        \draw [->] (0,-5)--(0,1)node[right]{$y$};
        \draw [domain=-3:3, samples=100, very thick]plot(\x,{-\x*\x*0.5});    
    \node at (2.5,-4.5)[below]{$y=ax^2,\; (a<0)$};
    \node at (-.3,.3){$O$};
    \end{tikzpicture}
    \caption{}
    \end{minipage}
    \end{figure}

事实上,这个性质是可以证明的,我们只证$a>0$的情
况,$a<0$的情况留给同学们自己证明。

证明:函数 $y=ax^2$当$a>0$时,在$(-\infty,0)$是递减的,在$(0,+\infty)$是递增的。

\begin{proof}
\begin{enumerate}
    \item 设$x_1,x_2\in(-\infty,0)$且$x_1<x_2$,则
    \[\begin{split}
        f(x_2)-f(x_1)&=ax^2_2-ax^2_1=a(x^2_2-x^2_1)\\
&=a(x_2-x_1)(x_2+x_1)
    \end{split}\]
$\because \quad x_1,x_2\in(-\infty,0)$,$\therefore\quad x_1<0,\; x_2<0$

$\therefore\quad x_1+x_2<0$,又$a>0$,$x_2-x_1>0$

$\therefore\quad a(x_2-x_1)(x_2+x_1)<0$,则$f(x_2)<f(x_1)$

$\therefore\quad f(x)$在$(-\infty,0)$上递减。

    \item 设$x_1,x_2\in(0,+\infty)$且$x_1<x_2$,则
    \[f(x_2)-f(x_1)=a(x_2-x_1)(x_2+x_1)>0\]
    $\therefore\quad f(x_1)<f(x_2)$

$\therefore\quad f(x)$在$(0,+\infty)$上递增。
\end{enumerate} 

这样,在$a>0$的情况下,函数$y=ax^2$在$(-\infty,0)$上递减,而在$(0,+\infty)$上递增。
\end{proof}

\begin{blk}{性质6}
对称轴和抛物线的交点叫做抛物线的顶点。抛物线$y=ax^2$的顶点是原点$(0,0)$。这就是说,
有序数对$(0,0)$适合关系$y=ax^2$。
\end{blk}

\begin{blk}{性质7}
    抛物线$y=ax^2$的顶点的特点是:曲线由下降通过它转变到上升$(a>0)$,或者曲线由上升通过它转变到下降的一点$(a<0)$,相应地函数$f(x)=ax^2$($a>0$或$a<0$),在顶点横坐标$x_0=0$的左邻递减(递增),但是在$x_0=0$的右邻改为递增(递减),$x_0=0$是$f(x)=ax^2$在点$x_0=0$的邻近取极小
    (大)值的一点,我们称点$x_0=0$是$f(x)=ax^2$的一个极小(大)
    点,$f(0)=0$叫做$f(x)=ax^2$在极小(大)点$x_0=0$的极小(大)值。
\end{blk}



    在这里需要明确的是,极值都是函数由递增转到递减,或
    由递减转到递增的那一点取得的,而函数的最大值或最小
    值,仅仅指的是函数值的最大或最小,并不要求由递增到递
    减或由递减到递增的转变条件。

    由性质5知道,二次函数$y=ax^2$,仅有一个极值点$x_0=0$,
    在这种情形下,二次函数在点$x_0=0$的极值$f(0)=0$,与它
    的最大值或最小值是一致的。事实上,当$a>0$时,$y=ax^2$对
    于一切$x\in(-\infty,+\infty)$, 都有$f(x)=ax^2\ge 0$, 所以$f(0)=0$
    是最小值;当$a<0$时,$y=ax^2$对于一切$x\in(-\infty,+\infty)$, 都
    有$f(x)=ax^2\le 0$, 所以$f(0)=0$是最大值。对于二次函数,我
    们常用实数平方不小零这个原理来求一般二次函数的极值
    点和极值(也是二次函数的最值)。

从以上讨论可以看到,对抛物线$y=ax^2$主要要掌握三件
东西:对称轴、顶点、开口方向,即$a$的正负。而
这三件东西又都和二次函数的极值点、极值有关,顶点的坐
标确定后,对称轴方程和极值点也就随之求出,故顶点位置
是个关键。

最后我们要指出,上面我们是由抛物线$y=ax^2$的特点来
看二次函数$y=ax^2$的性质的,但今后等我们逐步地学到了更
多的函数性质后,我们应该有意识地学会先研究函数的性
质,再由函数的性质去把握函数图象的大致形状,最后用描
点法画出图象,这时所画的函数图象就较为精确了。


\begin{ex}
\begin{enumerate}
    \item 设从固定的半径$R$的圆板上,挖掉半径为$r$的同心圆板,
    问所剩圆环面积$S$与$r$之间的关系是什么?
    \item 用16m长的篱笆,围成一个一边靠墙的矩形养鸡场,如
    果与墙垂直的一边长是$x$m,面积是$y{\rm m}^2$, 则$x$与$y$之间
    有什么关系?
    \item     用一块矩形空地来做花圃,这块地长20m,宽15m,
    如在四周留出宽度都是$x$米的小路,中间余下种花的空
地面积是$y{\rm m}^2$, 则$y$与$x$之间有什么关系?
\item  汽车在前8秒钟内以匀加速度$a=0.8{\rm m/s}^2$行驶。
\begin{enumerate}
\item 利用公式$s=\frac{1}{2}at^2$, 求$t=3$(s),$t=5.5$(s), $t=2.5$(s) 时所行的路程$s$(m);
\item 画出$s$和$t$之间函数关系的图象;
\item 根据图象,求汽车走5m、10m、15m所需时间。
\end{enumerate}

\item  
在坐标纸上画出函数$y=x^2$图象:
\begin{enumerate}
\item 根据图象,求当$x=1.5$; $x=2.3$; $x=-1.4$时,$y$
的值(精确到0.1);
\item 根据图象,求当$y=2$; $y=3$; $y=4.5$时,对应
的$x$的值(精确到0.1);
\item 利用图象求$\sqrt{5}$, $\sqrt{7}$的值(精确到0.1)。
\end{enumerate}

\item   在同一坐标系里作以下函数的图象:
$$y=3x^2,\qquad y=3x^2,\qquad y=-3x^2,\qquad y=-\frac{1}{3}x^2$$
这些图象有哪些相同的地方?哪些不同的地方?
\item  试证当$a<0$时,二次函数$y=ax^2$在$(-\infty,0)$上递增,
而在$(0,+\infty)$上递减。
\item   证明抛物线$y=ax^2$与抛物线$y=x^2$是位似形。
\end{enumerate}
\end{ex}

\subsection{函数$y=ax^2+bx+c\; (a\ne 0)$的图象}
\subsubsection{函数$y=ax^2+c\; (a\ne 0)$的图象}
为确定起见,假设$c>0$, 从解析式$y=ax^2+c$和$y=ax^2$
明显地看出,对于自变量的相同值,$y=ax^2+c$的对应值,
总可以由$y=ax^2$的对应值加上$c$得到,这表示$y=ax^2+c$的图
象上的一切点比抛物线$y=ax^2$上具有相同横坐标的点高出$c$
个单位。因此,$y=ax^2+c$的图象,可以由抛物线$y=ax^2$沿着$y$
轴向上平移$c$个单位得到。如果$c<0$, 那么$y=ax^2+c$的图象是
由抛物线$y=ax^2$, 沿着$y$轴向下平移$|c|$个单位得到。

例如,我们把函数$y=2x^2$的图象向上移动1个单位,就
可以得到函数$y=2x^2+1$的图象;向下移动3个单位,就可
以得到函数$y=2x^2-3$的图象。

所以函数$y=ax^2+c$的图象仍旧是一条抛物线。当$a>0$时,
开口向上;$a<0$时,开口向下,对称轴方程是$x=0$($y$轴为
对称轴);顶点坐标是$(0,c)$。 当$a>0$时,在$x=0$处取得
$y_{\min}=c$。当$a<0$时,在$x=0$处取得$y_{\max}=c$(注:以后我们用
$y_{\min}$表示$y$的极小值;用$y_{\max}$表示$y$的极大值)。

我们从图形的平移观点确定了$y=kx+b$的图象是一条平
行于直线$y=kx$的直线,也确定了$y=ax^2+c$的图象是抛物线,
它的顶点是$(0,c)$。更一般的结论是:

函数$y=f(x)+b$的图象是由$y=f(x)$的图象沿$y$轴平移而
来的,若$b>0$, 则向上平移$b$个单位。若$b<0$, 则向下平移
$|b|$个单位。

\subsubsection{函数$y=a(x+m)^2$的图象}
例如函数$y=\frac{1}{4}(x+2)^2$, $y=\frac{1}{4}(x-2)^2$都是这种类型
的函数。

我们把上面两个函数的图象与$y=\frac{1}{4}x^2$比较。分别列表如下:
\begin{center}
\begin{tabular}{c|ccccccccccc}
\hline
    $x$ & $-5$& $-4$& $-3$& $-2$& $-1$& 0& 1& 2& 3& 4& 5\\
\hline
$y=\tfrac{1}{4}x^2$ &  $6\tfrac{1}{4}$   &  $4$  &  $2\tfrac{1}{4}$  &  $1$  &  $\tfrac{1}{4}$  &  $0$  &  $\tfrac{1}{4}$  &  $1$ & $2\tfrac{1}{4}$ &4&$6\tfrac{1}{4}$\\
$y=\tfrac{1}{4}(x+2)^2$& $2\tfrac{1}{4}$ &  $1$  &  $\tfrac{1}{4}$  & 0& $\tfrac{1}{4}$  & 1& $2\tfrac{1}{4}$ &4 &  $6\tfrac{1}{4}$  & 9& $12\tfrac{1}{4}$\\ 
$y=\tfrac{1}{4}(x-2)^2$& $12\tfrac{1}{4}$ &  $9$  &  $6\tfrac{1}{4}$  &  $4$  &  $2\tfrac{1}{4}$  &  $1$  &  $\tfrac{1}{4}$  &  $0$  &  $\tfrac{1}{4}$   &1 &  $2\tfrac{1}{4}$\\ 
\hline
\end{tabular}
\end{center}


\begin{figure}[htp]
    \centering
\begin{tikzpicture}[>=latex, scale=.6]
\draw[->] (-7,0)--(7,0)node[right]{$x$};
\draw[->] (0,-1)--(0,9)node[right]{$y$};
\foreach \x in {-5,-4,...,-1,1,2,...,5}
{
    \draw (\x,0)node[below]{$\x$}--(\x,.2);
}
\node at (.3,-.3){$O$};

\draw [domain=-5:5, samples=100, thick] plot(\x,{0.25*\x*\x});
\draw [domain=-7:3, samples=100, very thick] plot(\x,{0.25*(\x+2)*(\x+2)});
\draw (-2,-1)--(-2,8)node[above]{$x=-2$};
\node at (5,6.25)[right]{$y=\frac{1}{4}x^2$};
\node at (3,6.25)[above]{$y=\frac{1}{4}(x+2)^2$};
\end{tikzpicture}
    \caption{}
\end{figure}

从表中可以看出,函数$y=\frac{1}{4}(x+2)^2$在自变量取某一值
$x=x_1$时,所对应的函数值$y=\frac{1}{4}(x+2)^2$, 恰巧和函数$y=\frac{1}{4}x^2$在自变量取值$x=x_1+2$时所对应的函数值$y=\frac{1}{4}(x+2)^2$
相同.这就告诉我们,在函数$y=\frac{1}{4}(x+2)^2$的图象上的点
$\left(x_1,\frac{1}{4}(x_1+2)^2\right)$与函数$y=\frac{1}{4}x^2$的图象上的点$\left(x_1+2,\frac{1}{4}(x_1+2)^2\right)$的纵坐标相等,因此这两点的连接线段平行$x$轴,
并且横坐标是$x_1$的点在横坐标是$x_1+2$的点的右边2个单位
处,利用这个关系,我们只需把函数$y=\frac{1}{4}x^2$的图象上的每一
点向左平移2个单位,就可以得到函数$y=\frac{1}{4}(x+2)^2$的图象
(图5.9)。


同样,函数$y=\frac{1}{4}(x-2)^2$在自变量取某一值$x$时,所对
应的函数值$y=\frac{1}{4}(x_1-2)^2$, 恰巧和函数$y=\frac{1}{4}x^2$在自变量取
值$x=x_1-2$时,所对应的函数值$y=\frac{1}{4}(x_1-2)^2$相同,这就
告诉我们,在函数$y=\frac{1}{4}(x-2)^2$的图象上,横坐标是$x_1$的
点的纵坐标就等于函数$y=\frac{1}{4}x^2$的图象上横坐标是$x_1-2$的
点的纵坐标。利用这个关系,我们只需把函数$y=\frac{1}{4}x^2$的图象
上的每一点向右平移2个单位,就可以得到函数$y=\frac{1}{4}(x-2)^2$的图象(图
5.10)。

\begin{figure}[htp]
    \centering
\begin{tikzpicture}[>=latex, scale=.6]
\draw[->] (-5,0)--(9,0)node[right]{$x$};
\draw[->] (0,-1)--(0,9)node[right]{$y$};
\foreach \x in {-4,-3,...,-1,1,2,...,7}
{
    \draw (\x,0)node[below]{$\x$}--(\x,.2);
}
\node at (.3,-.3){$O$};
\draw [domain=-5:5, samples=100, thick] plot(\x,{0.25*\x*\x});
\draw [domain=-3:7, samples=100, very thick] plot(\x,{0.25*(\x-2)*(\x-2)});
\draw (2,-1)--(2,8)node[above]{$x=2$};
\node at (5,6.25)[above]{$y=\frac{1}{4}x^2$};
\node at (7,6.25)[right]{$y=\frac{1}{4}(x-2)^2$};
\end{tikzpicture}
    \caption{}
\end{figure}


由此可见,函数$y=a(x+m)^2$的
图象,由函数$y=ax^2$的图象沿$x$轴方向左右平移得到。
当$m>0$时,向左平移$m$个单位;当$m<0$时,向右平
移$|m|$个单位。

因此函数$y=a(x+m)^2$的图象,仍然是一条抛物线,$a>0$
时,开口向上;$a<0$时,开口向下,对称轴方程是$x=-m$,
顶点坐标是$(-m,0)$. 当$a>0$时,在$x=-m$处取得$y_{\min}=0$, 当$a<0$时,在$x=-m$处取得$y_{\max}=0$。

应当指出:
\begin{enumerate}
    \item 上述的平移原理可以推广到一般情形。即函数$f(x+m)$的图象,是由函数$f(x)$的图象沿$x$轴方向左右平移得到。当
    $m>0$时,向左平移$m$个单位;当$m<0$时,向右平移$|m|$个单位。
    \item 具体作函数$y=a(x+m)^2$的图象时,不必先作出
    $y=ax^2\; (a\ne 0)$的图象,再作相应的平移得到它,而是先确
    定抛物线$y=a(x+m)^2$的顶点和对称轴,从顶点开始,左右
    取对称的点,再用平滑曲线去连接。
\end{enumerate}

\begin{example}
    作函数$y=2\left(x+2\frac{1}{2}\right)^2$的图象。
\end{example}

\begin{solution}
\begin{enumerate}
    \item 顶点坐标$\left(-2\frac{1}{2},0\right)$, 对称方程$x=-2\frac{1}{2}$,
    $a=2>0$, 开口向上。
    \item 列表:
\begin{center}
\begin{tabular}{c|ccccccccccc}
\hline
$x$&$\cdots$& $-4\tfrac{1}{2}$ &  $-4$ &  $-3\tfrac{1}{2}$ &  $-3$ &  $-2\tfrac{1}{2}$ &  $-2$ &  $-1\tfrac{1}{2}$ &  $-1$ &  $-\tfrac{1}{2}$ &$\cdots$\\
\hline
$y$&$\cdots$ & $8$& $4\tfrac{1}{2}$ &  $2$ &  $\tfrac{1}{2}$ &  $0$ &  $\tfrac{1}{2}$ &  $2$ &  $4\tfrac{1}{2}$ &  $8$ &  $\cdots$ \\
\hline
\end{tabular}
\end{center}

\item 完成图象如图5.11.
\end{enumerate}      

\begin{figure}[htp]
    \centering
\begin{tikzpicture}[>=latex, scale=.6]
\draw[->] (-6,0)--(2,0)node[right]{$x$};
\draw[->] (0,-1)--(0,9)node[right]{$y$};
\foreach \x in {-5,-4,...,-1}
{
    \draw (\x,0)node[below]{$\x$}--(\x,.1);
}
\foreach \y in {1,2,...,8}
{
    \draw (0,\y)--(.1,\y)node[right]{$\y$};
}
\node at (.5,-.5){$O$};
\draw [domain=-4.5:-.5, samples=100, very thick]plot(\x, {2*(\x+2.5)*(\x+2.5)});
\draw[dashed](-2.5,-1)--(-2.5,8)node[above]{$x=-2\frac{1}{2}$};
\end{tikzpicture}
    \caption{}
\end{figure}
\end{solution}


\subsubsection{函数$y=-\frac{1}{4}(x-2)^2-3$的图象}

函数$y=a(x+m)^2+k$的图象,是由函数$y=a(x+m)^2$的图象
沿$y$轴方向上下平移得
到.当$k>0$时,向上平移$k$个
单位;当$k<0$时,向下平移
$|k|$个单位。而$y=a(x+m)^2$的
图象,是由$y=ax^2$的图象沿$x$轴
方向左右平移得到.当$m>0$
时,向左平移$m$个单位;当$m<0$时,向右平移$|m|$个单位,
故函数$y=a(x+m)^2+k$的图象,是由函数$y=ax^2$的图象经上
下左右平移得到。

由此可见,函数$y=a(x+m)^2+k$的图象,也是一条抛
物线,当$a>0$时,开口向上,当$a<0$时,开口向下.对称轴
方程是$x=-m$, 顶点坐标是$(-m,k)$. 当$a>0$时,在$x=
-m$处取得$y_{\min}=k$,当$a<0$时,在$x=-m$处取得$y_{\max}=k$。

\begin{example}
    研究函数$y=\frac{1}{4}(x+2)^2+3$的图象。
\end{example}

\begin{solution}
函数$y=\frac{1}{4}(x+2)^2+3$的图象,就是抛物线
$y=\frac{1}{4}(x+2)^2$向上平移3个单位,也就是抛物线$y=\frac{1}{4}x^2$向
左平移2个单位后,再向上平移3个单位。这条抛物线对你
轴方程是$x=-2$,开口向上,顶点坐标是$(-2,3)$. 在$x=-2$时,取得$y=3$。

\textbf{另解:}$\because\quad (x+2)^2\ge 0$

$\therefore\quad y=\frac{1}{4}(x+2)^2+3\ge 3$, 等式在$x=-2$时
成立,即在$x=-2$时,$y_{\min}=3$

由此得知抛物线$y=\frac{1}{4}(x+2)^2+3$的顶点是$(-2,3)$. 
对称轴方程是$x=-2$,它可以由抛物线$y=\frac{1}{4}x^2$向左平移2
个单位,再向上平移3个单位得来。
\end{solution}


\begin{example}
    作函数$y=-\frac{1}{4}(x-2)^2-3$的图象。
\end{example}

\begin{solution}
\begin{enumerate}
    \item 函数$y=-\frac{1}{4}(x-2)^2-3$的图象是抛物线。由
    $a=-\frac{1}{4}<0$知抛物线开口向下,顶点坐标是$(2,-3)$对称轴
    方程是$x=2$。
  \item 列表:
\begin{center}
\begin{tabular}{c|ccccccccccc}
\hline
$x$&$\cdots$& $-2$ &  $-1$ & 0 &  1 & 2 &  3 &  4 &  5 &  6 &$\cdots$\\
\hline
$y$&$\cdots$ & $-7$& $-5\tfrac{1}{4}$ &  $-4$ &  $-3\tfrac{1}{4}$ &  $-3$ &  $-3\tfrac{1}{4}$ &  $-4$ &  $-5\tfrac{1}{4}$ &  $-7$ &  $\cdots$ \\
\hline
\end{tabular}
\end{center}  

\item 作图如图5.12。
\end{enumerate}

\begin{figure}[htp]
    \centering
\begin{tikzpicture}[>=latex, scale=.6]
\draw[->] (-3,0)--(7,0)node[right]{$x$};
\draw[->] (0,-8)--(0,1)node[right]{$y$};
\foreach \x in {-2,-1,1,2,...,6}
{
    \draw (\x,0)node[below]{$\x$}--(\x,.1);
}
\foreach \y in {-1,-2,...,-7}
{
    \draw (0,\y)--(.1,\y)node[right]{$\y$};
}
\node at (.5,-.5){$O$};
\draw [domain=-2:6, samples=100, very thick]plot(\x, {-.25*(\x-2)*(\x-2)-3});
\draw[dashed](2,1)--(2,-8)node[right]{$x=2$};
\node at (5,-2){$y=-\frac{1}{4}(x-2)^2-3$};
\end{tikzpicture}
    \caption{}
\end{figure}
\end{solution}


\begin{example}
    平移抛物线$y=ax^2$使顶点在$(2,4)$且$y$截距等于$-8$。
\end{example}

\begin{solution}
    平移抛物线$y=ax^2$使顶点在$(2,4)$, 因此新抛物线的方程是:
   \[ y=a(x-2)^2+4\]
    又$y$截距等于$-8$, 即抛物线通过$(0,-8)$点,因此,
\[\begin{split}
    -8&=a(0-2)^2+4\\
4a&=-12\\
a&=-3
\end{split}\]
所求抛物线方程是 $y=-3(x-2)^2+4$。
\end{solution}

\subsubsection{函数$y=ax^2+bx+c$的图象}
现在我们来研究函数$y=ax^2+bx+c$的图象。
因为
\[\begin{split}
ax^2+bx+c&=a\left(x^2+\frac{b}{a}x\right)+c\\
&=a\left[x^2+\frac{b}{a}x+\left(\frac{b}{2a}\right)^2\right]+c-\frac{b^2}{4a}\\
&=a\left(x+\frac{b}{2a}\right)^2+\frac{4ac-b^2}{4a}
\end{split}\]    
所以函数$y=ax^2+bx+c$可以化成$y=a(x+m)^2+k$的形式,这里$m=\frac{b}{2a}$, $k=\frac{4ac-b^2}{4a}$。

由此可知,函数$y=ax^2+bx+c$的图象和函数$y=ax^2$的
图象完全相同,只是位置不同,它们都是抛物线。

抛物线$y=ax^2+bx+c$, 对称轴方程是
$x=-\frac{b}{2a}$,
顶点坐标是$\left(-\frac{b}{2a},\frac{4ac-b^2}{4a}\right)$。
\begin{itemize}
    \item 当$a>0$时,二次函数$y=ax^2+bx+c$开口向上,在
$x=-\frac{b}{2a}$处取得$y_{\min}=\frac{4ac-b^2}{4a}$;
\item 当$a<0$时,函数开口向下,在$x=-\frac{b}{2a}$处取得$y_{\max}=\frac{4ac-b^2}{4a}$。
\end{itemize}

\begin{example}
    指出下面抛物线的开口
方向,顶点坐标和对称轴方程,
并画出图象:
\begin{multicols}{2}
\begin{enumerate}
    \item $y=2x^2+8x+5$
    \item $y=-x^2+2x+1$
\end{enumerate}
\end{multicols}
\end{example}

\begin{solution}
\begin{enumerate}
    \item 配方:
\[\begin{split}
    y=2x^2+8x+5&=2(x^2+4x)+5\\
    &=2(x^2+4x+4)+5-8\\
    &=2(x+2)^2-3
\end{split}\]

性质:$\because\quad a=2>0$

$\therefore\quad $
抛物线开口向上,顶点坐标是$(-2,-3)$, 对称轴方程是$x=-2$。

作图象:
\begin{center}
\begin{tabular}{c|ccccccccccc}
\hline
$x$&$\cdots$& $-4$ &  $-3\tfrac{1}{2}$ & $-3$ &  $-2\tfrac{1}{2}$ & $-2$ &  $-1\tfrac{1}{2}$ &  $-1$ &  $-\tfrac{1}{2}$ &  0 &$\cdots$\\
\hline
$y$&$\cdots$ & $5$& $1\tfrac{1}{2}$ &  $-1$ &  $-2\tfrac{1}{2}$ &  $-3$ &  $-2\tfrac{1}{2}$ &  $-1$ &  $1\tfrac{1}{2}$ &  $5$ &  $\cdots$ \\
\hline
\end{tabular}
\end{center}  
图象如图5.13。

\item 配方:
\[\begin{split}
    y=-x^2+2x+1&=-(x^2-2x-1)\\
    &=-(x^2-2x)+1\\
    &=-(x^2-2x+1)+2\\
&=-(x-1)^2+2
\end{split}\]

性质:$\because\quad a=-1<0$

$\therefore\quad $
抛物线开口向下,顶点坐标是$(1,2)$, 对称轴方程是$x=1$。

作图象:
\begin{center}
\begin{tabular}{c|ccccccccc}
\hline
$x$&$\cdots$& $-2$ &  $-1$ & 0 & 1 & 2 &  3 &  4 &$\cdots$\\
\hline
$y$&$\cdots$ & $-7$& $-2$ &  1 & 2 & 1 &  $-2$ &  $-7$  &  $\cdots$ \\
\hline
\end{tabular}
\end{center}  
图象如图5.14。

\begin{figure}[htp]\centering
    \begin{minipage}[t]{0.48\textwidth}
    \centering
\begin{tikzpicture}[>=latex, scale=.6]
 \draw[->] (-6,0)--(1,0)node[right]{$x$};
\draw[->] (0,-4)--(0,6.5)node[right]{$y$};
\foreach \x in {-4,-3,...,-1}
{
    \draw (\x,0)node[below]{$\x$}--(\x,.1);
}
\foreach \y in {-1,-2,-3,1,2,...,5}
{
    \draw (0,\y)--(.1,\y)node[right]{$\y$};
}
\node at (.5,-.5){$O$};   
\draw[dashed] (-2,-4)--(-2,6)node[above]{$x=-2$};
\node at (-2,-3)[below]{$(-2,-3)$};
\draw [domain=-4.1:0.1, samples=100, very thick]plot(\x, {2*(\x+2)*(\x+2)-3});
    \end{tikzpicture}
    \caption{}
    \end{minipage}
    \begin{minipage}[t]{0.48\textwidth}
    \centering
    \begin{tikzpicture}[>=latex, scale=.6]
 \draw[->] (-3,0)--(5,0)node[right]{$x$};
\draw[->] (0,-8)--(0,3)node[right]{$y$};
\foreach \x in {-2,-1,1,2,3,4}
{
    \draw (\x,0)node[below]{$\x$}--(\x,.1);
}
\foreach \y in {-1,-2,...,-7,1,2}
{
    \draw (0,\y)--(.1,\y)node[right]{$\y$};
}
\node at (.5,-.5){$O$};          
\draw[dashed] (1,-8)--(1,3)node[right]{$x=1$};
\draw [domain=-2:4, samples=100, very thick]plot(\x, {-(\x-1)*(\x-1)+2});
\node at (1,2)[right]{$(1,2)$};
    \end{tikzpicture}
    \caption{}
    \end{minipage}
    \end{figure}

\end{enumerate}
\end{solution}


\begin{example}
\begin{enumerate}
    \item $k$为何值时,抛物线$y=x^2+2kx+1$的顶点在直线$y=x$上?
    \item 说明上述情况下的抛物线是由怎样的抛物线作怎样的平移得到的。
\end{enumerate}
\end{example}

\begin{solution}
\begin{equation}
    y=x^2+2kx+1=(x+k)^2+1-k^2
\end{equation}
抛物线(5.5)的顶点坐标是$(-k,1-k)$. 顶点在直线$y=x$上
的充要条件是:
\[1-k2=-k\quad \Rightarrow\quad k^2-k-1=0\]
$\therefore\quad k=\frac{1\pm\sqrt{5}}{2}$

因此,当$k=\frac{1+\sqrt{5}}{2}$或$k=\frac{1-\sqrt{5}}{2}$时,抛物线(5.5)的顶点在直
线$y=x$上,这时抛物线方程是:
\begin{equation}
y=\left(x+\frac{1+\sqrt{5}}{2}\right)^2-\frac{1+\sqrt{5}}{2}
\end{equation}
和
\begin{equation}
    \begin{split}
 y&=\left(x+\frac{1-\sqrt{5}}{2}\right)^2-\frac{1-\sqrt{5}}{2}\\
&=\left(x-\frac{\sqrt{5}-1}{2}\right)+\frac{\sqrt{5}-1}{2}
    \end{split}
\end{equation}  

抛物线(5.6)是由抛物线$y=x^2$向左移$\frac{1+\sqrt{5}}{2}$个
单位再向下移$\frac{1+\sqrt{5}}{2}$个单位得来;抛物线(5.7)是由抛物线$y=x^2$向右移
$\frac{\sqrt{5}-1}{2}$个单位再向上移$\frac{\sqrt{5}-1}{2}$个单位
得来。
\end{solution}

\section*{习题5.1}
\addcontentsline{toc}{subsection}{习题5.1}

\begin{enumerate}
    \item 把下列各图象画在同一坐标系里进行比较:
\begin{enumerate}
    \item $y=x,\qquad  y=x+2,\qquad  y=x^2-2$
    \item $y=-x^2,\qquad y=-x^2+2,\qquad y=-x^2-2$
\end{enumerate}

    \item 把下列各图象画在同一坐标系里进行比较:
  \[  y=x^2,\quad     y=(x-1)^2,\quad    y=(x-2)^2,\quad     y=(x-3)^2\]
 \[   y=(x+1)^2,\quad     y=(x+2)^2,\quad     y=(x+3)^2\]
    \item 已知函数$y=2(x-3)^2$, 不作出图象而说出:
\begin{enumerate}
\item 图象的顶点,对称轴方程,开口方向;
\item 函数有没有极大值或极小值?这些值是多少?$x$等
    于什么值时,函数有这些值?
    \item $x$在什么区间时函数是递减的,递增的?
    \item $x$取什么值时函数等于零?
    \item 图象和$y$轴的交点的坐标。
\end{enumerate}

    \item 照上题那样研究$y=-3(x+5)^2$。
    \item 求把抛物线$y=2x^2$向左平移3个单位,再向上平移5个
    单位后的抛物线方程。
    \item 求把抛物线$y=\frac{1}{2}x^2$向右平移5个单位,再向下平移3个
    单位后的抛物线方程。
    \item 将抛物线$y=ax^2$向左平移2个单位后,再向上平移3个单
    位且知这个抛物线与$x$轴的一个交点的坐标是$(2,0)$。
    求这个抛物线方程和它与$x$轴的另一个交点的坐标。
    \item 求下列各函数的图象,并指出所求抛物线的开口方向,
    对称轴方程,顶点坐标,又当$x$为何值,二次函数取得
    什么极值:
\begin{multicols}{2}
    \begin{enumerate}
        \item $y=x^2+6x-3$
        \item $y=2x^2-5x+2$
        \item $y=5-x-x^2$
        \item $y=6+12x-3x^2$
        \item $y=-2x^2-5x+7$
        \item $y=3x^2+2x$
        \item $y=\frac{5}{2}x-2-3x^2$
        \item $y=\frac{1}{2}x^2+3x+\frac{5}{2}$
    \end{enumerate}
\end{multicols}

\item 处于静止状态的物体从40米的高处下落,计算$t$秒后物
体离地面的高度$h$米的公式是$h=40-5t^2$
\begin{enumerate}
    \item 经过两秒钟物体离地面的高度是多少?
    \item 经过多少时间物体落到地面?
    \item 求出时间$t$的取值范围。
    \item 作出$h$和$t$之间的函数关系的图象。
\end{enumerate}

\item 一石子从井口落下,经过7秒后听到碰击井底声,如果
石子下降距离$s$(米)与经过的时间$t$(秒)的关系是$s=5t^2$,
又声音在空气中的平均速度是每秒340米,求井深。
\end{enumerate}

\section{和二次函数有关的课题}
\subsection{根据已知条件确定二次函数}
在上一节里我们研究了二次函数的图象和它的性质,现
在我们进一步来研究,如何根据二次函数满足的条件来确定
这个二次函数的问题,下面我们来看几个例题:


\begin{example}
    已经知道函数$y=f(x)$是一个二次函数,并且知
道它的图象通过$A(0,1)$, $B(1,3)$, $C(-1,1)$三点,写出
这个二次函数。
\end{example}

\begin{solution}
二次函数的一般形式是
\begin{equation}
    y=ax^2+bx+c
\end{equation}
要确定这个函数,必须知道二次三项式里三个系数$a,b,c$的
值。由函数图象的定义知道,图象上的点的坐标必适合函数
关系式,现在已知$A,B,C$三点在图象上,故它们的坐标必适
合关系式(5.8), 因此可以列出关于$a,b,c$的三元一次方程
组:
\[\begin{cases}
    1=a\cdot 0^2+b\cdot 0+c\\
3=a\cdot 1^2+b\cdot 1+c\\
1=a(-1)^2+b(-1)+c
\end{cases}\]
即:
\begin{equation}
    \begin{cases}
        c=1\\
a+b+c=3\\
a-b+c=1
    \end{cases}
\end{equation}
解方程组(5.9)得
\[a=1,\qquad b=1,\qquad c=1\]
所求的二次函数是
$y=x^2+x+1$。
\end{solution}
    

\begin{example}
    已知二次函数的图象与$x$轴交于$(-2,0)$和$(1,
0)$两点,又通过点$(3,-5)$, 求这个二次函数的表达式、它
的极值点和极值。
\end{example}

\begin{solution}
二次函数$f(x)=ax^2+bx+c$的图象与$x$轴交于两点
$(-2,0)$, $(1,0)$的意思,是说函数值$f(-2)=0$和$f(1)=
0$。根据余式定理的推论2, $(x+2)(x-1)$必能整除$f(x)=
ax^2+bx+c$. 因此这个二次函数表达式可以写成:
\[f(x)=a(x+2)(x-1)\]
又它的图象通过点$(3,-5)$, 即$f(3)=-5$, 将$x=3$和$x=-5$
代入上式得
\[-5=a(3+2)(3-1)\]
$\therefore\quad a=-\frac{1}{2}$。

因此所求二次函数表达式是:
\begin{equation}
    \begin{split}
        f(x)&=-\frac{1}{2}(x+2)(x-1)\\
        &=-\frac{1}{2}(x^2+x-2)\\
        &=-\frac{1}{2}x^2-\frac{1}{2}+1
    \end{split}
\end{equation}

因为抛物线顶点的横坐标等于对称轴与$x$轴的交点的横
坐标,设顶点横坐标是$x_0$, 于是在$x$轴上有
    \[x_0-(-2)=1-x_0\]
    即:$x_0=\frac{(-2)+1}{2}=-\frac{1}{2}$(图5.15)
代入(5.10)得顶点纵坐标:
\[y_0=f\left(-\frac{1}{2}\right)=-\frac{1}{2}\left(\frac{1}{4}-\frac{1}{2}-2\right)=\frac{9}{8}\]

$\because\quad a=-\frac{1}{2}<0$

$\therefore\quad $在$x_0=-\frac{1}{2}$处$y_{\max}=\frac{9}{8}$。

\begin{figure}[htp]
    \centering
    \begin{tikzpicture}[>=latex]
 \draw[->] (-3,0)--(3,0)node[right]{$x$};
\draw[->] (0,-3)--(0,2)node[right]{$y$};
\draw[dashed] (-.5,-3)--(-.5,2);
\draw [domain=-3:2, samples=100, thick]plot(\x, {-0.5*(\x*\x+\x-2)});
\foreach \x in {-2,1}
{
    \node at (\x, 0)[below]{$\x$};
}
\node at (-.5,0)[below]{$x_0$};
\node at (.25,-.25){$O$};       
    \end{tikzpicture}

    \caption{}
\end{figure}
\end{solution}

\begin{rmk}
    我们在求函数的极值点与极值时没有应用前面给
出的公式,而是借助于二次函
数的图象的顶点在对称轴上。
\end{rmk}

结合图象来研究二次函数
的性质是解决问题的一个途径。

\begin{example}
    已知函数$y=ax^2+bx+c$的图象是以点$(2,3)$为
顶点的抛物线,并且这图象通过$(3,1)$, 写出这个函数。
\end{example}

\begin{solution}
\textbf{解法1:} 抛物线$y=ax^2+bx+c$的顶点坐标是:$\left(-\frac{b}{2a},\frac{4ac-b^2}{4a}\right)$,根据已知条件得:
\begin{align}
    -\frac{b}{2a}&=2\\
\frac{4ac-b^2}{4a}&=3
\end{align}
另外根据抛物线通过点$(3,1)$, 又可得到一个方程:
\begin{equation}
    1=9a+3b+c
\end{equation}
把(5.11), (5.12), (5.13)联立,就可得到关于$a,b,c$的方程组:
\begin{equation}
    \begin{cases}
        -\frac{b}{2a}&=2\\
\frac{4ac-b^2}{4a}&=3\\
9a+3b+c=1
    \end{cases}
\end{equation}
解方程组(5.14), 得
\[a=-2,\qquad b=8,\qquad c=-5\]
所求的二次函数是$y=-2x^3+8x-5$。

\textbf{解法2:} 以点$(m,k)$为顶点的抛物线方程是$y=a(x-m)^2+k$。
这样根据已知条件,就可以写出所求的二次函数是
\begin{equation}
    y=a(x-2)^2+3
\end{equation}
因为点$(3,1)$在图象上,所以把$x=3$, $y=1$代入(5.15)得到
\[1=a(3-2)^2+3\]
由此得$a=-2$. 把它代入(5.15)就得
\[y=-2(x-2)^2+3=-2x^2+8x-5\]
故所求二次函数是$y=-2x^2+8x-5$。

解法2要比解法1方便些。
\end{solution}

由上面讨论可知,要确定一个二次函数需要三个独立条
件去确定系数$a,b,c$, 若条件中有顶点坐标(如例5.13),那么
这个顶点坐标算两个独立条件,这是因为顶点已知的话,所
求抛物线的位置已经确定,所剩就是确定抛物线的开口情况
(即系数$a$),故只要再有一个条件就可确定了。

\begin{ex}
\begin{enumerate}
\item 求经过$A(0,1)$, $B(-1,1)$, $C(1,-1)$三点,且对称
    轴平行于$y$轴的抛物线,并求其顶点坐标和对称轴。
    \item 设有函数$y=x^2+px+q$, 按照下列条件,求$p$和$q$的值。
    \begin{enumerate}
 \item 在$x=2$时,$y=12$, 在$x=-3$时,$y=2$;
    \item 在$x=5$时,函数有极小值$-2$;
    \item 函数的图象和$x$轴的交点的坐标是$(-4,0)$和$(-1,
    0)$.
    \end{enumerate}
   
    \item 设有二次函数$y=ax^2+bx+c$, 按照下列条件,求出$a,b,
    c$的值,然后写出这个二次函数:
    \begin{center}
        \begin{tabular}{c|ccc}
            \hline
$x$&1&2&3\\
            \hline
$y$&0&0&4\\
            \hline
        \end{tabular}
    \end{center}
    函数的图象是以点$A(-1,-8)$为顶点的抛物线,
   并且和$y$轴交于点$B(0,-6)$。

    \item 抛物线$y=x^2+2ax+b$经过点$(2,4)$, 并且其顶点在
    $y-2x-1=0$上,求$a,b$。
    \item 若$f(x)$是二次函数,当$x=\frac{1}{2}$时有极大值25,又方程
    $f(x)=0$的二根平方和等于13, 求$f(x)$。
\end{enumerate} 
\end{ex}

\subsection{二次函数极值}
根据实数的平方不小于零,容易求得二次函数的极值如
下:
\[y=ax^2+bx+c=a\left(x+\frac{b}{2a}\right)^2+\frac{4ac-b^2}{4a}\]
\begin{enumerate}
    \item 如果$a>0$, 那么$a\left(x+\frac{b}{2a}\right)^2\ge 0$
\[y=a\left(x+\frac{b}{2a}\right)^2+\frac{4ac-b^2}{4a}\ge \frac{4ac-b^2}{4a}\]
即当$x=-\frac{b}{2a}$时,函数有极小值,$y_{\min}=\frac{4ac-b^2}{4a}$

\item 如果$a<0$, 那么$a\left(x+\frac{b}{2a} \right)^2\le 0$
\[y=a\left(x+\frac{b}{2a}\right)^2+\frac{4ac-b^2}{4a}\le \frac{4ac-b^2}{4a}\]
即当$x=-\frac{b}{2a}$时,函数有极大值,$y_{\max}=\frac{4ac-b^2}{4a}$
\end{enumerate}

求二次函数的极值有着许多实际的应用,下面我们举几
个例子。

\begin{example}
    某工厂为了存放材料,需要围一个周长为160米
的矩形场地,问矩形的长和宽各取多少米,才能使存放场地
的面积最大?
\end{example}

\begin{solution}
    设一边为$x$m,则另一边长为$(80-x)$m,如果$y{\rm m}^2$
是矩形的面积,则
\[\begin{split}
    y=x(80-x)&=-x2+80x,\qquad (0<x<80)\\
    &=-(x^2-80x+1600-1600)\\
    &=-(x-40)^2+1600
\end{split}\]
因此,当边长是40m的正
方形时,有最大面积1600${\rm m}^2$。
\end{solution}

\begin{example}
    窗的形状是矩形上
面加一个半圆,它的周长等于
6米,要使窗能透过最多的光
线,它的尺寸应该怎样设计?
\end{example}

\begin{solution}
设半圆的半径是$x$米(图5.16),那么半圆的长就是
$\pi x$米,矩形的底$BC$就是$2x$米,而矩形的高$AB$和$CD$就是
$\frac{6-\pi x-2x}{2}$米。

\begin{figure}[htp]
    \centering
\begin{tikzpicture}[>=latex]
\draw (0,0) rectangle (2,3) node [right]{$D$};
\node at (0,0)[below]{$B$};
\node at (2,0)[below]{$C$};
\node at (0,3)[left]{$A$};
\node at (1,3)[below]{$O$};
\draw (0,3) arc (180:0:1);
\draw[->] (1,3)--node[left]{$x$} +(60:1) ;
\draw [<->](0,.5)--node[fill=white]{$2x$}(2,.5);
\end{tikzpicture}
    \caption{}
\end{figure}

设图形的总面积是$y$平方米,那么在开区间$\left(0,\frac{6}{\pi+2}\right)$上,
\[y=\frac{6-\pi x-2x}{2}\cdot 2x+\frac{1}{2}\pi x^2\]
就是
\[\begin{split}
    y&=6x-\left(\frac{\pi}{2}+2\right)x^2\\
    &=-\frac{\pi+4}{2}\left[x^2-\frac{12}{\pi+4}x+\left(\frac{6}{\pi+4}\right)^2-\left(\frac{6}{\pi+4}\right)^2\right]\\
    &=-\frac{\pi+4}{2}\left(x-\frac{6}{\pi+4}\right)^2+\frac{18}{\pi+4}
\end{split}\]
由此可知,当$x=\frac{6}{\pi+4}$的时候,$y_{\max}=\frac{18}{\pi+4}$。

所以尺寸应该这样来设计:半圆的半径是$\frac{6}{\pi+4}\approx 0.84$米,或者说矩形的底边长是$\frac{12}{\pi+4}\approx 1.68$米时,窗能透过最
多的光线。
\end{solution}

\begin{example}
    用一块宽为1.2米的长方形铁板弯起两边做一个
水槽,水槽的横截面为底角是$120^{\circ}$的等腰梯形(图5.17),要
使水權的横截面积最大,它的侧面的宽应该是多少?
\end{example}

\begin{figure}[htp]
    \centering
\begin{tikzpicture}
\draw[thick] (120:2)--(0,0)node[below]{$B$}--(3,0)node[below]{$C$}--+(60:2);
\draw[dashed] (-1,1.732)node[left]{$A$}--(4,1.732)node[right]{$D$};
\draw [dashed](0,0)--(0,1.732)node[above]{$H$};
\draw (0,.7) arc (90:120:.7)node[above=6pt]{$30^{\circ}$};
\draw(2.5,0) arc (180:60:.5)node [left=6pt]{$120^{\circ}$};
\end{tikzpicture}   
    \caption{}
\end{figure}


\begin{solution}
    设侧的宽$AB$为$x$米,作$BH\bot AD$, 则
\[\angle ABH=30^{\circ},\qquad AH=x\sin 30^{\circ}=\frac{1}{2}x\]
\[BH=x\cdot \cos 30^{\circ}=\frac{\sqrt{3}}{2}x,\qquad BC=(1.2-2x)\]
\[AD=1.2-2x+2\cdot \frac{x}{2}=1.2-x\]
所以,水槽的横截面面积为:
\[\begin{split}
    S&=\frac{1}{2}\cdot \frac{\sqrt{3}}{2}x(1.2-2x+1.2-x)\\
    &=\frac{\sqrt{3}}{4}x(2.4-3x)\qquad (0<x<0.6)\\
    &=-\frac{3\sqrt{3}}{4}x^2+\frac{3\sqrt{3}}{5}x=-\frac{3\sqrt{3}}{4}\left(x^2-\frac{4}{5}x\right)\\
&=-\frac{3\sqrt{3}}{4}\left(x^2-\frac{4}{5}x+\frac{4}{25}-\frac{4}{25}\right)\\
&=-\frac{3\sqrt{3}}{4}\left(x-\frac{2}{5}\right)^2+\frac{3\sqrt{3}}{25}
\end{split}\]
所以$x=\frac{2}{5}=0.4$(米)时,水槽有最大的横截面面积$\frac{3\sqrt{3}}{25}$(平方米)。
\end{solution}


\begin{example}
    快艇和轮船分别从$A$地和$C$地同时开出,各沿着
    箭头所指方向航行(图5.18),快艇和轮船的速度分别是40公
    里/小时和16公里/小时,已知$AC=145$公里,经过多少时间
    以后,快艇和轮船之间的距离最短(图中$AC\bot CD$)?
\end{example}

\begin{figure}[htp]
    \centering
\begin{tikzpicture}[>=latex, scale=.9]
\draw (0,0)--(8,0)node[right]{$A$};
\draw[->, very thick] (0,0)--(0,-2)node[below]{$D$};
\draw[->, very thick] (8,0)--(3,0)node[above]{$B$};
\draw (0,-2)--(3,0);
\draw (3,0)--(3,-1); \draw (8,0)--(8,-1); 
\draw[<->] (3,-.5)--node[fill=white]{$40t$}(8,-.5);
\draw (-1,0)--(0,0); \draw (-1,-2)--(0,-2);
\draw[<->]  (-.5,0)--node[fill=white]{$16t$}(-.5,-2);
\node at (-.25,.25) {$C$};
\draw(0,0)--(0,1.2); \draw (8,0)--(8,1.2);
\draw[<->] (0,1) --node[fill=white]{$145$公里}  (8,1) ;

\end{tikzpicture}
    \caption{}
\end{figure}

\begin{solution}
    设经过$t$小时以后,快艇的位置在$B$, 轮船的位置在
$D$. 这时
\[\begin{split}
  AB&=40t \text{(公里)}\\
CD&=16t \text{(公里)}\\
BC&=(145-40t) \text{(公里)}\\  
\end{split}\]
根据勾股定理得
\[BD=\sqrt{BC^2+CD^2}=\sqrt{(145-40t)^2+(16t)^2}\]
现在要使$BD$最短,因$(145-40t)^2+(16t)^2>0$

故只需使被开方数$(145-40t)^2+(16t)^2$有最小的值。

令 $y=(145-40t)^2+(16t)^2\qquad \left(0<t<3\frac{5}{8}\right)$
则有:$$y=1856t^2-11600t+21025$$
这个二次函数在
$t=\frac{11600}{3712}=3\frac{1}{8}$(小时)
的时候有极小值。所以,快艇和轮船分别从$A$地和$C$地开出
$3\frac{1}{8}$(小时)的时候,它们间的距离最短。
\end{solution}

有时要在闭区间$[a,b]$上讨论二次函数的最大值或最小
值,这时要把开区间$(a,b)$内的极值和两端点处的函数值作
比较,再确定出最大值或最小值。


\begin{example}    
设$0\le x\le 3$, 讨论$y=x^2-4x+5$的最大值和最小
值。
\end{example}

\begin{solution}    
$$y=(x-2)^2+1$$

当$x=2$时,$y_{\min}=1$;又当$x=0$时,$y=5$;当$x=3$时,$y=2$。

所以当$x=2$时,$y$取最小值
$y_{\min}=1$;当$x=0$时,$y$取最大值$y_{\max}=5$(图5.19)。

\begin{figure}[htp]
    \centering
\begin{tikzpicture}[>=latex,scale=.8]
\draw[->] (-1,0)--(4,0)node[right]{$x$};
\draw[->] (0,-1)--(0,6)node[right]{$y$};
\foreach \x in {1,2,3}
{
    \draw(\x,0)node[below]{$\x$}--(\x,.1);
    \draw(0,\x)node[left]{$\x$}--(.1,\x);
}
\draw(0,4)node[left]{$4$}--(.1,4);
\draw(0,5)node[left]{$5$}--(.1,5);

\draw [domain=0:3, samples=50, very thick]plot(\x, {(\x-2)*(\x-2)+1});
\draw[dashed] (2,-1)--(2,6);
\draw[dashed] (3,0)--(3,2)--(0,2);
\end{tikzpicture}
    \caption{}
\end{figure}
\end{solution}    

\section*{习题5.2}
\addcontentsline{toc}{subsection}{习题5.2}
\begin{enumerate}
    \item 求下列各函数的最大值和最小值,并且求这时的$x$的值:
\begin{multicols}{2}
\begin{enumerate}
    \item $y=x^2-6x+10$
    \item $y=3x^2+6x-2$
    \item $y=5-4x-x^2$
    \item $y=3+2x-2x^2$
\end{enumerate}
\end{multicols}
    \item 求下列各函数的最大值和最小值:
\begin{enumerate}
    \item $y=2x^2+5x+4,\qquad    0\le x\le 1$
    \item $y=x^2-3x+4,\qquad      0\le x\le 1$
    \item $y=x^2-x+1,\qquad      0\le x\le 1$
\end{enumerate}
\item 求证在周长相同的所有矩形中,正方形面积是最大的。
\item 求证在一边长固定,且周长固定的所有三角形中,等腰
三角形面积是最大的。
\item 把宽40厘米的铁皮制成U字形的雨水槽,要使它的断面
积成为最大,深度应为多少?
\item 一农民准备用篱笆围成一面靠墙的矩形,养鸡场共分五
间,每间大小相等,现有可编120米长的篱笆竹料,养
鸡场每间的长宽各是多少米,养鸡场面积最大。
\item 炮弹以初速度$U_0=600$米/秒,仰角$\theta=30^{\circ}$射出时,上升
的高度$h$与相应的水平距离$x$之间的函数关系是:
\[h=-\frac{1}{54000}x^2+\frac{1}{\sqrt{3}}x\]
试求炮弹能达到的最大高度。
\item 设有边长为$a$的等边三角形,要作内接矩形(如图),问如
何作法使这矩形面积最大?
\item 在半径是20厘米的圆内作一内接矩形,这个矩形的面积
最大可以是多少平方厘米?
\item 给了一个正方形$ABCD$(如图)。由它的各顶点量下相等
的线段$Aa,Bb,Cc,Dd$, 并且以直线连结$a,b,c,d$各
点,问$Aa$多长时正方形$abcd$的面积最小?
\item 已知三角形的两边之和为4, 夹角是$60^{\circ}$, 求:最大
面积;最小周长.

\begin{center}
\begin{tikzpicture}[>=latex]
\begin{scope}
    \draw (0,0)--(2,0)--(1,1.732)--(0,0);
\draw[pattern=north east lines] (.5,0) rectangle (1.5,1.732/2);
\node at (1,-1){第8题};
\draw[|<->|](1-.25,1.732+.1)--node[fill=white]{$a$}(0-.25,0+.1);

\end{scope}    
\begin{scope}[xshift=5cm, yshift=1cm]
\draw (0,0) circle (1.5);
\draw (0,0)node[left]{$O$}--node[below]{20cm}(1.5,0)--(120:1.5)--(180:1.5)--(-60:1.5)--(1.5,0);
\node at (0,-2){第9题};
\end{scope}  
\begin{scope}[xshift=8cm]
    \draw (0,0)node[below]{$A$} rectangle(2.1,2.1)node[above]{$C$};
\node at (2.1,0)[below]{$B$}; 
\node at (0,2.1)[above]{$D$}; 

\node at (1.05,-1){第10题};

\draw (0.9,0)node[below]{$a$}--(2.1,.9)node[right]{$b$}--(1.2,2.1)node[above]{$c$}--(0,1.2)node[left]{$d$}--(.9,0); 
\end{scope}  
\end{tikzpicture} 
\end{center}
\end{enumerate}

\subsection{一元二次方程图象解法}
我们来观察二次函数$y=x^2-2x-3=(x-1)^2-4$的图
象(图5.21),这条抛物线交$x$轴于两点:$A(-1,0)$, $B(3,0)$.

这就是说,当$x=-1$或$x=3$时,函数$y=x^2-2x-3$的
值是0, 换句话说,也就是$x=-1$和$x=3$是方程
$x^2-2x-3=0$
的两个根,这个例子告诉我们,二次方程
$ax^2+bx+c=0$
的求根问题可归结为二次函数$y=ax^2+bx+c$对于怎样的$x$取
值为零的问题,这样的$x$称作二次函数的零点,因此二次方程
$ax^2+bx+c=0$的(实)根就等同于二次函数$y=ax^2+bx+c$的零点,这样我们就可用图象来解二次方程,如

\begin{figure}[htp]\centering
    \begin{minipage}[t]{0.48\textwidth}
    \centering
\begin{tikzpicture}[>=latex, scale=.7]
\draw[->] (-2,0)--(4,0)node[right]{$x$};
\draw[->] (0,-4.5)--(0,4)node[right]{$y$};
\foreach \x in {1,2,3,-1}
{
    \draw(\x,0)node[below]{$\x$}--(\x,.1);
    \draw(0,\x)node[left]{$\x$}--(.1,\x);
}       

\draw[dashed] (1,-4.5)--(1,4);
\draw [domain=-1.5:3.5, samples=100, very thick] plot(\x, {(\x-1)*(\x-1)-4});
\node at (1,-4)[below]{$(1,-4)$};
\node at (-.25,-.25){$O$};
    \end{tikzpicture}
    \caption{}
    \end{minipage}
    \begin{minipage}[t]{0.48\textwidth}
    \centering
    \begin{tikzpicture}[>=latex, scale=.7]
\draw[->] (-2,0)--(4,0)node[right]{$x$};
\draw[->] (0,-4)--(0,4)node[right]{$y$};
\foreach \x in {1,2,3,-1}
{
    \draw(\x,0)node[below]{$\x$}--(\x,.1);
    \draw(0,\x)node[left]{$\x$}--(.1,\x);
}          

\draw[dashed] (1,-4)--(1,4);
\draw [domain=-1.5:3.5, samples=100, very thick] plot(\x, {(\x-1)*(\x-1)-3});
\node at (.3,-.3){$O$};
    \end{tikzpicture}
    \caption{}
    \end{minipage}
    \end{figure}


\begin{example}
    用图象法解一元二次方程
$x^2-2x-2=0$。
\end{example}

\begin{solution}
    先作出函数$y=x^2-2x-2=(x-1)^2-3$的图象(图
5.21),从图中读出二次函数零点:$x_1\approx-0.7$; $x_2\approx 2.7$, 这
就是二次方程$x^2-2x-2=0$的两个近似实根。
\end{solution}

从函数图象的观点看来,二次函数$y=ax^2+bx+c$的零
点,就是它的图象与$x$轴的交点的横坐标,因此二次方程
$ax^2+bx+c=0$能否有实数解,或者有一个(实)重根,或者
有两个不同的(实)根,实际上就等于抛物线$y=ax^2+bx+c$是
否与$x$轴相交,或者与$x$轴相切于一点;或者与$x$轴相交于两
个不同的点(这两个点必对称地位于抛物线对称轴的左、右两
侧)。

当$a>0$时,抛物线$y=ax^2+bx+c$是开口向上的,并且
其顶点是它的最低点,因此二次方程$ax^2+bx+c=0$能否有
解,有一解或二解,完全取决于抛物线$y=ax^2+bx+c$的顶
点位于$x$轴的上方,位于$x$轴上,或位于$x$轴的下方。

抛物线$y=ax^2+bx+c$的顶点坐标是$\left(-\frac{b}{2a},\frac{4ac-b^2}{4a}\right)$
由于$a>0$, 因此抛物线的顶点位于$x$轴的上方,位于$x$轴上,
或者位于$x$轴的下方,分别取决于$b^2-4ac<0$, $b^2-4ac=0$,
或$b^2-4ac>0$ (图5.22),以前我们曾谈过:二次方程$ax^2+bx+c=0$能否有解,取决于判别式$b^2-4ac$的符号,现在我们又
可以看到,抛物线$y=ax^2+bx+c$能否与$x$轴有交点,也取决
于判别式$b^2-4ac$的符号。

当$a<0$时,情况与上面相同,我们留给同学去考虑。

\begin{figure}[htp]
    \centering
\begin{tikzpicture}[>=latex, scale=.8]
\begin{scope}
    \draw[->] (-1,0)--(5,0)node[right]{$x$};
    \draw[->] (0,-4)--(0,6)node[right]{$y$};
    \draw[domain=.2:4.2, samples=50, thick]plot(\x, {(\x-2.2)*(\x-2.2)*.8});
    \draw[domain=.2:4.2, samples=50, thick]plot(\x, {(\x-2.2)*(\x-2.2)*.8+2.5});
    \draw[domain=.2:4.2, samples=50, thick]plot(\x, {(\x-2.2)*(\x-2.2)*.8-3});
    \draw[dashed] (2.2,-4)--(2.2,6);
    \node at (-.3,-.3){$O$};
    \node at (2,-4.5){$a>0$};
\node at (3.2,-3)[right]{$b^2-4ac>0$};
\node at (3.2,1)[right]{$b^2-4ac=0$};
\node at (3.2,3.5)[right]{$b^2-4ac<0$};
\end{scope}

\begin{scope}[xshift=8cm, yshift=2cm]
    \draw[->] (-1,0)--(5,0)node[right]{$x$};
    \draw[->] (0,-6)--(0,4)node[right]{$y$};
    \draw[domain=.2:4.2, samples=50, thick]plot(\x, {-(\x-2.2)*(\x-2.2)*.8});
    \draw[domain=.2:4.2, samples=50, thick]plot(\x, {-(\x-2.2)*(\x-2.2)*.8-2.5});
    \draw[domain=.2:4.2, samples=50, thick]plot(\x, {-(\x-2.2)*(\x-2.2)*.8+3});
    \draw[dashed] (2.2,-6)--(2.2,4);
    \node at (-.3,-.3){$O$};
    \node at (2,-6.5){$a<0$};
    \node at (3.2,-4.5)[right]{$b^2-4ac<0$};
    \node at (3.2,-.7)[right]{$b^2-4ac=0$};
    \node at (3.2,2.5)[right]{$b^2-4ac>0$};
\end{scope}
\end{tikzpicture}
    \caption{}
\end{figure}

前面考虑二次方程$ax^2+bx+c=0$的根为二次函数$y=
ax^2+bx+c$的图象与$x$轴的交点的横坐标,我们也可用另外一种观点来考虑。由于
$ax^2+bx+c=0$与$ax^2=-bx-c$
的解是一样的,故我们可以考虑抛物线
$y=ax^2$与直线$y=-bx-c$的交点。

为此,在同一坐标系里,把抛物线和直线画出来,如果直
线与抛物线有交点(至多两个)那么交点的横坐标就是二次方
程的根。这是因为,如果交点是$(x_1,y_1)$, 那么必有$y_1=ax_1^2$和
$y_1=-bx_1-c$, 从而有$ax_1^2=-bx_1-c$, 即$ax_1^2+bx_1+c=0$。
如果没有交点,则二次方程
没有解。当我们使用函数图象求解二次方程近似根时,
这种方法是比较方便的,因
为$y=ax^2$与$y=-bx-c$都比
较容易描绘,如果我们还是
以例5.19的二次方程$x^2-2x-2=0$为例,如图5.23,
求出近似根$x_1\approx -0.7$,$x_2\approx 2.7$。

\begin{figure}[htp]
    \centering
\begin{tikzpicture}[>=latex, scale=.7]
\draw[->] (-4,0)--(4,0)node[right]{$x$};
\draw[->] (0,-1)--(0,9)node[right]{$y$};
\foreach \x in {1}
{
    \draw(\x,0)node[below]{$\x$}--(\x,.1);
    \draw(0,\x)node[left]{$\x$}--(.1,\x);
}          

\draw [domain=-3:3, samples=100, very thick] plot(\x, {\x*\x});
\draw [domain=-1.5:3.5, samples=10, very thick] plot(\x, {2*\x+2});
\node at (-1.5,-1)[below]{$y=2x+2$};
\node at (-1.7,8){$y=x^2$};
\node at (2.7,7.4)[right]{$(2.7,7.4)$};
\node at (-.7,.6)[left]{$(-0.7,0.6)$};
\node at (.3,-.3){$O$};
    \end{tikzpicture}
    \caption{}
\end{figure}




\begin{example}
    
\end{example}

\begin{solution}
    
\end{solution}


\begin{example}
    
\end{example}

\begin{solution}
    
\end{solution}

\begin{example}
    
\end{example}

\begin{solution}
    
\end{solution}


\begin{example}
    
\end{example}

\begin{solution}
    
\end{solution}

\begin{example}
    
\end{example}

\begin{solution}
    
\end{solution}








\begin{example}
    
\end{example}

\begin{solution}
    
\end{solution}


\begin{example}
    
\end{example}

\begin{solution}
    
\end{solution}


\begin{example}
    
\end{example}

\begin{solution}
    
\end{solution}

\begin{example}
    
\end{example}

\begin{solution}
    
\end{solution}


\begin{example}
    
\end{example}

\begin{solution}
    
\end{solution}

\begin{example}
    
\end{example}

\begin{solution}
    
\end{solution}















\begin{ex}
\begin{enumerate}
    \item 证明函数$y=x^3+x$处处递增。
    \item 求下面函数的极大值和极小值:
\begin{enumerate}
    \item $y=2x^3-9x^2+12x+5$
    \item $y=4x^3-6x^2-9x+1$
\end{enumerate}
    \item 求内接于半径为$R$的球中的直圆柱的最大体积。
    \item 某自来水厂设计一座圆柱形自来水塔,它的全面积合计
    为$150\pi$平方米,要使这座水塔有最大的容量,应该怎样
    定出水塔的高和底面圆的半径,才能达到最大容量的要
    求,并算出水塔的容量。
\end{enumerate}
\end{ex}

\section*{复习题五}
\addcontentsline{toc}{section}{复习题五}
\begin{enumerate}
    \item 指出下列函数图象的顶点,对称轴和开口方向:
\begin{multicols}{2}
\begin{enumerate}
    \item $y=2x^2-16x+32$
    \item $y=4x^2+12x-7$
    \item $y=-2x^2-4x+6$
    \item $y=1+4x-x^2$
    \item $y=(1-x)^2+2$
    \item $y=2(x-2)(x+3)$
\end{enumerate}
\end{multicols}
   \item  已知下列函数:
\begin{multicols}{2}
\begin{enumerate}
\item $y=x^2+4x-5$
\item $y=x^2-4x+1$
\item $y=6-4x-2x^2$
\item $y=-\frac{1}{4}x^2+x-1$
\item $y=(x-2)(x-3)$
\item $y=(2x+1)(3-x)$
\end{enumerate}
\end{multicols}
   问: \begin{enumerate}
       \item $x$取什么值时,$y=0$? $y>0$? $y<0$?
       \item 当$x$取什么值时,函数是增函数,减函数,并求出
   极值点和极值。
   \item 画出它们的图象。
   \end{enumerate}
\item 把24, 27, 34, 37各分成两个正整数,使它们的乘积最
大。
\item 抛物线$y=-\frac{1}{2}x^2+\frac{9}{2}x-6$, 当$x$的取值在什么范围
内,位于直线$y=\frac{1}{2}x$的下方。
\item 平移抛物线$y=\frac{1}{2}x^2$, 使顶点到$P(t,t^2)$点,并通过$A(2,
4)$点,求平移后的抛物线方程。


\item 求函数$y=\frac{1}{\sqrt{x^2+4x+7}}$的定义域和值域。

\item 通过作函数$y=|x^2+2x-3|$的图象,写出函数的递增
区间和递减区间,并给出函数的极值点和极值。
\item 按照自变量$x$可取值的范围,画出函数$y=\sqrt{-x^2+3x-2}$
的图象,并求最大值和最小值。
\item 已知二次函数$y=x^2+px+q$的图象和$x$轴交于$(1,0)$和
$(-6,0)$两点,求$p,q$的值。
\item 已知二次函数$y=ax^2+bx+c$, 按照下面的条件确定$a,
b,c$的值。
\begin{enumerate}
\item $x=6$时,$y=0$; $x=4$时,函数有极小值$-8$;
\item $x=\frac{1}{2}$时,函数有极大值25; $x=0$时,$y=24$;
\item 顶点是$(6,-12)$, 开口向上,且和$x$轴的一个交点是
$(8,0)$;
\item 顶点是$(2,-7)$, 开口向下,且和$y$轴有一个交点
$(0,-15)$。
\end{enumerate}

\item  $m$为何值使二次方程$x^2+(m-2)x-(m+3)=0$的二
根平方和有最小值。
\item 求$m$的范围,使方程$x^2+(m-2)x-(m+3)=0$的二根
都是正数。
\item  求$p$的范围,使方程$px^2-x+p=0$的一个根在区间$(0,1)$
内,又它的另一个根在什么范围内。
\item 用一根长为6米的木料,做一个分成上下两部分的矩形
窗框(如图),问窗的宽和高各取多
少米时,才能使通过窗的光线最多。
\begin{figure}[htp]
    \centering
    \begin{tikzpicture}
\draw (0,0) rectangle (3,4);
\draw (0,0) rectangle (3,3.2);        
    \end{tikzpicture}
    \caption{第14题}
\end{figure}

\item 扇形的周长为4厘米,问这扇形
所在圆的半径$r$为多少时,扇形面积最大。
\item 将长为$\ell$的铁丝剪成两段,各围成长与宽之比为2:1的矩形,问面积
之和的最小值是多少?
\item 在半径为$R$的半圆中,作一个内接等腰梯形,使它的
一个底边是半圆的直径,问其它三边为何值时等腰梯形
周长最大。
\item 求顶点在原点、对称轴为$y$轴的抛物线,使它和直线
$x+y=1$在第一象限的交点的坐标的乘积最大。
\item 抛物线$y=4-x^2$和直线$y=3x$相 交于点$A,B$,且使
$\triangle PAB$的面积为最大时,点$P$的坐标是$(p,q)$, 求$p$的值和
最大面积。
\item 求下列函数的极大值或极小值:
\begin{multicols}{2}
\begin{enumerate}
    \item $y=\frac{1}{x^2+6x+11}$
    \item $y=\frac{1}{-2x^2+8x-9}$
    \item $y=\frac{x^2}{x^2-2x-3}$
    \item $y=\frac{4x}{x-2}+\frac{x}{2}$
\end{enumerate}
\end{multicols}

\item 在闭区间$[-2,3]$上,求下列各函数的最大值和最小
值。
\begin{multicols}{2}
    \begin{enumerate}
        \item $f(x)=x^3-3x+1$
        \item $f(x)=-x^3+3x^2+2$
    \end{enumerate}
    \end{multicols}
\end{enumerate}
  \chapter{任意角的三角函数}

\section{弧和角的概念及其度量}

\subsection{任意大小的角}
在平面几何里,每一个角可以看作是由一条射线绕着它
的端点旋转而形成的。射线的端点叫做角的顶点,射线旋转的
开始位置叫做角的始边,终止位置叫做角的终边。如图6.1所
示的角$\alpha$是射线$OA$绕着端点$O$,按着箭头所示的方向旋转
到$OB$所形成。$O$点是角$\alpha$的顶点,射线$OA$和$OB$分别是
角$\alpha$的始边和终边。
\begin{figure}[htp]
    \centering
\begin{tikzpicture}[>=latex]
    \draw (0,0)node[left]{$O$}--(3,0)node[right]{$A$};
    \draw (0,0)--(50:3) node [right]{$B$};
\draw[->, thick] (1,0) arc (0:50:1);
\node at (25:1.25){$\alpha$};
\end{tikzpicture}
    \caption{}
\end{figure}

\begin{figure}[htp]
    \centering
    
        \begin{tikzpicture}[>=latex, scale=.8]
            \begin{scope}
            \draw (0,0)node[left]{$O$}--(3,0)node[right]{$A$};
            \draw (0,0)--(-30:3) node [right]{$B$};
        \draw[->, thick] (1,0) arc (0:360-30:1);
        \node at (1.5,-2){(a)};
    \end{scope}
    \begin{scope}[xshift=5cm]
            \draw (0,0)node[left]{$O$}--(3,0)node[above]{$A$};
          \node at   (3,0)[below]{$B$};
        \draw[thick] (.3,0) arc (0:180:.3);
        \draw[thick] (-.3,0) arc (180:360:.5);
        \draw[thick] (.7,0) arc (0:180:.7);
        \draw[thick,->] (-.7,0) arc (180:360:.9);

        \node at (1.5,-2){(b)};
    \end{scope}
    \begin{scope}[xshift=10cm]
        \draw (0,0)node[left]{$O$}--(3,0)node[right]{$A$};
        \draw (0,0)--(50:3) node [right]{$B$};
        \draw[thick] (50:.3) arc (50:180:.3);
        \draw[thick] (-.3,0) arc (180:360:.5);
        \draw[thick] (.7,0) arc (0:180:.7);
        \draw[thick,->] (-.7,0) arc (180:360+60:.9);
    \node at (1.5,-2){(c)};
\end{scope}
        \end{tikzpicture}
    \caption{}
\end{figure}

射线旋转所形成的角可以是任意大小的角,这也就是
说,一条射线旋转所成的角可以是锐角,钝角,平角,也可
以大于一个平角(图6.2a),也可以绕端点若干周后和开始
的位置重合(图6.2b),也可以旋转若干周又一周的部分
(图6.2c)。

我们还看到射线有两种相反的旋转方向:逆时针方向和
顺时针方向。为了加以区别,我们把按逆时针方向旋转所形
成的角叫做正角,按顺时针方向旋转所形成的角叫做负角。
例如图6.3中以$OA$为始边的角$\alpha=210^{\circ}$, $\beta=-150^{\circ}$, $\gamma=-660^{\circ}$.
\begin{figure}[htp]
    \centering
\begin{tikzpicture}[>=latex, scale=.8]
\draw (0,0)node[below]{$O$}--(4,0)node[right]{$A$};
\draw (0,0)--(-150:4)node[left]{$B_1$};    
\draw (0,0)--(60:4)node[right]{$B_2$};    
\draw[->, very thick] (.5,0) arc (0:-150:.5);
\node at (-75:.8){$\beta=-150^{\circ}$};

\draw[->, very thick] (.85,0) arc (0:210:.85);
\node at (105:1.2){$\alpha=210^{\circ}$};

\draw[thick] (2,0) arc (0:-150:2) ;
\draw[thick] (-150:2) arc (-150:-150-150:2.1) ;
\draw[thick] (60:2.2) arc (60:-150:2.2);
\draw[->, thick] (-150:2.2) arc (-150:-150-149:2.3);
\node at (0,2.6){$\gamma=-660^{\circ}$};


\end{tikzpicture}
    \caption{}
\end{figure}

如果射线$OA$没有
作任何旋转,仍留在开
始的位置,那么我们也
把它看成一个角,叫做\textbf{零角}。

这样,我们把角的概念推广到了任意的
角,包括正角、负角和零角。

我们这样引进来的广义角的概念,是由下列三个因素组
成:“始边”、“旋转方向”、“旋转量”。旋转量的大小
通常是以度数或弧度数来表示。

和角的概念对应的是弧的概念。

我们已经讨论了任意大小的角,现在再来讨论任意大小
的弧。

圆弧可以看做是射线上的一点(不与端点重合),随着
射线旋转所形成的轨迹。

\begin{figure}[htp]
    \centering
\begin{tikzpicture}
\draw (4,0)node[right]{$A$}--(0,0)node[left]{$O$}--(40:4)node[above]{$B$};
\draw (1.5,0)node[below]{$M$} arc (0:40:1.5) node[above]{$M'$};
\end{tikzpicture}
    \caption{}
\end{figure}




如图6.4所示,弧$\wideparen{MM'}$是射线$OA$上的$M$点,随着射
线$OA$旋转,由起始位置到$OB$时所形成的轨迹。显然,对
于任意角$\alpha$的终边的每个位置,都有$M$点划出的弧$\wideparen{MM'}$
和它对应。和规定角的正负一样,我们规定:当射线上的一
点按逆时针方向旋转时,该点所划出的弧为正的;按顺时针
方向旋转时,该点所划出的弧为负的,这样规定就使正、负
角和正、负弧对应起来。

再来规定角和它所对应弧的量数。在平面几何里,我们曾规定把圆
周分成360等分,每一份叫做一度的弧,一度弧所对的圆心角叫做一度的
角。因此,一个圆弧含有多少度、分、
秒,它所对圆心角也含有多少度、
分、秒,即弧与其所对应的圆心角有完全相同的量数。例如
圆心角是$500^{\circ}$的角时,它所对的弧就是$500^{\circ}$的弧;圆心角是$-300^{\circ}$时,它所对的弧也是$-300^{\circ}$.

\subsection{角的度量}
角的度量是取一个确定的角作为度量单位,利用它来量
所有的角。用周角的$\frac{1}{360}$
作为度量单位的叫做“度”。在高
等数学和其它基础科学理论系统中也常用弧度作为度量圆弧
和角的单位。

在弧度制中,取等于半径长的圆弧作为单位弧长。这样
的弧叫做一弧度弧。用一弧度弧度量同一个圆上的圆弧所得
到的量数叫做这个圆弧的弧度数,这也就是说给定圆弧的弧
度数等于圆弧的弧长和半径的比值:
\begin{equation}
    \alpha=\frac{\ell}{R}
\end{equation}
这里$\alpha$是圆弧的弧度数,$\ell$是弧长,$R$是圆的半径。

我们指出圆心角所张的圆弧的弧度数由这个角的大小决
定,而和圆的半径长短无关。

事实上,从几何里知道,在
圆心角相同时,两个圆上的弧长
的比等于它们的半径长的比(图6.5), 即
\[\frac{\wideparen{A_1B_1}}{\wideparen{A_2B_2}}=\frac{R_1}{R_2}\]
或\[\frac{\wideparen{A_1B_1}}{R_1}=\frac{\wideparen{A_2B_2}}{R_2}\]
这就是说,两个圆弧$\wideparen{A_1B_1}$, $\wideparen{A_2B_2}$的弧度数是相同的。

\begin{figure}[htp]
    \centering
\begin{tikzpicture}[scale=.8]
\draw (0,0) circle (2);
\draw (0,0) circle (3);
\draw (0,0)--(40:3);
\draw (0,0)--(140:3);

\node at (40:2)[right]{$A_2$};
\node at (40:3)[right]{$A_1$};
\node at (140:2)[left]{$B_2$};
\node at (140:3)[left]{$B_1$};
\node at (0,0)[below]{$O$};

\draw[decorate,decoration={brace,raise=1pt}] (0,0)--node[above=3pt]{$R_1$}(40:3);
\draw[decorate,decoration={brace,raise=1pt}] (140:2)--node[above=3pt]{$R_2$}(0,0);
\end{tikzpicture}
    \caption{}
\end{figure}

因此,一个圆心角所对的弧的弧度数可以表示这个角的
大小,我们也把圆心角所对的圆弧的弧度数称为这个角的弧
度数。

\begin{blk}{定义}
    以一个角为圆心角,这个角所对的弧的长和这个
弧的半径长之比,叫做这个角的弧度数。
\end{blk}

当弧长等于半径时,这个比值等于1, 因此,在弧度制
里,度量一个角时,我们规定:

长度等于半径的圆弧所对的圆心角叫做1弧度角。换言
之,一弧度圆弧所对的圆心角叫做1弧度角。

这样,由(6.1)推得
\begin{equation}
    \ell=aR
\end{equation}
即圆弧长等于这圆弧的弧度数(或这弧所对圆心角的弧度数)
和半径长的乘积。特别地,单位圆上的弧长等于它的弧度数。

利用(6.1)还可以接计算一些特殊角的弧度数。

当弧长等于圆周长$C=2\pi R$时,这个比值等于$2\pi$, 因
此,
\[\begin{split}
    \text{周角}&=  \frac{2\pi R}{R}=2\pi \text{弧度}\\
    \text{平角}&=\frac{1}{2} \text{周角}=\pi \text{弧度}\\
    \text{直角}&=\frac{1}{4}\text{周角}=\frac{\pi}{2}\text{弧度}\\
    45^{\circ}&=\frac{1}{2}\text{直角}=\frac{\pi}{4}\text{弧度}\\
    30^{\circ}&=\frac{1}{3}\text{直角}=\frac{\pi}{6}\text{弧度}\\
    60^{\circ}&=\frac{1}{3}\text{平角}=\frac{\pi}{8}\text{弧度}\\
\end{split}\]

\begin{rmk}
角的量数是以弧度数表示的,通常只写出数值不
写出单位,以后我们都将单位“弧度”二字省略不写。例如
平角$=\pi$弧度就写成平角$=\pi$。但是千万不要误解平角就是
圆周率$3.1415926\cdots$。
\end{rmk}

度与弧度的互化。

因为平角$=180^{\circ}=\pi$, 所以$1^{\circ}=\frac{\pi}{180}
\approx 0.017453$。$A^{\circ}$的角相应的弧度数:
\[\begin{split}
    \alpha&=\frac{A\pi}{180}\\
    1'&=\left(\frac{1}{60}\right)^{\circ}=\frac{1}{60}\left(\frac{\pi}{180}\right)\approx 0.00029088\\
    1\text{(弧度)}&=\frac{180^{\circ}}{\pi}=57.295^{\circ}\approx 3438'\approx 206265''
    = 57^{\circ}17'45''
\end{split}\]
$\alpha$弧度的角相应的度数:
\[A^{\circ}=\frac{a\cdot 180^{\circ}}{\pi}\]

下表给出一些常见角的弧度和它们的近似值:
\begin{center}
\begin{tabular}{cccccccc}
\hline
度 & $30^{\circ}$ & $45^{\circ}$ & $60^{\circ}$ & $90^{\circ}$ & $180^{\circ}$ & $270^{\circ}$ & $360^{\circ}$\\
\hline
弧度 & $\frac{\pi}{6}$   & $\frac{\pi}{4}$   & $\frac{\pi}{3}$   & $\frac{\pi}{2}$   & $\pi$   & $\frac{3}{2}\pi$  & $2\pi$\\
近似值 & 0.5236   & 0.7854   & 1.0472   & 1.5708   & 3.1416   & 4.7124   & 6.2832\\
\hline
\end{tabular}
\end{center}

\begin{example}
    化$67^{\circ}30'$为弧度。
\end{example}

\begin{solution}
\[67^{\circ}30'=67.5^{\circ}=\frac{\pi}{180}\x 67.5=\frac{3}{8}\pi \; \text{(弧度)}\]
\end{solution}

\begin{example}
    化$\frac{3}{5}\pi$弧度为度。
\end{example}

\begin{solution}
  \[\frac{3}{5}\pi= \frac{180^{\circ}}{\pi}\x\frac{3}{5}\pi=108^{\circ}  \]  
\end{solution}

\begin{example}
    两皮带轮的半径$R_1=20$, $R_2=30$, 求它们的转速
    之比(图6.6)。
\begin{figure}[htp]
    \centering
    \begin{tikzpicture}[>=latex]
\draw (0,0) circle (1);
\draw (4,0) circle (1.5);
\node at  (0,0) [below]{$O_1$};
\node at  (4,0) [below]{$O_2$};
\draw (140:1)node[left]{$S_1$}--(0,0)--(97.2:1)--+(7.2:3.97)--(4,0)--+(130:1.5)node[left]{$S_2$};
\draw (-97.2:1)--+(-7.2:3.97);
\draw[->] (97.2:.5) arc (97.2:140:.5);
\draw[->] (4,0)--+(97.2:.8) arc (97.2:130:.8);
\node at (-.25,.5)[above]{$\alpha_1$};
\node at (4-.3,.8)[above]{$\alpha_2$};

\draw[<-]  (1,1.35)--(2,1.5);
\draw [->] (1,-1.35)--(2,-1.5);

    \end{tikzpicture}
    \caption{}
\end{figure}
\end{example}

\begin{solution}
    因为在相同的时间内,两轮周上转过的弧长相等,
    即$S_1=S_2$, 在弧度制下:
\[S_1=\alpha_1 R_1,\qquad S_2=\alpha_2 R_2\]
$\therefore\quad \alpha_1R_1=\alpha_2R_2 \quad \Rightarrow\quad \frac{\alpha_1}{\alpha_2}=\frac{R_2}{R_1}=\frac{30}{20}$

$\therefore\quad \alpha_1:\alpha_2=3:2 $
\end{solution}

\begin{example}
    地球的半径为6400公里,在同一经线上,甲、乙两
地的距离为150公里,试求甲、乙两地纬度差。
\end{example}


\begin{solution}
设$\theta$为甲、乙两地纬度差,则
\[\begin{split}
    \theta &= \frac{150}{6400}\approx 0.0234\\
    &=0.0234\x\frac{180^{\circ}}{\pi}\\
    &\approx 0.0234\x 3438'\\
    &\approx 13^{\circ}14'
\end{split}\]

答: 两地纬度差为$13^{\circ}14'$。
\end{solution}

\begin{ex}
\begin{enumerate}
    \item 把下列各角的度数化为弧度数:
\begin{multicols}{3}
    \begin{enumerate}
    \item $2^{\circ}$
    \item $5^{\circ}$
    \item $7^{\circ}30'$
    \item $12^{\circ}30'$
    \item $22.5^{\circ}$
    \item $200^{\circ}$
    \item $320^{\circ}$
    \item $14^{\circ}24'$
    \item $86^{\circ}45'$
    \item  $157^{\circ}30'$
\end{enumerate}
\end{multicols}

\item 把下列各角的弧度数化为度数:
 \begin{multicols}{3}
    \begin{enumerate}
        \item $0.4800$
        \item $0.0099$ 
        \item $2.6400$
        \item $\frac{3}{5}\pi$
        \item $\frac{4}{5}\pi$
        \item $\frac{\pi}{15}$
        \item $\frac{\pi}{10}$
        \item $3\pi$
\end{enumerate}
\end{multicols}

\item 已知$200^{\circ}$的圆心角所对的弧长等于50cm, 求圆的
半径。
\item 轮子每秒旋转$\frac{5}{18}$
弧度,20秒钟内转了多大角度?
\item 一个大钟的长针长2尺8寸.20秒间针端走了几寸?
\item 扇形弧长为20cm, 半径为15cm, 求扇形面积。
\item 地球半径为6400公里,地面上一弧所对球心角为$1'$,
问弧长若干公里?
\end{enumerate}
\end{ex}

\subsection{始边和终边相同的角}
今后我们常在直角坐标系里讨论角,并把角放在下面的
标准位置:使角的顶点与坐标原点重合,角的始边与$x$轴的
正半轴重合,角的终边在第几象限,就把这个角叫做第几象
限角(或说这个角属于第几象限)。如图6.7(1)中,
$\frac{\pi}{6}$, $\frac{13\pi}{6}$
和$-\frac{11\pi}{6}$
都是第一象限的角。在图6.7(2)中,
$-\frac{\pi}{3}$, $\frac{5\pi}{3}$
都是第四象限的角。

\begin{figure}[htp]
    \centering
\begin{tikzpicture}[>=latex]
\begin{scope}
    \draw[->](-2,0)--(2,0)node[right]{$x$};
    \draw[->](0,-2)--(0,2)node[right]{$y$};
\draw (0,0)--(30:2.5)node[right]{$B$};
\draw[->] (.75,0) arc (0:30:.75);
\draw[->] (1,0) arc (0:-330:1);
\node at (.2,-.2){$O$};
\node at (15:1){$\tfrac{\pi}{6}$};
\node at (2,.5){$\tfrac{13\pi}{6}$};
\node at (.5,1.2){$-\tfrac{11\pi}{6}$};
\draw(1.5,0) arc (0:270:1.5);
\draw [->] (0,-1.5) arc (270:384:1.7);
\node at (0,-2.5){(1)};
\end{scope}
\begin{scope}[xshift=6cm]
    \draw[->](-2,0)--(2,0)node[right]{$x$};
    \draw[->](0,-2)--(0,2)node[right]{$y$};
    \draw (0,0)--(-60:2.5)node[right]{$B$};
    \draw[->] (.75,0) arc (0:-60:.75);
    \draw[->] (1,0) arc (0:300:1);
    \node at (.2,.2){$O$};
    \node at (-30:1){$-\tfrac{\pi}{3}$};
    
\node at (1,1){$\tfrac{5\pi}{3}$};
\node at (0,-2.5){(2)};

\end{scope}
\end{tikzpicture}   
    \caption{}
\end{figure}


在图6.7(1)中可以看到$\frac{13\pi}{6}$与$-\frac{11\pi}{6}$都和$\frac{\pi}{6}$的角终边相同。
$\frac{13\pi}{6}$和$-\frac{11\pi}{6}$
可以写成下列形式:
\[2\pi+\frac{\pi}{6},\qquad -2\pi+\frac{\pi}{6}\]
显然,除了这两个角以外,与
的角终边相同的角还有:
\[\begin{split}
    2\x 2\pi+\frac{\pi}{6},&\qquad -2\x 2\pi+\frac{\pi}{6}\\
    3\x 2\pi+\frac{\pi}{6},&\qquad -3\x 2\pi+\frac{\pi}{6}\\
\cdots\cdots\qquad &\qquad\qquad \cdots\cdots
\end{split}\]
所有和$\frac{\pi}{6}$
的角终边相同的角,连同$\frac{\pi}{6}$
在内,可以用下式表示:
\[2k\pi+\frac{\pi}{6},\quad (k\in\mathbb{Z})\]
当$k=1$时,它表示$\frac{\pi}{6}$的角;$k=1$时,它表示$\frac{13\pi}{6}$的角;$k=-1$时,它表示$-\frac{11\pi}{6}$的角。

一般地,所有和$\alpha$角终边相同的角,连同$\alpha$在内,可
以用式子$2k\pi+\alpha\; (k\in\mathbb{Z})$来表示。

由此可见,具有相同始边和终边的角不止一个,而是无
穷多个,它们之间彼此相差整数周(正的或负的)即$2\pi$的整数
倍。实际上,相同始边和终边的角是由无穷多个角组成的集
合。与$\alpha$终边相同的角($\alpha$角处在标准位置)的集合可记作:
\[\{\beta|\beta=2k\pi+\alpha,\; k\in\mathbb{Z}\}\quad  \text{(若$\alpha$以弧度制给出)}\]
或
\[\{\beta|\beta=k\cdot 360^{\circ}+\alpha,\; k\in\mathbb{Z}\}\quad  \text{(若$\alpha$以度数制给出)}\]


\begin{example}
    在$0^{\circ}$到$360^{\circ}$的范围内,找出与下列各角终边相同
的角,并判定下列各角是哪个象限的角。
\[-120^{\circ},\qquad 640^{\circ},\qquad -950^{\circ}12'\]

\end{example}


\begin{solution}
\begin{enumerate}
    \item $\because\quad -120^{\circ}=-360^{\circ}+240^{\circ}$
    
$\therefore\quad -120^{\circ}$的角与$240^{\circ}$的角的终边相同,它是第
三象限的角。
\item $\because\quad 640^{\circ}=360^{\circ}+280^{\circ}$

$\therefore\quad 640^{\circ}$的角与$280^{\circ}$的角的终边相同,它是第四象限
的角。

\item $\because\quad -950^{\circ}12'=-3x360^{\circ}+129^{\circ}48'$

$\therefore\quad -950^{\circ}12'$的角与$129^{\circ}48'$的角的终边相同,它是
第二象限的角。
\end{enumerate}
\end{solution}

\begin{example}
写出与下列各角终边相同的角的集合$S$, 并把$S$
中$-2\pi$到$4\pi$间的角写出来:
\[\frac{\pi}{3},\qquad -\frac{\pi}{4},\qquad \frac{15\pi}{7}\]
\end{example}

\begin{solution}
\begin{enumerate}
    \item 
$S=\left\{\beta \Big|\beta =2k\pi+\frac{\pi}{3},\; k\in\mathbb{Z}\right\}$

$S$中在$-2\pi$到$4\pi$间的角:
\begin{itemize}
    \item 当$k=-1$时,$\beta =-2\pi+\frac{\pi}{3}=\frac{-5\pi}{8}$
    \item 当$k=0$时,$\beta=\frac{\pi}{3}$ 
    \item 当$k=1$时,$\beta =2\pi+\frac{\pi}{3}=\frac{7\pi}{3}$
\end{itemize}

\item $S=\left\{\beta \Big|\beta =2k\pi-\frac{\pi}{4},\;k\in\mathbb{Z}\right\}$

$S$中在$-2\pi$到$4\pi$间的角:
\begin{itemize}
    \item 当$k=0$时,$\beta =-\frac{\pi}{4}$
    \item 当$k=1$时,$\beta =2\pi-\frac{\pi}{4}=\frac{7\pi}{4}$
    \item 当$k=2$时,$\beta =4\pi-\frac{\pi}{4}=\frac{15\pi}{4}$
\end{itemize}

\item $S=\left\{\beta \Big|\beta =2k\pi+\frac{15\pi}{7},\; k\in\mathbb{Z}\right\}$

$S$中在$-2\pi$到$4\pi$间的角:
\begin{itemize}
    \item 当$k=-2$时,$\beta =-4\pi+\frac{15\pi}{7}=-\frac{13\pi}{7}$
    \item 当$k=-1$时,$\beta =-2\pi+\frac{15\pi}{7}=\frac{\pi}{7}$
    \item 当$k=0$时,$\beta=\frac{15\pi}{7}$
\end{itemize}
\end{enumerate}    
\end{solution}

如果处在标准位置的角的终边落在坐标轴上,那么如何
写出终边相同的角呢?下面我们来研究这个问题。

\begin{enumerate}
    \item 终边落在$x$轴的正向上,这些角的量数为
\[2n\pi\quad  (n\in\mathbb{Z})\]
\item 终边落在x轴的负向上,这些角的量数
\[2n\pi +\pi =(2n+1)\pi\quad  (n\in\mathbb{Z})\]
把1、2结合起来,量数为$k\pi\;  (k\in\mathbb{Z})$的角的终边
落在$x$轴上,在$k$为偶数时,终边落在$x$轴的正向上;在$k$
为奇数时,终边落在$x$轴的负向上。
\item 终边落在$y$轴的正向上,这些角的量数为
\[2n\pi +\frac{\pi}{2}\quad (n\in\mathbb{Z})\]
\item 终边落在$y$轴的负向上,这些角的量数为
\[-\frac{\pi}{2}+2nx=(2n-1)\pi +\frac{\pi}{2}\quad (n\in\mathbb{Z})\]
把3、4结合起来,量数为$k\pi +\frac{\pi}{2}\; (k\in\mathbb{Z})$的
角的终边落在$y$轴上,在$k$为偶数时,终边落在$y$轴的正向
上,在$k$为奇数时,终边落在$y$轴的负向上。
\end{enumerate}

量数为$k\cdot \frac{\pi}{2}\; (k\in\mathbb{Z})$的角的终边,或落在$x$轴上,或落在$y$轴上。在$k=0,1,2,3,4,5,\ldots$时,终边依次落
在$z$轴正向、$y$轴正向、$x$轴负向、$y$轴负向、$x$轴正向、$y$
轴正向……在$k=-1,-2,-3,-4,-5,\ldots$
时,终边依次落在$y$轴负向、$x$轴负向、$y$轴正向、$x$轴正
向、$y$轴负向……

此外,若$\alpha$的终边落在右半平面(一、四象限),则满足
\[-\frac{\pi}{2}+2k\pi<\alpha<\frac{\pi}{2}+2k\pi\quad (k\in\mathbb{Z})\]
若$\alpha$的终边落在上半平面(一、二象限),则满足
\[2k\pi <\alpha<(2k+1)\pi \]

以上这些表示法希望大家熟悉,因为以后经常要用到。

\begin{ex}
\begin{enumerate}
    \item 在度数制下,写出下面处在标准位置的终边相同的
    角。
        \begin{enumerate}    \begin{multicols}{2}
    \item $30^{\circ}$
    \item 终边落在$x$轴正向上;
    \item 终边落在$x$轴负向上;
    \item 终边落在$x$轴上;
    \item 终边落在$y$轴正向上;
    \item 终边落在$y$轴负向上;
    \item 终边落在$y$轴上;    \end{multicols}
    \item 角$\alpha$的终边落在左半平面上;
    \item 角$\alpha$的终边落在下半平面上。     
    \end{enumerate}

    
    \item 把下列角放在标准位置上,用量角器作出下列各角,
    并指出它们是哪个象限的角。
    \[-55^{\circ},\qquad -265^{\circ},\qquad 400^{\circ} ,\qquad 1000^{\circ},\qquad -512^{\circ} \]

    \item 当时钟上指出3点,6点和8点的时候,写出分针
    与时针所成角的一般形式。
    \item 试求出下列处在标准位置的各角的最小正同边角及
    最大负同边角,并说明各角为何象限角:
    \[1140^{\circ},\qquad 1680^{\circ},\qquad -1290^{\circ},\qquad -1510^{\circ}\]
\item 写出与下列各角终边相同的角的集合,并把集合中
在$-4\pi$ 到$2\pi$ 间的角写出来:
\[\frac{\pi}{4},\qquad -\frac{\pi}{6},\qquad \frac{36\pi}{5},\qquad -\frac{8\pi}{7}\]
\end{enumerate} 
\end{ex}

\subsection{单位圆}
\begin{blk}{定义}
    以坐标原点为圆心,半径长为1的圆叫做单位圆。
它上面任一弧的长度恰好等于弧度数。
\end{blk}
 
如图6.8, 单位圆交坐标轴于四个点$A(1,0)$、$B(0,1)$、
$A_1(-1,0)$、$B_1(0,-1)$。过$A$作单位圆的切线$\ell$, 在$\ell$上这
样来建立坐标系:取$A$为原点,取向上的方向为正向,单
位等于半径长,这样$\ell$就是一条实数轴。我们已经知道实数
轴上的点和全体实数是一一对应的。

\begin{figure}[htp]
    \centering
\includegraphics[scale=.6]{fig/6-8.PNG}
    \caption{}
\end{figure}



现在把数轴$\ell$设想为一
条无限长而没有伸缩性的丝
线,把数轴$\ell$正的那一半按
反时针方向来包卷单位圆,
而用这条数轴负的那一半按
顺时针方向包卷于单位圆上。设$S_1$是数轴$\ell$正的那一半上的一点(图6.8),当
$\ell$包卷到圆上后,此点就落到
单位圆上的$P_1$点,此时$S_1$
点的坐标是单位圆上弧$\wideparen{AP_1}$
的长或$\wideparen{AP_1}$的弧度数,也是
$\wideparen{AP_1}$所对圆心角$\theta$的弧度数。如
果数轴$\ell$上的一点坐标是负的,那么它就是$\ell$的负半轴按顺
时针方向包卷在单位圆上的负弧或它所对负圆心角的弧度
数,通过数轴在单位圆上的包卷,我们建立了数轴上一切点的
坐标和处在标准位置的圆心角$\theta$的弧度数之间的一一对应,
并且由于它们的基本单位相等,于是$\theta$角的弧度数就可以从
包卷在单位圆上的数轴$\ell$上的点的坐标直接读出来。

我们必须注意,在数轴$\ell$上,坐标相差$2\pi$ 的或相差$2\pi$ 
整数倍的那些点,当把$\ell$包卷在单位圆上时,都位于同一
点,例如$P_1$点是弧长$S_1$达到的一点,那么弧长等于$S_1\pm 2\pi,
S_1\pm 4\pi ,\ldots$的弧,当$\ell$包卷在单位圆上时也达到同一个
点$P_1$, 这就说明了数轴上的点和单位圆上的点是多一对应。
于是数轴$\ell$上的任意两个实数$S_1,S_2$和单位圆上同一个点
对应的充要条件是:
\[S_1-S_2=2n\pi \qquad  (n\in\mathbb{Z})\]
我们把上述两种对应复合在一起得到:
\begin{align*}
    \text{实数}\mathbb{R}& \longleftrightarrow  \{\theta|\text{标准位置有向角的弧度数}\}  \tag{一一对应}\\
& \longrightarrow \{(x,y)|x^2+y^2=1\} \tag{多一对应}
\end{align*}

这里$(x,y)$是单位圆上点的坐标。
我们在下面将应用这种对应关系来研究三角函数的许多
性质。并且把角的三角函数与实变数的三角函数统一起来。


\begin{example}
    在单位圆上作出对应于下列各数的点:
\[0,\quad \frac{\pi}{6},\quad \frac{\pi}{3},\quad \frac{\pi}{2},\quad \frac{2\pi}{3},\quad \frac{5\pi}{6},\quad \pi,\quad \frac{7\pi}{6},\quad \frac{4\pi}{3},\quad \frac{3\pi}{2},\quad \frac{5\pi}{3},\quad \frac{11\pi}{6},\quad 2\pi\]
\end{example}

\begin{solution}
    这些数中每相邻两数的差是$\frac{\pi}{6}$,
即在单位圆上以
相邻两数为端点的弧都相等,因此我们将单位圆12等分后,就
得到对应于上列各数的点(图6.9)。
\begin{figure}[htp]
    \centering
\begin{tikzpicture}[>=latex]
    \draw[->] (-3,0)--(3,0)node[right]{$x$};
    \draw[->] (0,-3)--(0,3)node[right]{$y$};
\draw[thick] (0,0) circle(2);

    \draw (1*30:2) [fill=black] circle (1.5pt) node[above=5pt]{$P_1$};
    \draw (2*30:2) [fill=black] circle (1.5pt) node[above=5pt]{$P_2$};
    \draw (3*30:2) [fill=black] circle (1.5pt);
    \draw (4*30:2) [fill=black] circle (1.5pt) node[above=5pt]{$P_3$};
    \draw (5*30:2) [fill=black] circle (1.5pt) node[above=5pt]{$P_4$};
    \draw (6*30:2) [fill=black] circle (1.5pt);
    \draw (7*30:2) [fill=black] circle (1.5pt) node[below=5pt]{$P_5$};
    \draw (8*30:2) [fill=black] circle (1.5pt) node[below=5pt]{$P_6$};
    \draw (9*30:2) [fill=black] circle (1.5pt);
    \draw (10*30:2) [fill=black] circle (1.5pt) node[below=5pt]{$P_7$};
    \draw (11*30:2) [fill=black] circle (1.5pt) node[below=5pt]{$P_8$};
   
\node at (2.2,0)[above]{$A$};   
\node at (-2.2,0)[above]{$A_1$};
\node at (0,2.2)[left]{$B$};   
\node at (0,-2.2)[left]{$B_1$};
\node at (0.25,-.25){$O$};  

\node at (6,3){$\frac{\pi}{6}\to P_1,\qquad \frac{\pi}{3}\to P_2$};
\node at (6,2){$ \frac{\pi}{2}\to B,\qquad \frac{2\pi}{3}\to P_3$};
\node at (6,1){$\frac{5\pi}{6}\to P_4,\qquad \pi\to A_1$};
\node at (6,0){$\frac{7\pi}{6}\to P_5,\qquad \frac{4\pi}{3}\to P_6$};
\node at (6,-1){$\frac{3\pi}{2}\to B_1,\qquad \frac{5\pi}{3}\to P_7$};
\node at (6,-2){$\frac{11\pi}{6}\to P_8,\qquad 2\pi\to A$};

\end{tikzpicture}
    \caption{}
\end{figure}
\end{solution}

\begin{example}
在数轴$\ell$上找到和单位圆上$(0,1)$点对应的一切实
数。
\end{example}

\begin{solution}
在单位圆上$B$点的坐标是$(0,1)$, $\wideparen{AB}$的弧长是$\frac{\pi}{2}$,
因此在数轴$\ell$上和$(0,1)$点对应的一切实数是:$\frac{\pi}{2}+2k\pi\; (k\in\mathbb{Z})$。

\end{solution}

\begin{ex}
\begin{enumerate}
   \item 在单位圆上找出与实数0, $\frac{\pi}{2}$, $-\frac{\pi}{2}$,$\pi$对应的点$P_1$、$P_2$、$P_3$、$P_4$. 并写出与$\angle AOP_1$、$\angle AOP_2$、$\angle AOP_3$、
$\angle AOP_4$对应的一切实数的一般形式。


\item \begin{enumerate}
\item 在单位圆上找出分别与下面各实数0、$\frac{\pi}{6}$、$\frac{\pi}{4}$、$\frac{\pi}{3}$、$\frac{\pi}{2}$对应的点$P_0$、$P_1$、$P_2$、$P_3$、$P_4$。
\item 分别写出$P_0$、$P_1$、$P_2$、$P_3$、$P_4$各点的直角坐标。
\item 与下面各实数对应的点,哪些和$P_0$、$P_1$、$P_2$、$P_3$、$P_4$关于坐标轴对称?哪些关于原点对称?并写出它们的直
角坐标。
\end{enumerate} 
\end{enumerate}
\[\frac{2\pi}{3},\quad \frac{3\pi}{4},\quad \frac{5\pi}{6},\quad \pi,\quad \frac{7\pi}{6},\quad \frac{5\pi}{4},\quad \frac{4\pi}{3},\quad \frac{3\pi}{2},\quad \frac{5\pi}{3},\quad -\frac{\pi}{4},\quad \frac{11\pi}{6}\]
\end{ex}

\section{任意角的三角函数}
在描述和研究有关转动和振动的实际问题的时候,我们
就要研究任意角的三角函数,下面我们来研究任意角的三角
函数。

\subsection{任意角三角函数的定义}
如图6.10, 在角$\alpha$的终边上任意取一点$P$(不是坐标
系的原点)。

以坐标系的原点为起点,$P$为终点的有向线段$\Vec{OP}$叫做
$P$点的向量半径或旋转半径。

设$P$点的坐标是$(x,y)$, 它和原点的距离为$r>0$(即
旋转半径$\Vec{OP}$的长),横坐标$x$与纵坐标$y$的正负是由
$P$点所在的象限来确定的。距离$r$总是正的,并且
\[r=\sqrt{x^2+y^2}\]

\begin{figure}[htp]
    \centering
\begin{tikzpicture}[>=latex]
    \draw[->] (-3,0)--(3,0)node[right]{$x$};
    \draw[->] (0,-3)--(0,3)node[right]{$y$};
\draw[thick] (0,0) circle(2);
\draw (0,0)--(120:3);
\draw (0,0)--(120:2)node[left]{$P(x,y)$}--(-1,0)node[below]{$M(x,y)$};
   
\node at (2.2,0)[above]{$A$};   
\node at (.25,-.25){$O$};
\draw[->] (.5,0) arc (0:120:.5);
\node at (60:.8){$\alpha$};
\end{tikzpicture}
    \caption{}
\end{figure}

\begin{blk}{定义}
    \begin{enumerate}
\item $\frac{y}{r}$叫做角$\alpha$的正弦,记作$\sin\alpha$, 即
$\sin\alpha=\frac{y}{r}$;
\item $\frac{x}{r}$叫做角$\alpha$的余弦,记作$\cos\alpha$, 即$\cos\alpha=\frac{x}{r}$;
\item $\frac{y}{x}$
叫做角$\alpha$的正切,记作$\tan\alpha$, 即$\tan\alpha=\frac{y}{x}$;
\item $\frac{x}{y}$叫做角$\alpha$的余切,记作$\cot\alpha$, 即$\cot\alpha=\frac{x}{y}$;
\item $\frac{r}{x}$
叫做角$\alpha$的正割,记作$\sec\alpha$, 即$\sec\alpha=\frac{r}{x}$;
\item $\frac{r}{y}$
叫做角$\alpha$的余割,记作$\csc\alpha$, 即$\csc\alpha=\frac{r}{y}$。        
    \end{enumerate}
\end{blk}

对于确定的角$\alpha$,$\frac{y}{r},\; \frac{x}{r},\; \frac{y}{x},\; \frac{x}{y},\; \frac{r}{x},\; \frac{r}{y}$
这六个比值的大小,和我们在$\alpha$角的终边上所取$P$点的位置没有关系,如图6.11中,$P_1(x_1,y_1)$点为角$\alpha$终边上
另一点,$P_1$到原点$O$的距离为$r_1$,$x$和$x_1$、$y$和$y_1$的符号
相同,因为$\triangle POM\sim \triangle P_1OM_1$,所以
\[\begin{split}
   & \frac{y_1}{r_1}=\frac{y}{r},\qquad \frac{x_1}{r_1}=\frac{x}{r},\qquad \frac{y_1}{x_1}=\frac{y}{x}\\
   &\frac{x_1}{y_1}=\frac{x}{y},\qquad  \frac{r_1}{x_1}= \frac{r}{x},\qquad \frac{r_1}{y_1}=\frac{r}{y}
\end{split}\]

\begin{figure}[htp]
    \centering
\begin{tikzpicture}[>=latex]
    \draw[->] (-2.5,0)--(2.5,0)node[right]{$x$};
    \draw[->] (0,-1)--(0,3)node[right]{$y$};
\draw (0,0)--(120:3)node[left]{$P_1(x_1,y_1)$}--(-1.5,0)node [below]{$M_1$};
\draw (0,0)--(120:2)node[right]{$P(x,y)$}--(-1,0)node [below]{$M$};
\node at (.25,-.25){$O$};
\end{tikzpicture}
    \caption{}
\end{figure}

这就是说,对于确定的角 $\alpha$, $\sin\alpha$、$\cos \alpha$、$\tan \alpha$、$\cot \alpha$、
$\sec\alpha$、$\csc\alpha$
都有确定的值,因为它们的值是随着 $\alpha$变化而
变化的,当 $\alpha$角取确定值的时候,它们的值也相应地唯一确
定,所以,角 $\alpha$的正弦、余弦、正切、余切、正割和余割都
是角 $\alpha$的函数,这些函数都叫做三角函数。

很明显,当角 $\alpha$是锐角或钝角时,上面三角函数的定义
和锐角、钝角三角函数的定义完全一样,所以锐角、钝角三
角函数定义是任意角三角函数定义的特例。

根据任意角三角函数定义,可以看出
\[\sec\alpha=\frac{1}{\cos\alpha},\qquad \csc \alpha=\frac{1}{\sin\alpha},\qquad \cot \alpha=\frac{1}{\tan\alpha}\]

今后我们主要研究
$\sin \alpha$、 $\cos \alpha$、$\tan \alpha$、$\cot \alpha$ 四个函数。

\begin{example}
    已知单位圆中旋转半径$OP$和$OX$轴正方向成
$300^{\circ}$角,求 $\sin300^{\circ}$, $\cos300^{\circ}$, $\tan 300^{\circ}$, $\cot300^{\circ}$。

\end{example}

\begin{solution}
    设$OP$的端点$P$的坐标是$(x,y)$. 作直线$PM\bot OX$轴于$M$点,$A$点是单位圆与$OX$轴的交点$(1,0)$, 联结
$P,A$(图6.12). 在$\triangle OPA$中,

$\because\quad OP=OA=1,\quad \angle POA=360^{\circ}-300^{\circ}=60^{\circ}$

$\therefore\quad \triangle OPA$是等边三角形,因此,$PM$垂直平分$OA$,
$|x|=|OM|=\frac{1}{2}$
\[\begin{split}
    |y|&=|PM|=\sqrt{|OP|^2-|OM|^2}\\
&=\sqrt{1-\left(\frac{1}{2}\right)^2}=\frac{\sqrt{3}}{2}
\end{split}\]

\begin{figure}[htp]
    \centering
\begin{tikzpicture}[>=latex]
    \draw[->] (-3,0)--(3,0)node[right]{$x$};
    \draw[->] (0,-3)--(0,3)node[right]{$y$};
\draw[thick] (0,0) circle(2);
\draw (0,0)--(-60:2)--(2,0);
\draw (1,0)node[above]{$M(\tfrac{1}{2},0)$}--(-60:2)node[below]{$P(x,y)$};
   
\node at (2.5,0)[below]{$A(1,0)$};   
\node at (-.25,-.25){$O$};
\draw[->] (.5,0) arc (0:300:.5);
\node at (150:.8){$300^{\circ}$};
\end{tikzpicture}
    \caption{}
\end{figure}

又$\because\quad P$点在第四象限,$\therefore\quad P$点坐标是
\[x=\frac{1}{2},\qquad y=-\frac{\sqrt{3}}{2}\]
由此求得
\[\sin300^{\circ}=-\frac{\sqrt{3}}{2},\qquad 
\cos300^{\circ}=\frac{1}{2}\]
\[
\tan300^{\circ}=-\sqrt{3},\qquad 
\cot300^{\circ}=-\frac{1}{\sqrt{3}}=-\frac{\sqrt{3}}{3}\]
\end{solution}


\begin{example}
    $$\cos 0=1,\qquad \sin 0=0,\qquad \cos\frac{\pi}{2}=0,\qquad \sin\frac{\pi}{2}=1$$
    \[\cos\pi=-1,\qquad \sin\pi=0,\qquad \cos\frac{3\pi}{2}=0,\qquad \sin\frac{3\pi}{2}=-1\]
\end{example}

\subsection{数值变数的三角函数与三角函数的定义域}
在数学中两个变数之间的函数可以表示不同物理量或几
何量之间的函数关系,例如,数学中的二次函数$y=ax^2$,
如果$a=1$, $a$表示正方形的边长的数值,$y$就表示正方形的
面积数值;但是当
$a=\frac{1}{2}g$
时,$x$表示自由落体下降的时
间,则$y$表示下降的距离。同样,有许多物理或技术问题,
常常要用到的三角函数,其中的自变量就不一定是角或弧而
是时间或长度等,所以为了满足科学技术上的需要,就必需
把角的三角函数扩充为变数$x$的三角函数。

假设$x$是在函数定义域内的任意实数,根据前节
讨论知,对应此实数有一个量数是$x$的角或弧(用弧度作
单位),而对应于该角又有它的各三角函数值,由于这种关
系,对于任意实数$x$就有完全确定的三角函数值$y$与之对
应,于是得到了一个数值变数的三角函数。

\begin{blk}{定义}
 变数$x$的三角函数就是具有弧度数$x$的角(或
弧)的三角函数。    
\end{blk}


\begin{example}
    若$x=1.54$, 求$\sin x$的值。
\end{example}

\begin{solution}
    因为$\sin1.54=\sin1.54\text{弧度}$,
又 $1.54\text{弧度}\approx 88^{\circ}14'$,

所以
$\sin1.54\approx \sin88^{\circ}14'\approx 0.9995$
\end{solution}

在任意大小的角、弧及数之间所能建立的对应,使得我
们可以认为三角函数是角的函数,或是弧的函数,或是数的
函数,其中变数由我们处理,可以解释为角或解释为弧,或
解释为数。

现在给每个三角函数确定它的定义域:

设数值$\alpha$, 有单位圆上的点$P$与之对应(如图6.13),
那么$P$点的坐标是
\[x=\cos\alpha,\qquad y=\sin\alpha\]

\begin{figure}[htp]
    \centering
\begin{tikzpicture}[>=latex]
\draw[->] (-2,0)--(2,0)node[right]{$x$};
\draw[->] (0,-2)--(0,2)node[right]{$y$};
\draw (0,0) circle (1.5);
\draw (0,0)--(60:1.5)node[right]{$P(\cos\alpha,\sin\alpha)$}--(1.5/2,0)node[below]{$M$};
\node at (-.25,-.25){$O$};\node at (2,0)[below]{$A(1,0)$};
\draw[->] (.5,0) arc (0:60:.5);
\node at (30:.7){$\alpha$};
\end{tikzpicture}
    \caption{}
\end{figure}


因此:
\[\tan\alpha=\frac{y}{x}=\frac{\sin\alpha}{\cos\alpha},\qquad \cot\alpha=\frac{x}{y}=\frac{\cos\alpha}{\sin\alpha}\]
\[ \sec\alpha=\frac{1}{x}=\frac{1}{\cos\alpha},\qquad \csc\alpha=\frac{1}{y}=\frac{1}{\sin\alpha}\]

\begin{enumerate}
    \item 函数 $\cos\alpha$和$\sin\alpha$的定义域是开区间$-\infty<\alpha<+\infty$,
    即$(-\infty,+\infty)$, 这是因为 $\cos\alpha$和$\sin\alpha$是单位圆上对应
    于数$\alpha$的点$P$的横坐标和纵坐标,它们对于任何实数$\alpha$都
    有明确的值。
\item 函数$\tan\alpha$的定义域是除去形如$\frac{\pi}{2}+k\pi\; (k\in\mathbb{Z})$的数的实数集:
    \[\left\{\alpha\Big|\alpha\in\mathbb{R},\; \alpha\ne \frac{\pi}{2}+k\pi, \; k\in\mathbb{Z} \right\} \]
    即正切的定义域是无限个开区间组成的一个集:
\[\cdots, \left(-\frac{3\pi}{2},-\frac{\pi}{2}\right), \left(-\frac{\pi}{2},\frac{\pi}{2}\right), \left(\frac{\pi}{2},\frac{3\pi}{2}\right),\left(\frac{3\pi}{2},\frac{5\pi}{2}\right),\cdots\]
这是因为$\tan\alpha =\frac{y}{x}$
是单位圆上$P$点
的纵坐标对横坐标之比,唯有当$x=0$时它失去意义,单位圆
上与$x=0$对应的点只有$(0,1)$和$(0,-1)$, 与$(0,1)$点对应的
一切实数是$\alpha=\frac{\pi}{2}+2k\pi\; (k\in\mathbb{Z})$, 与$(0,-1)$点对应的一
切实数是$\alpha=-\frac{\pi}{2}+2k\pi\; (k\in\mathbb{Z})$, 和$(0,1)$点、$(0,-1)$
点这两点对应的一切实数可以合并写成:
\[\alpha=\frac{\pi}{2}+k\pi\qquad  (k\in\mathbb{Z})\]
\item 函数$\cot\alpha$的定义域是除去形如$k\pi\;  (k\in\mathbb{Z})$的数的实
数集:$\{\alpha|\alpha\in\mathbb{R},\; \alpha\ne k\pi ,k\in\mathbb{Z}\}$, 即余切的定义域是无
限个开区间组成的一个集:
\[\cdots, (-\pi ,0),(0,\pi ),(\pi ,2\pi ),\cdots\]
\item 函数$\sec\alpha$的定义域与正切函数$\tan\alpha$的定义域$\left\{\alpha\Big|\alpha\in\mathbb{R},\;  \alpha\ne \frac{\pi}{2}+k\pi ,k\in\mathbb{Z}\right\}$相同。
\item 函数$\csc\alpha$的定义域与余切函数$\cot\alpha$的定义域$\{\alpha|\alpha\in\mathbb{R},\; \alpha\ne k\pi ,k\in\mathbb{Z}\}$相同。
\end{enumerate}

\subsection{三角函数的正负}
设单位圆上点$P$与数$\alpha$对应,以后我们把与$\alpha$对应,
处在标准位置,以$\alpha$(弧度)为量数的角$\angle AOP$简称为角$\alpha$。

\begin{enumerate}
    \item 若$P$点在第一象限(或者角$\alpha$终边在第一象限),
则$P$点的横坐标$x>0$, 纵坐标$y>0$, 因此,
$$\cos\alpha>0,\qquad \sin\alpha>0, \qquad\tan\alpha=\frac{\sin\alpha}{\cos\alpha}>0,\qquad \cot\alpha=\frac{\cos\alpha}{\sin\alpha}>0$$
\item 若$P$点在第二象限(或者角$\alpha$终边在第二象限),
则$P$点的横坐标$x<0$, 纵坐标$y>0$, 因此,
$$\cos\alpha<0,\qquad \sin\alpha>0, \qquad\tan\alpha=\frac{\sin\alpha}{\cos\alpha}<0,\qquad \cot\alpha=\frac{\cos\alpha}{\sin\alpha}<0$$
\item 若$P$点在第三象限(或者角$\alpha$终边在第三象限),
则$P$点的横坐标$x<0$, 纵坐标$y<0$, 因此,
$$\cos\alpha<0,\qquad \sin\alpha<0, \qquad\tan\alpha=\frac{\sin\alpha}{\cos\alpha}>0,\qquad \cot\alpha=\frac{\cos\alpha}{\sin\alpha}>0$$
\item 若$P$点在第四象限(或者角$\alpha$终边在第四象限),
则$P$点的横坐标$x>0$, 纵坐标$y<0$, 因此,
$$\cos\alpha>0,\qquad \sin\alpha<0, \qquad\tan\alpha=\frac{\sin\alpha}{\cos\alpha}<0,\qquad \cot\alpha=\frac{\cos\alpha}{\sin\alpha}<0$$
\end{enumerate}

总之,三角函数的符号可以由单位圆上$P$点在哪一象
限,或者由角$\alpha$的终边在哪一象限决定,如图6.14所示。
\begin{figure}[htp]
    \centering
\begin{tikzpicture}[>=latex]
\begin{scope}
\draw[->] (-1.5,0)--(1.5,0)node[right]{$x$};
\draw[->] (0,-1.5)--(0,1.5)node[right]{$y$};
\node at (-.25,-.25){$O$};
\draw (0,0) circle(1);
\foreach \x/\xcorr in {+/{.5,.5}, -/{-.5,.5}, -/{-.5,-.5}, +/{.5,-.5}}
{
    \node at (\xcorr) {$\x$};
}
\node at (0,-2){$\cos\alpha$和$\sec\alpha$};
\end{scope}
\begin{scope}[xshift=4cm]
    \draw[->] (-1.5,0)--(1.5,0)node[right]{$x$};
    \draw[->] (0,-1.5)--(0,1.5)node[right]{$y$};
    \node at (-.25,-.25){$O$};
    \draw (0,0) circle(1);
    \foreach \x/\xcorr in {+/{.5,.5}, +/{-.5,.5}, -/{-.5,-.5}, -/{.5,-.5}}
    {
        \node at (\xcorr) {$\x$};
    }
    \node at (0,-2){$\sin\alpha$和$\csc\alpha$};
    \end{scope}
    \begin{scope}[xshift=8cm]
\draw[->] (-1.5,0)--(1.5,0)node[right]{$x$};
\draw[->] (0,-1.5)--(0,1.5)node[right]{$y$};
\node at (-.25,-.25){$O$};
\draw (0,0) circle(1);
\foreach \x/\xcorr in {+/{.5,.5}, -/{-.5,.5}, +/{-.5,-.5}, -/{.5,-.5}}
{
    \node at (\xcorr) {$\x$};
}
\node at (0,-2){$\tan\alpha$和$\cot\alpha$};
\end{scope}
\end{tikzpicture}
    \caption{}
\end{figure}

\begin{example}
    例1 决定下列三角函数的符号:
\[\cos120^{\circ},\qquad \sin(-465^{\circ}),\qquad \csc\left(-\frac{4\pi}{3}\right)\]
\end{example}


\begin{solution}
\begin{enumerate}
    \item $120^{\circ}$的角是第二象限的角,而第二象限的角的
余弦为负,所以$$\cos120^{\circ}<0$$
\item $-465^{\circ}$的角是第三象限的角,而第三象限的角的正
弦为负,所以$$\sin(-465^{\circ})<0$$
\item $-\frac{4\pi}{3}$的角是第二象限的角,而第二象限的角的余
割为正,所以$$\csc\left(-\frac{4\pi}{3}\right)>0$$
\end{enumerate}    
\end{solution}



\begin{example}
    如果$\alpha$ 是第四象限的角,那么
$\sin\alpha \tan \alpha$和$\frac{cos\alpha }{\cot \alpha}$
各取什么符号?
\end{example}

\begin{solution}
    因为$\alpha$是第四象限的角,根据第四象限角的三角
函数的符号得:
\[\sin\alpha <0,\qquad \cos\alpha>0,\qquad  \tan \alpha<0,\qquad \cot \alpha<0\]
所以
\[\sin\alpha \tan \alpha>0,\qquad  \frac{\cos\alpha}{\cot a}<0\]
\end{solution}


\begin{example}
    若$\alpha$ 是第三象限的角,求证:
$\tan \alpha+\cot \alpha\ge 2$
\end{example}

\begin{solution}
    $\because\quad \alpha$ 是第三象限的角,$\tan \alpha>0$, $\cot \alpha>0$. 根据
两个正数的算术平均值不小于它的几何平均值,因此有:
\[\frac{\tan a+\cot a}{2}\ge \sqrt{\tan \alpha \cot \alpha}\]
即:$\frac{\tan a+\cot a}{2}\ge 1$

$\therefore\quad \tan a+\cot a\ge 2$
\end{solution}

\begin{ex}
\begin{enumerate}
    \item 图示下列各点位置,并求出与原点的距离。写出角
的终边通过这些点的三角函数值:$P_1(-3,4)$, $P_2(4,-5)$,
$P_3(-5,-3)$。
\item 若$P$点的坐标为$(-4,y)$、$OP$长为5, 试求$y$值,
并写出角的终边通过$P$点的三角函数值。
\item $\sec\alpha$ 能否小于$\tan \alpha$? $\csc\alpha$ 能否小于$\cot\alpha$?
\item 就绝对值而言能否$\cos\alpha$  大于 
$\cot\alpha$? $\sin\alpha$ 大
于$\tan a$?
\item $x$为何值,不等式$|\sin x|+|\cos x|<1$成立?
\item 决定下列各角正弦、余弦、正切的符号:
\[885^{\circ},\qquad -395^{\circ},\qquad \frac{19\pi}{6}\]
\item 若$\cos A<0$且$\tan A<0$, 试决定$A$为何象限角?
\item 设$\frac{\sin\alpha}{\tan\alpha}<0$, $\frac{\cot\alpha}{\cos\alpha}<0$, $\sin\alpha \cos\alpha <0$, 角$\alpha$ 的终边应当在哪些象限?
\item 设$\alpha$ 的终边在第三象限内,决定下列各式的符
号:
\begin{multicols}{2}
\begin{enumerate}
    \item $\sin\alpha  +\cos\alpha$
    \item $\tan \alpha-\sin\alpha$
    \item $\cos\alpha +\cot \alpha$
    \item $\sec\alpha  +\tan \alpha  +\cot \alpha$
\end{enumerate}    
\end{multicols}

\item $\alpha$ 为何值时,下面式子失去意义:
\begin{multicols}{2}
\begin{enumerate}
    \item $\cos\alpha  +\frac{1}{\cos\alpha}$
    \item $\tan \alpha+\sin\alpha$
    \item  $\tan \alpha+\cot\alpha$
    \item $\frac{1}{\cos\alpha} +\tan \alpha$
    \item $\frac{1}{\sin\alpha\cos\alpha}$
\end{enumerate}    
\end{multicols}
\end{enumerate}
\end{ex}

\subsection{一些特殊角的三角函数值}
我们常常要考虑在$[0,2\pi]$ 范围的一些特殊角的三角函
数值,仍利用单位圆来讨论,设向量半径$\Vec{OP}$的端点为$P(x,y)$, 于是
\begin{itemize}
    \item 当$\alpha =0$时,则$P$的坐标是$(1,0)$, 那么,$\cos0=1$, 
$\sin0=0$, $\tan0=0$, $\cot0$不存在,$\sec 0=1$, $\csc0$不存在。
\item 当$\alpha =\frac{\pi}{2}$
时,则$P$的坐标是$(0,1)$, 那么,
$\cos\frac{\pi}{2}=0$, 
$\sin\frac{\pi}{2}=1$, $\tan\frac{\pi}{2}$不存在, $\cot\frac{\pi}{2}=0$,$\sec \frac{\pi}{2}$不存在, $\csc\frac{\pi}{2}=1$。

\item 当$\alpha =\pi$ 时,则$P$的坐标是$(-1,0)$, 那么,$\cos\pi =-1$, $\sin\pi =0$, $\tan\pi =0$, $\cot\pi$不存在,$\sec \pi =-1$, $\csc\pi$ 不存在。
\item 当$\alpha=\frac{3\pi}{2}$时,则P的坐标是$(0,-1)$, 那么,$\cos\frac{3\pi}{2}=0$, 
$\sin\frac{3\pi}{2}=-1$, $\tan\frac{3\pi}{2}$不存在, $\cot\frac{3\pi}{2}=0$,$\sec \frac{3\pi}{2}$不存在, $\csc\frac{3\pi}{2}=-1$。
\end{itemize}

把上面结果列成表。要记着表上结果并不难,只要记着特殊点$P$的相应位置,根据三角函数的定义就很快得到函数值。
\begin{center}
\begin{tabular}{c|c|cccccc}
\hline
$P(x,y)$ & $\alpha$& $\cos\alpha$& $\sin\alpha$& $\tan\alpha$& $\cot\alpha$& $\sec\alpha$& $\csc\alpha$\\
\hline
$P(1,0)$  &0&1&0&0&不存在&1&不存在\\
$P(0,1)$ &$\frac{\pi}{2}$&0&1&不存在&0&不存在&1\\
$P(-1,0)$ &$\pi$&$-1$&0&0&不存在&$-1$&不存在\\
$P(0,-1)$ &$\frac{3\pi}{2}$&0&$-1$&不存在&0&不存在&$-1$\\
\hline
\end{tabular}
\end{center}

在$\left(0,\frac{\pi}{2}\right)$之间的几个特殊角的函数值,以前见过了,
现在再复习一下(见图6.15):
\begin{figure}[htp]
    \centering
\begin{tikzpicture}[>=latex, scale=1.5]
\begin{scope}
 \draw[->] (-1.5,0)--(1.5,0)node[right]{$x$};
 \draw[->] (0,-1.5)--(0,1.5) node[right]{$y$};
\draw[->, thick] (0,0)--(30:1)node[right]{$P\left(\tfrac{\sqrt{3}}{2},\tfrac{1}{2}\right)$};
\draw(30:1)--(1.732/2,0);
\node at (-.2,-.2){$O$};
\draw (0,0) circle(1);
\draw[->] (.5,0) arc (0:30:.5)node[right]{$\tfrac{\pi}{6}$};
\end{scope}

\begin{scope}[xshift=4cm]
    \draw[->] (-1.5,0)--(1.5,0)node[right]{$x$};
    \draw[->] (0,-1.5)--(0,1.5) node[right]{$y$};
   \draw[->, thick] (0,0)--(45:1)node[right]{$P\left(\tfrac{\sqrt{2}}{2},\tfrac{\sqrt{2}}{2}\right)$};
   \draw(45:1)--(1.414/2,0);
   \node at (-.2,-.2){$O$};
   \draw (0,0) circle(1);
   \draw[->] (.35,0) arc (0:45:.35)node[right]{$\tfrac{\pi}{4}$};
   \end{scope}


\begin{scope}[xshift=2cm, yshift=-2.5cm]
    \draw[->] (-1.5,0)--(1.5,0)node[right]{$x$};
    \draw[->] (0,-1.5)--(0,1.5) node[right]{$y$};
   \draw[->, thick] (0,0)--(60:1)node[right]{$P\left(\tfrac{1}{2},\tfrac{\sqrt{3}}{2}\right)$};
   \draw(60:1)--(.5,0);
   \node at (-.2,-.2){$O$};
   \draw (0,0) circle(1);
   \draw[->] (.35,0) arc (0:60:.35)node[right]{$\tfrac{\pi}{3}$};
   \end{scope}

\end{tikzpicture}
    \caption{}
\end{figure}

\begin{itemize}
   \item  设
$\alpha =\frac{\pi}{6}(=30^{\circ})$, 根据直角三角形性质($30^{\circ}$角的对边等于斜边的一半),得$P$点坐标
$P\left(\frac{\sqrt{3}}{2},\frac{1}{2}\right)$,那么
\[\cos\frac{\pi}{6}=\frac{\sqrt{3}}{2},\qquad \sin\frac{\pi}{6}=\frac{1}{2},\qquad \tan\frac{\pi}{6}=\frac{1}{\sqrt{3}}=\frac{\sqrt{3}}{3},\qquad \cot \frac{\pi}{6}=\sqrt{3}\]
\item 设$\alpha =\frac{\pi}{4}(=45^{\circ})$, 这时得
$P\left(\frac{\sqrt{2}}{2},\frac{\sqrt{2}}{2}\right)$,
那么
\[\cos\frac{\pi}{4}=\frac{\sqrt{2}}{2},\qquad \sin\frac{\pi}{4}=\frac{\sqrt{2}}{2},\qquad \tan\frac{\pi}{4}=1,\qquad \cot \frac{\pi}{4}=1\]
\item 设$\alpha =\frac{\pi}{3}(=60^{\circ})$, 这时得
$P\left(\frac{1}{2},\frac{\sqrt{3}}{2}\right)$,那么
\[\cos\frac{\pi}{6}=\frac{1}{2},\qquad \sin\frac{\pi}{6}=\frac{\sqrt{3}}{2},\qquad \tan\frac{\pi}{6}=\sqrt{3},\qquad \cot \frac{\pi}{6}=\frac{1}{\sqrt{3}}=\frac{\sqrt{3}}{3}\]
\end{itemize}

把上述结果列表如下:
\begin{center}
\begin{tabular}{c|ccc}
\hline
$\alpha$  &  $\frac{\pi}{6}(30^{\circ})$   &  $\frac{\pi}{4}(45^{\circ})$   &  $\frac{\pi}{3}(60^{\circ})$   \\
\hline
$\sin\alpha$ & $\frac{1}{2}$& $\frac{\sqrt{2}}{2}$& $\frac{\sqrt{3}}{2}$\\
$\cos\alpha$ &  $\frac{\sqrt{3}}{2}$& $\frac{\sqrt{2}}{2}$& $\frac{1}{2}$\\
$\tan\alpha$ & $\frac{1}{2}$ & 1 &$\sqrt{3}$\\
$\cot\alpha$& $\sqrt{3}$ &1& $\frac{\sqrt{3}}{3}$\\
\hline
\end{tabular}
\end{center}


\begin{ex}
\begin{enumerate}
    \item  求下列各式的值:
\begin{enumerate}
    \item $5\sin90^{\circ}+2\cos0^{\circ}-3\sin270^{\circ}+10\cos180^{\circ}$
    \item $a^2\cos270^{\circ}+b^2\sin0^{\circ}+2ab\cot270^{\circ}$
    \item $m\sin270^{\circ}-\frac{n\sin90^{\circ}}{\cos180^{\circ}}+k\tan 180^{\circ}$
    \item $a^2\cos0^{\circ}-b^2\sin270^{\circ}+ab\cos180^{\circ} - ab\cos0^{\circ}$
    \item $a^2\sin90^{\circ}+2ab\cos180^{\circ}+\frac{b^2}{\cos^2 0^{\circ}}
    $
\end{enumerate}

    \item 求下列各式的值:
\begin{enumerate}
    \item $p^2\sin90^{\circ}-2pq\cos0^{\circ}-q^2\sec180^{\circ}$
    
    其中:$p=\frac{1}{3}$, $q=\frac{1}{2}$
    \item $\cos\frac{\pi}{3}-\tan\frac{\pi}{4}+\frac{3}{4}\tan^2\frac{\pi}{6}-\sin\frac{\pi}{6}+\cos^2\frac{\pi}{6}+\sin\frac{3\pi}{2}$
    \item $\cos\frac{\pi}{3}-\sin^2\frac{\pi}{4}\cos\pi-\frac{1}{3}\tan^2\frac{\pi}{3}\sin \frac{3\pi}{2}+\cos0$
\end{enumerate}
\end{enumerate}
\end{ex}

\section{三角函数的诱导公式}
\subsection{三角函数的奇偶性}
关于函数的奇偶性的概念我们在以前已经学习过了,现
在我们来研究三角函数的奇偶性,以便在计算函数值时得到
一些简便的法则。

我们有下面的定理。
\begin{blk}{定理}
    余弦函数是偶函数,正弦函数、正切函数和余
    切函数都是函数,这就是说,对于一切容许的$\alpha$ 值,都有
  \begin{equation}
 \begin{split}
        \sin(-\alpha )=-\sin\alpha,&\qquad \cos(-\alpha )=\cos\alpha\\
          \tan(-\alpha )=-\tan\alpha,&\qquad  \cot(-\alpha )=-\cot\alpha
    \end{split}     
  \end{equation}  
\end{blk}

\begin{proof}
设$\alpha$ 和$-\alpha$ 是处在标准位置(即如图6.16所示)
的,按相反的旋转方向且有相同绝对值的角,于是这两个角
的终边关于$Ox$轴对称。

\begin{figure}[htp]
    \centering
\begin{tikzpicture}[>=latex]
\draw[->] (-2,0)--(2,0)node[right]{$x$};
\draw[->] (0,-2)--(0,2)node[right]{$y$};
\draw (0,0) circle (1.5);
\draw[->](0,0)--(120:1.5)node[above]{$P_{\alpha}$};
\draw[->](0,0)--(-120:1.5)node[below]{$P_{-\alpha}$};
\draw[->] (.5,0) arc (0:120:.5);
\draw[->] (.75,0) arc (0:-120:.75);
\node at (60:.5)[above] {$\alpha$};
\node at (-60:.75)[below] {$-\alpha$};
\node at (2,0)[above]{$A(1,0)$};
\node at (-.25,-.25){$O$};
\end{tikzpicture}
    \caption{}
\end{figure}


设它们的终边与单位圆交于
$P_{\alpha}$和$P_{-\alpha}$ 两点,此两点也必定
关于$Ox$轴对称。

根据三角函数的定义,
\begin{itemize}
    \item $P_{\alpha}$点的坐标是$(\cos\alpha ,\sin\alpha)$;
    \item $P_{-\alpha}$
点的坐标是$(\cos(-\alpha ),\sin(-\alpha ))$。
\end{itemize}

因此,由它们关于$Ox$轴的
对称性便知道它们的横坐标相等,纵坐标是相反数,也就
是说
\[\cos(-\alpha )=\cos\alpha ,\qquad \sin(-\alpha )= -\sin\alpha\]

由此,
\[\begin{split}
    \tan(-\alpha)&=\frac{\sin(-\alpha)}{\cos(-\alpha)}=\frac{-\sin\alpha}{\cos\alpha}=-\tan\alpha\\
    \cot(-\alpha)&=\frac{\cos(-\alpha)}{\sin(-\alpha)}=\frac{\cos\alpha}{-\sin\alpha}=-\cot\alpha\\
\end{split}\]
\end{proof}

\begin{example}
求$\cos\left(-\frac{\pi}{6}\right)$,$\sin \left(-\frac{\pi}{6}\right)$,$\tan\left(-\frac{\pi}{6}\right)$,$\cot\left(-\frac{\pi}{6}\right)$的值。    
\end{example}

\begin{solution}
\[\begin{split}
\cos\left(-\frac{\pi}{6}\right)&=\cos\frac{\pi}{6}=\frac{\sqrt{3}}{2}  \\
\sin\left(-\frac{\pi}{6}\right)&=-\sin\frac{\pi}{6}=-\frac{1}{2} \\
\tan\left(-\frac{\pi}{6}\right)&=-\tan\frac{\pi}{6}=-\frac{\sqrt{3}}{3}   \\
\cot\left(-\frac{\pi}{6}\right)&=-\cot\frac{\pi}{6}=-\sqrt{3}
\end{split}\]
\end{solution}

\begin{example}
    下面函数哪些是偶函数,哪些是奇函数,哪些都
不是?
\begin{multicols}{2}
\begin{enumerate}
    \item $g(x)=1-\cos x$
    \item $F(x)=x-\sin x$
    \item $h(x)=x^2\cos x$
    \item $y(x)=\frac{x+\sin x}{x-\sin x}$
\end{enumerate}
\end{multicols}
\end{example}

\begin{solution}
\begin{enumerate}
    \item $\because\quad g(-x)=1-\cos(-x)=1-\cos x=g(x)$
    
    $\therefore\quad g(x)$是偶函数。
    \item $\because\quad F(-x)=(-x)-\sin(-x)=-x-(-\sin x)=-(x-\sin x)=-F(x)$
    
    $\therefore\quad F(x)$是奇函数。
    \item $\because\quad h(-x)=(-x)^2\cos(-x)=x^2\cos x=h(x)$
    
    $\therefore\quad h(x)$是偶函数。
    \item $\because\quad y(-x)=\frac{(-x)+\sin (-x)}{(-x)-\sin (-x)}=\frac{-(x+\sin x)}{-(x-\sin x)}=\frac{x+\sin x}{x-\sin x}=y(x)$
    
    $\therefore\quad y(x)$是偶函数。
\end{enumerate}
\end{solution}

\begin{ex}
    下面函数哪些是偶函数,哪些是奇函数,哪些都不是?
\begin{multicols}{2}
\begin{enumerate}
    \item $f(x) =|\sin x|$
    \item $H(x)=x^2 +\sin x$
    \item $\phi(x)=\cos x\sin x$
    \item $G(x)=\frac{1-\cos x}{1+\cos x}$
    \item $f(t)=\frac{t^2+\sin t^2}{1+\sin^2 t}$
    \item $y=\tan (x+2)$
    \item $y=-\tan 2x+2$
    \item $y=\tan (3-x)$
\end{enumerate}
\end{multicols}
\end{ex}

\subsection{$2\pi\pm\alpha$与$\alpha$的三角函数间的关系}
根据三角函数的定义可以知道,终边相同的角的同一三
角函数的值相等,即
\begin{equation}
\begin{split}
    \sin(\alpha+2k\pi )=\sin\alpha,&\qquad \cos(\alpha+2k\pi )=\cos\alpha \\
\tan(\alpha+2k\pi ) =\tan\alpha,&\qquad \cot(\alpha+2k\pi )=\cot\alpha
\end{split}
\end{equation}
其中$k\in\mathbb{Z}$。

利用上面的公式可以把求任意角的三角函数值的问题,
转化为求0到$2\pi$间角的三角函数值的问题。

\begin{example}
    求下列三角函数值:
\begin{multicols}{3}
\begin{enumerate}
    \item $\sin(-1480^{\circ}10')$
    \item $\cos\left(\frac{9\pi}{4}\right)$
    \item $\tan\left(-\frac{11\pi}{3}\right)$
\end{enumerate}
\end{multicols}
\end{example}

\begin{solution}
\begin{enumerate}
    \item 
$\sin(-1480^{\circ}10')=-\sin(1480^{\circ}10')=-\sin(4\x 360^{\circ}+40^{\circ}10')=-\sin 40^{\circ}10'=-0.6451$
\item $\cos\left(\frac{9\pi}{4}\right)=\cos\left(2\pi+\frac{\pi}{4}\right)=\cos\frac{\pi}{4}=\frac{\sqrt{2}}{2}$
\item $\tan\left(-\frac{11\pi}{3}\right)=\tan\left(-4\pi+\frac{\pi}{3}\right)=\tan\frac{\pi}{3}=\sqrt{3}$
\end{enumerate}
\end{solution}

在上面公式中,若令$k=1$, 则有
\begin{equation}
    \begin{split}
        \sin(2\pi+\alpha)=\sin\alpha,&\qquad \cos(2\pi+\alpha)=\cos\alpha \\
    \tan(2\pi+\alpha) =\tan\alpha,&\qquad \cot(2\pi+\alpha)=\cot\alpha
    \end{split}
    \end{equation}

现在来看$2\pi-\alpha$的各三角函数值:
\[\begin{split}
    \sin(2\pi-\alpha)&=\sin[2\pi+(-\alpha)]=\sin(-\alpha)=-\sin\alpha\\
    \cos(2\pi-\alpha)&=\cos[2\pi+(-\alpha)]=\cos(-\alpha)=\cos\alpha\\
    \tan(2\pi-\alpha)&=\tan[2\pi+(-\alpha)]=\tan(-\alpha)=-\tan\alpha\\
    \cot(2\pi-\alpha)&=\cot[2\pi+(-\alpha)]=\cot(-\alpha)=-\cot\alpha\\    
\end{split}\]

于是有
\begin{equation}
    \begin{split}
        \sin(2\pi-\alpha)=-\sin\alpha,&\qquad \cos(2\pi-\alpha)=\cos\alpha \\
    \tan(2\pi-\alpha) =-\tan\alpha,&\qquad \cot(2\pi-\alpha)=-\cot\alpha
    \end{split}
    \end{equation}

\begin{example}
    求下列各三角函数值:
\begin{multicols}{3}
\begin{enumerate}
    \item $\sin\frac{11\pi}{6}$
    \item $\cos\frac{13\pi}{8}$
    \item $\cot310^{\circ}18'$
    \item $\sin\left(-\frac{17}{3}\pi\right)$
    \item $\tan(-324^{\circ}18')$
\end{enumerate}
\end{multicols}
\end{example}

\begin{solution}
\begin{enumerate}
    \item $\sin\frac{11\pi}{6}=\sin\left(2\pi-\frac{\pi}{6}\right)=-\sin\frac{\pi}{6}=-\frac{1}{2}$
    \item $\cos\frac{13\pi}{8}=\cos\left(2\pi-\frac{3\pi}{8}\right)=\cos\frac{3\pi}{8}=\cos67^{\circ}30'=0.3827$
    \item $\cot310^{\circ}18'=\cot(360^{\circ}-49^{\circ}42')=-\cot49^{\circ}42'=-0.8481$
    \item $\sin\left(-\frac{17}{3}\pi\right)=\sin\left(-6\pi+\frac{\pi}{3}\right)=\sin\frac{\pi}{3}=\frac{\sqrt{3}}{2}$
    
或者
\[\sin\left(-\frac{17}{3}\pi\right)=-\sin\frac{17\pi}{3}=-\sin\left(6\pi-\frac{\pi}{3}\right)=-\sin\left(-\frac{\pi}{3}\right)=\sin\frac{\pi}{3}=\frac{\sqrt{3}}{2}\]

    \item $\tan(-324^{\circ}18')=-\tan324^{\circ}18'=-\tan(360^{\circ}-35^{\circ}42')=-(-\tan 35^{\circ}42')=\tan 35^{\circ}42'=0.7186$
\end{enumerate}    
\end{solution}

\subsection{$\frac{\pi}{2}\pm\alpha$与$\alpha$的三角函数间的关系}

$\frac{\pi}{2}\pm\alpha$与$\alpha$的三角函数间有下述关系:
\begin{equation}
    \begin{split}
\sin\left(\frac{\pi}{2}+\alpha\right)=\cos\alpha,&\qquad \cos\left(\frac{\pi}{2}+\alpha\right)=-\sin\alpha\\
\tan\left(\frac{\pi}{2}+\alpha\right)=-\cot\alpha,&\qquad \cot\left(\frac{\pi}{2}+\alpha\right)=-\tan\alpha        
    \end{split}
\end{equation}

\begin{proof}
    设$P$和$P'$是单位圆上的点,且$\angle POP'=\frac{\pi}{2}$。
    若$P$点分别在第一、二、三、四象限,那么$P'$点就依次在
    第二、三、四、一象限,如图6.17所示。
\begin{figure}[htp]
    \centering
\begin{tikzpicture}[>=latex]
    \draw[->] (-2,0)--(2,0)node[right]{$x$};
    \draw[->] (0,-2)--(0,2)node[right]{$y$};
    \draw (0,0) circle (1.5);
\draw [thick](0,0)--(30:1.5)node[right]{$P$}--(1.5*1.732/2,0)node[below]{$M$};
\draw [thick](0,0)--(30+90:1.5)node[above]{$P'$}--(0,1.5*1.732/2)node[right]{$M'$};
\node at (-.25,-.25){$O$};
\draw[->] (1,0) arc (0:30:1)node[right]{$\alpha$};
\draw[->] (.5,0) arc (0:30+90:.5);
\node at (60:.7){$\alpha+\tfrac{\pi}{2}$};
\end{tikzpicture}
    \caption{}
\end{figure}

    从$P$点和$P'$点分别向$Ox$轴和$Oy$轴引垂线$PM$, 

    
\end{proof}

\begin{example}
    
\end{example}


\begin{solution}
    
\end{solution}

\begin{example}
    
\end{example}

\begin{solution}
    
\end{solution}



\begin{example}
    
\end{example}

\begin{solution}
    
\end{solution}

\begin{example}
    
\end{example}

\begin{solution}
    
\end{solution}


\begin{example}
    
\end{example}























\begin{solution}
    
\end{solution}

\begin{solution}
    
\end{solution}

\begin{solution}
    
\end{solution}

\begin{solution}
    
\end{solution}
%  \chapter{三角函数的图象和性质}
我们知道,函数图象能把函数性质形象地表现出来,为了便于研究三角函数的性质,我们现在来做出各三角函数的图象.
\section{正弦函数的图象和性质}
\subsection{正弦函数的图象}

我们知道正弦函数的定义域是$(-\infty, +\infty)$, 且它是个奇函数,故它的图象可在$x$轴的正、负方向无限延伸,且图象关于原点对称.

我们先用描点法作出它的图象,列出$x$由0到$2\pi$每隔$\frac{\pi}{6}$
取值的正弦值表如下:

\begin{center}
\begin{tabular}{c|ccccccc}
    \hline
$x$ & 0& $\frac{\pi}{6}$& $\frac{\pi}{3}$& $\frac{\pi}{2}$& $\frac{2\pi}{3}$& $\frac{5\pi}{6}$& $\pi$\\
$\sin x$ & 0  & $\frac{1}{2}$  & $\frac{\sqrt{3}}{2}$  & 1  & $\frac{\sqrt{3}}{2}$  & $\frac{1}{2}$ & 0\\
\hline
$x$ && $\frac{7\pi}{6}$& $\frac{4\pi}{3}$& $\frac{3\pi}{2}$& $\frac{5\pi}{3}$& $\frac{11\pi}{6}$&$2\pi$\\
$\sin x$ && $-\frac{1}{2}$  & $-\frac{\sqrt{3}}{2}$  & $-1$  & $-\frac{\sqrt{3}}{2}$  & $-\frac{1}{2}$  & 0\\
\hline
\end{tabular}
\end{center}

把表内$x,y$的每一对值作为点的坐标,在直角坐标系内作出对应的点,将它们依次连结成平滑曲线,这条曲线就是$[0, 2\pi]$上正弦函数$y=\sin x$的图象(图7.1).

\begin{figure}[htp]
    \centering
\begin{tikzpicture}[>=latex, scale=1.3]
\draw[->](-1,0)--(7.5,0)node[right]{$x$}; 
\draw [->](0,-2)--(0,2)node[right]{$y$};
    
\draw[domain=-pi/6 :13*pi/6,  samples=1000, very thick] plot(\x, {sin(\x r)});
\foreach \x in {-1,...,13}
{
    \draw (\x/6*pi,0)--(\x/6*pi,.1);
    \draw  (\x/6*pi, {sin(\x/6*pi r)})[fill=black] circle(1pt);
}
\draw (0,1)--(.1,1);
\draw (0,-1)--(.1,-1);
\node at (0,1)[left]{1};
\node at (0,-1)[right]{$-1$};


\node at (-.2,.2){$O$};
\node at (-1/6*pi,0)[below]{$-\frac{\pi}{6}$};
\node at (1/6*pi,0)[below]{$\frac{\pi}{6}$};
\node at (2/6*pi,0)[above]{$\frac{\pi}{3}$};
\node at (3/6*pi,0)[below]{$\frac{\pi}{2}$};
\node at (4/6*pi,0)[above]{$\frac{2\pi}{3}$};
\node at (5/6*pi,0)[below]{$\frac{5\pi}{6}$};
\node at (6/6*pi,0)[above]{$\pi$};
\node at (7/6*pi,0)[above]{$\frac{7\pi}{6}$};
\node at (8/6*pi,0)[below]{$\frac{4\pi}{3}$};
\node at (9/6*pi,0)[above]{$\frac{3\pi}{2}$};
\node at (10/6*pi,0)[below]{$\frac{5\pi}{3}$};
\node at (11/6*pi,0)[above]{$\frac{11\pi}{6}$};
\node at (12/6*pi,0)[below]{$2\pi$};
\node at (13/6*pi,0)[below]{$\frac{13\pi}{6}$};
\end{tikzpicture}
    \caption{}
\end{figure}


因为终边相同的角的三角函数值相等,所以正弦函数$y=\sin x$在区间$[2k\pi,2(k+1)\pi]\; (k=\pm1,\pm2,\pm3,\ldots)$上的图象,与它在$[0, 2\pi]$上的图象完全一样,因此,为了要作出整个定义域上的正弦函数的图象,我们只要把它在$[0, 2\pi]$上的图象向左或向右平移$2\pi,4\pi,\ldots$就可以得到$y=\sin x,\; x\in\mathbb{R}$的图象(图7.2).

正弦函数$y=\sin x$的图象叫做正弦曲线.

由上面描点法可以看出,要作出整个定义域上正弦函数的图象,关键要作出$[0, 2\pi]$上的正弦函数的图象,而要作出$[0, 2\pi]$上正弦函数图象,有五个关键点:$(0,0)$, $\left(\frac{\pi}{2},1\right)$, $(\pi,0)$, $\left(\frac{3\pi}{2},-1\right)$, $(2\pi,0)$就可以把图象基本确定了.


\begin{figure}[htp]
    \centering
\begin{tikzpicture}[>=latex, xscale=.5]
\draw[->](-9,0)--(15,0)node[right]{$x$}; 
\draw [->](0,-2)--(0,2)node[right]{$y$};
    
\draw[domain=-2.5*pi : pi*4.5,  samples=2000, very thick] plot(\x, {sin(\x r)});

\foreach \x in {-4,-3,...,8}
{
    \draw (\x/2*pi,0)--(\x/2*pi,0.1);
}
\node at (-2*pi,0)[above]{$-2\pi$};
\node at (-1.5*pi,0)[below]{$-\frac{3\pi}{2}$};
\node at (-1*pi,0)[above]{$-\pi$};
\node at (-.5*pi,0)[above]{$-\frac{\pi}{2}$};
\node at (.5*pi,0)[below]{$\frac{\pi}{2}$};
\node at (1*pi,0)[above]{$\pi$};
\node at (1.5*pi,0)[above]{$\frac{3\pi}{2}$};
\node at (2*pi,0)[below]{${2\pi}$};
\node at (2.5*pi,0)[below]{$\frac{5\pi}{2}$};    
\node at (3*pi,0)[above]{$3\pi$};
\node at (3.5*pi,0)[above]{$\frac{7\pi}{2}$};
\node at (4*pi,0)[below]{$4\pi$};


\draw (0,1)--(.2,1);
\draw (0,-1)--(.2,-1);
\node at (0,1)[left]{1};
\node at (0,-1)[right]{$-1$};



\end{tikzpicture}
    \caption{}
\end{figure}

因此,在精确度要求不太高时,我们常用“五点法”作出关于正弦函数在$[0, 2\pi]$上的图象.

\begin{example}
    用五点法作出$y=1+\sin x,\quad x\in [0, 2\pi]$的图象.
\end{example}

\begin{solution}
    列表    
\begin{center}
\begin{tabular}{c|ccccc}
    \hline
    $x$ & 0  & $\frac{\pi}{2}$  & $\pi$  & $\frac{3\pi}{2}$  & $2\pi$  \\
    \hline
    $\sin x$   &  0 & 1  & 0  & $-1$  &0\\
    $1+\sin x$   & 1  & 2  & 1  & 0  & 1\\
    \hline
\end{tabular}    
\end{center}

描点作图:(如图7.3)
\begin{figure}[htp]
    \centering
\begin{tikzpicture}[>=latex, scale=1.3]
\draw[->](-1,0)--(7.5,0)node[right]{$x$}; 
\draw [->](0,-.5)--(0,2.5)node[right]{$y$};
    
\draw[domain=0 :2*pi,  samples=1000, very thick] plot(\x, {sin(\x r)+1});
\foreach \x in {1,...,4}
{
    \draw (\x/2*pi,0)--(\x/2*pi,.1);
}
\node at (0,1)[left]{1};
\node at (0,2)[left]{$2$};
\draw (0,1)--(.1,1);
\draw (0,2)--(.1,2);

\node at (-.2,-.2){$O$};
\node at (1/2*pi,0)[below]{$\frac{\pi}{2}$};
\node at (pi,0)[below]{$\pi$};
\node at (3/2*pi,0)[below]{$\frac{3\pi}{2}$};
\node at (2*pi,0)[below]{$2\pi$};

\node at (3,1.5)[right]{$y=1+\sin x,\quad x\in [0,2\pi]$};

\end{tikzpicture}
    \caption{}
\end{figure}

\end{solution}

我们也可以用几何法作出$[0, 2\pi]$上正弦函数的图象.如图7.4所示,在$O-x$轴的负半轴上任意取一点,以这点为圆心,单位长为半径作圆,从这个圆的右半圆和$O-x$轴的交点$P_0$量起,把这个圆分成12等份,并在$O-x$轴上,从原点起向右取长等于$2\pi$(即单位圆的周长)的一段,也分成12等份,过圆上的各个分点,分别向$Ox$轴作垂线,便得到各分点上的纵坐标,显然,这些点的纵坐标就是对应各角(数)的正弦值,因此过各分点作平行于$O-x$轴的直线,它们分别与由$O-x$轴上各个对应点处所作$Ox$轴的垂线相交,这些交点就是$y=\sin x$图象上的点.把这些点依次连结成平滑的曲线,就得到正弦函数$y=\sin x$在$[0, 2\pi]$区间上的图象.

如果把曲线在$[0, 2\pi]$间的一段,沿着$Ox$轴向左、右连续移动,每次移动$2\pi$个单位,就可以得到如图7.2所示的连续不断的正弦曲线.

\begin{figure}[htp]
    \centering
\begin{tikzpicture}[>=latex,scale=1.1]

\draw[->] (-4,0)--(7,0)node[right]{$x$};
\draw [->](0,-2)--(0,2)node[right]{$y$};
\draw (-2.5,0) circle (1);

\draw[domain=0 : pi*2,  samples=1000, very thick] plot(\x, {sin(\x r)});

\foreach \x in {1,2,...,11}
{
    \draw (\x*pi/6,1.5)node[above]{\x}--(\x*pi/6,-1.5);
}

\foreach \x in {1,2,...,5}
{
    \draw (-2.5,0)--+ (\x*30:1)node[above]{\x};
}
\foreach \x in {7,8,...,11}
{
    \draw (-2.5,0)--+ (\x*30:1)node[below]{\x};
}



\draw (-2.5+.5, .5*1.732)--(-2.5+.5, -.5*1.732);
\draw (-2.5-.5, .5*1.732)--(-2.5-.5, -.5*1.732);
\draw (-2.5-.5*1.732, .5)--(-2.5-.5*1.732, -.5);
\draw (-2.5+.5*1.732, .5)--(-2.5+.5*1.732, -.5);
\draw [dashed] (-2.5-.5, .5*1.732)--(2*pi/3, .5*1.732);
\draw [dashed] (-2.5-.5*1.732, .5)--(5*pi/6, .5);
\draw [dashed] (-2.5-.5, -.5*1.732)--(5*pi/3, -.5*1.732);
\draw [dashed] (-2.5-.5*1.732, -.5)--(11*pi/6, -.5);

\draw[dashed] (-2.5, 1)--(pi/2, 1);
\draw [dashed](-2.5, -1)--(1.5*pi, -1);
\node at (-1*pi/6,-1.8){$-\frac{\pi}{6}$};
\node at (13*pi/6,-1.8){$\frac{13\pi}{6}$};
\node at (1*pi/6,-1.8){$\frac{\pi}{6}$};
\node at (2*pi/6,-1.8){$\frac{\pi}{3}$};
\node at (3*pi/6,-1.8){$\frac{\pi}{2}$};
\node at (4*pi/6,-1.8){$\frac{2\pi}{3}$};
\node at (5*pi/6,-1.8){$\frac{5\pi}{6}$};
\node at (6*pi/6,-1.8){$\pi$};
\node at (7*pi/6,-1.8){$\frac{7\pi}{6}$};
\node at (8*pi/6,-1.8){$\frac{4\pi}{3}$};
\node at (9*pi/6,-1.8){$\frac{3\pi}{2}$};
\node at (10*pi/6,-1.8){$\frac{5\pi}{3}$};
\node at (11*pi/6,-1.8){$\frac{11\pi}{6}$};
\node at (12*pi/6,-1.8){$2\pi$};

\draw (-pi/6,1.5)node[above]{11}--(-pi/6,-1.5);
\draw (2*pi,1.5)node[above]{0}--(2*pi,-1.5);
\draw (13*pi/6,1.5)node[above]{1}--(13*pi/6,-1.5);

\draw[domain=-pi/6: 0,  samples=100, very thick, dashed] plot(\x, {sin(\x r)});
\draw[domain=pi*2:13*pi/6,  samples=100, very thick, dashed] plot(\x, {sin(\x r)});

\node at (-1.4, 0)[below]{0};
\node at (-3.6, 0)[below]{6};

\node at (.2,-.2){$O$};
\end{tikzpicture}
    \caption{}
\end{figure}

\begin{ex}
    \begin{enumerate}
        \item 作$y=|\sin x|,\quad x\in\mathbb{R}$的图象.
        \item 用“五点法”作出下列各函数的图象($0\le x\le 2\pi$), 并且和$y=\sin x$的图象比较,说明这些图象 和$y=\sin x$的图象的区别.
        \begin{enumerate}
            \item $y=\sin x-1$
            \item $y=1-\sin x$
          \item $y=2\sin x$
        \end{enumerate}
    \end{enumerate}
\end{ex}

\subsection{正弦函数的主要性质}

由上一章的讨论和正弦函数图象,我们可以得到正弦函数$y=\sin x$的主要性质如下:
\begin{enumerate}
    \item 定义域\quad 正弦函数的定义域是一切实数,也就是说,当自变量$x$取任何实数值时,正弦函数$y$都有唯一确定的值与之对应,从图象上看曲线随着$x$轴连续不断地无限延伸.
    \item 值域\quad 由图7.2看出,曲线上点的纵坐标最小是-1,最大是1,正弦函数值是在$-1$与$+1$之间,这说明正
    弦函数的值域是闭区间$[-1, 1]$, 或$|\sin x|\le 1$.
    \item 奇偶性\quad 正弦函数是奇函数,因此,正弦曲线关于原点对称.
    \item 函数的符号\quad 终边落在$x$轴的上半平面时,正弦函数为正;落在轴的下半平面时,正弦函数为负.也就是说,在区间$(0,\pi)$内,$\sin x>0$. 一般地,当$2k\pi <x<(2k+1)\pi$时($k\in\mathbb{Z}$),$\sin x>0$.在区间$(\pi,2\pi)$内,$\sin x<0$. 一般地,当$(2k+1)\pi<x<2(k+1)\pi$时,$\sin x<0$.这反映在图象上,在区间$(2k\pi, (2k+1)\pi),\;\; k=0,\pm1,\pm2,\ldots$上,曲线在$x$轴的上方;在区间$((2k+1)\pi, 2(k+1)\pi),\;\; k=0,\pm1,\pm2,\ldots$上,曲线在$x$轴下方.
    
    当横坐标$x=0$, $x=\pi$和$x=2\pi$时,正弦函数值为零.一般地,当$x=k\pi\; (k\in \mathbb{Z})$时,$\sin x=0$, 这时曲线与$x$轴相交.

    \item 增减性\quad 由正弦曲线容易看出,随着$x$增加正弦函数在区间$\left(-\frac{\pi}{2},\frac{\pi}{2}\right)$内是递增的;在区间$\left(-\frac{\pi}{2},\frac{3\pi}{2}\right)$内是递减的.一般的情况是$\sin x$在区间$\left(-\frac{\pi}{2}+2k\pi,\frac{\pi}{2}+2k\pi\right)$内是递增的,在区间$\left(\frac{\pi}{2}+2k\pi,\frac{\pi}{2}+(2k+1)\pi\right)$内是递减的,这里$k\in \mathbb{Z}$.
    
    由$\sin x$的增减性看出,在$x=\frac{\pi}{2}$一处,正弦函数由递增变为递减,因此在$x=\frac{\pi}{2}$处,$\sin x$取得极大值1. 一般地,当$x=\frac{\pi}{2}+2k\pi\;\; (k\in\mathbb{Z})$时,$\sin x=1$是极大值.同时还看
    出,在$x=\frac{3\pi}{2}$处,正弦函数由递减变为递增,因此在
    $x=\frac{3\pi}{2}$处,$\sin x$取得极小值$-1$, 一般地,当
    $x=\frac{3\pi}{2}+2k\pi\;\; (k\in\mathbb{Z})$
    时,$\sin x=-1$是极小值.
    \item 周期性\quad 当横坐标$x$每隔$2\pi$ 时,曲线重复出现,也就是说,正弦曲线上任何一点的横坐标加上或减去$2\pi$ 时,对应的纵坐标相等,即
\[\sin(x\pm 2\pi)=\sin x\]
一般地:
\[\sin(x+2k\pi)=\sin x\quad (k\in\mathbb{Z})\]

从上面的公式知道正弦函数是周期函数,它的周期有无穷多个,即$2\pi$ 的整数倍,但是我们所关心的是最小正周期.

下面我们来研究一般的周期函数的定义,并证明正弦函数的最小正周期是$2\pi$.
\end{enumerate}

\begin{blk}{定义}
    设有$x$的函数$f(x)$, 若存在不等于0的一个常
数$p$, 使对于函数定义域中的任何实数$x$, 等式
\begin{equation}
    f (x) =f (x+p) 
\end{equation}
成立,则称$f(x)$是周期函数,常数$p$叫做函数$f(x)$的一个周期.
\end{blk}

下面我们来说明,任何周期函数一定有正周期.

在等式(7.1)中,以$x-p$替换$x$, 就得到
\[f (x-p) =f (x)\]
因此有$f (x) =f (x\pm p)$.这也就是说,$\pm p$都是函数$f(x)$的周期,故$f(x)$必有正周期.

函数$f(x)$的最小正周期应满足:
\begin{enumerate}
    \item $p>0$;
    \item 对于任意实数$x$, 正数$p$须使$f(x+p)=f(x)$成立;
    \item $p$为满足1、2的最小正数.
\end{enumerate}

下面我们来证明$\sin x$的最小正周期等于$2\pi$.

在恒等式 $\sin(x+p)=\sin x$ 中,令$x=0$, 得到$\sin p=0$,
在单位圆上,弧的始点为$(1, 0)$, 而弧长分别等于0和$\pi$的这两个弧的端点$P_0,P_{\pi}$的纵坐标等于0, 又和这两个点对应的最小正数分别是$2\pi$和$\pi$.在这两个数中,$\pi$显然不能是周期,因为,$\sin\frac{\pi}{2}=1$,但$\sin\left(\frac{\pi}{2}+\pi\right)=-1$
,因此最小
正周期只可能是$2\pi$, 由于$\sin(x+2\pi)=\sin x$, 对于任何数$x$都成立,所以$\sin x$的最小正周期等于$2\pi$.

\begin{example}
    求下列函数的最小正周期:
\begin{multicols}{2}
\begin{enumerate}
    \item $y=\sin2x $ 
    \item $y=2\sin\left(4x-\frac{\pi}{6}\right)$
\end{enumerate}
\end{multicols}
\end{example}

\begin{solution}
\begin{enumerate}
\item
    因为 $\sin2x =\sin(2x+2\pi)=\sin 2(x+\pi), \;\; (x\in\mathbb{R})$, 即当自变量$x$改变成$x+\pi$时,函数值不变,所以$y=\sin2x$的周期$T=\pi$.

为证明$\pi$是$\sin2x$的最小正周期,我们用反证法.假设$y=\sin2x$还有一个比$\pi$小的正周期$T'$, 即$0<T'<\pi$,根据周期$T'$的定义,我们有
\[\sin2 (x+T') =\sin2x\]
即
$$\sin (2x+2T') =\sin2x$$
令$x=0$, 代入上式得$\sin 2T'=0$,
依不等式$0<T'<\pi$,从而$0<2T'<2\pi$,得到$2T'=\pi$
即
\[T'=\frac{\pi}{2}\]

今验知,$\sin2\left(x+\frac{\pi}{2}\right)=\sin(2x+\pi)=-\sin2x$. 故$T'$不是$y=\sin2x$的周期,因此得到矛盾.这就是说 $\sin 2x$的最小正周期是$\pi$.



\item 由于:
\[\begin{split}
    2\sin\left(4x-\frac{\pi}{6}\right)&=2\sin\left(4x-\frac{\pi}{6}+2\pi\right)\\
    &=2\sin\left[4\left(x+\frac{\pi}{2}\right)-\frac{\pi}{6}\right]\quad (x\in\mathbb{R})
\end{split}\]
即当自变量$x$改变成$x+\frac{\pi}{2}$时,函数值不变,所以$y=
2\sin\left(4x-\frac{\pi}{8}\right)$的周期$T=\frac{\pi}{2}$. 再证$T=\frac{\pi}{2}$是$y=2\sin\left(4x-\frac{\pi}{8}\right)$的最小正周期.

假设$y=2\sin\left(4x-\frac{\pi}{8}\right)$还有一个比$\frac{\pi}{2}$小的正周期
$T'$,即$0<T'<\frac{\pi}{2}$,
从而得到
\begin{equation}
    0<4T'<2\pi
\end{equation}

根据周期$T'$的定义,我们有
\begin{equation}
    2\sin \left[4(x+T')-\frac{\pi}{6}\right]=2\sin \left(4x-\frac{\pi}{6}\right)
\end{equation}
即
\begin{equation}
    2\sin \left[\left(4x-\frac{\pi}{6}\right)+4T'\right]=2\sin \left(4x-\frac{\pi}{6}\right)
\end{equation}

令 $4x-\frac{\pi}{6}=0$, 即$x=\frac{\pi}{24}$, 代入(7.4), 得
$$2\sin 4T'=0$$
依不等式(7.2),$4T'$的值只能是$\pi$,即$T'=\frac{\pi}{4}$.今验证知
\[2\sin\left[4\left(x+\frac{\pi}{4}\right)-\frac{\pi}{6}\right]=2\sin \left(4x+\pi-\frac{\pi}{6}\right)=-2\sin\left(4x-\frac{\pi}{6}\right)\]
故$T'=\frac{\pi}{4}$不是$2\sin\left(4x-\frac{\pi}{6}\right)$的周期,因此得到矛盾.这就证明了$y=2\sin\left(4x-\frac{\pi}{6}\right)$的最小正周期是$\frac{\pi}{2}$.
\end{enumerate}
\end{solution}

一般地,对于函数$y=A\sin(\omega x+\varphi)$ ($\omega,\varphi$为常数,且$\omega\ne 0$, $x\in\mathbb{R}$),由于
\[\begin{split}
    A\sin(\omega x+\varphi)&=A\sin(\omega x+\varphi+2\pi)\\
    &=A\sin\left[\omega\left(x+\frac{2\pi}{\omega}\right)+\varphi\right]  
\end{split}\]
故$\frac{2\pi}{|\omega|}$是$y=A\sin(\omega x+\varphi)$的一个正周期,它只与自变量的系数有关,而与$A,\varphi$无关.用上面同样的方法可以证明$\frac{2\pi}{|\omega|}$是$y=A\sin(\omega x+\varphi)$的最小正周期(证明留给读者去完成).

\begin{example}
不求值决定下列各差的符号:
\begin{enumerate}
    \item $\sin 20^{\circ}12'-\sin20^{\circ}13'$
    \item $\sin\left(-\frac{\pi}{18}\right)-\sin\left(-\frac{\pi}{10}\right)$
    \item $ \sin1605^{\circ}-\sin1657^{\circ}$
\end{enumerate}  
\end{example}

\begin{solution}
\begin{enumerate}
    \item 正弦函数在第一象限是增函数,
    
$\therefore\quad  \sin 20^{\circ}12'<\sin20^{\circ}13'$
即
    \[\sin 20^{\circ}12'-\sin20^{\circ}13'<0\]
\item $\because\quad -\frac{\pi}{2}<-\frac{\pi}{10}<-\frac{\pi}{18}<\frac{\pi}{2}$

正弦函数$y=\sin x$在$-\frac{\pi}{2}<x<\frac{\pi}{2}$上是增函数

$\therefore\quad \sin\left(-\frac{\pi}{10}\right)<\sin\left(-\frac{\pi}{18}\right)$
即
\[\sin\left(-\frac{\pi}{18}\right)-\sin\left(-\frac{\pi}{10}\right)>0\]

\item $\because\quad \sin 1605^{\circ}=\sin 165^{\circ},\qquad \sin 1657^{\circ}=\sin 217^{\circ}$

又$\because\quad 90^{\circ}<165^{\circ}<217^{\circ}<270^{\circ}$,正弦函数$y=\sin x$在$90^{\circ}<x<270^{\circ}$上是减函数.

$\therefore\quad \sin 165^{\circ}>\sin 217^{\circ}$,即
\[\sin 1605^{\circ}-\sin 1657^{\circ}>0\]
\end{enumerate}
\end{solution}

\begin{example}
    求证定圆的外切菱形中以正方形的面积最小.
\end{example}

\begin{solution}
如图7.5, 设定圆$O$的半径为$r$, 它的外切菱形中的$\angle A=\theta$, 由于对边切点连线必过圆心,故外切菱形的高等于$2r$, 外切菱形的边长为$\frac{2r}{\sin\theta}$.

\begin{figure}[htp]
    \centering
    \begin{tikzpicture}[>=latex]
 \draw (-4,0)node[left]{$A$}--(0,2)node[above]{$D$}--(4,0)node[right]{$C$}--(0,-2)node[below]{$B$}--(-4,0)    ;
\draw (-4,0)--(4,0);
\draw (0,0) circle (1.78);       
\draw (.8,1.6)--(-.8,-1.6);
\draw (0,2)--(-1.6,-1.2);
\node at (.2,-.2){$O$};
\draw[->](-3.5,0) arc (0:28:.5);
\draw[->](-3.5,0) arc (0:-28:.5);
\node at (-3.2,0.2){$\theta$};
    \end{tikzpicture}
    \caption{}
\end{figure}

    于是,菱形面积$S=2r\cdot \frac{2r}{\sin\theta}$,当$\sin\theta=1$时,菱形面积$S$最小,这时$\theta=90^{\circ}\;\;(0^{\circ}<0<180^{\circ})$.
    
因此,定圆的外切菱形中以正方形的面积最小.
\end{solution}

\begin{example}
求函数$y=\left(\sin x-\frac{1}{2}\right)^2+2$的最大值和最小值,
并求取最大值或最小值时的$x$值.
\end{example}

\begin{solution}
    把$y$看作$\sin x$的二次函数,这样问题变成求闭区间$-1\le \sin x\le 1$上的$y$的最大值和最小值.也就是要把开区间$(-1, 1)$内的极值和两端点处的函数值作比较.

\begin{enumerate}
    \item 当$\sin x=-1$时,$y=\left(-1-\frac{1}{2}\right)^2+2=4\frac{1}{4}$
    \item 当$\sin x=1$时,$y=\left(1-\frac{1}{2}\right)^2+2=2\frac{1}{4}$
    \item 当$\sin x=\frac{1}{2}$时,极小值$y=2$, 这是函数在$(-1, 1)$中的唯一极值点.
\end{enumerate}

因此,
\begin{enumerate}
    \item 当$\sin x=-1$, 即$x=-\frac{\pi}{2}+2k\pi\quad (k\in\mathbb{Z})$时,
最大值$y=4\frac{1}{4}$;
\item 当$\sin x=\frac{1}{2}$,即$x=\frac{\pi}{6}+2k\pi$或$\frac{5\pi}{6}+2k\pi \quad (k\in\mathbb{Z})$时,
最小值$y=2$.
\end{enumerate}


\end{solution}


\section*{习题7.1}
\addcontentsline{toc}{subsection}{习题7.1}
\begin{enumerate}
    \item 比较下列各组中两个三角函数值的大小(不求值):
    \begin{enumerate}
        \item $\sin250^{\circ}$和$\sin260^{\circ}$
        \item $\sin\left(-\frac{54}{7}\pi\right)$和$\sin\left(-\frac{63}{8}\pi\right)$
        \item $\sin380^{\circ}$和 $\sin480^{\circ}$
    \end{enumerate}
    
    \item 说出下列各函数的最小正周期:
\begin{multicols}{2}
    \begin{enumerate}
        \item $y=\sin 3x$
        \item $y=\sin\frac{x}{2}$
        \item $y=\sin \left(x+\frac{\pi}{3}\right)$
        \item $y=\cos \left(2 x-\frac{\pi}{6}\right)$
        \item $y=3 \sin \left(\frac{x}{3}+\frac{\pi}{4}\right)$
        \item $y=3 \sin \left(\pi x+\frac{\pi}{3}\right)+1$
    \end{enumerate}
\end{multicols}

\item 求下列函数的最大值和最小值,又在何时有最大值
或最小值:
\begin{multicols}{2}
    \begin{enumerate}
\item $y=|\sin x|$
\item $y=1+\sin x$
\item $y=1+5 \sin ^{2} x$
\item $y=\frac{1}{1-\sin x}$
\item $y=\left(\sin x-\frac{3}{2}\right)^{2}-2 $
\item $y=2-\left(\sin x-\frac{\sqrt{3}}{2}\right)^{2}$
    \end{enumerate}
\end{multicols}

\item 求证等腰三角形中,若腰长一定则等腰直角三角形
的面积最大.
\item $y=|\sin x|$是周期函
数吗?如果是,说出它的最小正周期.
\item 作$y=\sin|x|$的图象,试从它的图象说明函数$y=\sin|x|$不是周期函数:

\item 利用单位圆,容易看出$\sin x<\frac{1}{2}$的解的范围,如下图所示.
\[\frac{5\pi}{6}+2k\pi<x<\frac{\pi}{6}+2(k+1)\pi\]
\begin{center}
\begin{tikzpicture}[>=latex, scale=1.5]

\draw [thick] (0,0) circle (1);

\draw[->] (-1.5,0)--(2,0)node[right]{$x$};
\draw[->] (0,-1.5)--(0,2)node[right]{$y$};    
\draw (1.732/2,.5)--(-1.732/2,.5);
\node at (0,.55)[right]{$(0,\tfrac{1}{2})$};
\foreach \x in {15,30,150,165}
{
    \draw (0,0)--(\x:1);
}
\foreach \x in {198,216,...,342}
{
    \draw (0,0)--(\x:1);
}

\node at (30:1)[right]{$P_{\tfrac{\pi}{6}}$};
\node at (150:1)[left]{$P_{\tfrac{5\pi}{6}}$};


\end{tikzpicture}    
\end{center}


用同样的方法解下面不等式:
\[\sin 2x>\frac{1}{2},\qquad \sin 3x\ge -\frac{\sqrt{2}}{2}\]
\end{enumerate}  

\section{余弦函数的图象和性质}
\subsection{余弦函数的图象}

我们知道余弦函数的定义域是$(-\infty,+\infty)$, 它是个偶函数,故它的图象向$x$轴的正负方向无限延伸,图象关于$y$轴对称.

我们从第五章中知道,函数$f(x+\ell)$的图象是由函数$f(x)$的图象向左、右平移$|\ell|$个单位得到,当$\ell>0$时,向左平移1个单位,当$\ell<0$时,向右平移$|\ell|$个单位.

在上一章我们又知道,$\cos x=\sin\left(\frac{\pi}{2}+x\right)$,
故$y=\cos x$的
图象就是$y=\sin\left(x+\frac{\pi}{2}\right)$的图象,而正弦型函数$y=\sin\left(x+\frac{\pi}{2}\right)$的图象恰是正弦函数$y=\sin x$的图象向左
平移$\frac{\pi}{2}$个单位得到,这就是说,我们只须把正弦曲线
$y=\sin x$沿着$x$轴向左平移$\frac{\pi}{2}$个单位就得到余弦函数$y=\cos x$
的图象.如图7.6所示.

 \begin{figure}[htp]
     \centering
     \begin{tikzpicture}[scale=1.3, >=latex]

\draw[->] (-2,0)--(7,0)node[right]{$x$};
\draw [->](0,-1.5)--(0,2)node[right]{$y$};
\draw[domain=0 : pi*2,  samples=1000, very thick, dashed] plot(\x, {sin(\x r)});
\draw[domain=-0.5*pi : pi*1.75,  samples=1000, very thick] plot(\x, {cos(\x r)});

\foreach \x in {-3,-2,...,12}
{
    \draw (\x*pi/6, 0)--(\x*pi/6, .1);
}
\node at (-1,1){$y=\cos x$};  \node at (3,1){$y=\sin x$};

\node at (-3*pi/6, 0)[below]{$-\frac{\pi}{2}$};
\node at (-2*pi/6, 0)[below]{$-\frac{\pi}{3}$};
\node at (-1*pi/6, 0)[below]{$-\frac{\pi}{6}$};
\node at (1*pi/6, 0)[below]{$\frac{\pi}{6}$};
\node at (2*pi/6, 0)[below]{$\frac{\pi}{3}$};
\node at (3*pi/6, 0)[below]{$\frac{\pi}{2}$};
\node at (4*pi/6, 0)[below]{$\frac{2\pi}{3}$};
\node at (5*pi/6, 0)[below]{$\frac{5\pi}{6}$};
\node at (6*pi/6, 0)[below]{$\pi$};
\node at (7*pi/6, 0)[below]{$\frac{7\pi}{6}$};
\node at (8*pi/6, 0)[below]{$\frac{4\pi}{3}$};
\node at (9*pi/6, 0)[below]{$\frac{3\pi}{2}$};
\node at (10*pi/6, 0)[below]{$\frac{5\pi}{3}$};
\node at (11*pi/6, 0)[below]{$\frac{11\pi}{6}$};
\node at (12*pi/6, 0)[below]{$2\pi$};
\node at (.15,-.15){$O$};
     \end{tikzpicture}
     \caption{}
 \end{figure}

 当然,也可以用描点法来作余弦函数的图象.根据上面所说的我们可以借助第一节中的正弦函数值表,将自变数$x$的取值分别减去$\frac{\pi}{2}$
 而对应的$y$值仍不变就能够得到余弦函数
 值表.

 把表内$x$、$y$的每一对值作为点的坐标,在直角坐标系内作出对应的点,将它们依次连结成平滑曲线,这样就得到余弦函数在$\left[\frac{\pi}{2},\frac{3\pi}{2}\right]$上的图象.如果把曲线$\left[\frac{\pi}{2},\frac{3\pi}{2}\right]$
 上的一段,沿着$x$轴向左、右推移,每次移动$2\pi$个单
 位,就可以得到连续不断的余弦函数的图象(图7.7).

\begin{center}
\begin{tabular}{c|ccccccc}
    \hline
$x$ & $-\frac{\pi}{2}$& $-\frac{\pi}{3}$& $-\frac{\pi}{6}$& 0 & $\frac{\pi}{6}$& $\frac{\pi}{3}$& $\frac{\pi}{2}$\\
$y=\cos x$ & 0  & $\frac{1}{2}$  & $\frac{\sqrt{3}}{2}$  & 1  & $\frac{\sqrt{3}}{2}$  & $\frac{1}{2}$ & 0\\
\hline
$x$ && $\frac{2\pi}{3}$& $\frac{5\pi}{6}$& $\pi$& $\frac{7\pi}{6}$& $\frac{4\pi}{3}$&$\frac{3\pi}{2}$\\
$y=\cos x$ && $-\frac{1}{2}$  & $-\frac{\sqrt{3}}{2}$  & $-1$  & $-\frac{\sqrt{3}}{2}$  & $-\frac{1}{2}$  & 0\\
\hline
\end{tabular}
\end{center}

\begin{figure}[htp]
    \centering
\begin{tikzpicture}[>=latex, xscale=.6]
\draw[->](-10,0)--(11,0)node[right]{$x$}; 
\draw [->](0,-1.5)--(0,2)node[right]{$y$};
    
\draw[domain=-3*pi :3*pi,  samples=2000, very thick] plot(\x, {cos(\x r)});
\foreach \x in {-3,-2.5,...,3}
{
    \draw (\x*pi,0)--(\x*pi,.1);
}

\foreach \x in {1,-1}
{
    \draw (0,\x)node[left]{$\x$}--(.2,\x);
}


\node at (-.2,-.2){$O$};
\node at (-2.5*pi,0)[above]{$-\frac{5\pi}{2}$};
\node at (-.5*pi,0)[above]{$-\frac{\pi}{2}$};
\node at (2.5*pi,0)[above]{$\frac{5\pi}{2}$};
\node at (-1.5*pi,0)[below]{$-\frac{3\pi}{2}$};
\node at (.5*pi,0)[below]{$\frac{\pi}{2}$};
\node at (1.5*pi,0)[below]{$\frac{3\pi}{2}$};

\node at (3,1.5)[right]{$y=\cos x,\quad x\in \mathbb{R}$};

\end{tikzpicture}
    \caption{}
\end{figure}

余弦函数$y=\cos x$的图象叫做余弦曲线.

如作正弦函数的图象那样,只要把$\left(-\frac{\pi}{2},0\right)$、$(0, 1)$、$\left(\frac{\pi}{2},0\right)$、$(\pi,-1)$、$\left(\frac{3\pi}{2},0\right)$这五个点作出后,余弦函数$y=\cos x,\;\; x\in\left[-\frac{\pi}{2},\frac{3\pi}{2}\right]$的图象就基本确定了.因此,
在精确度要求不太高的情况下,也可用“五点法”作出关于余弦函数的图象.

\begin{example}
 用“五点法”作出$y=-\cos x,\;\; x\in[0,2\pi]$ 的图象.   
\end{example}

\begin{solution}
    列表并作图(如图7.8所示)
\begin{center}
\begin{tabular}{cccccc}
\hline
    $x$  &0&$\frac{\pi}{2}$&$\pi$&$\frac{3\pi}{2}$&$2\pi$\\
    \hline
   $\cos x$&1&0&$-1$&0&1\\ 
$-\cos x$&$-1$&0&1&0&$-1$\\
\hline
\end{tabular}    
\end{center}

\begin{figure}[htp]
    \centering
\begin{tikzpicture}[>=latex, xscale=1]
\draw[->](-1,0)--(7.5,0)node[right]{$x$}; 
\draw [->](0,-1.5)--(0,2)node[right]{$y$};
    
\draw[domain=0:2*pi,  samples=1000, very thick] plot(\x, {-cos(\x r)});
\foreach \x in {1,2,...,4}
{
    \draw (\x*pi/2,0)--(\x*pi/2,.1);
}

\foreach \x in {1,-1}
{
    \draw (0,\x)node[left]{$\x$}--(.2,\x);
}


\node at (-.2,-.2){$O$};
\node at (.5*pi,0)[below]{$\frac{\pi}{2}$};
\node at (1*pi,0)[below]{$\pi$};
\node at (1.5*pi,0)[below]{$\frac{3\pi}{2}$};
\node at (2*pi,0)[below]{$2\pi$};

\node at (3,-1.5){$y=-\cos x,\quad x\in [0,2\pi]$};

\end{tikzpicture}
    \caption{}
\end{figure}
\end{solution}

\subsection{余弦函数的主要性质}

由上一章的讨论和余弦函数图象,我们可以得到余弦函数$y=\cos x$的主要性质如下:
\begin{enumerate}
\item 定义域\quad  余弦函数$y=\cos x$的定义域是一切实数,即$-\infty<x<+\infty$或$(-\infty, +\infty)$;
\item 值域\quad 余弦函数$y=\cos x$的值域是$[-1, 1]$, 或
$|\cos x|\le 1$;
\item 奇偶性\quad 余弦函数是偶函数;
\item 函数的符号
\begin{itemize}
    \item 当$-\frac{\pi}{2}+2k\pi<x<\frac{\pi}{2}+2k\pi$时,$\cos x>0$;
    \item 当$\frac{\pi}{2}+2k\pi<x<\frac{\pi}{2}+(2k+1)\pi$时,$\cos x<0$;
    \item 当$x=\frac{\pi}{2}+k\pi$时,$\cos x=0$,这里$k\in\mathbb{Z}$.
\end{itemize}

\item 增减性
\begin{itemize}
    \item $y=\cos x$在区间$[2k\pi,(2k+1)\pi]$内是递减的;
    \item $y=\cos x$在区间$[(2k+1)\pi,2(k+1)\pi]$内是递增的.
\end{itemize}
因此,
\begin{itemize}
    \item 当$x=2k\pi$时,$\cos x=1$是极大值.
    \item 当$x=(2k+1)\pi$时,$\cos x=-1$是极小值,这里$k\in\mathbb{Z}$.
\end{itemize}

\item 周期性\quad 余弦函数$y=\cos x$的最小正周期(以后简称周期)是$2\pi$.

函数$y=\cos x$的周期是$\frac{2\pi}{|\omega|}\quad (\omega\ne 0,\;\;x\in \mathbb{R})$一般地,函数$y=A\cos(\omega x+\varphi)$的周期是$\frac{2\pi}{|\omega|}$
($\omega,\varphi$为常数,且$\omega\ne 0,\;\;x\in \mathbb{R}$).

因为$\cos(\omega x+\varphi)=\sin\left[\omega x+\left(\varphi+\frac{\pi}{2}\right)\right]$, 这里$\omega$是不等于0的常数,$\varphi+\frac{\pi}{2}$仍是常数,根据函数$y=A\sin\left[\omega x+\left(\varphi+\frac{\pi}{2}\right)\right]$的最小正周期是$\frac{2\pi}{|\omega|}$,因此
$\cos(\omega x+\varphi)$的最小正周期也是$\frac{2\pi}{|\omega|}$.

\end{enumerate}

\begin{example}
    求函数$y=4\cos(2x+3)$的周期,极值和极值点.
\end{example}

\begin{solution}
    函数$y=4\cos(2x+3)$的周期是$\frac{2\pi}{2}=\pi$

把$2x+3$看作一个变数,并根据余弦函数的增减性知:
\begin{enumerate}
    \item 当$2x+3=2k\pi$ 时,$y$达到极大值,这时$x=k\pi -\frac{3}{2}$,
    极大值$y=4$;
    \item 当$2x+3=(2k+1)\pi$ 时,$y$达到极小值,这时$x=k\pi +\frac{\pi}{2}-\frac{3}{2}$
    极小值$y=-4$.
\end{enumerate}
\end{solution}


\begin{example}
    求$4-2\cos\alpha -\sin^2\alpha$ 的最大值和最小值.
\end{example}


\begin{solution}
    \[4-2\cos\alpha -\sin^2\alpha=3-2\cos\alpha+\cos^2\alpha = 2+(1-\cos\alpha)^2\]

$1-\cos\alpha$ 的最小值为0, 最大值为2, 故知$4-2\cos\alpha -\sin^2\alpha$的最小值为2, 最大值为6, 且当$\alpha=2k\pi\quad  (k\in\mathbb{Z})$时,$4-2\cos\alpha -\sin^2\alpha$ 有最小值;当$\alpha =(2k+1)\pi\quad  (k\in\mathbb{Z})$时,$4-2\cos\alpha -\sin^2\alpha$ 有最大值.
\end{solution}

\begin{example}
    已知函数
\begin{enumerate}
    \item $\sin\left(\omega x+\frac{\pi}{4}\right)$的周期是$\frac{2\pi}{3}$
    \item $\cos\left(\omega x+\frac{\pi}{3}\right)$的周期是$\pi$
\end{enumerate}    
试确定函数.    
\end{example}

\begin{solution}
\begin{enumerate}
    \item $\because\quad \frac{2\pi}{|\omega|}=\frac{2\pi}{3}$
    
$\therefore\quad |\omega|=3,\quad \omega=\pm 3$

故所求函数为:$\sin\left(\pm 3x+\frac{\pi}{4}\right)$

\item $\because\quad \frac{2\pi}{|\omega|}=\pi$
    
$\therefore\quad |\omega|=2,\quad \omega=\pm 2$

故所求函数为:$\cos\left(\pm 2x+\frac{\pi}{3}\right)=\cos\left(2x\pm \frac{\pi}{3}\right)$
\end{enumerate}  
\end{solution}

\section*{习题7.2}
\addcontentsline{toc}{subsection}{习题7.2}
\begin{enumerate}
    \item 确定差的符号(不查表):
 \begin{multicols}{2}
\begin{enumerate}
    \item     $\sin 72^{\circ}-\sin 80^{\circ}$
    \item  $\cos 15^{\circ}-\cos 16^{\circ}$
    \item  $\sin 200^{\circ}-\sin 250^{\circ}$
    \item  $\cos 300^{\circ}-\cos 340^{\circ}$
\end{enumerate}
 \end{multicols}

 \item 用 “五点法”作出下列函数的图象 $(x \in [0,2])$:
$$y=-\sin x,     \qquad y=1+\cos x,\qquad  y=1+|\cos x|$$


\item 求下列各函数周期:
\begin{multicols}{2}
\begin{enumerate}
    \item $y=\sin 3 x$
    \item $y=\cos \frac{x}{6}$
    \item $y=3 \sin \frac{x}{4}$
    \item $y=\sin \left(x+\frac{\pi}{10}\right)$
    \item $y=\cos \left(2 x+\frac{\pi}{3}\right)$
    \item $y=\sqrt{8} \sin \left(\frac{1}{2} x-\frac{\pi}{4}\right) $
\end{enumerate}
\end{multicols}

    
    \item 求下面函数的极大值和极小值以及取极值时的 $x$ 值:
    $$ y=2+\cos x, \qquad y=2-\cos x,\qquad y=\frac{1}{1+\cos^2 x}$$

\item 求函数$y=-\cos^2\alpha -0.1\sin\alpha+1.15$的最大值和最小值.又当
$\alpha\;\; (0<\alpha<2\pi)$为何值时,函数有最大值和最小值.

\item 求$y=\sqrt{2\cos 2x-\sqrt{3}}$的定义域.

\end{enumerate}

\section{正切函数的图象和性质}
\subsection{正切函数的图象}
我们知道正切函数的定义域是除去$\frac{\pi}{2}+k\pi\quad (k\in\mathbb{Z})$的实数集.也就是由下面无数个开区间
\[\ldots, \left(-\frac{3\pi}{2},-\frac{\pi}{2}\right),\left(-\frac{\pi}{2},\frac{\pi}{2}\right), \left(\frac{\pi}{2},\frac{3\pi}{2}\right), \left(\frac{3\pi}{2},\frac{5\pi}{2}\right), \ldots\]
组成的一个集,图象在这些点:$\frac{\pi}{2}+k\pi\quad (k\in\mathbb{Z})$处断开.

我们又知道正切函数是奇函数,故它的图象关于原点对称.

现在,我们用描点法作出正切函数的图象.

列表:
\begin{center}
\begin{tabular}{c|ccccccc}
\hline
$x$  &    $-\frac{\pi}{2}$  &    $-\frac{5\pi}{12}$  &    $-\frac{\pi}{3}$  &    $-\frac{\pi}{4}$  &    $-\frac{\pi}{6}$  &    $-\frac{\pi}{12}$  \\    
\hline
$y=\tan x$   & 不存在  & $-3.73$  & $-1.73$  & $-1$  & $-0.58$  & $-0.27$  \\
\hline
$x$  &    $0$  &    $\frac{\pi}{12}$  &    $\frac{\pi}{6}$  &    $\frac{\pi}{4}$  &    $\frac{\pi}{3}$  &    $\frac{5\pi}{12}$ & $\frac{\pi}{2}$ \\    
\hline
$y=\tan x$   & 0  & $0.27$  & $0.58$  & $1$  & $1.73$ & 3.73 & 不存在  \\
\hline
\end{tabular}    
\end{center}

把表内$x$、$y$的每一对值作为点的坐标,在直角坐标系内作出对应的点,将它们依次连结成平滑曲线,这样就得到正切函数在$\left(-\frac{\pi}{2},\frac{\pi}{2}\right)$上的图象,如果把图象向左、右扩展出去,就得出$y=\tan x,\quad x\in\left(-\frac{\pi}{2}+k\pi,\frac{\pi}{2}+k\pi\right),\;\; k\in \mathbb{Z}$ 的图象(图7.9).

\begin{figure}[htp]
    \centering
\begin{tikzpicture}[>=latex, xscale=.7]
    \draw[->] (-4,0)--(9,0)node[right]{$x$};
\draw[->] (0,-3)--(0,3)node[right]{$y$};
\foreach \x in {-.5,.5,1.5,2.5}
{
    \draw[dashed] (\x*pi,-3)--(\x*pi,3);
}

\foreach \y in {-.5,.5,1.5}
{
    \draw [domain=\y*pi+.35:(\y+1)*pi-.35, samples=1000, very thick] plot(\x, {tan(\x r)});
}

\draw [domain=-4:-.5*pi-.35, samples=1000, very thick] plot(\x, {tan(\x r)});

\foreach \z in {-2,-1,2,1}
{
    \draw (0,\z)node[left]{$\z$}--(.2,\z);
}

\node at (.25,-.2){$O$};
\node at (-pi,0)[below]{$-\pi$};
\node at (pi,0)[below]{$\pi$};
\node at (2*pi,0)[below]{$2\pi$};
\node at (-.5*pi,0)[below]{$-\frac{\pi}{2}$};
\node at (.5*pi,0)[below]{$\frac{\pi}{2}$};
\node at (1.5*pi,0)[below]{$\frac{3\pi}{2}$};
\node at (2.5*pi,0)[below]{$\frac{5\pi}{2}$};

\node at (3,-3.5){$y=\tan x\qquad x\in\left(-\frac{\pi}{2}+k\pi,\; \frac{\pi}{2}+k\pi\right)$};

\end{tikzpicture}
    \caption{}

\end{figure}

正切函数$y=\tan x$的图象叫做\textbf{正切曲线},由图7.9可以看出,正切曲线是由互相平行的直线$x=\frac{\pi}{2}+k\pi\;\;(k\in\mathbb{Z})$隔
开的无穷多支曲线所组成.

下面我们说明$y=\tan x$图象的几何画法.

应用单位圆上的正切线,我们在开区间$\left(-\frac{\pi}{2},\; \frac{\pi}{2}\right)$和$\left(\frac{\pi}{2},\; \frac{3\pi}{2}\right)$内作出正切函数$y=\tan x$的图象.画图象时
让横坐标每隔$\frac{\pi}{12}$
取点,作法如图7.10.

\begin{figure}[htp]
    \centering
    \begin{tikzpicture}[>=latex, scale=.9]
        \draw[->, thick] (-6,0)--(6,0)node[right]{$x$};   
\draw[->, thick] (0,-5)--(0,5)node[right]{$y$};
\draw[thick] (-4,0) circle(1);
\draw (-4,-4)--(-4,4);
\draw[thick] (-3,-4)--(-3,4);
\foreach \x in {-5,-4,...,5}
{
    \draw (-4,0) -- (-3, {tan(\x*pi/12 r)});
}

\foreach \y in {-.5,.5}
{
    \draw [domain=\y*pi+.25:(\y+1)*pi-.25, samples=1000, very thick] plot(\x, {tan(\x r)});
}

\foreach \x in {-5,-4,...,-1,7,8,...,11}
{
    \draw[dashed] (\x*pi/12,0)--(\x*pi/12,-4.5);
}
\foreach \x in {1,2,...,5,13,14,...,17}
{
    \draw[dashed] (\x*pi/12,0)--(\x*pi/12,4.5);
}
\foreach \x in {-1,1,2,3}
{
    \draw [dashed] (\x*pi/2, -4.5)--(\x*pi/2, 4.5);
}

\foreach \x in {-5,-4,...,5}
{
    \draw[dashed] (-3, {tan(\x*pi/12 r)})--(\x*pi/12, {tan(\x*pi/12 r)});
}
\node at (-4.3,-.3){$C$};
\node at (-4.2,1)[above]{$\frac{\pi}{2}$};
\node at (-4.2,-1)[below]{$-\frac{\pi}{2}$};
\node at (.3,-.3){$O$};
\node at (-pi/2-.25,0)[below]{$-\frac{\pi}{2}$};
\node at (pi/2-.25,0)[below]{$\frac{\pi}{2}$};
\node at (3*pi/2+.25,0)[below]{$\frac{3\pi}{2}$};
\node at (pi+.25,0)[below]{$\pi$};
    \end{tikzpicture}
    \caption{}
\end{figure}

\subsection{正切函数的主要性质}

由上一章的讨论和正切函数图象,我们可以得到正切函数$y=\tan x$的主要性质如下:

\begin{enumerate}
    \item 定义域\quad  正切函数$y=\tan x$的定义域是$x\ne \frac{\pi}{2}+k\pi, 
(k\in\mathbb{Z})$的一切实数,也就是由下面无数个开区间:
\[\left(-\frac{\pi}{2}+k\pi,\; \frac{\pi}{2}+k\pi\right),\quad k=0,\pm1,\pm2,\pm3,\ldots \]
组成的一个集.
\item 值域\quad  正切函数$y=\tan x$的值域为一切实数.
\item 奇偶性\quad 正切函数是奇函数.
\item 函数的符号\quad 当$x$在一、三象限时,$\tan x>0$; 在二、四象限时,$\tan x<0$. 一般地,
\begin{itemize}
    \item 若$x\in\left(2k\pi,\frac{\pi}{2}+2k\pi\right)$或$\left((2k+1)\pi,\frac{\pi}{2}+(2k+1)\pi\right)$时,$\tan x>0$;
    \item 若$x\in\left(\frac{\pi}{2}+2k\pi,(2k+1)\pi\right)$或$\left(\frac{\pi}{2}+(2k+1)\pi, 2k\pi\right)$时,$\tan x<0$.(这里$k\in\mathbb{Z}$)
\end{itemize}

 \item 增减性\quad 正切函数$y=tan x$在$x\in\left(-\frac{\pi}{2}+k\pi,\frac{\pi}{2}+k\pi\right)\;\; (k\in\mathbb{Z})$的每一个开区间内,都是递增的.

 但是要注意,正切函数在整个定义域内并不是增函数.事实上,设$x_1=\frac{\pi}{4}$, $x_2=\frac{3\pi}{4}$,那么
 \[\begin{split}
     \tan x_1&=\tan\frac{\pi}{4}=1\\
     \tan x_2&=\tan\frac{3\pi}{4}=\tan\left(\pi-\frac{\pi}{4}\right)=-\tan\frac{\pi}{4}=-1
 \end{split}\]
这样,$x_1<x_2$时,有$\tan x_1>\tan x_2$.
 当$x=k\pi$时 ($k\in\mathbb{Z}$), $\tan x=0$.
 
 \item 周期性\quad 由诱导公式$\tan (x+\pi)=\tan x$(这里$x$为定义域内任意数),知正切函数的周期是$\pi$, 现在我们证明$\pi$是正切函数的最小正周期.
 
\begin{proof}
    设$p$是$\tan x$的正周期且$0<p<\pi$,根据周期$p$的定义,我们有
$\tan (x+p)=\tan x$ (这里$x$是定义域内任意数).

令$x=0$, 则$\tan p=\tan 0=0$, 由此得到$p=k\pi\;\; (k\in\mathbb{Z}, \text{且 }k\ne 0)$, 这就是说,如果$p$是$\tan x$的周期,$p$只能是$\pi$的整数倍,这就与存在比$\pi$小的正周期$p$的假设矛盾.

因此,$\pi$就是$\tan x$的最小正周期.
\end{proof}

\begin{itemize}
    \item 函数$y=\tan \omega x$的最小正周期是$\frac{\pi}{|\omega|}$ ($\omega\ne 0$,  $\omega x$为定义域内的数).

    \item    函数$y=\tan (\omega x+\varphi)$的最小正周期是$\frac{\pi}{|\omega|}$ ($\omega,\varphi$为
    常数,且$\omega\ne 0$, $\omega x$为定义域内的数).
\end{itemize}

\item  渐近线\quad 由图7.9可以看到,当$0<x<\frac{\pi}{2}$时,
$\tan x>0$, 又当$x<\frac{\pi}{2}$而$x$又无限地趋近$\frac{\pi}{2}$时,(记作$x\to \frac{\pi^-}{2}$),
正切曲线无限地下降但与直线$x=\frac{\pi}{2}$
永远不相交,我们把这个性质说成当$x$由小于$\frac{\pi}{2}$
的方面无限趋近$\frac{\pi}{2}$时,$\tan x$的值增大并超出任何指定的正数,并且写成
\[\lim_{x\to\tfrac{\pi^-}{2}}\tan x=+\infty  \]

当$\frac{\pi}{2}<x<\pi$时,$\tan x<0$,又当$x>\frac{\pi}{2}$,而且$x$无限地趋近$\frac{\pi}{2}$时(记作$x\to\frac{\pi^+}{2}$),
正切曲线无限地上升,但与直线$x=\frac{\pi}{2}$
永远不相交,我们把这个性质说成当$x$由大于$\frac{\pi}{2}$的方面无限趋近
$\frac{\pi}{2}$时,$\tan x$取负值减小但其绝对值增
大并超出任何指定的正数,并且写成
\[\lim_{x\to\tfrac{\pi^+}{2}}\tan x=-\infty  \]
同样地,还有
\[\lim_{x\to -\tfrac{\pi^-}{2}}\tan x=+\infty,\qquad  \lim_{x\to -\tfrac{\pi^+}{2}}\tan x=-\infty \]

我们把直线$x=-\frac{\pi}{2}$和$x=\frac{\pi}{2}$叫做正切曲线的渐近线.一般地,直线$x=(2k+1)\frac{\pi}{2},\;\; k\in\mathbb{Z}$都是正切曲线的渐近线.
\end{enumerate}

\begin{rmk}
    这里对渐近线的叙述,同学们只要从图象上了解其意义就可以了,这个问题到高中还要详细地介绍. 
\end{rmk}

\begin{ex}
\begin{enumerate}
    \item 证明$y=\tan\left(x-\frac{\pi}{2}\right)$是奇函数,作出它的图象.
    \item 作$y=-|\tan x|$的图象,并说出它的周期.
    \item 求下列函数的周期:
    \[\tan\left(2x-\frac{x}{4}\right),\qquad \tan \frac{x}{2} \]
\end{enumerate}
\end{ex}

\section{余切函数的图象和性质}
\subsection{余切函数的图象}

我们知道余切函数的定义域是除去$x=k\pi\;\; (k\in\mathbb{Z})$的实数集.也就是由下面无数个开区间$\ldots,(-2\pi ,-\pi ),\;(-\pi ,0),\;(0,\pi ),\;(\pi ,2\pi ),\ldots$ 组成的一个集,图象都在$k\pi\;\; (k\in\mathbb{Z})$处断开.

我们又知道余切函数是奇函数,故它的图象关于原点对称.

由关系式$\cot x=-\tan\left(x+\frac{\pi}{2}\right)$, 知道余切函数$\cot x$的
图象就是正切函数型$y=-\tan\left(x+\frac{\pi}{2}\right)$的图象,又$y=-\tan\left(x+\frac{\pi}{2}\right)$的图象
,可以通过将正切曲线$y=\tan x$沿$x$轴向左平移$\frac{\pi}{2}$个单位,再把所得曲线作关于$Ox$轴反射,这个最后得到的曲线就是余切函数$y=\cot x$的图象(如图7.11).

\begin{figure}[htp]
    \centering
\begin{tikzpicture}[>=latex, scale=1]
\draw[->](-1,0)--(6,0)node[right]{$x$};
\draw[->](0,-4)--(0,4)node[right]{$y$};
\draw [domain=0+.3:pi-.3, thick, samples=1000] plot (\x, {tan(pi*.5 r+\x r)});
\draw [domain=0+.3:pi-.3, very thick, samples=1000] plot (\x, {-tan(\x r +pi*.5 r)});
\draw [domain=.5*pi+.3: 1.5*pi-.3, samples=1000] plot (\x, {tan(\x r)});

\foreach \x in {3,2,1,-1,-2,-3}
{
    \draw (0,\x)node[left]{$\x$}--(.1,\x);
}

\foreach \x/\xtext in {1/\tfrac{\pi}{2},2/\pi,3/\tfrac{3\pi}{2}}
{
    \draw [dashed] (\x*pi/2,-3.5)--(\x*pi/2,3.5);
    \node at (\x*pi/2+.2,0)[below]{$\xtext$};
}

\node at (-.2,-.2){$O$};
\foreach \x in {-1,-2,-3}
{
    \draw[dashed] (0,\x)--(pi,\x);
}
\foreach \x in {1,2,3}
{
    \draw[dashed] (0,\x)--(1.5*pi,\x);
}
\node at (3*pi/2-.35,3)[right]{$y=\tan x$};
\node at (2*pi/2,3.5){$y=\tan \left(x+\frac{\pi}{2}\right)$};
\node at (2*pi/2-.5,-3.5)[right]{$y=-\tan \left(x+\frac{\pi}{2}\right)=\cot x$};

\end{tikzpicture}
    \caption{}
\end{figure}


在具体画图时,我们也可以将正切函数$y=\tan x$的数值表作相应的改变得到余切函数$y=\cot x$的数值表,然后用描点法
作图.为此,只须将$\tan x$的自变数$x$的取值各减去$\frac{\pi}{2}$而使
原来的对应函数值都变为相应的相反数就可以了.今由正切函数$y=\tan x$, 在区间$\left(\frac{\pi}{2},\frac{3\pi}{2}\right)$上的数值表:

\begin{center}
\begin{tabular}{c|ccccccc}
\hline
$x$  &    $\frac{6\pi}{12}$  &    $\frac{7\pi}{12}$  &    $\frac{8\pi}{12}$  &    $\frac{9\pi}{12}$  &    $\frac{10\pi}{12}$  &    $\frac{11\pi}{12}$  \\    
\hline
$y=\tan x$   & 不存在  & $-3.73$  & $-1.73$  & $-1$  & $-0.58$  & $-0.27$  \\
\hline
$x$  &    $\pi$  &    $\frac{13\pi}{12}$  &    $\frac{14\pi}{12}$  &    $\frac{15\pi}{12}$  &    $\frac{16\pi}{12}$  &    $\frac{17\pi}{12}$ & $\frac{18\pi}{12}$ \\    
\hline
$y=\tan x$   & 0  & $0.27$  & $0.58$  & $1$  & $1.73$ & 3.73 & 不存在  \\
\hline
\end{tabular}    
\end{center}

作相应的改变后得到余切函数$y=\cot x$的数值表:
\begin{center}
\begin{tabular}{c|ccccccc}
\hline
$x$  &    $0$  &    $\frac{\pi}{12}$  &    $\frac{2\pi}{12}$  &    $\frac{3\pi}{12}$  &    $\frac{4\pi}{12}$  &    $\frac{5\pi}{12}$  \\    
\hline
$y=\cot x$   & 不存在  & $3.73$  & $1.73$  & $1$  & $0.58$  & $0.27$  \\
\hline
$x$  &    $\frac{6\pi}{12}=\frac{\pi}{2}$  &    $\frac{7\pi}{12}$  &    $\frac{8\pi}{12}$  &    $\frac{9\pi}{12}$  &    $\frac{10\pi}{12}$  &    $\frac{11\pi}{12}$ & $\frac{12\pi}{12}=\pi$ \\    
\hline
$y=\cot x$   & 0  & $-0.27$  & $-0.58$  & $-1$  & $-1.73$ & -3.73 & 不存在  \\
\hline
\end{tabular}    
\end{center}

根据此表,我们可以画出在区间$(0,\pi)$上的余切函数$y=\cot x$的图象.

余切函数$y=\cot x$的图象叫做\textbf{余切曲线}.整个余切曲线在点$x=k\pi\;\;(k\in\mathbb{Z})$处间断开,分成无数多个分支(图7.12).

\begin{figure}[htp]
    \centering
\begin{tikzpicture}[>=latex, scale=.7]
\draw[->] (-4,0)--(10,0)node[right]{$x$};
\draw[->] (0,-4)--(0,4)node[right]{$y$};
\foreach \x/\xtext in {-2/-\pi,-1/-\frac{\pi}{2},1/\frac{\pi}{2},2/\pi,3/\frac{3\pi}{2},4/2\pi,5/\frac{5\pi}{2}}
{
    \node at (\x*.5*pi-.3,0)[below]{$\xtext$};
}

\foreach \x in {-1,1,2,3}
{
    \draw[dashed] (\x*pi,-4)--(\x*pi,4);
}

\node at (-.3,-.3){$O$};

\foreach \y in {-1,1,3,5}
{
    \draw [domain=\y*0.5*pi-1.3 : \y*0.5*pi+1.3, samples=1000, thick] plot(\x, {cot (\x r) });
}

\end{tikzpicture}
    \caption{}
\end{figure}

\subsection{余切函数的主要性质}

由上一章的讨论和余切函数图象,我们可以得到余切函数$y=\cot x$的主要性质如下:
\begin{enumerate}
    \item 定义域\quad 余切函数$y=\cot x$的定义域是$x\ne k\pi\;\;(k\in\mathbb{Z})$的实数集,也就是由下面
无数个开区间:
\[\Bigl(k\pi,\; (k+1)\pi\Bigr), \qquad k=0,\pm1,\pm2,\pm3,\ldots\]
组成的一个集.

\item 值域\quad 余切函数$y=\cot x$的值域为一切实数.
\item 奇偶性\quad 余切函数是奇函数.
\item 函数的符号\quad 当$x$在第一、三象限时,$\cot x>0$;在第二、四象限时,$\cot x<0$,一般地:
\begin{itemize}
    \item 若$x\in\left(2k\pi,\;\frac{\pi}{2}+2k\pi\right)$或$\left((2k+1)\pi,\;\frac{\pi}{2}+(2k+1)\pi\right)$时,$\cot x>0$;
    \item 若$x\in\left(\frac{\pi}{2}+2k\pi,\; (2k+1)\pi\right)$或$\left(\frac{\pi}{2}+(2k+1)\pi,\; (2k+2)\pi\right)$时,$\cot x<0$,这里$k\in\mathbb{Z}$
\end{itemize}

\item 增减性\quad 余切函数$y=\cot x$在区间$(k\pi,\; (k+1)\pi)\;\; (k\in\mathbb{Z})$都是减函数.

当$x=\frac{\pi}{2}+2k\pi\;\; (k\in\mathbb{Z})$时,$\cot x=0$.

\item 周期性\quad 余切函数的周期是$\pi$.

事实上,假设$p>0$是$\cot x$的一个周期,根据周期$p$的定义,有$\cot (x+p)=\cot x$,这里x是$\cot x$的定义域中任何一个数.

令$x=\frac{\pi}{2}$,则
\[\cot\left(\frac{\pi}{2}+p\right)=\cot\frac{\pi}{2}\]
即$$\tan p=0$$
$\therefore\quad p=k\pi\quad (k\in\mathbb{Z},\; k\ne 0)$

由于$\cot(x+k\pi)=\cot x$对于$x$的任何容许值都成立,所以能使上面等式成立的最小正数$p=x$. 即$\cot x$的最小正周期等于$\pi$.

\begin{itemize}
    \item 函数$y=\cot \omega x$的最小正周期是$\frac{\pi}{|\omega|}$,($\omega\ne 0, \omega x$为定义域内的数).
    \item 函数$y=A\cot (\omega x+\varphi)$的最小正周期是$\frac{\pi}{|\omega|}$,($\omega,\varphi$为常数,$\omega\ne 0, \omega x$为定义域内的数).
\end{itemize}

\item 渐近线\quad 因为
\[\begin{split}
    \lim_{x\to 0^+} \cot x=+\infty,&\qquad \lim_{x\to 0^-} \cot x=-\infty\\
    \lim_{x\to \pi^-} \cot x=-\infty, &\qquad \lim_{x\to \pi^+} \cot x=+\infty
\end{split}\]
故$x=0$, $x=\pi$的直线就是余切曲线的渐近线.

一般地,直线$x=k\pi,\; (k\in\mathbb{Z})$,都是余切曲线的渐近线.

\end{enumerate}


\section*{习题7.3}
\addcontentsline{toc}{subsection}{习题7.3}

\begin{enumerate}
    \item 确定差的符号:
\begin{multicols}{2}
    \begin{enumerate}
\item $\tan 72^{\circ}-\tan 80^{\circ}$
\item $\cot 15^{\circ}-\cot 16^{\circ}$
\item $\tan 70^{\circ}-\cot 70^{\circ}$
\item $\cot 100^{\circ}-\cot 90^{\circ}$
    \end{enumerate}
\end{multicols}

\item 求下列各式子的符号:
\begin{multicols}{2}
    \begin{enumerate}
\item $\sin 3\cdot \tan 5$
\item $\cos 8\cdot \cos 5\cdot \tan 1$
\item $\tan 5\cdot \cot 3\cdot \tan 1$
\item $\sin(-5)\cos(-3)\tan(-2)\cot2$
    \end{enumerate}
\end{multicols}

\item 在怎样的区间内随着不超过$360^{\circ}$的正角$\alpha$的增大:
\begin{enumerate}
    \item $\sin\alpha$和$\cot\alpha$同时增大?同时减小?
    \item $\cos\alpha$和$\tan\alpha$同时增大?同时减小?
    \item $\sin\alpha$ 和 $\cos\alpha$同时增大?同时减小?
\end{enumerate}


\item 作下面函数的图象:
\[y=\cot\left(x-\frac{\pi}{2}\right),\qquad y=\cot\left(x+\frac{\pi}{3}\right) \]
\item 利用单位圆和余切函数线,解下面不等式
\[\cot x> 1,\qquad 3\cot x+\sqrt{3}<0\]

\item 作下面函数的图象:
\[y=\sec x,\qquad y=\csc x\]
\end{enumerate}


为了便于比较和查阅,我们把正弦函数,余弦函数,正切函数,余切函数的主要性质列表如下:

{\small
\begin{longtable}{p{.1\textwidth}|p{.18\textwidth}p{.18\textwidth}p{.18\textwidth}p{.18\textwidth}}
    \hline
函数 & $y=\sin x$ & $y=\cos x$    & $y=\tan x$ & $y=\cot x$\\
\hline
定义域 & $\mathbb{R}$ & $\mathbb{R}$ & $\{x|x\in \mathbb{R},\;\; x\ne\frac{\pi}{2}+k\pi\}$ & $\{x|x\in \mathbb{R},\;\; x\ne k\pi\}$\\
\hline
值域 & $[-1,1]$& $[-1,1]$& $\mathbb{R}$ &$\mathbb{R}$ \\
\hline
奇偶性 & 奇函数 &偶函数& 奇函数 &奇函数\\
\hline
函数的符号 & 在一、二象限为正;在三、四象限为负
& 在一、四象限为正;在二、三象限为负
& 在一、三象限为正;在二、四象限为负
& 在一、三象限为正;在二、四象限为负\\
\hline
增减性 & 在一、二象限是增函数;在三、四象限是减函数 
& 在一、二象限是减函数;在三、四象限是增函数 
& 在各象限都为增函数 
& 在各象限都为减函数 \\
\hline
函数的极值& $x=\frac{\pi}{2}+2k\pi$时,取极大值1;$x=\frac{3\pi}{2}+2k\pi$时,取极小值$-1$
& $x=2k\pi$时,取极大值1;$x=(2k+1)\pi$时,取极小值$-1$&无& 无\\
\hline
周期性& $2\pi$& $2\pi$& $\pi$& $\pi$\\
\hline
渐近线 & 无& 无& 直线$x=(2k+1)\frac{\pi}{2}$& 直线$x=k\pi$\\
\hline
\end{longtable}
}

\section{正弦型曲线}
在这一节,我们来研究正弦型曲线,即函数$y=A\sin(mx+a)$的图象,这个图象在学习物理、电工和力学时将会遇到,下面我们先研究几种特殊正弦型曲线的作法,然后再归结到一般情形.

\subsection{函数$y=A\sin x$ 的图象}

我们取$A=\frac{1}{2}$, $A=1$, $A=2$三种情形来讨论,即讨论
\[y=\frac{1}{2}\sin x,\qquad  y=\sin x,\qquad  y=2\sin x\]
考虑到函数$\sin x$的周期是$2\pi$, 我们只须画出闭区间$[0,
2\pi]$上的图象,先列出$x$由0到
$\frac{\pi}{2}$,每隔$\frac{\pi}{6}$取值的正弦
值表,然后,应用诱导公式$\sin(\pi-x)=\sin x$,
列出$\frac{\pi}{2}$
到$\pi$之间的正弦值;最后应用诱导公式
$\sin (\pi+x) =-\sin x$
列出$\pi$到$2\pi$之间的正弦值.
对于同一个$x$值,函数$\frac{1}{2}\sin x$的值为函数$\sin x$的值的$\frac{1}{2}$,而函数$2\sin x$的值为函数$\sin x$的值的2倍.

把它们的自变量及对应函数值列表于下:
\begin{center}
\begin{tabular}{cccc}
    \hline
$x$ & $y=\sin x$ & $y=\tfrac{1}{2}\sin x$ & $y=2\sin x$\\
\hline
$\cdots$ & $\cdots$&$\cdots$&$\cdots$\\
$0$   &    0   &   0 &  0\\
$\tfrac{\pi}{6}$   &    0.5   &   0.25 &  1\\
$\tfrac{\pi}{3}$   &    0.87   &   0.44 &  1.74\\
$\tfrac{\pi}{2}$   &    1   &   0.5 &  2\\
$\tfrac{2\pi}{3}$   &    0.87   &   0.44 &  1.74\\
$\tfrac{5\pi}{6}$   &    0.5   &   0.25 &  1\\
$\pi$   &    0   &   0 &  0\\
$\tfrac{7\pi}{6}$   &    $-0.5$   &   $-0.25$ &  $-1$\\
$\tfrac{4\pi}{3}$   &    $-0.87$   &   $-0.44$ &  $-1.74$\\
$\tfrac{3\pi}{2}$   &    $-1$   &   $-0.5$ &  $-2$\\
$\tfrac{5\pi}{3}$   &    $-0.87$   &   $-0.44$ &  $-1.74$\\
$\tfrac{11\pi}{6}$   &    $-0.5$   &   $-0.25$ &  $-1$\\
$2\pi$   &    0   &   0 &  0\\
$\cdots$ & $\cdots$&$\cdots$&$\cdots$\\
\hline
\end{tabular}
\end{center}

对于每一个函数,把表内$x$、$y$的每一对值作为点的坐标,在直角坐标系内作出对应点,将它们依次连结成平滑曲
线,这三条曲线就是正弦函数$y=\sin x$, 及正弦型函数$y=\frac{1}{2}\sin x$和$y=2\sin x$的图象(如图7.13).

\begin{figure}[htp]
    \centering
\begin{tikzpicture}[>=latex]
\draw [->](-1,0)--(7.8,0)node[right]{$x$};
\draw [->](0,-3)--(0,3)node[right]{$y$};
\foreach \x in {-1,-2,1,2}
{
    \draw(0,\x)node[left]{$\x$}--(.1,\x);
}

\node at (-.2,-.2){$O$};

\foreach \y in {1,2,.5}
{
    \draw [domain=0:7, samples=1000, thick] plot(\x, {\y*sin(\x r)});
}

\draw (.5*pi,0)node[below]{$\frac{\pi}{2}$}--(.5*pi,.1);

\draw (1.5*pi,0)--(1.5*pi,.1)node[above]{$\frac{3\pi}{2}$};

\node at (pi-.2,0)[below]{$\pi$};
\node at (2*pi+.2,0)[below]{$2\pi$};

\node at (7,.3)[right]{$y=\tfrac{1}{2}\sin x$};
\node at (7,.9)[right]{$y=\sin x$};
\node at (7,1.5)[right]{$y=2\sin x$};
\end{tikzpicture}
    \caption{}
\end{figure}


由图象显然看出:
\begin{enumerate}
    \item 对于同一横坐标$x$, $y=\frac{1}{2}\sin x$的纵坐标为$y=\sin x$的纵坐标的$\frac{1}{2}$,
    $y=2\sin x$的纵坐标为$y=\sin x$的纵坐标的2倍,
    因此,可以说,把曲线$y=\sin x$ 沿纵轴方向
    压缩到$\frac{1}{2}$,就得到曲线$y=\frac{1}{2}\sin x$; 沿纵轴方向拉长2
倍,就得到曲线$y=2\sin x$.

\item $y=\sin x$的最大纵坐标是1; $y=2\sin x$的最大纵坐标是2; $y=\frac{1}{2}\sin x$的最大纵坐标是$\frac{1}{2}$. 我们这个最大的纵坐标叫做该曲线的振幅.
\item 它们的周期相同,都是$2\pi$.
\end{enumerate}

推广到一般情形,就得出下面的结论:

\begin{blk}{}
    函数$y=A\sin x$的图象是把$y=\sin x$的图象沿着纵轴方向压缩到$A\; (0<A<1)$倍或拉长到$A\; (A>1)$倍而得到,曲线$y=A\sin x$的振幅为$A$, 周期为$2\pi$.
\end{blk}

\subsection{函数$y=\sin mx$的图象}

我们取$m=\frac{1}{2}$, $m=1$, $m=2$三种情形来讨论,即讨论$y=\sin\frac{x}{2}$, $y=\sin x$, $y=\sin2x$.

三个函数的周期分别是$4\pi$, $2\pi$和$\pi$, 它们有相同的振
幅$A=1$. 所有这些函数在自变量$mx$取相同值的变化过程中对应的函数值也相同,但现在$m$值不同,因此,只有当取不同的值时,才能使这些函数的对应值相同.
比如,
\begin{itemize}
    \item 当$x=\frac{\pi}{6}$时,$\sin x$的值为0.5;
    \item 当$x=\frac{\pi}{3}$时,$\sin \frac{1}{2}x$的值为0.5;
    \item 当$x=\frac{\pi}{12}$时,$\sin 2x$的值为0.5.
\end{itemize}

这就是说,只要我们有一个函数$y=\sin x$在闭区间$[0,2\pi]$上的函数值表,我们就可以通过把这个数值表中自变量
$x$的取值乘以2, 就得到函数$y=\sin\frac{1}{2}x$在闭区间$[0,4\pi]$上的函数值表;如果把$y=\sin x$的数值表中自变量$x$的取值除以2, 就可以得到$y=\sin 2x$在闭区间$[0,\pi]$上的函数值表.

列表作图于下:

将$y=\sin x$的函数值表
\begin{center}
\begin{tabular}{c|cccccccc}
\hline
$x$ &  $\cdots$   &  0    &  $\frac{\pi}{6}$     &   $\frac{\pi}{3}$     &   $\frac{\pi}{2}$      &  $\frac{2\pi}{3}$         &  $\frac{5\pi}{6}$ &  $\pi$  \\
\hline
$y=\sin x$ &  $\cdots$   &    0  & 0.5      &  0.87     &  1      &   0.87       &  0.5   & 0\\
\hline
$x$ &     & $\frac{7\pi}{6}$      &   $\frac{4\pi}{3}$     &  $\frac{3\pi}{2}$      &  $\frac{5\pi}{3}$       &    $\frac{11\pi}{6}$       &  $2\pi$ & $\cdots$   \\
\hline
$y=\sin x$ &      &  $-0.5$   &  $-0.87$    &   $-1$    & $-0.87$      &   $-0.5$     &   0      & $\cdots$\\
\hline
\end{tabular}
\end{center}
里的$x$的取值分别乘以2得到$y=\sin\frac{1}{2}x$的函数值表如下:
\begin{center}
\begin{tabular}{c|cccccccc}
\hline
$x$ &  $\cdots$   &  0    &  $\frac{\pi}{3}$     &   $\frac{2\pi}{3}$     &   $\pi$      &  $\frac{4\pi}{3}$         &  $\frac{5\pi}{3}$ &  $2\pi$  \\
\hline
$y=\sin\frac{1}{2}x$ &  $\cdots$   &    0  & 0.5      &  0.87     &  1      &   0.87       &  0.5   & 0\\
\hline
$x$ &     & $\frac{7\pi}{3}$      &   $\frac{8\pi}{3}$     &  $3\pi$      &  $\frac{10\pi}{3}$       &    $\frac{11\pi}{3}$       &  $4\pi$ & $\cdots$   \\
\hline
$y=\sin\frac{1}{2}x$ &      &  $-0.5$   &  $-0.87$    &   $-1$    & $-0.87$      &   $-0.5$     &   0      & $\cdots$\\
\hline
\end{tabular}
\end{center}

作图:
\begin{figure}[htp]
    \centering
\begin{tikzpicture}[>=latex, scale=.8]
\draw[->] (-1,0)--(14,0)node[right]{$x$};
\draw[->]  (0,-2)--(0,2)node[right]{$y$};
\draw [domain=0:6.8, samples=1000, thick] plot(\x, {sin(\x r)});
\draw [domain=0:13.6, samples=1000, very thick] plot(\x, {sin(0.5*\x r)});
\node at (-.25,-.25){$O$};
\foreach \x in {-1,1}
{
    \draw (0,\x)node[left]{$\x$}--(.2,\x);
}
\foreach \x in {1,2,...,8}
{
    \draw (pi*\x/2,0)--(pi*\x/2,0.2);
}
\node at (pi/2,0)[below]{$\frac{\pi}{2}$};
\node at (pi,0)[below]{$\pi$};
\node at (3*pi/2,0)[below]{$\frac{3\pi}{2}$};
\node at (2*pi,0)[below]{$2\pi$};
\node at (5*pi/2,0.2)[above]{$\frac{5\pi}{2}$};
\node at (3*pi,0.2)[above]{$3\pi$};
\node at (7*pi/2,0.2)[above]{$\frac{7\pi}{2}$};
\node at (4*pi,0.2)[above]{$4\pi$};
\node at (3*pi/2,-1.5){$y=\sin x$};
\node at (3*pi,-1.5){$y=\sin \frac{1}{2}x$};
\end{tikzpicture}
    \caption{}
\end{figure}

由上面的表和图7.14可以看出:
\begin{enumerate}
    \item 当$y=\sin\frac{1}{2}x$的横坐标为$y=\sin x$的横坐标的2倍时,它们的纵坐标相等.
    \item 它们的振幅相同,都是1.
    \item $y=\sin\frac{1}{2}x$的周期是$4\pi=\frac{2\pi}{1/2}$,
二倍于$y=\sin x$的周期.
\end{enumerate}

因此,$y=\sin\frac{1}{2}x$的图象是把$y=\sin x$的图象沿横轴方向拉长2倍而得到.

将$y=\sin x$的函数值表里的$x$取值分别除以2, 得到$y=\sin 2x$的函数值如下:
\begin{center}
\begin{tabular}{c|cccccccc}
\hline
$x$ &  $\cdots$   &  0    &  $\frac{\pi}{12}$     &   $\frac{\pi}{6}$     &   $\frac{\pi}{4}$      &  $\frac{\pi}{3}$         &  $\frac{5\pi}{12}$ &  $\frac{\pi}{2}$  \\
\hline
$y=\sin\frac{1}{2}x$ &  $\cdots$   &    0  & 0.5      &  0.87     &  1      &   0.87       &  0.5   & 0\\
\hline
$x$ &     & $\frac{7\pi}{12}$      &   $\frac{2\pi}{3}$     &  $\frac{3\pi}{4}$      &  $\frac{5\pi}{6}$       &    $\frac{11\pi}{12}$       &  $\pi$ & $\cdots$   \\
\hline
$y=\sin\frac{1}{2}x$ &      &  $-0.5$   &  $-0.87$    &   $-1$    & $-0.87$      &   $-0.5$     &   0      & $\cdots$\\
\hline
\end{tabular}
\end{center}

作图象:
\begin{figure}[htp]
    \centering
\begin{tikzpicture}[>=latex, scale=1.4]
\draw[->] (-1,0)--(7,0)node[right]{$x$};
\draw[->]  (0,-2)--(0,2)node[right]{$y$};
\draw [domain=0:6.8, samples=1000, thick] plot(\x, {sin(\x r)});
\draw [domain=0:3.4, samples=1000, very thick] plot(\x, {sin(2*\x r)});
\node at (-.25,-.25){$O$};
\foreach \x in {-1,1}
{
    \draw (0,\x)node[left]{$\x$}--(.1,\x);
}
\foreach \x in {1,2,3,4}
{
    \draw (pi*\x/2,0)--(pi*\x/2,0.1);
}
\node at (pi/2,0)[below]{$\frac{\pi}{2}$};
\node at (pi,0)[below]{$\pi$};
\node at (3*pi/2,0)[above]{$\frac{3\pi}{2}$};
\node at (2*pi,0)[below]{$2\pi$};
\node at (pi/4,0)[below]{$\frac{\pi}{4}$};
\node at (pi/4*3,0)[above]{$\frac{3\pi}{4}$};
\draw[dashed] (pi/4,0)--(pi/4,1);
\draw[dashed] (pi/4*3,0)--(pi/4*3,-1);
\node at (7,.6){$y=\sin x$};
\node at (3.5,.6){$y=\sin 2x$};


\end{tikzpicture}
    \caption{}
\end{figure}

由上面的表和图7.15可以看出:
\begin{enumerate}
    \item 当$y=\sin2x$的横坐标为$y=\sin x$的横坐标的一半时,它们的纵坐标相等.
    \item 它们的振幅相同,都是1.
    \item $y=\sin2x$的周期是$\pi\left(=\frac{\pi}{2}\right)$
\end{enumerate}

因此,$y=\sin2x$的图象是把曲线$y=\sin x$沿横轴方向压缩$\frac{1}{2}$倍,就得到曲线$y=\sin2x$.

根据上述情况,可得一般结论如下:

\begin{blk}{}
   函数$y=\sin mx$的图象是把$y=\sin x$的图象沿横轴方向拉长$\frac{1}{m}$倍$(0<m<1)$,或压缩到$\frac{1}{m},\; (m>1)$而得到.曲线$y=\sin mx$的振幅为1, 周期为$\frac{2\pi}{m}$. 
\end{blk}

\subsection{函数$y=A\sin(mx+\alpha)$的图象}

现在我们来说明如何画一般的正弦型曲线$y=A\sin(mx+\alpha)$. 这里$A$和$m$为已给的正实数,$\alpha$为已给的实数.

把函数
\begin{equation}
    y=A\sin(mx+\alpha)
\end{equation}
改写成
\begin{equation}
    y=A\sin m\left(x+\frac{\alpha}{m}\right)
\end{equation}
函数(7.5), (7.6)和函数$y=A\sin mx$的振幅相同,都等于$A$. 而且它们的周期也相同,都等于$\frac{2\pi}{m}$.

显然,$y=A\sin m\left(x+\frac{\alpha}{m}\right)$的图象上一点的横坐标$x_1$所对应的纵坐标是$y=A\sin m\left(x_1+\frac{\alpha}{m}\right)$, 并且恒等于$y=A\sin mx$的图象上横坐标是
$x_1+\frac{\alpha}{m}$的一点的纵坐标.由于在$Ox$轴上点$x_1$是在点$x_1+\frac{\alpha}{m},\;(\alpha>0)$的左方$\left|\frac{\alpha}{m}\right|$个单位处,或者点$x_1$是在点$x_1+\frac{\alpha}{m},\;(\alpha<0)$的右方的$\left|\frac{\alpha}{m}\right|$个单位处,所以我们只须把
$y=A\sin mx$的图象沿横轴方向左移$x_1+\frac{\alpha}{m},\;(\alpha>0)$个单位,或者右移$x_1+\frac{\alpha}{m},\;(\alpha<0)$个单位,就得到$y=A\sin m\left(x+\frac{\alpha}{m}\right)$的图象.

\begin{example}
作函数$y=1.5\sin \left(2t+\frac{\pi}{4}\right)$的图象.
\end{example}

\begin{solution}
所求函数$y=1.5\sin \left(2t+\frac{\pi}{4}\right)$的图象,可以由函数$y=1.5\sin 2t$的图象沿横轴方向左移$\frac{\pi}{8}$个单位得到.这时,曲线$y=1.5\sin2t$上的$(0, 0)$点就平移到$\left(-\frac{\pi}{8},0\right)$
点,我们把它作为曲线$y=1.5\sin2 \left(t+\frac{\pi}{8}\right)$的起点.

列出函数$y=\sin t$在闭区间$[0, 2\pi]$上自变量$t$每隔$\frac{\pi}{4}$的函数值表:
\begin{center}
\begin{tabular}{c|ccccccccc}
    \hline
$t$  &  0  &  $\frac{\pi}{4}$ & $\frac{\pi}{2}$   & $\frac{3\pi}{4}$  & $\pi$   & $\frac{5\pi}{4}$  & $\frac{3\pi}{2}$   & $\frac{7\pi}{4}$ &$2\pi$ \\ 
    \hline
$y=\sin t$  & 0   &   0.71 &  1 & 0.71   & 0  & $-$0.71   &  $-$1 & $-$0.71   & 0\\
    \hline
\end{tabular}
\end{center}

按照给出的曲线方程作相应的变更,分别得到函数$y=\sin2t$, $y=1.5\sin2t$和$y=1.5\sin2\left(t+\frac{\pi}{8}\right)$在一个周期内的函数值表如下:
\begin{center}
\begin{tabular}{c|ccccccccc}
    \hline
$t$  &  0  &  $\frac{\pi}{8}$ & $\frac{\pi}{4}$   & $\frac{3\pi}{8}$  & $\frac{\pi}{2}$   & $\frac{5\pi}{8}$  & $\frac{3\pi}{4}$   & $\frac{7\pi}{8}$ &$\pi$ \\ 
    \hline
$y=\sin 2t$  & 0   &   0.71 &  1 & 0.71   & 0  & $-$0.71   &  $-$1 & $-$0.71   & 0\\
$y=1.5\sin 2t$ & 0   &   1.07 &  1.5 & 1.07   & 0  & $-$1.07   &  $-$1.5 & $-$1.07   & 0\\
    \hline
\end{tabular}
\end{center}

\begin{center}
\begin{tabular}{c|ccccc}
    \hline
$t$  &  $-\frac{\pi}{8}$ & 0 & $\frac{\pi}{8}$   & $\frac{\pi}{4}$  & $\frac{3\pi}{8}$   \\     \hline
$y=1.5\sin \left(2t+\frac{\pi}{4}\right)$  & 0   &   1.07 &  1.5 & 1.07   & 0 \\
    \hline
    $t$  && $\frac{\pi}{2}$  & $\frac{5\pi}{8}$   & $\frac{3\pi}{4}$ &$\frac{7\pi}{8}$ \\
    \hline
    $y=1.5\sin \left(2t+\frac{\pi}{4}\right)$  &  & $-$1.07   &  $-$1.5 & $-$1.07   & 0\\
    \hline
\end{tabular}
\end{center}

作图:
\begin{figure}[htp]
    \centering
\begin{tikzpicture}[>=latex, xscale=2]
\draw[->](-.8,0)--(3.5,0)node[right]{$x$};
\draw[->] (0,-2.5)--(0,2.5)node[right]{$y$};    

\draw [domain=0:pi, samples=1000, thick] plot(\x, {sin(2*\x  r)});
\draw [domain=0:pi, samples=1000, thick] plot(\x, {1.5*sin(2*\x  r)});
\draw [domain=-pi/8:pi*7/8, samples=1000, thick] plot(\x, {1.5*sin(2*\x r +pi/4  r)});
\node at (-.1,-.2){$O$};

\foreach \x/\xtext in {-1/-\frac{\pi}{8},1/\frac{\pi}{8},2/\frac{\pi}{4},3/\frac{3\pi}{8},4/\frac{\pi}{2}}
{
    \node at (\x*pi/8,0)[below]{$\xtext$};
}

\foreach \x/\xtext in {5/\frac{5\pi}{8},6/\frac{3\pi}{4},7/\frac{7\pi}{8},8/\pi}
{
    \node at (\x*pi/8,0)[above]{$\xtext$};
}

\foreach \x in {1,2}
{
    \draw[dashed] (\x*pi/8,0)--(\x*pi/8,1.5);
    \draw[dashed] (\x*pi/8+0.5*pi,0)--(\x*pi/8+0.5*pi,-1.5);
}

\draw[dashed](pi/4,1.5)--(0,1.5);  \draw[dashed](pi/4,1)--(0,1);
\draw[dashed](3*pi/4,-1.5)--(0,-1.5);  \draw[dashed](3*pi/4,-1)--(0,-1);

\foreach \x in {-1.5,-1,1,1.5}
{
    \draw (0,\x)node[left]{$\x$}--(.05,\x);
}
\end{tikzpicture}
    \caption{}
\end{figure}
\end{solution}

上面这种作图方法,把几种曲线之间的关系明确的揭露出来,使我们能够了解到它们之间的内在联系,但是,这样作图比较费事,在实际问题中往往只须知道函数
$y=A\sin(mx+\alpha)$的草图,只要这个图能够表示出它的
三个特征量:
振幅$A$, 周期$\frac{2\pi}{m}$,起点$\left(-\frac{\alpha}{m},0\right)$就可以了.

$y=A\sin(mx+\alpha)$的图象的简便作法如下:
\begin{enumerate}
    \item 化函数$y=A\sin(mx+\alpha)$为下面的形式
    \[y=A\sin m\left(x+\frac{\alpha}{m}\right)\]
    \item 确定:振幅为$A$; 周期为$\frac{2\pi}{m}$;起点为$\left(-\frac{\alpha}{m},0\right)$.
    \item 选取比例尺:1个单位为几厘米.
    \item 在$Ox$轴上以$\left(-\frac{\alpha}{m},0\right)$为起点,并把$x=-\frac{\alpha}{m}$
到$x=-\frac{\alpha}{m}+\frac{2\pi}{m}$的一段分成4等份,每等份长为$\frac{2\pi}{m}\div 4=\frac{\pi}{2m}$, 各分点的横坐标是
\[-\frac{\alpha}{m},\quad  -\frac{\alpha}{m}+\frac{\pi}{2m},\quad -\frac{\alpha}{m}+\frac{\pi}{m},\quad -\frac{\alpha}{m}+\frac{3\pi}{2m},\quad -\frac{\alpha}{m}+\frac{2\pi}{m}  \]
它们对应的纵坐标分别是$0, A,0,-A,0$.
\end{enumerate}

根据上述结果,便可作出函数$y=A\sin (mx+\alpha)$在区间$\left[-\frac{\alpha}{m},\; -\frac{\alpha}{m}+\frac{2\pi}{m}\right]$上的图象.


\begin{example}
求作函数$y=\frac{3}{2}\sin(3t-\pi)$的图象.
\end{example}

\begin{solution}
\begin{enumerate}
    \item 化函数$y=\frac{3}{2}\sin(3t-\pi)$为下面的形式:
    \[y=\frac{3}{2}\sin 3\left(t-\frac{\pi}{3}\right)\]
\item 确定:振幅为$\frac{3}{2}$,周期为$\frac{2\pi}{8}$,起点为$\left(\frac{\pi}{8},\; 0\right)$
\item 选取比例尺:1个单位长$=0.8$厘米.
\item 列表:
\begin{center}
    \begin{tabular}{c|ccccccc}
  \hline      
$t$   &  $\cdots$  & $\frac{\pi}{3}$  & $\frac{\pi}{2}$  & $\frac{2\pi}{3}$  & $\frac{5\pi}{6}$  & $\pi$ &$\cdots$\\
\hline
$y=\frac{3}{2}\sin(3t-\pi)$ &  $\cdots$  & 0 & $\frac{3}{2}$ &0 &$-\frac{3}{2}$ & 0&$\cdots$\\
\hline
    \end{tabular}
\end{center}
  \end{enumerate}  


\begin{figure}[htp]
    \centering
\begin{tikzpicture}[scale=1.3, >=latex]
\draw[->] (-2,0)--(7,0)node[right]{$x$};
\draw[->] (0,-2.5)--(0,2.5)node[right]{$y$};
\draw [domain=-pi/3:2*pi, samples=1000, dashed] plot(\x, {1.5*sin(3*\x r - pi r)});
\draw [domain=pi/3:pi, samples=1000, very thick] plot(\x, {1.5*sin(3*\x r - pi r)});
\node at (-.2,-.2){$O$};

\foreach \x/\xtext in {-2/-\frac{\pi}{3},-1/-\frac{\pi}{6},1/\frac{\pi}{6},3/\frac{\pi}{2},5/\frac{5\pi}{6},7/\frac{7\pi}{6},9/\frac{3\pi}{2},11/\frac{11\pi}{6}}
{
    \draw (\x*pi/6,0)node[below]{$\xtext$}--(\x*pi/6,.1);
}

\foreach \x/\xtext in {2/\frac{\pi}{3},4/\frac{2\pi}{3},6/\pi,8/\frac{4\pi}{3},10/\frac{5\pi}{3},12/2\pi}
{
    \draw (\x*pi/6,0)--(\x*pi/6,.1)node[above]{$\xtext$};
}

\foreach \x in {-1.5,-1,1,1.5}
{
    \draw(0,\x)node[left]{$\x$}--(.1,\x);
}


\end{tikzpicture}
    \caption{}
\end{figure}
\end{solution}


图中虚线表示曲线$y=\frac{3}{2}\sin(3t-\pi)$上未画出的部分,我们把图象变换总结如下:

\begin{blk}{位置变换}
\begin{enumerate}
    \item $y=\sin(x+b)$的图象可由$y=\sin x$的图象沿$x$轴的方向左、右平移而得到(当$b>0$时向左,$b<0$时向右).
    \item $y=\sin x+\ell$的图象可由$y=\sin x$的图象沿$y$轴的方向上、下平移而得到(当$\ell>0$时向上,$\ell<0$时向下).
    \item $y=-\sin x$的图象可由$y=\sin x$的图象作关于$Ox$轴的反射得到.
\end{enumerate}
\end{blk}

\begin{blk}{形状变换}
    \begin{enumerate}
        \item $y=A\sin x\; (A>0)$的图象可由$y=\sin x$的图象的振幅扩大$A$倍而得到.
\item $y=\sin mx\; (m>0)$的图象可由$y=\sin x$的图象沿$x$轴方向拉长($0<m<1$)或压缩($m>1$)到$\frac{1}{m}$倍而得到(周期$T$为$\frac{2\pi}{m}$).
    \end{enumerate}
\end{blk}


\begin{blk}{位置形状全变换}
\[y=A\sin(mx+\alpha)+\ell=A\sin m\left(x+\frac{\alpha}{m}\right) +\ell\quad  (A> 0,\; m>0)\]
可由$y=\sin x$的图象沿轴方向拉长或压缩到原来的
$\frac{1}{m}$倍,(周期变为$\frac{2\pi}{m}$), 沿$x$轴向左或向右平移$\left|\frac{\alpha}{m}\right|$
个单位,然后把振幅扩大到原来的$A$倍;最后沿$y$轴向上或向下平移$|\ell|$个单位.
\end{blk}

\begin{ex}
\begin{enumerate}
    \item 作下面三角函数的图象:
    \begin{multicols}{2}
\begin{enumerate}
    \item $y=1+\sin2x$
    \item $y=\sin\left(x+\frac{\pi}{4}\right) $
    \item $y=2\sin\left(3t-\frac{\pi}{3}\right)$
    \item $y=\frac{2}{3}\cos\left(\frac{1}{2}x+\frac{\pi}{4}\right)$
\end{enumerate}
    \end{multicols}
    \item 求下面函数的图象在$(0, 2\pi)$区间的交点坐标:
    \begin{enumerate}
        \item $y=\sin 2x$和$y=2\cos x$
        \item $y=\sin 3x$和$y=2\sin x$
    \end{enumerate}
\end{enumerate}
\end{ex}

\section*{复习题七}
\addcontentsline{toc}{section}{复习题七}
\begin{enumerate}
    \item 求$\theta$所在的象限,已知:
\begin{multicols}{2}
\begin{enumerate}
    \item $\sin\theta$和$\cot \theta$同号
    \item $\cos\theta$和$\tan \theta$异号
    \item $\frac{\tan\theta}{\cot\theta}$是正值
    \item $\sin\theta <0$,而$\cos\theta>0$
\end{enumerate}
\end{multicols}
\item 根据三角函数性质,比较两函数值的大小:
\begin{multicols}{2}
    \begin{enumerate}
        \item $\sin 20^{\circ},\qquad \sin 21^{\circ}$
        \item $\cos 20^{\circ},\qquad \cos 120^{\circ}$
        \item $\tan 120^{\circ},\qquad \tan 121^{\circ}$
        \item $\cot175^{\circ},\qquad \cot 185^{\circ}$
        \item $\cos 1^{\circ},\qquad \cos 1$
        \item $\sin 1,\qquad \cos 1$
        \item $\sin(-310^{\circ}),\qquad \cos(-310^{\circ})$
        \item $\sin\frac{4\pi}{5},\qquad \cos\frac{4\pi}{5}$
    \end{enumerate}
    \end{multicols}


\item 说出下列各函数的周期:
\begin{multicols}{2}
\begin{enumerate}
    \item $y=\sin 4x$
    \item $y=\cos\frac{x}{2}$
    \item $y=\tan ax \quad (a>0)$
    \item $y=\sin\left(x+\frac{\pi}{8}\right)$
    \item $y=\cos\left(2x-\frac{\pi}{6}\right)$
    \item $y=3\tan\left(\frac{x}{2}+\frac{\pi}{4}\right)$
\end{enumerate}
\end{multicols}

\item \begin{enumerate}
    \item 等式$\sin\left(\frac{7\pi}{6}+\frac{2\pi}{3}\right)=\sin\frac{7\pi}{6}$,能不能成立?
    \item 如果上面的等式成立,能不能说$\frac{2\pi}{3}$是$\sin x$的周期?为什么?
\end{enumerate}

\item \begin{enumerate}
    \item 等式$\sin(x+4\pi)=\sin x$是不是对于$x$的一切值都能成立?
    \item 如果上面的等式对于$x$的一切值都能成立,能不能说$4\pi$是$\sin x$的周期?为什么?
\end{enumerate}

\item 用下面给出的周期,各举出一个周期函数来:
\[3\pi,\qquad \frac{\pi}{6},\qquad \frac{2\pi}{3}\]

\item 说明正弦曲线$y=\sin x$经过怎样的变化能得到下列的
函数曲线:
\begin{multicols}{2}
\begin{enumerate}
    \item $y=\frac{1}{2}\sin 3x$
    \item $y=2\sin\left(x+\frac{\pi}{4}\right)$
    \item $y=\sin\left(2x+\frac{\pi}{2}\right)$
    \item $y=2\sin\left(3x-\frac{\pi}{8}\right)$
\end{enumerate}
\end{multicols}

\item 电流强度$I$随时间$t$变化的函数关系是$I=A\sin\omega t$.设$\omega=100\pi$(弧度/秒),$A=5$(安培):
\begin{enumerate}
    \item 求电流强度变化的周期和频率(往复振动一次所需
的时间$T=\frac{2\pi}{\omega}$, 叫做振动的周期,单位时间内往复振动的
次数$f=\frac{1}{T}=\frac{\omega}{2\pi}$,叫做振动的频率);
\item 当$t=0$、$\frac{1}{200}$、$\frac{1}{100}$、$\frac{3}{200}$、$\frac{1}{50}$(秒)时,求电流强度;
\item 画出图象表示电流强度$I$随时间$t$的变化情况(以
$I$为纵坐标,0.5cm表示1安培;以$t$为横坐标,1cm表示$\frac{1}{200}$秒).
\end{enumerate}

\item 如图,一质点$P$在线段$CC'$上作往复运动,取原点
$O$为$CC'$中点,又$CC'=16$cm, 设$P$点离开原点$O$的坐
标$x$是时间$t$的函数$x=8\sin\left(\frac{\pi}{4}t+\frac{\pi}{6}\right)$
\begin{center}
    \begin{tikzpicture}
   \draw (-5,0)--(5,0);
\foreach \x/\xtext in {-4/C,0/O,4/C'}
{
    \draw (\x,0) [fill=black] circle(1.5pt)node[below]{$\xtext$};
}        

\draw (1,0) [fill=black] circle(1.5pt)node[above]{$P$};
    \end{tikzpicture}
\end{center}
问: \begin{enumerate}
\item 当$t=0, 2, 4$秒时,$P$点在何位置?
\item 从0秒算起,经过多少秒,$P$点第一次到达$C$点.
\item 经过多少秒后$P$点开始重复运动.
\end{enumerate}


\end{enumerate}

 \input{tikz-tmp.tex}


\end{document}