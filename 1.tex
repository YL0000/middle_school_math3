\chapter{指数概念普遍化与对数}

\section{指数概念普遍化}
\subsection{引言}
这一节我们要把指数的概念加以推广,除了第一册学过
的整数指数、
零指数外,还要引入负整数指数,正、负分数
指数,为能够应用对数来简化乘法、除法、乘方、开方的计
算建立初步理论基础。

在推广过程中要注意以下几个方面:
\begin{enumerate}
\item 各种定义的条件。
\item 探索各种定义产生的逻辑过程,
\item 要熟练运用各种定义去计算。
\end{enumerate}


我们来回忆一下正整数指数幂的定义:

\begin{blk}{定义1}
$a^n=\overbrace{a\cdot a\cdots a}^{\text{$n$个}}$ ($n$是大于1的正整数),其中$a$称
为\textbf{底数},$n$称为\textbf{指数},$a^n$称为以$a$为底数、$n$为指数的\textbf{幂}.当$n=1$
时,规定$a^1=a$。
\end{blk}

在定义1中,对于底数$a$,没有任何限制,$a$
可以是正数,也可以是负数,也可以是零;而指数$n$,必须
是正整数,否则定义1就无意义!这样,我们称按定义1定
义的$a^n$为正整数指数幂。

我们可以证明正整数指数幂有下列性质:
\begin{blk}{性质}
\begin{enumerate}
    \item $a^m\cdot a^n=a^{m+n}$ \quad ($m,n$为正整数);
    \item $a^m\div a^n=a^{m-n}$\quad ($m,n$为正整数,且$m>n,\; a\ne 0$);
    \item $(a^m)^n=a^{mn}$\quad ($m,n$为正整数);
    \item $(ab)^n=a^nb^n$ \quad($n$为正整数);
    \item $\left(\frac{a}{b}\right)^n=\frac{a^n}{b^n}$\quad ($n$为正整数,$b\ne 0$)。
\end{enumerate}
\end{blk}

以后将会看到,将指数概念扩充到新的范围以后,这几
条性质依然保留,并且可以适当地合并。

\subsection{零指数与负整数指数}

【I】从正整数指数幂的性质可以看到:
\begin{itemize}
    \item 幂的\textbf{乘法}可以转化为指数\textbf{加法},(性质1)
    \item 幂的\textbf{除法}可以转化为指数\textbf{减法},(性质2)
    \item 幂的\textbf{乘方}可以转化指数\textbf{相乘}。(性质3)
\end{itemize}

要使正整数的加、减、乘、除,尤其是减、除通行无阻,
那么指数只限制在正整数范围是不行的,例如:$a^3\div a^2=
a^{3-2}=a^1=a$, 这个转化是没有问题的,但是
\[a^3\div a^3 \mathop{=}^{?} a^{3-3} \mathop{=}^{?} a^0=?\]
\[a^2\div a^4 \mathop{=}^{?} a^{2-4} \mathop{=}^{?} a^{-2}=?\]

要使指数的减法通行无阻,就会出现零指数和负整数指
数,而零指数和负整数指数是什么,过去从未见过,这就有
必要将指数的概念加以推广,给零指数和负整数下定义。

【II】探索零指数和负整数指数如何定义。

我们知道$a^m\div a^n=a^{m-n}$,
这里限制$a\ne 0$, 并且$m>n$, $m$、$n$均为正整数,现在把$m>
n$的条件取消,这样$m$可以等于$n$. 在$m=n$的时候,上面的公
式就是:
\[a^m\div a^n=a^n\div a^n=a^{n-n}=a^0\quad (a\ne 0)\]
而$a^m\div a^n$的实际内容是
\[a^m\div a^n=\frac{a^m}{a^n}=\frac{a^n}{a^n}=1\]

为了使公式$a^m\div a^n=a^{m-n}$
也适用于$m=n$的情况,就应该
使形式上的运算结果“$a^0$”与实际内容“1”一致起来,我们规
定$a$的零次幂等于1, 即$a^0=1\;\;  (a\ne 0)$就合理了。
用同样的方法,来探索$m<n$时的情况。

当$m<n$时,$a^m\div a^n=a^{m-n}=a^{-(n-m)}$,这里$-(n-m)$是个负整数,我们还没有规定$a^{-(n-m)}$的意
义。但另一方面,在$m<n$的条件下,$a^m\div a^n$的实际意义是:
\[a^m\div a^n=\frac{a^m}{a^n}=\frac{a^m\div a^n}{a^n\div a^n}=\frac{1}{a^{n-m}}\qquad (a\ne 0)\]
这样看来,为了使公式$a^m\div a^n=a^{m-n}$在$m<n$时也适用,规
定:
\[a^{-(n-m)}=\frac{1}{a^{n-m}}\quad (a\ne 0)  \]
就可以了,也就是说,$n$是正整数的时候,应规定:


【III】对零指数和负整数指数下定义

\begin{blk}{定义2}
    \[a^0=1\qquad  (a\ne 0)\]
\end{blk}

\begin{blk}{定义3}
    \[a^{-n}=\frac{1}{a^n} \qquad  (a\ne 0, \quad n\text{是正整数})\]
\end{blk}

这两个定义中特别要注意“$a\ne 0$”这个条件,也就是说零
的零次幂是没有意义的,零的负整数次幂也是没有意义的。

有了定义2和定义3, 对于同底数的整数指数幂相除的
运算法则,就可通行无阻,而不必再局限于$m>n$了,例
如
\[(-2)^2\div (-2)^2=(-2)^0=1\]
\[4^5\div 4^7=4^{5-7}=4^{-2}=\frac{1}{4^2}=\frac{1}{16}\]

定义1、2、3说明现在我们的指数已经扩充到整数范
围了。

【IV】性质的证明

一般来说,当一个概念被推广以后,原来具有的性质可
能有些仍然保持成立,有些就不再成立了。所以必须对原来
的性质逐条加以研究,对那些仍然成立的性质,加以证明肯
定,对那些不再成立的性质也应通过举反例加以否定。另
外,新的概念是否又带来了新的性质,这是研究推广了的概
念应该注意的一个问题,只有通过这样的研究,才能掌握推
广了的概念。

把指数概念推广到整数范围以后,上述五条性质,因条
件起了变化,就成为:

\begin{blk}{}
\begin{enumerate}
    \item $a^m\cdot a^n=a^{m+n}\quad (m,n\in\mathbb{Z},\;\; a\ne 0)$
    \item $a^m\div a^n=a^{m-n}\quad (m,n\in\mathbb{Z},\;\; a\ne 0)$
    \item $(a^m)^n=a^{mn}\quad (m,n\in\mathbb{Z},\;\; a\ne 0)$
    \item $(a\cdot b)^n=a^{n}\cdot b^n\quad (ab\ne 0,\;\; n\in\mathbb{Z})$
    \item $\left(\frac{a}{b}\right)^n=\frac{a^n}{b^n}\quad (ab\ne 0,\;\; n\in\mathbb{Z})$
\end{enumerate}    
\end{blk}

现在我们来证明性质1。

已知:$m,n$是任意两个整数,$a\ne 0$.

求证:$a^m\cdot a^n=a^{m+n}$

我们要对$m,n$分各种情况来讨论,由于$m,n$均可取正整
数、负整数和零,即
\[m=\begin{cases}
    0\\ \text{正整数}\\ \text{负整数}
\end{cases},\qquad n=\begin{cases}
    0\\ \text{正整数}\\ \text{负整数}
\end{cases}  \]
因而从$m$中取出一种情况,从$n$中取出一种情况,组成一种
情况,那么一共有$3\x3=9$种,在这九种情况中,因为$m
=$正整数,$n=$正整数的情况以前已证过,可以略去。又由
于实数乘法适合交换律,因而$m,n$是对称的,这样,我们只
就下面五种情况来证明:
\[\begin{cases}
    m=0\\n=0
\end{cases}\begin{cases}
    m=\text{正整数}\\n=0
\end{cases}\begin{cases}
    m=\text{负整数}\\n=0
\end{cases}\begin{cases}
m=\text{正整数}\\  n=\text{负整数}
\end{cases}\begin{cases}
    m=\text{负整数}\\  n=\text{负整数}
\end{cases}  \]

\textbf{情况1:} 当$m=0$, $n=0$时,由零指数的定义,
\[a^m\cdot a^n=a^0\cdot a^0=1\cdot 1=1=a^0=a^{m+n} \]
$\therefore\quad a^m\cdot a^n=a^{m+n}$成立。

\textbf{情况2:} 当$m$是正整数,$n=0$时,
\[a^m\cdot a^n=a^m\cdot a^0=a^m\cdot 1=a^m=a^{m+0}=a^{m+n}\]
$\therefore\quad a^m\cdot a^n=a^{m+n}$成立。


\textbf{情况3:} 当$m$是负整数,$n=0$时,
\[a^m\cdot a^n=a^m\cdot a^0=a^m\cdot 1=a^m=a^{m+0}=a^{m+n}\]
$\therefore\quad a^m\cdot a^n=a^{m+n}$成立。

\textbf{情况4:} 当$m$是正整数,$n$是负整数,那么$n=-|n|$, $|n|$为正整数。

$\because\quad a^m\cdot a^n=a^m\cdot a^{-|n|}=a^m\cdot\frac{1}{a^{|n|} }=\frac{a^m}{a^{|n|}}=a^{m-|n|}=a^{m+(-|n|)}=a^{m+n} $

$\therefore\quad a^m\cdot a^n=a^{m+n}$成立。

\textbf{情况5:} 当$m,n$都是负整数时,那么
$m=-|m|$, $n=-|n|$。因此:
\[\begin{split}
    a^m\cdot a^n&=a^{-|m|}\cdot a^{-|n|}=\frac{1}{a^{|m|}}\cdot \frac{1}{a^{|n|}}\\
    &=\frac{1}{a^{|m|}\cdot a^{|n|}}=\frac{1}{a^{|m|+|n|}}\\
    &=a^{-(|m|+|n|)}\\
    &=a^{-(|m|)+(-|n|)}=a^{m+n}
\end{split}\]
$\therefore\quad a^m\cdot a^n=a^{m+n}$成立。

综合五种情况及前面的分析,就证明了当$m,n$为任意整
数时,$a^m\cdot a^n=a^{m+n}$成立。

其它几条性质可以类似地证明。应当指出,有了定义2、3以后,对于任何整数$n$, 都有:
\[\frac{1}{a^n}=a^{-n}\qquad (a\ne 0)\]
这是因为:当$n$是正整数时,这就是定义3, 当$n$为零时,
 \[\frac{1}{a^n}=\frac{1}{a^0}=\frac{1}{1}=a^0=a^{-n}\]
 当$n$为负整数时,$n=-|n|$
\[\frac{1}{a^n}=\frac{1}{a^{-|n|}}=a^{|n|}=a^{-n}\]
这样,定义3虽然是对$n$为正整数来说的,有$a^{-n}=\frac{1}{a^n}$,
但实
际上暗示了$n$为任意整数时,$a^{-n}=\frac{1}{a^n}$
都有了确定的意义。

有了这个公式,上面的性质2就可归结到性质1
上。
事实上,只要1成立,那么对于任何整数$m,n$我们
有:
\[a^m\div a^n=a^m\cdot \frac{1}{a^n}=a^m\cdot a^{-n}=a^{m+(-n)}=a^{m-n}\]

有了这个公式,上面的性质5也可归结为性质3和4。
事实上,只要3和4成立,那么对于任何整数$m,n$,
我们有:
\[\begin{split}
    \left(\frac{a}{b}\right)^n&=\left(a\cdot\frac{1}{b}\right)^n=(a\cdot b^{-1})^n=a^n\cdot (b^{-1})^n\\
    &=a^n \cdot b^{-n}=a^n\cdot \frac{1}{b^n}=\frac{a^n}{b^n}
\end{split}\]
这样,五条性质可归结为三条性质:
\begin{enumerate}
    \item $a^m\cdot a^n=a^{m+n}\quad (a\ne 0;\;\;m,n\in\mathbb{Z})$
    \item $(a^m)^n=a^{mn}\quad (a\ne 0;\;\;m,n\in\mathbb{Z})$
    \item $(a\cdot b)^n=a^n\cdot b^n \quad (ab\ne 0;\;\;n\in\mathbb{Z})$
\end{enumerate}


\begin{example}
\begin{multicols}{2}  
    \begin{enumerate}
        \item $2^0=1$
        \item $(0.75)^0=1$
        \item $\left(-\sqrt{3}\right)^0=1$
        \item $0^0$无意义
        \item $10^{-3}=\frac{1}{10^3}=0.001$
        \item $3(-2)^{-4}=\frac{3}{(-2)^{4}}=\frac{3}{16}$
        \item $\left(\frac{1}{2}\right)^{-5}=2^5=32$
        \item $(-0.25)^{-1}=\left(-\frac{1}{4}\right)^{-1}=-4$
    \end{enumerate}
\end{multicols}   
\end{example}

\begin{example}
    把下列单项分式化成整式形式:
    \[\frac{a}{3bc^2}=\frac{1}{3}ab^{-1}c^{-2},\qquad \frac{3}{2a^{-3}b^{-1}c^2}=\frac{3}{2}a^3bc^{-2}  \]
\end{example}

\begin{example}
    把下面的数写成$a\cdot 10^n$的形式,这里$a$是含有一
位整数的数,$n$是任意整数(把一个数写成这种形式叫 科学
记数法)。
    \begin{enumerate}
        \item 1吨(t)$=1000$公斤(kg)$=10^3$公斤(kg)
        \item  1毫米(mm)$=0.001$米(m)$=10^{-3}$米(m)
        \item  1年$=31,556,925,975$秒(s)=$3.1556925975\x10^7$(s)$\approx 3.156\x10^7$(s)
        \item   地球到太阳的距离$=149,640,000$公里(km)
        $=1.4964\x10^8$(km)
        \item    地球的质量$=5,970,000,000,000,000,000,000$(t)
        $=5.97\x10^{21}$(t)
        \item    原子核$U_{288}$的半径$=0.000,000,000,000,93$(cm)$=9.3\x10^{-13}$(cm)
    \end{enumerate}
\end{example}

\begin{example}
\begin{enumerate}
    \item $(b^{-3})^{-2}=b^{(-3)(-2)}=b^6$
    \item \[\begin{split}
        \left(3a^{-2}b^2c^{-3}\right)\left(\frac{4}{5}ab^{-3}c^3\right)&=\frac{12}{5}a^{-2+ 1}b^{2+(-3)}c^{-3+3}\\
        &=\frac{12}{5}a^{-1}b^{-1}\\
        &=\frac{12}{5ab}
    \end{split}\]
    \item \[\begin{split}
        \left(\frac{a+b}{a-b}\right)^{-3}\left(\frac{a-b}{a+b}\right)^{-2}&=\left[\left(\frac{a+b}{a-b}\right)^{-1}\right]^{3} \left(\frac{a-b}{a+b}\right)^{-2}\\
        &=\left(\frac{a-b}{a+b}\right)^{3}\left(\frac{a-b}{a+b}\right)^{-2}\\
        &=\left(\frac{a-b}{a+b}\right)^{3-2}=\left(\frac{a-b}{a+b}\right)^{1}\\
        &=\frac{a-b}{a+b}
    \end{split}\]
\end{enumerate}
\end{example}

    
\begin{example}
\begin{enumerate}
    \item \[\begin{split}
        \frac{\left(a^{-2} b^{-3}\right)\left(-4 a^{-1} b\right)}{12 a^{-4} b^{-2}}&=\frac{-4 a^{-2-1} b^{-3+1}}{12 a^{-4} b^{-2}} \\
    &=\frac{-4 a^{-3} b^{-2}}{12 a^{-4} b^{-2}}\\
    &=-\frac{1}{3} a
    \end{split}\]
    \item \[\begin{split}
        \left(x^{2}-y^{-2}\right)\div\left(x-y^{-1}\right) 
    &=\left[x^{2}-\left(y^{-1}\right)^{2}\right] \div\left(x-y^{-1}\right) \\
    &=\left(x+y^{-1}\right)\left(x-y^{-1}\right)\div \left(x-y^{-1}\right) \\
    &=x+y^{-1}=x+\frac{1}{y}
    \end{split}\]
\end{enumerate}
\end{example}


\section*{习题1.1}
\addcontentsline{toc}{subsection}{习题1.1}

\begin{enumerate}
    \item 换去下列各算式中的负指数:
    \begin{multicols}{2}
        \begin{enumerate}
        \item $4x^{-3}y^3$
        \item $\frac{1}{5c^{-3}}$
        \item $\frac{4a^{-2}}{5b^{-3}}$
        \item $\frac{3a^{-3}x^2}{5b^3y^{-4}}$
    \end{enumerate}
    \end{multicols}
    
    \item 证明下面的运算法则,如果$m,n$为任意整数,而$a\ne 0$,
    则$(a^m)^n=a^{m\cdot n}$
    \item 把下列的数写成$a\cdot 10^n$的形式,这里$a$是含有一位整数的数,$n$是任意整数,
    \begin{multicols}{2}
        \begin{enumerate}
        \item 8900
        \item  3,200,000
        \item 0.000,015
        \item 0.000,000,025
    \end{enumerate}
    \end{multicols}
    
    
\item 下面是物理学常用的长度单位,将它化成mm并以10的幂表
示出来:
\begin{enumerate}
    \item $1\text{微米}(1\mu {\rm m})=\frac{1}{1000}{\rm mm}$
    \item $1\text{毫微米}(1 {\rm nm})=\frac{1}{1,000,000}{\rm mm}$ 
    \item  $1\text{微微米}(1 {\rm pm})=\frac{1}{1,000,000,000}{\rm mm}$ 
\end{enumerate}


\item 以10的幂来表示:
\begin{enumerate}
    \item 1mm是多少cm;
    \item 1${\rm cm}^3$是多少${\rm m}^3$ (Litre即升);
    \item 1g是多少kg, 是多少吨(t)。
\end{enumerate}
\item 将下面的数据用小数形式表示出来:
\begin{enumerate}
    \item 红血球的直径$=0.7\x10^{-8}$(cm);
    \item 最小的细菌的长度$\approx 10^{-4}$(cm);
    \item 钠光(黄色)的波长$=5.89\x10^{-7}$(cm);
    \item 氢原子的直径$\approx 10^{-8}$(cm);
    \item 氢原子的质量$=1.64\x10^{-24}$(g);
    \item 电子的质量$=9\x10^{-28}$(g);
    \item 铀矿石的含镭量$=3.328\x10^{-6}$\%.
\end{enumerate}

\item 完成下列运算并将所得结果中的负指数变换成正指数:
\begin{multicols}{2}
 \begin{enumerate}
    \item $(2x^2)\div (3x^{-3})$
    \item $ax^2\div x^{-1}$
    \item $(a^{-2}x^2)\div a^4$
    \item $(ax)^{-3}\div (bx)^{-3}$
    \item $(2^n)^{n-1} \div (2^{n-1})^{n+1}\quad (n>1)$
\end{enumerate}   
\end{multicols}

\item 利用整数指数幂的指数法则,计算
\begin{multicols}{2}\begin{enumerate}  
\item $\left\{\left[\frac{5}{3}-\left(\frac{6}{5}\right)^{-1}\right]^{-2}-\left(\frac{25}{11}\right)^{-1}\right\}^{-3}$\item $\left[\left(\frac{5 a^{-2} c^{3}}{3 x^{-3} y^{4}}\right)^{-2}\right]^{4}$
\item $\frac{\left(3 x^{2} y^{-3}\right)^4}{\left(-2 x^{-3} y^{2}\right)^{-3}\left(-27 x^{-5} y^{2}\right)}$
\item $\left(\frac{2}{a^{n}+a^{-n}}\right)^{-2}-\left(\frac{2}{a^{n}-a^{-n}}\right)^{-2}$
\end{enumerate}\end{multicols}

\item 求证:
\begin{enumerate}
    \item $\left(a+a^{-1}\right)^{2}\left(a-a^{-1}\right)^{2}=a^{4}-2+\frac{1}{a^{4}}$
    \item $\frac{a^{-3}+b^{-3}}{a^{-1}+b^{-1}}+\frac{a^{-3}-b^{-3}}{a^{-1}-b^{-1}}=2\left(\frac{1}{a^{2}}+\frac{1}{b^{2}}\right)$
\end{enumerate}

\item 化简:
 \begin{multicols}{2}
\begin{enumerate}
    \item  $\frac{a^{2}+a^{-2}-2}{a^{2}-a^{-2}}$
    \item  $\frac{m^{3}+n^{-3}}{m+n^{-1}}+\left(m-n^{-1}\right)^{2}$
\end{enumerate}
\end{multicols}
\item 求$32x^{-6}+12x^{-4}+10x^{-2}-12$除
以$2x^{-2}-1$所得的商式
和余数。
\end{enumerate}

\subsection{$n$次算术根}
为了进一步把指数概念从整数范围扩充到有理数范围,
我们先介绍一下$n$次算术根及其运算性质。

\begin{blk}{定义}
 设$a$是一个实数,$n$是正整数,如果存在着实数
$x$, 使得
$$x^n=a$$
那么,$x$就叫做\textbf{$a$的$n$次方根},$a$叫做\textbf{被开方数},$n$叫做\textbf{根指数}。   
\end{blk}
 
由方根的定义推知:
\begin{enumerate}
    \item 如果$x_1$是正数$a$的偶次方根,那么$x_1$也是$a$的偶次
    方根。

    因为如果存在$x_1$使得$x_1^n=a\; (>0)$, 那么,$(-x_1)^n=(x_1)^n=a$ ($n$为偶数),即$x_1$和$-x_1$都是正数$a$的偶次方根.比如,
$x^4=16$, 则$x=\pm 2$都是16的四次方根。
\item 负数的偶次方根不存在,这是因为任何数的偶次方
都是非负数的缘故。
\item 不等于零的任何数的奇次方根的符号与被开方数的
符号相同。

因为如果被开方数是负数$-a\; (a>0)$, 那么它的奇次方
根,就不能是0或正数,由于0的任何奇次方是0, 正数的
任何奇次方是正数,这就说明负数的奇次方根应该是负数;
同样说明正数的奇次方根应该是正数。
\item 0的方根是0。
\end{enumerate}

在本教程的第六册中,我们将证明下面的存在定理。


\begin{blk}{定理}
    如果$a>0$, 方程$x^n=a$存在唯一的正实数根。
\end{blk}

为明确起见,我们把这个正实数方根叫做$a$的算术根,下
面给出它的定义和记法。


\begin{blk}{定义}
    如果$a$是正实数,那么$a$的正的$n$次方根叫做\textbf{$a$的$n$次
算术根},而零也叫零的$n$次算术根,记为$\sqrt[n]{a}$。$a$的算术平方根
用不带根指数的符号$\sqrt{n}$表示。$\sqrt[1]{a}$这个表示$a$的符号,我们不用。
\end{blk}

例如:$2^6=32$, 2叫做32的5次方根,也是32的5次算术根,
因此$\sqrt[5]{32}=2$; $(\pm 2)^4=16$,$2$和$(-2)$都是16的4次方根,其中2是16的4次算术根,因此$\sqrt[4]{16}=2$; $(-2)$不是16的4次算术根。因此$\sqrt[4]{16}\ne -2$, 但$-2$可写成$-\sqrt[4]{16}$。$(-3)^3=-27$, $-3$
叫做$-27$的3次方根,但不是算术根。有的课本也用符号$\sqrt[n]{a}$
($a<0$, $n$是奇数)表示负数的奇次方根,例如$\sqrt[3]{-27}=-3$。
但是负数的奇次方根总可以用对应的算术根表示出来,例如
$\sqrt[3]{-27}=-\sqrt[3]{27}$, 一般地,$\sqrt[2n+1]{-a}=-\sqrt[2n+1]{a}\quad  (a>0)$。

请注意以下几点:

\begin{enumerate}
    \item 算术根的概念含有两个条件:被开方数是非负的;
算术根只代表方根中的非负值。以后在实数范围内,符号
$\sqrt[n]{a}$($a>0$)只代表算术根,因此,正数$a$的偶次方根的负值
用$-\sqrt[n]{a}$表示。
\item 在等式$x^n=a$中,已知$x$和$n$求$a$的过程叫做乘方运算。
反过来,已知$a$和$n$求$x$的过程叫做开方运算,乘方运算和开
方运算互为逆运算,由于当$a\ge 0$时,$\sqrt[n]{a}$是非负值,因此对
于一切大于1的正整数$n$都有:
\begin{align}
    \left(\sqrt[n]{a}\right)^n&=a\\
    \sqrt[n]{a^n}&=a
\end{align}

事实上,设$\sqrt[n]{a}=x$, 按定义,$x^n=a$, 所以$\left(\sqrt[n]{a}\right)^n=a$。

在$a<0$的场合,可以验知:当$n$是奇数时,等式(1.1)、
(1.2)对于负数的奇次方根仍保持;而当$n$是偶数时,等式(1.1)
无意义,而(1.2)的右端是$|a|$, 不是$a$, 即$a=|a|$, 这是因
为算术根只代表非负数的缘故。

例如:$\sqrt{(-2)^2}+\sqrt[3]{-27}+\sqrt[5]{243}=\sqrt{(-2)^2}+\sqrt[3]{(-3)^3}+\sqrt[5]{3^5}=2-3+3=2$

\item 符号$\sqrt[n]{\quad }$叫做$n$次根号,如果根号下是一个算式,例如$\sqrt{1-a}$,$\sqrt[3]{\frac{x}{y}}$
等等,我们称它为根式。当根号下的算式
的值不为负数时,它的$n$次算术根才有意义。

例如:$\sqrt{(a-2)^2}=\begin{cases}
    a-2 & a\ge 2\\
    2-a & a<2
\end{cases},\quad \sqrt{(1-a)} \text{ 在$a\le 1$的情形下有意义}$

\end{enumerate}

下面我们介绍$n$次算术根的性质。

\begin{blk}{性质1:乘积的开方法则}
设$n$为正整数,$a,b$为正实数,那么
$$\sqrt[n]{ab}=\sqrt[n]{a}\cdot \sqrt[n]{b}$$
\end{blk}

\begin{proof}
    因为$\left(\sqrt[n]{a}\cdot \sqrt[n]{b}\right)^n=\left(\sqrt[n]{a}\right)^n\cdot \left(\sqrt[n]{b}\right)^n=ab$
    
    按方根定义,$\sqrt[n]{a}\cdot \sqrt[n]{b}$是$ab$的$n$次方根,又因为$a,b$是正
    数,当然积$ab$仍是正数,且算术根的积$\sqrt[n]{a}\cdot \sqrt[n]{b}$也是正数,
    这就是说,$\sqrt[n]{a}\cdot \sqrt[n]{b}$是正数$ab$的$n$次算术根;另一方面,$ab$
    的$n$次算术根是$\sqrt[n]{ab}$, 根据$ab$的$n$次算术根的唯一性得到
    $$\sqrt[n]{ab}=\sqrt[n]{a}\cdot \sqrt[n]{b}$$
\end{proof}

\begin{blk}{推论}
\[\sqrt[n]{a_1a_2\cdots a_n}=\sqrt[n]{a_1}\cdot \sqrt[n]{a_2}\cdots \sqrt[n]{a_n}\]
\end{blk}

例如:\[\begin{split}
    \sqrt[3]{60\x 18\x 25}&=\sqrt[3]{(2^2\x 3\x 5)(2\x 3^2)(5^2)}\\
    &=\sqrt[3]{2^3\x 3^3\x 5^3}\\
    &=2\x 3\x 5=30
\end{split}\]

\begin{blk}{性质2:幂的开方法则}
设$m,n$是正整数,$a>0$, 那么,$\sqrt[n]{a^m}=\left(\sqrt[n]{a}\right)^m$。

\end{blk}

\begin{proof}
由性质1的推论得,
\[\sqrt[n]{a^m}=\underbrace{\sqrt[n]{a\cdot a\cdots a}}_{\text{$m$个}}=\underbrace{\sqrt[n]{a}\cdots \sqrt[n]{a}}_{\text{$m$个}}=\left(\sqrt[n]{a}\right)^m\]
\end{proof}

\begin{blk}{性质3:商的开方法则}
    设$n$为正整数,$a,b>0$, 那么,
\[\sqrt[n]{\frac{a}{b}}=\frac{\sqrt[n]{a}}{\sqrt[n]{b}}\] 
\end{blk}

\begin{proof}
$\because\quad \left(\frac{\sqrt[n]{a}}{\sqrt[n]{b}}\right)^n=\frac{\left(\sqrt[n]{a}\right)^n}{\left(\sqrt[n]{b}\right)^n}=\frac{a}{b}$

又$\because\quad a>0,\quad   b>0,\quad \frac{a}{b}>0,\quad \frac{\sqrt[n]{a}}{\sqrt[n]{b}}>0$

$\therefore\quad \frac{\sqrt[n]{a}}{\sqrt[n]{b}}$是$\frac{a}{b}$的$n$次算术根

$\therefore\quad \sqrt[n]{\frac{a}{b}}=\frac{\sqrt[n]{a}}{\sqrt[n]{b}}$

\end{proof}

\begin{blk}{性质4:根式的开方法则}
    设$m,n$是正整数,$a>0$, 那么,
\[\sqrt[m]{\sqrt[n]{a}}=\sqrt[mn]{a}\]
\end{blk}

\begin{proof}
    $\because\quad \left(\sqrt[m]{\sqrt[n]{a}}\right)^{mn}=\left[\left(\sqrt[m]{\sqrt[n]{a}}\right)^m\right]^n=\left(\sqrt[n]{a}\right)^n=a$

    又$\because\quad a>0,\quad \sqrt[m]{\sqrt[n]{a}}>0 $
    
    $\therefore\quad \sqrt[n]{a}$是$a$的$mn$次算术根,

    $\therefore\quad \sqrt[m]{\sqrt[n]{a}}=\sqrt[mn]{a}$
\end{proof}


\begin{blk}{性质5:幂指数与根指数相约法则}
设$m,n,k$为正整数,$a>0$, 那么,
\[\sqrt[nk]{a^{mk}}=\sqrt[n]{a^m} \]  
\end{blk}

\begin{proof}
$\because\quad \left(\sqrt[n]{a^m}\right)^{nk}=\left[\left(\sqrt[n]{a^m}\right)^n\right]^k=\left(a^m\right)^k=a^{mk}$

又$\because\quad a>0,\quad \sqrt[n]{a^m}>0,\quad a^{mk}>0 $
    
$\therefore\quad \sqrt[n]{a^m}$是$a^{mk}$的$nk$次算术根,

$\therefore\quad \sqrt[nk]{a^{mk}}=\sqrt[n]{a^m} $
\end{proof}

最后强调,这些性质是在$a>0$的前提下成立的,对于
$a<0$, 有些根式也许仍有意义,但法则就不适用了,例如,性
质5:

设$a=-8$, 那么,
$\sqrt[6]{(-8)^2}=\sqrt[6]{64}=2$, 而$\sqrt[3]{-8}=\sqrt[3]{(-2)^3}=-2$,
因此,$\sqrt[6]{(-8)^2}\ne \sqrt[3]{-8}$。



\begin{example}
    化简下列各式:
\begin{enumerate}
\item $\sqrt[4]{25 x^{2} y^{2}} \quad(x, y \ge 0)$
\item  $\sqrt[6]{5^{4} a^{4} b^{2}}\quad (a, b \ge 0)$
\item $\sqrt{3 x^{2 a+2}}\quad (x \ge 0)$
\item $\sqrt[6]{27(u+v)^{18}}\quad (u,v\ge 0)$
\item $(a-b) \sqrt{\frac{a^{2}+a b}{a^{2}-2 a b+b^{2}}}\quad (a, b>0)$
\item $\sqrt[3]{16 \sqrt{2}}$
\item $\sqrt[4]{4 x^{6}}\quad (x \in \mathbb{R})$
\end{enumerate}
\end{example}

\begin{solution}
\begin{enumerate}
    \item $\sqrt[4]{25 x^{2} y^{4}}=\sqrt[4]{\left(5 x y^{2}\right)^{2}}=\sqrt{5 x y^{2}}$
    \item $\sqrt[6]{5^{4} a^{4} b^{2}}=\sqrt[6]{\left(5^{2} a^{2} b\right)^{2}}=\sqrt[3]{5^{2} a^{2} b}=\sqrt[3]{25 a^{2} b}$
    \item $\sqrt{3 x^{2 n+2}}=\sqrt{3} \cdot \sqrt{\left(x^{n+1}\right)^{2}}=\sqrt{3} x^{n+1}$
    \item \[\begin{split}
        \sqrt[6]{27(u+v)^{18}}&=\sqrt[6]{3^3\left[(u+v)^{6}\right]^{3}}\\
        &=\sqrt[6]{[3 (u+v)^{6}]^3}=\sqrt{3(u+v)^{6}}
        \\
        &=\sqrt{3}\sqrt{(u+v)^6}=\sqrt{3}(u+v)^3
    \end{split}\]
    \item \[\begin{split}
        (a-b) \sqrt{\frac{a^{2}+a b}{a^{2}-2 a b+b^{2}}}&=(a-b) \frac{\sqrt{a^2+ab}}{\sqrt{(a-b)^2}}\\
        &=(a-b)\frac{\sqrt{a(a+b)}}{|a-b|}\\
        &=\begin{cases}
            \sqrt{a(a+b)}, & a>b>0\\
            -\sqrt{a(a+b)}, & b>a>0
        \end{cases}
    \end{split}\]
\item \[\begin{split}
    \sqrt[3]{16 \sqrt{2}}&=\sqrt[3]{\sqrt{16^2}\cdot \sqrt{2}}\\
    &=\sqrt[3]{\sqrt{16^2 \cdot 2}}=\sqrt[6]{16^2 \cdot 2}\\
    &=\sqrt[6]{2^8\cdot 2}=\sqrt[6]{2^9}=\sqrt{2^3}=2\sqrt{2}
\end{split}\]
\item \[\begin{split}
    \sqrt[4]{4 x^{6}}&=\sqrt[4]{2^2 |x|^{6}}=\sqrt{2|x|^3}\\
    &=\sqrt{2\cdot |x|^2\cdot |x|}=\sqrt{2}|x|\sqrt{|x|}\\
    &=\begin{cases}
        \sqrt{2}x\sqrt{x} , & x\ge 0\\
        -\sqrt{2}x\sqrt{-x} , & x<0
    \end{cases}
\end{split}\]
\end{enumerate}
\end{solution}


\begin{example}
    把$\sqrt{2}$,$\sqrt[3]{-3}$,$\sqrt[4]{4}$化成同次式(指数相同的
    根式叫做同次根式)
\end{example}

\begin{solution}
$\because\quad \sqrt[3]{-3}=-\sqrt[3]{3},\quad \sqrt[4]{4}=\sqrt{2}$

$\therefore\quad $只需要考虑将$\sqrt{2}$, $\sqrt[3]{3}$化成同次根式
\[\begin{split}
\sqrt{2}&=\sqrt[6]{2^3}=\sqrt[6]{8}\\
\sqrt[3]{-3}&=-\sqrt[3]{3}=-\sqrt[6]{3^2}=-\sqrt[6]{9}\\
\sqrt[4]{4}&=\sqrt{2}=\sqrt[6]{8}
\end{split}\]
\end{solution}

\begin{ex}
\begin{enumerate}
    \item 求出下列方根的值:
    \begin{multicols}{2}
\begin{enumerate}
    \item $\sqrt[4]{10000}$
    \item $\sqrt[4]{1}$
    \item $\sqrt[4]{256}$
    \item $\sqrt[4]{\frac{1}{16}}$
    \item $\sqrt[4]{\frac{1}{625}}$
    \item $\sqrt[5]{100000}$
    \item $\sqrt[5]{0}$
    \item $\sqrt[5]{0.00001}$
    \item $\sqrt{\frac{4}{9}}$
    \item $\sqrt[4]{\frac{1}{81}}$
    \item $\sqrt[6]{64}$
    \item $\sqrt[3]{0.064}$
    \item $\sqrt[5]{0.00243}$
    \item $\sqrt[10]{0}$
    \item $\sqrt[3]{-2 \frac{10}{27}}$
\end{enumerate}        
    \end{multicols}

\item 计算下面方根,准确到0.1
\[\sqrt[4]{25},\qquad \sqrt[6]{27},\qquad \sqrt[8]{16},\qquad \sqrt[10]{32} \]

\item 约简下列根式中被开方数的指数和根指数:
\begin{multicols}{2}
\begin{enumerate}
    \item $\sqrt[6]{36 x^{2}}$
    \item $\sqrt[4]{25 y^{2}}$
    \item $\sqrt[8]{2^{4} a^{4} b^{6}}$
    \item $\sqrt[16]{14^{4^{4m} b^{8m}}}$
\end{enumerate}
\end{multicols}
这里,$a>0,\;\; b>0,\quad x,y\in\mathbb{R}$


\item 计算下列各题(题中字母都是正的):
\begin{multicols}{2}
    \begin{enumerate}
    \item $\sqrt[3]{\frac{8 x^{3} y^{6}}{27 x^{6} y^{9}}}$
    \item $\sqrt[n]{\frac{a^{n} b^{2n}}{x^{3 n} y^{n}}}$
    \item $\sqrt{\frac{25(a+b)^{2}}{(c-d)^{4}}}$
    \item $\sqrt[3]{-\frac{(a-b)^{3 n}}{(x+y)^{6n}}}$
    \item $\sqrt[n]{\frac{(a+b)^{2 n}}{a^{3 n}(a-b)^{n}}}$
\end{enumerate}
\end{multicols}

\item 化简下列各式:
\begin{multicols}{2}
    \begin{enumerate}
    \item $\sqrt{2 \sqrt{3}}$
    \item $\sqrt[3]{2 \sqrt{5}}$
    \item $\sqrt{3\sqrt[3]{2}}$
    \item $\sqrt[4]{5 \sqrt{2}}$
    \item $\sqrt{\sqrt[3]{a^{4} b^{2}}}$
    \item $\sqrt{x^{3} \sqrt[3]{x}}$
    \item $\sqrt{\frac{x}{y} \sqrt{\frac{x}{y}}}$
    \end{enumerate}
\end{multicols}
\end{enumerate}
     
\end{ex}


\subsection{最简根式和同类根式}
根式作运算,计算结果用根式表示时,根式应为最简根
式,最简根式指:
\begin{enumerate}
    \item 被开方数的每一个因式的指数都小于根指数;
    \item 根号内的式子不含分母;
    \item 根指数与被开方
数的指数互质。
\end{enumerate}

把根式化为最简根式的方法是:
\begin{enumerate}
    \item 移因式到根号外,例如:\[\sqrt[n]{a^nb}=\sqrt[n]{a^n}\cdot \sqrt[n]{b}=a\sqrt[n]{b}\;\;(a\ge 0)\]
    \item 移因式到根号内,例如:\[a\sqrt[3]{\frac{1}{a^2}}=\sqrt[3]{a^3}\cdot \sqrt[3]{\frac{1}{a^2}}=\sqrt[3]{\frac{a^3}{a^2}}=\sqrt[3]{a}\;\;(a>0)\]
    \item 化去根号内式子的分母,例如
    \begin{enumerate}
        \item 如果$ab>0$, 那么$\sqrt{\frac{a}{b}}=\sqrt{\frac{ab}{b^2}}=\frac{\sqrt{ab}}{\sqrt{b^2}}=\frac{\sqrt{ab}}{|b|}$
        \item 如果$a>0$, $b>0$, 且$n>m$, 那么
        \[\sqrt[n]{\frac{a}{b^m}}=\sqrt[n]{\frac{ab^{n-m}}{b^mb^{n-m}}}=\frac{\sqrt[n]{ab^{n-m}}}{\sqrt[n]{b^n}}=\frac{\sqrt[n]{ab^{n-m}}}{b} \]
    \end{enumerate}
\item 约去根指数与被开方数的指数的公因数,例如,
\[\sqrt[6]{8x^3}=\sqrt[6]{(2x)^3}=\sqrt{2x}\quad (x>0)\]
\end{enumerate}

\begin{example}
    化下面根式为最简根式:
\begin{enumerate}
    \item \[\begin{split}
        \frac{5 a^{2}}{7 b}\sqrt{\frac{49 b^{3}}{5 a}} &=\frac{5 a^{2}}{7 b} \sqrt{\frac{\left(7^{2} b^{2} b\right)(5 a)}{5^{2} a^{2}}} \\
    &=\frac{5 a^{2}}{7 b} \cdot \frac{7 b}{5 a} \sqrt{5 a b} \\
    &=a \sqrt{5 a b} \qquad(a>0, b>0) \\
    \end{split}\]
    \item \[\begin{split}
        \frac{2 a^{2}}{3 b} \sqrt[3]{\frac{b^{3}}{a^{4}}-\frac{b^{5}}{a^{6}}} &=\frac{2 a^{2}}{3 b}\sqrt[3]{\frac{a^{2} b^{3}-b^{5}}{a^{6}}} \\
    &=\frac{2 a^{2}}{3 b} \sqrt[3]{\frac{b^{3}\left(a^{2}-b^{2}\right)}{a^{6}}} \\
    &=\frac{2 a^{2}}{3 b} \cdot \frac{b}{a^{2}} \cdot \sqrt[3]{a^{2}-b^{2}} \\
    &=\frac{2}{3} \sqrt[3]{a^{2}-b^{2}} \qquad(a \neq 0, b>0)
    \end{split}\]
\end{enumerate}
\end{example}

\begin{example}
    作下面根式的乘法和除法:
\begin{enumerate}
    \item $$5\sqrt[4]{2a}\cdot \sqrt[4]{8a^3}=5\sqrt[4]{16a^4}=5\sqrt[4]{(2a)^4}=5\x 2a=10a\qquad (a\ge 0)$$
    \item \[\begin{split}
        \sqrt{\frac{3}{4}} \cdot \sqrt[4]{\frac{4}{3}}&=\sqrt[4]{\left(\frac{3}{4}\right)^{2}} \cdot \sqrt[4]{\frac{4}{3}}=\sqrt[4]{\left(\frac{3}{4}\right)^{2} \cdot \frac{4}{3}}\\
&=\sqrt[4]{\frac{3}{4}}=\sqrt{\frac{3}{2^{2}}}=\sqrt[4]{\frac{3 \times 2^{2}}{2^{4}}}=\frac{\sqrt[4]{12}}{2}
    \end{split}\]
    \item $\frac{16 \sqrt{3}}{\sqrt{2}}=\frac{16 \sqrt{3} \cdot \sqrt{2}}{\sqrt{2} \cdot \sqrt{2}}=8 \sqrt{6}$
    \item $\frac{5}{\sqrt[3]{4}}=\frac{5 \sqrt[3]{2}}{\sqrt[3]{2^2} \sqrt[3]{2}}=\frac{5 \sqrt[3]{2}}{\sqrt[3]{2^{3}}}=\frac{5}{2} \sqrt[3]{2}$
\end{enumerate}
\end{example}

几个根式化成最简根式后,如果根指数相同,根号内的
式子也相同,这几个根式叫做同类根式。


\begin{example}
    $\sqrt[3]{8ax^3}$和$\sqrt[6]{64a^2y^{12}}$是同类根式吗?
\end{example}
    
\begin{solution}
    由于:
\[\begin{split}
    \sqrt[3]{8ax^3}&=2x\sqrt[3]{a}\\
    \sqrt[6]{64a^2y^{12}}&=2y^2\sqrt[6]{a^2}=2y^2\sqrt[3]{a}\qquad (x\ge 0,\;\; a\ge 0)
\end{split}\]    
    $\therefore\quad \sqrt[3]{8ax^3}$和$\sqrt[6]{64a^2y^{12}}$同类根式。
\end{solution}


\begin{example}
    $\sqrt{\frac{2x}{3}}$, $\sqrt{\frac{6}{x}}$, $\sqrt{6x}$是同类根式吗?
\end{example}


\begin{solution}
    由于:
\[\begin{split}
    \sqrt{\frac{2x}{3}}&=\sqrt{\frac{2x\cdot 3}{3\cdot 3}}=\sqrt{\frac{6x}{3^2}}=\frac{\sqrt{6x}}{3}=\frac{1}{3}\sqrt{6x} \\
    \sqrt{\frac{6}{x}}&= \sqrt{\frac{6x}{x^2}}=\frac{\sqrt{6x}}{x}=\frac{1}{x}\sqrt{6x}
\end{split}\]    
    $\therefore\quad \sqrt{\frac{2x}{3}},\; \sqrt{\frac{6}{x}},\;  \sqrt{6x}$是同类根式。
\end{solution}

根式相加减,就是把同类根式分别合并。


\begin{example}
    \[\begin{split}
       &\quad  \frac{2}{3}x\sqrt{9x}+6x\sqrt{\frac{x}{4}}-x^2\sqrt{\frac{1}{x}}\\
&=2x\sqrt{x}+3x\sqrt{x}-x\sqrt{x}\\
&=4x\sqrt{x}
    \end{split}\]
\end{example}
    
\begin{example}
    \[\begin{split}
        &\quad 15\sqrt[3]{4}-3\sqrt[3]{32}-16\sqrt[3]{\frac{1}{16}}-\sqrt[3]{108}\\
        &=15\sqrt[3]{4}-6\sqrt[3]{4}-4\sqrt[3]{4}-3\sqrt[3]{4}\\
        &=2\sqrt[3]{4}
    \end{split}\] 
\end{example}

\begin{ex}
\begin{enumerate}
    \item 说明下面根式是同类根式:
\begin{multicols}{2}
\begin{enumerate}
    \item $\sqrt[3]{24}$和$\sqrt[3]{81}$
    \item $\sqrt[3]{54}$和$\sqrt[3]{16}$
    \item $\sqrt{216}$和$\sqrt{\frac{3}{8}}$
    \item $\sqrt[3]{\frac{72}{343}}$和$\sqrt[3]{41\frac{2}{3}}$
    \item $\sqrt[4]{\frac{1}{27}}$和$\sqrt[4]{0.1875}$
\end{enumerate}
\end{multicols}
    \item  将下面根式化为同次根式:
    \begin{multicols}{2}
\begin{enumerate}
    \item $\sqrt{2},\; \sqrt[3]{5}$
    \item $\sqrt[3]{2},\; \sqrt[4]{3}$
    \item $\sqrt[4]{x^8},\; \sqrt[6]{y^5}$
    \item $\sqrt[3]{xy^2},\; \sqrt{yz},\; \sqrt[4]{xz^3}$
    \item $\sqrt{\frac{1}{a}+\frac{1}{x}},\; \sqrt[3]{\frac{1}{a}-\frac{1}{x}}$
\end{enumerate}
\end{multicols}
    \item 作下面根式的加减法:
\begin{enumerate}
    \item $\sqrt[3]{40}+\left(\frac{3}{2}\sqrt[3]{-5}-2\sqrt[3]{\frac{1}{5}}\right)$
    \item $\left(3\sqrt[3]{32}+\sqrt[3]{\frac{1}{9}}-\sqrt[3]{108}\right)-\left(16\sqrt[3]{\frac{1}{16}}-4\sqrt[3]{\frac{1}{72}}\right)$
\end{enumerate}
    \item 作下面根式的乘除法:
\begin{multicols}{2}
\begin{enumerate}
    \item $\sqrt{a}\cdot \sqrt[4]{\frac{x}{a}}$
    \item $\sqrt{\frac{2}{3}}\cdot \sqrt[3]{\frac{3}{2}}\cdot \sqrt[6]{\frac{1}{2}}$
    \item $\left(\sqrt{2}-\sqrt[3]{4}+\sqrt[4]{8}\right)\cdot \sqrt{2}$
    \item $\sqrt{140}\div \sqrt{20}$
    \item $\sqrt[4]{8}\div \sqrt{2}$
    \item $\sqrt[6]{3}\div \sqrt[12]{18}$
    \item $\left(\sqrt[9]{8}\right)^4$
    \item $\left(\sqrt[9]{27}\right)^4$
\end{enumerate}
\end{multicols}
\end{enumerate}
\end{ex}

\subsection{分数指数幂}
我们再把指数幂的概念由整数指数幂推广到分数指数
幂,先来探索如何合理地定义$a^{\tfrac{1}{4}}$,$a^{\tfrac{3}{4}}$的意义。

假设符号$a^{\tfrac{1}{4}}$,$a^{\tfrac{3}{4}}$
有意义,并且适合整数指数幂法则$(a^m)^n=a^{mn}$,
那么对于$a^{\tfrac{1}{4}}$,$a^{\tfrac{3}{4}}$
应用这个法则就得到
$\left(a^{\tfrac{1}{4}}\right)^4=a$和$\left(a^{\tfrac{3}{4}}\right)^4=a^3$。这就是说,可以把$a^{\tfrac{1}{4}}$,$a^{\tfrac{3}{4}}$
看作方程$x^4=a,\; x^4=a^3$的根。实际上这两个方程的唯一正实数解分别是$\sqrt[4]{a}$和$\sqrt[4]{a^8}$, 即有等式$\left(\sqrt[4]{a}\right)^4=a$, $\left(\sqrt[4]{a^3}\right)^4=a^3$成
立。因此,我们定义$a^{\tfrac{1}{4}}=\sqrt[4]{a}$,$a^{\tfrac{3}{4}}=\sqrt[4]{a^3}$是合理的。

下面给出一般的定义:
\begin{blk}{定义4}
若$a>0$,$m,n$是正整数,我们规定:
 \[a^{\tfrac{m}{n}}=\sqrt[n]{a^m},\qquad a^{-\tfrac{m}{n}}=\frac{1}{a^{\tfrac{m}{n}}}\]
\end{blk}

例如:
\[5^{\tfrac{3}{4}}=\sqrt[4]{5^3},\quad 2^{\tfrac{3}{2}}=\sqrt{a^3},\quad 4^{\tfrac{1}{3}}=\sqrt[3]{4},\quad a^{-\tfrac{4}{3}}=\frac{1}{a^{\tfrac{4}{3}}}=\frac{1}{\sqrt[3]{a^4}} \]

有了定义4, 我们的指数就推广到了有理数了。

对于分数指数,还需要讨论它的合理性问题。这个问题
的提法是这样的。设有一个正有理数$r$, 按照有理数的性质,
一定有两个正整数$m,n$, 使得$r=\frac{m}{n}$,
那么按照定义4,
\[a^r=a^{\tfrac{m}{n}}=\sqrt[n]{a^m}\qquad (a>0)\]
但另一方面,$r$还有其它正分数表示法,例如,
$\frac{2m}{2n}$
就是另一个不同的表示法,设
$\frac{m_1}{n_1}$是$r$的另一个任意正分数表示法,
那么按定义4,
\[a^r=a^{\tfrac{m_1}{n_1}}=\sqrt[n_1]{a^{m_1}}\qquad (a>0)\]

于是,对于同一个正有理数$r$, $a^r$就有很多(实际上是无
穷多)个形式上不同的表示式,而$a^r$当然被规定为一个确定
的数,所以必须证明$a^r$的任何两个不同的表示式是相等的。
这就是下面的定理。

\begin{blk}{定理}
    设$a$是任意给定的正实数,$m,n,m_1,n_1$是正整
    数且$\frac{m}{n}=\frac{m_1}{n_1}$,
    则$a^{\tfrac{m}{n}}=a^{\tfrac{m_1}{n_1}}$   
\end{blk}

\begin{proof}
由于$\frac{m}{n}=\frac{m_1}{n_1}$,因而$mn_1=m_1n$,故
\[a^{mn_1}=a^{m_1n}\]

$\because\quad a^{\tfrac{m}{n}}=\sqrt[n]{a^m}=\sqrt[nn_1]{a^{mn_1}}\qquad 
a^{\tfrac{m_1}{n_1}}=\sqrt[n_1]{a^{m_1}}=\sqrt[nn_1]{a^{m_1n}}$

$\therefore\quad \sqrt[nn_1]{a^{mn_1}}=\sqrt[nn_1]{a^{m_1n}}$,即:
\[a^{\tfrac{m}{n}}=a^{\tfrac{m_1}{n_1}}\]
\end{proof}
    
这就解决了定义的合理性的问题。不仅如此,它还告诉
我们,可以改变有理数的分母以适应各种不同的需要。

例如:$5^{\tfrac{1}{2}}=5^{\tfrac{2}{4}}=5^{\tfrac{3}{6}}=\cdots$

\begin{rmk}
    分指数幂的定义不考虑底是负数的情形,因为这
    时分指数幂不再具有上述的重要性质.例如,$(-1)^{\tfrac{1}{3}}=\sqrt[3]{-1}=-1$, 同时$(-1)^{\tfrac{2}{6}}=\sqrt[6]{(-1)^2}$=1, 所以$(-1)^{\tfrac{1}{3}}\ne (-1)^{\tfrac{2}{6}}$。又$0^{-\tfrac{m}{n}}$没有
意义,因此分指数幂的底限制为正数。
\end{rmk}

有了分数指数幂的定义,上面讲过的三条性质在新的范
围内就可叙述为:
\begin{blk}{性质}
\begin{enumerate}
    \item  $a^r\cdot a'^s=a^{r+s}$\quad ($r,s$是有理数,$a>0$);
    \item  $(a^r)^s=a^{rs}$\quad ($r,s$是有理数,$a>0$);
    \item  $(ab)^r=a^r\cdot b^r$\quad ($r$是有理数,$a,b>0$)。
\end{enumerate}  
\end{blk}

我们只对性质1作出证明,其它性质的证明留给同学
自己去考虑。

性质1的证明:设$a>0$, $r,s$为有理数,我们证明
$a^r a^s=a^{r+s}$

\begin{proof}
\textbf{情形1 } 若$r,s$都是正有理数时,则$r=\frac{m}{n}$, $s=\frac{\ell}{k}$, 
这里$m,n,\ell,k$都是正整数。
\[\begin{split}
a^r a^s&= a^{\tfrac{m}{n}}\cdot a^{\tfrac{\ell}{k}} =\sqrt[n]{a^m}\cdot \sqrt[k]{a^{\ell}}\\
&=\sqrt[nk]{a^{mk}}\cdot \sqrt[nk]{a^{\ell n}}\\
&=\sqrt[nk]{a^{mk}\cdot a^{n\ell}}=\sqrt[nk]{a^{mk+n\ell}}\\
&=a^{\tfrac{mk+n\ell}{nk}}=a^{\tfrac{mk}{nk}+\tfrac{n\ell}{nk}}\\
&=a^{\tfrac{m}{n}+\tfrac{1}{k}}=a^{r+s}
\end{split}\]
$\therefore\quad a^r a^s=a^{r+s}$

\textbf{情形2 } 若$r<0$, $s<0$, 则$r=-|r|$, $s=-|s|$.
\[\begin{split}
a^r\cdot a^s &= a^{-|r|}a^{-|s|}=\frac{1}{a^{|r|}}\cdot \frac{1}{a^{|s|}}\\
&=\frac{1}{a^{|r|}a^{|s|}}=\frac{1}{a^{|r|+|s|}}\\
&=a^{-(|r|+|s|)}=a^{-|r|+(-|s|)}\\
&=a^{r+s}
\end{split}\]
$\therefore\quad a^r a^s=a^{r+s}$


\textbf{情形3 } 若$r>0$, $s<0$, 则$|s|=\frac{\ell}{k}$, $r=\frac{m}{n}$, 
$m,n,\ell,k$都是正整数,则
\[\begin{split}
a^r a^s&=a^r\cdot  a^{-|s|}=\frac{a^r}{a^{|s|}}\\
&=\frac{a^{\tfrac{m}{n}}}{a^{\tfrac{\ell}{k}}}=\frac{\sqrt[n]{a^m}}{\sqrt[k]{a^{\ell}}}\\
&=\frac{\sqrt[nk]{a^{mk}}}{\sqrt[nk]{a^{n\ell}}}=\sqrt[nk]{\frac{a^{mk}}{a^{n\ell}}}\\
&=\sqrt[nk]{a^{mk-n\ell}}=a^{\tfrac{mk-n\ell}{nk}}\\
&=a^{\tfrac{m}{n}-\tfrac{\ell}{k}}=a^{r-|s|}\\
&=a^{r+s}
\end{split}\]
$\therefore\quad a^r\cdot  a^s=a^{r+s}$

(若$r<0$, $s<0$则根据交换律也是成立的)

此外,$r,s$有一个为零的情形,性质1显然成立,故
$a^ra^s=a^{r+s}$对任意有理数$r,s$成立。
\end{proof}

\begin{example}
求下面各分指数幂的值:
\begin{enumerate}
\item $18^{\tfrac{1}{2}}=\left(3^2\cdot 2\right)^{\tfrac{1}{2}}=3\cdot 2^{\tfrac{1}{2}}=3\sqrt{2}$
\item $100^{-\tfrac{3}{2}}=(10^2)^{-\tfrac{3}{2}}=10^{-3}=\frac{1}{10^3}=0.001$
\item $\left(\frac{81}{625}\right)^{-\tfrac{3}{4}}=\left[\left(\frac{3}{5}\right)^4\right]^{-\tfrac{3}{4}}=\left(\frac{3}{5}\right)^{-3}=\left(\frac{5}{3}\right)^3=\frac{125}{27}$
\end{enumerate}
\end{example}
    
\begin{example}
    用分指数幂作下面根式运算:
\begin{enumerate}
    \item $\sqrt[4]{a^3}\div \sqrt[3]{a}=a^{\tfrac{3}{4}}\cdot a^{-\tfrac{1}{3}}=a^{\tfrac{3}{4}-\tfrac{1}{3}}=a^{\tfrac{5}{12}}=\sqrt[12]{a^5}$
    \item $\sqrt{a^3\cdot a\sqrt{a}\cdot a^6 \sqrt[3]{a}}=\left(a^3\cdot a^{1\tfrac{1}{2}}\cdot a^{6\tfrac{1}{3}}\right)^{\tfrac{1}{2}}=\left(a^{10\tfrac{5}{6}}\right)^{\tfrac{1}{2}}=a^{5\tfrac{5}{12}}=a^{5}\cdot \sqrt[12]{a^5}$
\end{enumerate}
    
\end{example}

\begin{example}
    化简下面算式:
\begin{enumerate}
    \item \[\begin{split}
&\frac{5 x^{-\tfrac{2}{3}} y^{\tfrac{1}{2}}}{\left(-\frac{1}{4}x^{-1} y^{-\tfrac{1}{3}}\right)\left(-\frac{5}{6} x^{-\tfrac{1}{3}} y^{-\tfrac{1}{6}}\right)} \\
&=24 x^{-\tfrac{2}{3}+1+\tfrac{1}{3}} y^{\tfrac{1}{2}+\tfrac{1}{3}+\tfrac{1}{6}}=24 x^{\tfrac{2}{3}} y \\
&=24 y \sqrt[3]{x^{2}}
\end{split}\]
\item \[\begin{split}
& \quad   \left(a^{2} x+a x^{1.5} \right)\left(a^{1.5} x^{0.5}+a^{0.5} x\right)^{-1}\\
&=(a x)\left(a+x^{0.5}\right)\left[a^{0.5} x^{0.5} \left(a+x^{0.5}\right)\right]^{-1}\\
&=(a x)(a x)^{-0.5}\left(a+x^{0.5}\right)\left(a+x^{0.5}\right)^{-1}\\
&=(a x)^{0.5}\left(a+x^{0.5}\right)^{0}\\
&=(a x)^{0.5}=\sqrt{a x}
\end{split}\]
\item \[\begin{split}
 &\quad   \frac{m-n}{m^{\tfrac{1}{2}}-n^{\tfrac{1}{2}}}-\frac{m^{\tfrac{3}{4}}+n^{\tfrac{3}{4}}}{m^{\tfrac{1}{4}}+n^{\tfrac{1}{4}}} \\
&=\frac{\left(m^{\tfrac{1}{2}}\right)^2-\left(n^{\tfrac{1}{2}}\right)^2}{m^{\tfrac{1}{2}}-n^{\tfrac{1}{2}}}-\frac{\left(m^{\tfrac{1}{4}}\right)^3+\left(n^{\tfrac{1}{4}}\right)^3}{m^{\tfrac{1}{4}}+n^{\tfrac{1}{4}}} \\
&=m^{\tfrac{1}{2}}+n^{\tfrac{1}{2}}-\left[\left(^{\tfrac{1}{4}}\right)^2- m^{\tfrac{1}{4}}n^{\tfrac{1}{4}}+\left(n^{\tfrac{1}{4}}\right)^2\right]\\
&=m^{\tfrac{1}{4}}n^{\tfrac{1}{4}}=\sqrt[4]{mn}
\end{split}\]
\end{enumerate}
\end{example}

由例1.15看出,有了分数指数,根式就可转化为幂,根式
的运算转化为幂的运算,这就有可能简化计算工作,为以后
的对数计算在理论上作了准备。

\section*{习题1.2}

\addcontentsline{toc}{subsection}{习题1.2}
\begin{enumerate}
    \item 求下面分指数幂的值:
\begin{multicols}{3}
\begin{enumerate}
    \item $8^{\tfrac{4}{3}}$
    \item $16^{-\tfrac{3}{2}}$
    \item $9^{-\tfrac{5}{2}}$
    \item $\left(\frac{27}{8}\right)^{-\tfrac{1}{3}}$
    \item $25^{-\tfrac{1}{2}}$
    \item $\left(2\frac{1}{4}\right)^{-\tfrac{3}{2}}$
    \item $\left(3\frac{3}{8}\right)^{-\tfrac{1}{3}}$
    \item $(0.008)^{-\tfrac{2}{3}}$
    \item $54^{\tfrac{1}{3}}$
    \item $(500)^{\tfrac{1}{3}}$
\end{enumerate}
\end{multicols}
(i),(j)化为最简根式。

\item 按照$\left(\frac{2}{5}\right)^{\tfrac{1}{3}}=\left(\frac{2\cdot 25}{5\cdot 25}\right)^{\tfrac{1}{3}}=\frac{50^{\tfrac{1}{3}}}{5} $ 的样子将分母变成有理数:
\begin{multicols}{3}
\begin{enumerate}
    \item $\left(\frac{2}{3}\right)^{\tfrac{1}{2}}$
    \item $\left(\frac{5}{12}\right)^{\tfrac{1}{2}}$
    \item $\left(\frac{1}{2}\right)^{\tfrac{1}{3}}$
    \item $\left(\frac{4}{3}\right)^{\tfrac{1}{3}}$
    \item $\left(\frac{2}{5}\right)^{\tfrac{1}{4}}$
    \item $\left(\frac{4}{3y^2}\right)^{\tfrac{1}{4}}$
    \item $\left(\frac{1}{10x^3}\right)^{\tfrac{1}{4}}$
    \item $\left(\frac{2x}{3y^2}\right)^{\tfrac{1}{3}}$
\end{enumerate}
\end{multicols}

\item 通过扩分(应用分式的基本性质)将下列分式的分母变成有理数:

例:$\frac{1}{a^{\tfrac{2}{3}}}=\frac{1\cdot a^{\tfrac{1}{3}}}{a^{\tfrac{2}{3}}\cdot a^{\tfrac{1}{3}}}=\frac{a^{\tfrac{1}{3}}}{a}=\frac{\sqrt[3]{a}}{a}$
\begin{multicols}{3}
    \begin{enumerate}
\item $\frac{4}{8^{\tfrac{1}{2}}}$
\item $\frac{10}{5^{\tfrac{1}{3}}}$
\item $\frac{1}{x^{\tfrac{1}{4}}}$
\item $\frac{1}{a^{\tfrac{2}{3}}}$
\item $\frac{10a}{(2a^3)^{\tfrac{1}{3}}}$
    \end{enumerate}
\end{multicols}

\item 按照$2\cdot \left(\frac{1}{4}\right)^{\tfrac{1}{3}}=\left(\frac{8}{4}\right)^{\tfrac{1}{3}}=2^{\tfrac{1}{3}}$的方法做
\begin{multicols}{2}
    \begin{enumerate}
\item $2\cdot \left(\frac{1}{2}\right)^{\tfrac{1}{3}}$
\item $3\cdot \left(\frac{1}{3}\right)^{\tfrac{1}{3}}$
\item $\frac{1}{2} \left(64\right)^{\tfrac{1}{5}}$
\item $a \left(\frac{1}{a}\right)^{\tfrac{3}{2}}$
    \end{enumerate}
\end{multicols}
\item 用根式表示指数幂:
\begin{multicols}{3}
    \begin{enumerate}
\item $a^{\tfrac{4}{5}}$
\item $b^{-\tfrac{1}{2}}$
\item $c^{-\tfrac{3}{5}}$
\item $\left(a^{\tfrac{4}{7}}\right)^{-3}$
\item $a^{\tfrac{1}{3}}b^{\tfrac{2}{3}}c^{-\tfrac{1}{3}}$
    \end{enumerate}
\end{multicols}

\item 用正指数幂表示下列根式:
\begin{multicols}{2}
    \begin{enumerate}
\item $\sqrt[4]{b^{-3}}$
\item $\left(\sqrt{2}\right)^{-\tfrac{3}{5}}$
\item $\left(\frac{1}{\sqrt[4]{x^{-5}}}\right)^{-2}$
\item $\frac{\sqrt[3]{a^2}}{\sqrt[3]{b}}$
    \end{enumerate}
\end{multicols}
\item 用有理指数幂表示下列各式:
\begin{multicols}{2}
    \begin{enumerate}
\item $\sqrt[3]{a^4}$
\item $\sqrt[5]{b^{11}}$
\item $\frac{1}{c\sqrt[5]{c^4}}$
\item $\sqrt{\frac{1}{x^5}}$
\item $\frac{3}{\sqrt[5]{y^3}}$
\item $\frac{a^{\tfrac{1}{2}}}{\sqrt{5x^3}}$
\item $\frac{2\sqrt{a^{-3}}}{3\sqrt[3]{5^5}}$
    \end{enumerate}
\end{multicols}


\item 用分指数幂计算:
\begin{multicols}{2}
    \begin{enumerate}
\item $2\sqrt{2}\cdot \sqrt[4]{2}\cdot \sqrt[8]{2}$
\item $\frac{\sqrt{x}\cdot \sqrt[3]{x^2}}{x\sqrt[6]{x}}$
\item $\frac{a^5\sqrt[3]{a}}{\sqrt{a^3}\sqrt[6]{a^3}}$
\item $\sqrt{\sqrt[3]{4}}$
\item $\sqrt[3]{a\sqrt[4]{a^3}}$
\item $\left(\sqrt[3]{5}-\sqrt{125}\right)\div \sqrt[4]{5}$
\item $\sqrt[3]{a^{-2}}\cdot \sqrt{a^{-3}}$
    \end{enumerate}
\end{multicols}

\item 化简下列各式:
\begin{multicols}{2}
    \begin{enumerate}
\item $a^{\tfrac{1}{4}}a^{\tfrac{1}{3}}$
\item $\left(\frac{1}{2}x^{\tfrac{1}{2}}y^{-\tfrac{1}{2}}z\right)\left(-\frac{2}{3}x^{-1}y^{-\tfrac{1}{3}}z^{\tfrac{1}{2}}\right)$
\item $\left(x^{\tfrac{1}{2}}y^{-\tfrac{2}{3}}\right)^6$
\item $\frac{15a^{\tfrac{1}{2}}b^{\tfrac{1}{3}}c^{\tfrac{3}{4}}}{25a^{-\tfrac{1}{2}}b^{\tfrac{1}{3}}c^{\tfrac{5}{4}}}$
\item $a^{\tfrac{1}{3}}b^{\tfrac{2}{3}}\div a^{-\tfrac{2}{3}}b^{\tfrac{4}{3}}$
\item $\left(a^{\tfrac{1}{2}}\sqrt[3]{b^2}\right)^{-3}\div \sqrt{b^{-4}\sqrt{a^{-2}}}$
\item $\frac{a^{-\tfrac{1}{2}}b^{-\tfrac{3}{4}}}{a^{\tfrac{1}{3}}b^{\tfrac{1}{2}}}\div \frac{a^{\tfrac{5}{6}}}{b^{-\tfrac{1}{6}}}$
\item $\left(8b^{-\tfrac{1}{3}}x^{-\tfrac{4}{3}}b\sqrt[4]{x^{\tfrac{4}{3}}}\right)^{\tfrac{1}{3}}$
\item $\left[\left(x^{\tfrac{1}{m-n}}\right)^{m-\tfrac{n^2}{m}}\right]^{\tfrac{m}{m+n}}$
\item $x\sqrt{x\sqrt{x\sqrt{x}}}$
    \end{enumerate}
\end{multicols}

\item 分解下列各式为因式的积:
\[a^{\tfrac{2}{3}}-b^{\tfrac{2}{3}},\qquad x^{\tfrac{3}{2}}-y^{\tfrac{3}{2}},\qquad x^{-3}-27y^{-3} \]

\item 计算:
\begin{multicols}{2}
    \begin{enumerate}
\item $\left(a^{\tfrac{1}{2}}b^{\tfrac{1}{2}}\right)^2$
\item $\left(a^{\tfrac{1}{3}}-b^{-\tfrac{2}{3}}\right)\left(a^{\tfrac{2}{3}}+b^{-\tfrac{2}{3}}+b^{-\tfrac{4}{3}}\right)$
\item $\left(a^{m}+a^{\tfrac{m}{2}}+1\right)\left(a^{-m}+a^{\tfrac{m}{2}}+1\right)$
\item $\left(m^{\tfrac{3}{2}}+n^{\tfrac{3}{2}}\right)\div \left(m^{\tfrac{1}{2}}+n^{\tfrac{1}{2}}\right)$
\item $\frac{a-b}{a^{\tfrac{1}{3}}-b^{\tfrac{1}{3}}}-\frac{a+b}{a^{\tfrac{1}{3}}+b^{\tfrac{1}{3}}}$
    \end{enumerate}
\end{multicols}
\end{enumerate}

\subsection{无理指数幂}
由于无理指数幂的概念要用到实数完备性或极限存在定
理,我们这里不作详细介绍,只指出$a^{\alpha}$($a>0$, $\alpha$是无理数),例如$2^{\sqrt{2}}$, $10^{\sqrt{8}}$, $3^{\pi}$等等,仍有确定意义,即它仍
代表一个确定的实数,并且也满足指数运算的三个法则:
\[\begin{split}
    a^{\alpha}\cdot a^{\beta}&=a^{\alpha+\beta}\\
    \left(a^{\alpha}\right)^{\beta}&=a^{\alpha\beta}\\
    (ab)^{\alpha}&=a^{\alpha}\cdot b^{\alpha}
\end{split}\]
这里$a$, $b$大于零;$\alpha$, $\beta$是无理数。

于是指数法则可以进一步推广,得到下面的普遍定理。

\begin{blk}{定理}
    指数运算法则$a^{\alpha}\cdot a^{\beta}=a^{\alpha+\beta}$, $\left(a^{\alpha}\right)^{\beta}=a^{\alpha\beta}$, $(ab)^{\alpha}=a^{\alpha}\cdot b^{\alpha}$ ($a,b>0$)对于任何实数$\alpha, \beta$成立。
\end{blk}

在实际应用中,我们常用有理指数幂去近似地代替无理
指数幂,例如$\sqrt{2}\approx 1.414$, $10^{\sqrt{2}}\approx 10^{1.414}$。
因此,在这里
我们只要求同学知道上述结论就可以了。

\section{对数和常用对数}
\subsection{对数的定义}

在前面我们引入了有理指数的概念,并指出对于正实数
$a$与任意实数$\alpha$, $a^{\alpha}$都有明确的意义;在学习指数时,我们回
顾一下,同底指数幂有一些运算公式:
\[\begin{split}
    a^{m}\cdot a^n&=a^{m+n}\\
    a^{m}\div a^n&=a^{m-n}\\
    (a^m)^n&=a^{mn}\\
    \sqrt[n]{a}&=a^{\tfrac{m}{n}}
\end{split}\]
这里幂指数$m,n$为任意实数,根指数$n$是大于1的自然数,
$a$为正实数。

这就是说,同底的幂相乘转化为指数相加;同底的幂相
除转化为指数相减;幂的乘方转化为指数相乘;幂的开方转
化为指数相除,对于数值的计算,加法和减法显然比乘法和
除法容易得多,譬如我们用2的$n$次幂可以简捷地算出式子:
$M=\frac{512^2\x 64}{32^3\x 256}$的值。
\[M=\frac{(2^9)^2\x 2^6}{(2^5)^3\x 2^8}=2^{18+6-15-8}=2^1=2\]
因此,我们自然就希望把这一性质用到实际计算工作中去,
使计算简化。

现在的问题是,任意两个正数相乘,能否应用上述简化
的思想来计算?即,如果$M$和$N$均为任意正实数,能否实现
下述过程:
\begin{center}
\begin{tikzpicture}[>=latex]
\node at (0,0){\Large  $N\x M=a^b\x a^c=a^{b+c}\quad (a>0,\; \text{且} a\ne 1)$};
\draw[->](-4.7,-.3)--(-4.7,-1)--(-2.5,-1)--(-2.5,-.3);
\draw[->](-3.5,.3)--(-3.5,1)--(-1.3,1)--(-1.3,.3);
\node at (-3,0){\Large ?};
\end{tikzpicture}
\end{center}

\begin{rmk}
    因为1以外的正数不可能等于1的任何次幂,所
以这里必须限定$a\ne 1$。
\end{rmk}

要把任意正数的乘法转化为同底的幂的乘法,从而用指
数相加来完成,关键在于任意一个正数$N$能否写成一个已知
数$a$ ($a>0$且$a\ne1$)的幂$a$的形式,换句话说,就是给了一个
不等于1的正实数$a$作为底数,那么对于任意给定的正实数
$N$, 是否有唯一的实数$b$存在,使得$N=a$, 只要这个问题解
决了,上述的想法就可实现,回答是肯定的,这就是:

\begin{blk}{定理}
    设$a>0$且$a\ne 1$, 那么对于任意给定的正实数$N$
存在唯一的实数$b$, 使得
\[a^b=N\]
\end{blk}
 
定理的严格证明需要较多的理论,我们将在第六册给出
它的证明。

现在我们利用这个定理再介绍一个新的概念和一个新的
符号如下:

\begin{blk}{定义}
    设$a$是一个不等于1的正实数,$N$是任意给定的
正实数,如果实数$b$使得等式
\[a^b=N\]
成立,那么$b$就叫做\textbf{以$a$为底$N$的对数},记为$\log_a N=b$, $N$叫
做\textbf{真数}。
\end{blk}
 
从定义里可以看出,下面两个式子是等价的:
\[a^b=N  \Longleftrightarrow \log_a N=b\]
前者叫做\textbf{指数式},后者叫做\textbf{对数式}。

由定义知道,求对数是求方幂的一种逆运算。若给出底数
$a$和指数$b$就是求方幂$N$, 反过来,若给出底数$a$和方幂$N$, 
求指数$b$, 就是求对数$b=\log_a N$; 若给出指数$b$和方幂$N$, 求底数$a$, 这就是求方根$a=N^{\tfrac{1}{b}}$, 这三个等式,$\log_a N=b$, $a^b=N$
和$a=N^{\tfrac{1}{b}}$是等价的,即如果$a$、$b$、$N$三个数满足其中一
个等式,那么它们也满足另外两个等式。




\begin{example}
    将下列指数式换成对数式:
\begin{enumerate}
    \item $10^2=100\quad \Longleftrightarrow \quad \log_{10}100= 2$
    \item $3^5=243\quad \Longleftrightarrow \quad\log_3 243=5$
    \item $4^{\tfrac{1}{4}}=\sqrt{2}\quad \Longleftrightarrow \quad \log_4\sqrt{2}=\frac{1}{4}$
    \item $2^10=1024\quad \Longleftrightarrow \quad \log_2 1024=10$
    \item $3^{-1}=\frac{1}{3}\quad \Longleftrightarrow \quad \log_3\frac{1}{3}=-1$
\end{enumerate}
\end{example}



\begin{example}
    求下面等式中的$x$值:
\begin{enumerate}
    \item $\log_{64}x=-\frac{2}{3}  \quad \Longleftrightarrow \quad x=(64)^{-\tfrac{2}{3}}=\left(4^3\right)^{-\tfrac{2}{3}}=4^{-2}=\frac{1}{16} $
    \item  $ \log_x 8 =6 \quad \Longleftrightarrow \quad x^6=8,\quad x=(2^3)^{\tfrac{1}{6}}=\sqrt{2} $
    \item  $ \log_9 27=x \quad \Longleftrightarrow \quad 9^x=27,\quad 3^{2x}=3^3 $
    
    $\therefore\quad 2x=3,\quad x=\frac{3}{2}$
\end{enumerate}
\end{example}

同学们想一想怎样证明下面两个对数的重要性
质:
\[\log a=1\qquad (a>0,\; a\ne 1)\]
即底的对数恒等于1;
\[\log 1=0\qquad (a>0,\; a\ne 1)\]
即1的对数,对于任何底恒等于0。

对某数连续地完成两个互逆的运算得到的数就是原来这
个数,例如
$$(2+3)-3=2,\qquad (2\x3)\div 3=2$$
因此如
果给出数$N$, 求出以$a$为底$N$的对数以后,接着再求以$a$为底
该对数为指数的幂,结果仍等于$N$. 这一点可将对数式
$\log_a N=b$代入指数式$a^b=N$中的$b$得到下面的恒等式:
\begin{equation}
    a^{\log_a N}=N
\end{equation}

这个恒等式也说明$\log_a N$就是方程$a^b=N$的唯一解。

如果给出数$b$, 我们求出$a$的$b$次幂后,接着再求这个幂
以$a$为底的对数,这相当于把方幂$a^b=N$代入等式$\log_a N=b$中
的真数$N$得到恒等式:
\begin{equation}
    b=\log_a a^b
\end{equation}



\begin{example}
计算
\begin{multicols}{3}
\begin{enumerate}
    \item $3^{\log_3 243}$
    \item $\log_3 27$
    \item $4^{1+\log_4\sqrt{2}}$
    \item $\log_{10}0.1$
    \item $5^{\log_5 2 -1}$
\end{enumerate}
\end{multicols}
\end{example}

\begin{solution}
\begin{enumerate}
    \item $3^{\log_3 243}=243$
    \item $\log_3 27=\log_3 3^3=3$
    \item $4^{1+\log_4\sqrt{2}}=4\cdot 4^{\log_4\sqrt{2}}=4\sqrt{2}$
    \item $\log_{10}0.1=\log_{10}10^{-1}=-1$
    \item $5^{\log_5 2 -1}=\frac{5^{\log_5 2}}{5}=\frac{2}{5}$
\end{enumerate}    
\end{solution}

\begin{example}
 试将以下四式写成2的方幂:
 \[\sqrt{2},\qquad \sqrt{8},\qquad 3,\qquad \sqrt[3]{3}\]   
\end{example}
  
\begin{solution}
\begin{enumerate}
    \item $\sqrt{2}=2^{\tfrac{1}{2}}$
    \item $\sqrt{8}=2^{\tfrac{3}{2}}$
    \item $3=2^{\log_2 3}$
    \item $\sqrt[3]{3}=2^{\log_2 \sqrt[3]{3}}$
\end{enumerate}
\end{solution} 


\begin{ex}
\begin{enumerate}
    \item 计算下列各式的值:
\begin{multicols}{3}
\begin{enumerate}
    \item $2^{0.5}\cdot 8^{0.5}$
    \item $\frac{10^{-2.5}}{10^{0.5}}$
    \item $\left(5^{\tfrac{1}{3}}\right)^3$
    \item $\left(\frac{49}{225}\right)^{-\tfrac{1}{2}}$
    \item $(16\x 81)^{-0.25}$
\end{enumerate}
\end{multicols}   

\item 求下列各式中的$x$, 指出哪个是幂的运算?哪个是求对
数?哪个是开方运算?
\begin{multicols}{3}
    \begin{enumerate}
        \item $3^4=x$
        \item $x^3=1000$
        \item $10^x=0.001$
    \end{enumerate}
    \end{multicols}  
\item 求真数:
\begin{multicols}{3}
\begin{enumerate}
    \item $\log_2 x=3$
    \item $\log_4 x=-2$
    \item $\log_3 N=0$
    \item $\log_3 N=1$
    \item $\log_{\sqrt{2}} x=4$
    \item $\log_{0.1} x=-1$
\end{enumerate}
\end{multicols}  
\item 求底数:
\begin{multicols}{2}
\begin{enumerate}
    \item $\log_x 216 = 3 $
    \item $\log_x \frac{1}{81} = 4 $
    \item $\log_x \frac{1}{64} = -3 $
    \item $\log_x \sqrt{8} = \frac{3}{4} $
\end{enumerate}
\end{multicols}  

\item 求对数:
\begin{multicols}{3}
\begin{enumerate}
    \item $\log_{10} 10000=x$
    \item $\log_{10} 0.001=x$
    \item $\log_4 256=x$
    \item $\log_{\tfrac{1}{3}} 9=x$
    \item $\log_3 \frac{1}{27}=x$
    \item $\log_9\frac{1}{9}=x$
    \item $\log_5 1=x$
    \item $\log_{0.04} 5=x$
    \item $\log_{3\sqrt{3}}\frac{1}{27}=x$
\end{enumerate}
\end{multicols} 

\item 利用恒等式$\log_a a^b=b,\quad (a>0,\; a\ne 1)$计算:
\begin{multicols}{3}
    \begin{enumerate}
        \item $\log_3\sqrt[4]{3}$
        \item $\log_2\sqrt{2}$
        \item $\log_5\frac{1}{\sqrt[3]{5}}$
        \item $\log_3\sqrt{27}$
        \item $\log_a\frac{1}{a^n}$
        \item $\log_a \frac{1}{\sqrt[n]{a}}$
    \end{enumerate}
    \end{multicols} 

\item 求对数$x$:
\begin{multicols}{3}
    \begin{enumerate}
        \item $2^x=8$
        \item $2^x=0.125$
        \item $3^x=1$
        \item $4^x=2$
        \item $4^x=0.5$
        \item $10^x=10\sqrt{10}$
        \item $2^x=\frac{\sqrt[5]{2}}{2}$
        \item $5^x=\sqrt[3]{5^2}$
        \item $10^x=\frac{1}{\sqrt[4]{1000}}$
        \item $2^x=3$
        \item $\left(\sqrt{2}\right)^x=10$
    \end{enumerate}
    \end{multicols} 

\item 根据恒等式$a^{\log_a N}=N$, 求:
\begin{multicols}{3}
    \begin{enumerate}
        \item $2^{\log_2 8}$
        \item $3^{\log_3 7}$
        \item $36^{\log_6 2}$
        \item $25^{\log_5 3}$
        \item $81^{0.5\log_9 7}$
        \item $81^{\tfrac{1}{2}\log_3 7}$
        \item $5^{\log_5 10-1}$
        \item $2^{\log_2 5+1}$
    \end{enumerate}
    \end{multicols} 
\end{enumerate} 
\end{ex}

\subsection{对数的性质}
在前面我们说明了对于每个正实数$x$都有唯一的以$a$为底
的对数$\log_a x$和它对应,于是,
\[\begin{split}
   M(M>0)&\longrightarrow \log_a M,\quad \text{使得 } a^{\log_a M}=M \\
   N(N>0)&\longrightarrow \log_a N,\quad \text{使得 } a^{\log_a N}=N \\
   M\cdot N(M>0,\; N>0)&\longrightarrow \log_a (M\cdot N),\quad \text{使得 } a^{\log_a (MN)}=MN \\
   \frac{M}{N}(M>0,\; N>0)&\longrightarrow \log_a \frac{M}{N},\quad \text{使得 } a^{\log_a \tfrac{M}{N}}=\frac{M}{N} \\
   M^u(M>0,\; u\in\mathbb{R})&\longrightarrow \log_a M^u,\quad \text{使得 } a^{\log_a M^u}=M^u \\
   \sqrt[n]{M}(n\ge 1,\; n\in\mathbb{Z})&\longrightarrow \log_a \sqrt[n]{M},\quad \text{使得 } a^{\log_a \sqrt[n]{M}}=\sqrt[n]{M} \\
\end{split}\]

现在我们来研究怎样求:
\begin{enumerate}
    \item 积的对数;
    \item 商的对数;
    \item $u$次方幂的对数;
    \item $n$次方根的对数。
\end{enumerate}

显然对数的性质与指数法则有密切关系,譬如,对应于
指数法则$a^u\cdot a^v=a^{u+v}$,就有下面的对数法则:

\begin{blk}{性质1}
    两个正数乘积的对数,等于这个积的因数的对
数的和,即
\[\log_a(MN)=\log_a M+\log_a N\]
\end{blk}

\begin{proof}
    根据恒等式,有
    \begin{equation}
        M=a^{\log_a M},\qquad N=a^{\log_a N}
    \end{equation}
    那么将指数法则:$a^u\cdot a^v=a^{u+v}$应用到(1.5), 得到
\[MN=a^{\log_a M}\cdot a^{\log_a N}=a^{\log_a M+\log_a N}\]
    根据对数定义得到
    \[\log_a MN = \log_a M + \log_a N\]
\end{proof}    

对应于指数法则$\frac{a^u}{a^v}=a^{u-v}$,就有对数法则:
    
\begin{blk}{性质2}
    一个商的对数等于分子与分母的对数的差,即
    \[\log_a \frac{M}{N}=\log_a M-\log_a N\]
\end{blk}

\begin{proof}
将指数法则$\frac{a^u}{a^v}=a^{u-v}$
应用到(1.5), 得到:
\[\frac{M}{N}=\frac{a^{\log_a M}}{a^{\log_a N}}=a^{\log_a M-\log_a N}\]
根据对数定义得到:
\[\log\frac{M}{N}=\log_a M -\log_a N\]
\end{proof}    

对应于指数法则$(a^m)^n=a^{mn}$就有对数法则: 
\begin{blk}{性质3}
    一个幂的对数,等于幂底数的对数与指数的
积,即
\[\log_a M^u =u\log_a M\]
\end{blk}
  
\begin{proof}
    将指数法则$(a^m)^n=a^{mn}$应用到(1.5)得到
\[M^u=\left(a^{\log_a M}\right)^u=a^{u\log_a M}\]
根据对数定义得到
\[\log_a M^u = u\log_a M\]
\end{proof}
    
性质3的特殊情况是:

\begin{blk}{性质4}
 $M(M>0)$的$n$次算术根的对数等于根指数的
倒数乘以被开方数的对数,即
\[\log_a\sqrt[n]{M}=\frac{1}{n}\log_a M\]
\end{blk}

\begin{rmk}
    \begin{enumerate}
        \item 由性质2可得$\log_a\frac{1}{M}=-\log_a M\; (M>0)$;
        \item 一般来说,$\log_a(M\pm N)\ne \log_a M\pm \log_a N$;
        \item 上述四个性质说明,如果先对算式取对数,即把算式
        过渡到对数,然后根据对数运算法则,两个数相乘与相除可以
        化为它们的对数相加与相减,一个数的乘方与开方可以化为
        把它的对数乘以或除以指数,这样计算简捷得多,以后我们还
        要进一步说明,如何把计算对数的结果再还原到算式的结果。
    \end{enumerate}
\end{rmk}
    
\begin{example}
已知 $\log _{10} 2=0.3010$, 求 $\log_{10} 5, \log_{10} 40$,
    $\log_{10} 5000$
\end{example}

\begin{solution}
\[\begin{split}
  \log _{10} 5&=\log _{10} \frac{10}{2}=\log _{10} 10-\log _{10} 2=1-0.3010=0.6990 \\
   \log _{10} 40&=\log_{10}(4 \times 10)\\
   &=\log_{10} 2^{2}+\log_{10} 10=2 \log_{10} 2+1\\
    &=2 \times(0.3010)+1=1.6020\\
    \log _{10} 5000&=\log _{10}(5 \times 1000)=\log _{10} 5+\log _{10} 10^{3}\\
    &=\left(\log_{10} 10-\log_{10} 2\right)+3 \log _{10} 10\\
    &=4-0.3010=3.6990
\end{split}\]
\end{solution}


\begin{example}
已知$\log_{10}2=0.3010$,$\log_{10}3=0.4771$,求$\log _{10} 0.2$, $\log _{10} \sqrt{1.5}$, $\log _{10} 450$
\end{example}    

\begin{solution}
    \[\begin{split}
        \log _{10} 0.2 &=\log _{10} \frac{2}{10}=\log _{10} 2-\log _{10} 10\\ 
        &=0.3010-1 =-0.6990\\
        \log _{10} \sqrt{1.5} &=\frac{1}{2} \log _{10} \frac{3}{2}=\frac{1}{2}\left(\log _{10} 3-\log _{10} 2\right)\\
        &=\frac{1}{2}(0.4771-0.3010)=0.08805\\
        \log _{10} 450 &=\log_{10}\frac{3^2\x 10^2}{2}=\log_{10}(3\x 10)^2-\log_{10} 2\\
        &=2\left(\log_{10}3+\log_{10}10\right) -\log_{10}2\\
        &=2(0.4771+1)-0.3010=2.6532
\end{split}\]
注意:在解题时常用到$\log_a a=1$, $\log_a 1=0$。
\end{solution}    

求任何单项式的对数就是求该式中各因式的对数的代数
和。
    
\begin{example}
    已知$y=\frac{(a-b)^3\sqrt[3]{c}}{\sqrt[5]{(a+b)^2d^3}}$,求$\log_c y$
\end{example}

\begin{solution}
\[\begin{split}
    \log_c y&=\log_c \frac{(a-b)^3\sqrt[3]{c}}{\sqrt[5]{(a+b)^2d^3}}\\ 
&=\log_c (a-b)^3\cdot \sqrt[3]{c}  -\log_c \sqrt[5]{(a+b)^2d^3}\\
&=\log_c (a-b)^3+ \log_c \sqrt[3]{c}  -\frac{1}{5}\log_c {(a+b)^2d^3}\\
&=3\log_c (a-b)+\frac{1}{3}\log_c  c-\frac{1}{5}[2\log_c (a+b)+3\log_c d]\\
&=3\log_c (a-b)+\frac{1}{3}-\frac{2}{5}\log_c (a+b)-\frac{3}{5}\log_c d
\end{split}\]
\end{solution}

所谓代数式的还原法就是由一个单项式的对数式反求这
个单项式。

\begin{example}
已知$\log_2 x=\log_2 a+2\log_2 b-\frac{1}{3}\log_2 c$,求$x$。
\end{example}

\begin{solution}
    \[\log_2 x=\log_2 a+\log_2 b^2-\log_2 \sqrt[3]{c}=\log_2 ab^2-\log_2 \sqrt[3]{c}
=\log_2 \frac{ab^2}{\sqrt[3]{c}}\]
根据同底的两个对数相等,则它的真数相等,得出
\[x=\frac{ab^2}{\sqrt[3]{c}}\]
\end{solution}    


\begin{example}
    已知$\log_a N=\log_ac+b$, 求$N$。
\end{example}


\begin{solution}
\[\log_a N=\log_a c+\log_a a^b=\log_a c\cdot a^b\]
根据同底的两个对数相等则其真数相等,得出
\[N=c\cdot a^b\] 
\end{solution}

\begin{blk}{性质5}
    以$b$为底$N$的对数,除以以$b$为底$a$的对数的商,
可以换作以$a$为底$N$的对数
\[\log_a N=\frac{\log_b N}{\log_b a}\]
这个公式叫做换底公式。
\end{blk}


\begin{proof}
    因为$N=a^{\log_a N}$,两边取以b为底的对数,得到
    \[\log_b N = \log_b (a^{log_aN} )\]
    应用对数法则(性质3), 得到
\[\log_b N=(\log_a N)\log_ba\]
即
\[\log_a N=\frac{\log_b N}{\log_b a}\]

\end{proof}


\begin{blk}{推论}
    $$\log_ab=\frac{1}{\log_ba}$$
\end{blk}


\begin{example}
已知$\log_{10}2=0.3010$,求$\log_2 5$。
\end{example}

\begin{solution}
\[\log_2 5=\log_2 \frac{10}{2}=\log_2 10-\log_2 2=\frac{1}{\log_{10}2}-1=\frac{1}{0.3010}-1=\frac{699}{301}\] 
\end{solution}    

\begin{example}
    已知$\log_{10}2=a$, $\log_{10}7=b$,求$\log_{8}9.8$。
\end{example}

\begin{solution}
\[\begin{split}
    \log_{8}9.8&=\frac{\log_{10}9.8}{\log_{10}8}=\frac{\log_{10}\frac{2\x 7^2}{10}}{\log_{10}2^3}\\
    &=\frac{\log_{10}2+2\log_{10}7-\log_{10}10}{3\log_{10}2}\\
    &=\frac{a+2b-1}{3a}
\end{split}\]
\end{solution}

\section*{习题1.3}
\addcontentsline{toc}{subsection}{习题1.3}
\begin{enumerate}
    \item 利用对数性质将下面的式子变形:
    \begin{multicols}{2}
\begin{enumerate}
    \item $\log_a a^2b^2$
    \item $\log_x x\sqrt{y}$
    \item $\log_x\frac{\sqrt[3]{x}}{y^2}$
    \item $\log_a\frac{b^3}{\sqrt{a}}$
    \item $\log_a(ab)^3$
    \item $\log_a\left(\frac{x}{y}\right)^4$
    \item $\log_a\sqrt[4]{\frac{x^3}{y}}$
    \item $\log_a\frac{uv^3}{w^2}$
\end{enumerate}        
    \end{multicols}

\item 求下列各式中$\log_{10}x$的展开式:
\begin{enumerate}
    \item $x=6a\frac{\sqrt{2(a-b)c}}{5(a-b)^2}$
    \item $x=5m^{\tfrac{3}{4}}n^{\tfrac{1}{3}}\sqrt[3]{\frac{2\cos\alpha}{3}},\quad (0^{\circ}\le \alpha\le 90^{\circ})$
    \item $x=\sqrt[5]{\left(\frac{1}{a^2b^2\sqrt[4]{c^3}}\right)^3}$
    \item $x=\sqrt{\frac{\sqrt{ab}}{a^{-1}}\cdot \sqrt[3]{a^{-1}b^{-2}}}$
\end{enumerate}
\item 求证:
\begin{enumerate}
    \item $\log_a (e^x+2+e^{-x})+\log_a(e^x-2+e^{-x})=2\log_a(e^x-e^{-x})$
    \item $\log_a(m^3+3m^2+3m+1)-\log_a(n^3+3n^2+3n+1)=3\log_a\frac{m+1}{n+1}$
\end{enumerate}

\item 已知$\log_{10} x=\log_{10} a+\log_{10} b-3\log_{10} c$, 求$x$。
\item 求下列各式的值:
\begin{multicols}{2}
\begin{enumerate}
    \item $\frac{1}{2}\log_{3} 9$
    \item $\log_{4} 2+\log_{4} 8$
    \item $3\log_{10} 2-\log_{10} 80$
    \item $\log_{5} 10-\log_{5}\frac{2}{\sqrt{5}} $
\end{enumerate}
\end{multicols}
\item 若$\log_{10} 2=0.3010$,
计算$\log_{2} 13+\frac{1}{2}\log_{2} 25+\log_{2} \frac{4}{7}-\log_{2}\frac{13}{35} $
\item 已知$\log_{10} 2=0.3010$, $\log_{10}3=0.4771$,
求$\log_4 3$和$\log_3 2$

\item  试证$\log_{a^k}b=\frac{\log_ab}{k}$

\item 计算:
$(\log_23+\log_49)(\log_34 +\log_92)$
\item 计算:$\log_ab\cdot \log_b c\cdot \log_ca$ ($a,b,c$是不等于1的正
数)。
\end{enumerate}  

\subsection{常用对数}
\subsubsection{近似数计算常识}
本节要介绍常用对数表,对数表所列对数几乎全是近似
数,我们不仅在数学用表中遇到近似数,其实在日常生活
中,有关度量和计算大量物件或计算无理数的的结果,都得
用近似数来表示。因此,在介绍常用对数之前,我们特别介
绍下面几点关于近似数的常识。

在度量和计算中所遇到的数有两种情况:一种是能够
用一个数准确地表示某一个量。例如,在一堂课内出席学生
50人,一星期有7天,这些数都与实际完全符合,叫做该量
的准确量。但是在中午一小时内,经过北京天安门广场的人
数,就未必能用一个数准确地表示,又比如说约2000人出席
全校大会,“约”这个字说明2000这数是近似的.在计算大量数
目的物件时,常常要用近似数来表示。

度量一个量,都不能做到绝对准确,所得结果总会含有
一些误差的,这误差的大小,要看度量仪器的质量以及作度
量的人是否有经验而定。例如,若用最小分度为1毫米的尺
子量一金属杆的长度时,所得结果为1.234米,这表示真实长
度在1.2335米和1.2345米之间.

\begin{blk}{定义1}
    一个量的准确数(有时说准确值)与近似数(或
    近似值)的差,叫做这个\textbf{近似数(值)的误差}。
\end{blk}


设$x$是某量的准确数(值),并且$x=3.283$, 则称$a=3.2$
是$x$的一个不足近似数(值),$b=3.3$是$x$的一个过剩近似数
(值)。把$x$用它的不足近似数$a$来代替,所产生的误差是:
\[x-a=3.283-3.2=0.083\]
如果取$x$的过剩近似数(值),那么这个近似值的误差是:
\[x-b=3.283-3.3=-0.017\]

显然,不足近似值的误差总是正的,而过剩近似值的误
差总是负的。

\begin{blk}{定义2 }
    一个量的准确数与近似数的差的绝对值,叫做
这个\textbf{近似数的绝对误差}。
\end{blk}


在我们所举的例子中,不足近似值$a$的绝对误差:
\[|x-a|=|0.083|=0.083\]
而过剩近似值$b$的绝对误差:
\[|x-b|=|-0.017|=0.017\]

近似数(值)的绝对误差,表示近似数和该准确数相差多
少,绝对误差越小说明这个近似数越精确。

在度量长度时,一个量的准确长度是不知道的,但可以
估计它在哪个范围内。因此近似数的绝对误差的准确值也是
不知道的,但可以估计出近似数的绝对误差不超过什么数。

\begin{blk}{定义3}
    设$x$是某量的准确数,$a$是它的近似的数(值),如
    果这个近似数(值)的绝对误差不超过数$h$, 也就是说满足条
    件$|x-a|\le h$, 那么$a$叫做$x$的\textbf{精确到$h$的近似数(值)}。    
\end{blk}

例如:$|x-a|=0.083<0.1,\quad |x-b|=0.017<0.1$,因此:
3.2和3.3分别叫做3.283的精确到0.1的不足
近似值和过剩近似值。

前面用最小分度为1毫米的尺子度量金属杆所得到的近
似值1.234的绝对误差不超过0.0005(米),因此它是精确到
万分之五(米)的近似值。

\subsubsection{数的四舍五入法和近似数的有效数字}
准确数据四舍五入时,也会出现近似数,如果某校有
2003人,当参观者问校长说“学校有多少人?”时,他大概会
说四舍五入得约数2000人。

把分数$\frac{2}{13}$
化为小数时,得到无穷循环小数$\frac{2}{13}\approx 0.\dot{1}5384\dot{6}$

在实际计算上,我们常把这种小数取到小数点后某一位
为止,比方说取到千分之一位,而按四舍五入法舍去其后各
位数字,于是得:
$\frac{2}{13}\approx 0.154$。

四舍五入法就是略去几位数字时,如果略去的第一位数
字小于5, 那么把留下的最末一位数字保留不动;如果略去
的第一位数字大于或等于5, 那就把留下的最末一位数字加
上1, 用四舍五入法写出来的数,比单单略去最后几位数后写
出的数准确些。

例如,0.154就比0.153更接近$\frac{2}{13}$。

$\because\quad \left|\frac{2}{13}-0.154\right|<0.0002,\quad \left|\frac{2}{13}-0.153\right|<0.0009$

$\therefore\quad $0.151比0.153更接近$\frac{2}{13}$。

估计近似数的准确度的最简便的方法之一,是算出它的
有效数字的个数。

\begin{blk}{定义4}
    若近似数的误差不超过其末一位数字的一-个单
位,那么该数的一切数字;除了左起第一个非零数字之前的
荐以外,都叫有效数字。
\end{blk}


例如,0.154和0.153都是
的取3位有效数字的数,即
$\frac{2}{13}$
的近似值0.154或0.153具有三位有效数字。把准确数
1.9996四舍五入到千分之一位,得2.000, 这时所有四个数字,
连三个零在内,都是有效数字,即它是有四个有效数字的
数。

\begin{blk}{定义5}
    若近似数的误差大于最末一位数字的一个单
    位,那么它的最末一个数字叫不可靠的。  
\end{blk}

把0.1535作为$\frac{2}{13}$
的近似数时,那么它的最末一个数字5
就是不可靠的,因为
$\left|\frac{2}{13}-0.1535\right|>0.0003$

把学校人数2135四舍五入到百位数得2100,这里最末两
个零是不可靠的数字,2100是有两个有效数字的数。

如果近似数末后的零不是有效数字,我们规定把它写得
小一些,这样21{\small00}表示具有两个有效数字的近似数。

用科学记数法表示的数在10的方幂前面的数字代表有效
数字。

例如,0.0000437有三位有效数字,它的标准写法是
$4.37\x10^{-8}$, 前面2100的标准写法是$2.1\x10^3$.

在计算近似数时,我们必须指明精确到几位有效数字。
例如0.09235716, 若取三位有效数字的精确度,我们就写成:
$9.24\x10^{-2}$. 若是$70,454,620,000$取四位有效数字的精确度
就写成$7.045\x10^{10}$.

\begin{rmk}
前面例子有效数字中,最后一位数字都已经过四
舍五入的处理了,如果最后一例中,取二位有效数字的精
度,就要写作$7.0\x10^{10}$, 不可以写作$7\x10^{10}$。  

不要把“小数点后的数字”跟“有效数字”等同起来,数0.000175有三个有效数字,但总共有六个小数点后的数字。
\end{rmk}

关于近似数的计算规则如下:
\begin{blk}{法则1}
    近似数相加(减)时,其中小数位数最少的近似
    数有几个数位,它们的和(差)就保留几个小数数位。  
\end{blk}

\begin{example}
    已知$\pi\approx 3.1416$, $a=2.4$, $b=0.047$。
求$\pi+a+b$。
\end{example}    

\begin{solution}



    
\end{solution}

\begin{example}
    
\end{example}
    
\begin{example}
    
\end{example}
    
\begin{solution}
    
\end{solution}
\begin{example}
    
\end{example}



\begin{solution}
    
\end{solution}

\begin{example}
    
\end{example}

\begin{solution}
    
\end{solution}
    
\begin{example}
    
\end{example}

\begin{solution}
    
\end{solution}

\begin{example}
    
\end{example}

\begin{solution}
    
\end{solution}

\begin{example}
    
\end{example}


\begin{solution}
    
\end{solution}


\begin{example}
    
\end{example}

\begin{solution}
    
\end{solution}    
\begin{example}
    
\end{example}

\begin{solution}
    
\end{solution}


\begin{example}
    
\end{example}

\begin{solution}
    
\end{solution}    
\begin{example}
    
\end{example}

\begin{solution}
    
\end{solution}






































































































































































